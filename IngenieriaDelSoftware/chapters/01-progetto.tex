\chapter{Progetto - 100 Progetti per il comune di Trento}
    \section{Presentazione}
        \subsection{La struttura del comune}
            \subsubsection{La struttura politica}
                \paragraph{Il Sindaco} Eletto direttamente dai cittadini, è il massimo rappresentante del comune. è l'equivalente di un CEO di un'azienda.
                \paragraph{Consiglio comunale} Organo di indirizzo politico, composto da 40 consiglieri eletti dai cittadini.
                    \subsubsection{La struttura amministrazione}
                        \paragraph{24 Strutture} Sono i dipartimenti del comune, ognuno con un responsabile. LE strutture possono essere raggruppate in 3 macro categorie:
                            \subparagraph{Servizi di staff} Sono i servizi che supportano il funzionamento dell'ente, come ad esempio il servizio legale, il servizio informatico, il servizio finanziario, ecc.
                            \subparagraph{Servizi alla comunità/cittadino} Sono i servizi che si occupano di fornire servizi alla cittadinanza, come ad esempio il servizio sociale, il servizio scolastico, il servizio culturale, ecc.
                            \subparagraph{Servizi tecnici} Sono i servizi che si occupano di gestire il territorio, come ad esempio il servizio urbanistico, il servizio ambiente, il servizio lavori pubblici, ecc.
        \subsection{Il Servizio Innovazione e Transizione Digitale}
            \subsubsection{Il servizio}
                \paragraph{La composizione del servizio} Il servizio è composto da 3 uffici:
                    \begin{itemize}
                        \item Protocollo e Archivio
                        \item Servizi Cloud e infrastrutture IT
                        \item Servizi Applicativi e cartografici
                    \end{itemize}
        \subsection{Norme di riferimento} Le principali norme di riferimento sono:
            \begin{itemize}
                \item Codice dell'amministrazione digitale
                \item Codice dei contratti pubblici
                \item Codice del terzo settore
                \item Codice della privacy
                \item Codice della trasparenza
            \end{itemize}
        \subsection{La struttura del servizio}
            \{ Immagine della struttura del servizio \}
            \subsubsection{Architettura DB2}
                \{ Immagine dell'architettura DB2 \}
            \subsubsection{Architettura Cartografia}
                \{ Immagine dell'architettura cartografia \}
        \subsection{Aspetti da considerare}
            \paragraph{Cybersicurezza} Seguendo la normativa della strategia nazionale di cybersicurezza.
            \paragraph{Privacy} Usando il criterio della privacy by design.
        \subsection{Il digital hub}
        \paragraph{La storia} NEgli ultimi anni con la collaborazione di FBK e UniTN è stato creato un digital hub per rendere più efficaci i servizi digitali del comune e per prendere decisioni basate sui dati.
        \subsubsection{Architettura digital hub}
                \{ Immagine dell'architettura digital hub \}
        \subsubsection{Use Cases}
            \paragraph{Analisi Parcheggi} Grazie al digital hub è stato possibile verificare l'effettivo uso corretto dei parcheggi riservati disabili e carico scarico attorno al centro storico di Trento, i dati sono disponibili tramite apposite dashboard.
            \paragraph{Analisi dei parcheggi dei monopattini} è stato possibile analizzare l'uso dei parcheggi per i monopattini elettrici e verificare se pre/post normativa questi vengono effettivamente utilizzati.
            \paragraph{Servizio OnOff} è stato possibile analizzare l'uso del servizio OnOff e verificare quando viene utilizzato e se il servizio è attivato con efficienza ed se è necessario aggiungere e/o rimuovere facie orarie di servizio.
        \subsubsection{Open Data}
            \href{https://www.comune.trento.it/Aree-tematiche/Open-Data}{https://www.comune.trento.it/Aree-tematiche/Open-Data} 
        \subsubsection{Smart City Control Room}
            \paragraph{Obbiettivi}
                \begin{itemize}
                    \item Analisi della sosta
                    \item Eventi esterni
                    \item Mobilità leggera
                    \item TPL
                \end{itemize}
    \section{Requisiti}
