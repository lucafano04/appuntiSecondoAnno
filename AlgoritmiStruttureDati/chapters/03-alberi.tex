\chapter{Alberi}
\thispagestyle{chapterInit}
\section{Introduzione}
    \paragraph{Albero Radicato (\textit{Rooted Tree})}
        \begin{definition}
            Un albero intero consiste in un insieme di nodi orientati che connettono coppie di nodi, con le seguenti proprietà:
            \begin{enumerate}
                \item Un nodo dell'albero è designato come nodo \textbf{radice}
                \item Ogni nodo $ n $, a parte la radice, ha esattamente un arco entrante
                \item Esiste un cammino unico dalla radice ad ogni nodo
                \item L'albero è connesso
            \end{enumerate}
        \end{definition}
    \paragraph{Albero Radicato Ricorsivo}
        \begin{definition}
            Un albero è dato da:
            \begin{enumerate}
                \item Un insieme vuoto, oppure
                \item Un nodo \textbf{radice} e zero o più \textbf{sotto-alberi}, ognuno dei quali è un albero.
                    \subitem La radice è connessa ai sotto-alberi tramite archi orientati.
            \end{enumerate}
        \end{definition}
    \paragraph{Profondità nodi (\textit{Depth})}
        \begin{definition}
            Si definisce \textbf{profondità} di un nodo $ n $ la lunghezza del cammino semplice dalla radice al nodo $ n $ (misurato in numero di archi).
        \end{definition}
    \paragraph{Livello (\textit{Level})}
        \begin{definition}
            Si definisce come \textbf{livello} di un albero l'insieme di tutti i nodi alla stessa profondità.
        \end{definition}
    \paragraph{Altezza dell'albero (\textit{Height})}
        \begin{definition}
            Si definisce come \textbf{altezza} di un albero la profondità massima dei nodi dell'albero.
        \end{definition}
\section{Alberi Binari}
    \begin{definition}
        Un \textbf{albero binario} è un albero radicato nel quale ogni nodo ha al massimo due figli, identificati come figlio \textbf{sinistro} e figlio \textbf{destro}.
    \end{definition}
    \subsubsection{Specifica delle \texttt{API}}

        \begin{algorithm}[H]
            \caption{Tree}
            \% Costituisce unb nuovo nodo, contenente $v$, senza figli o genitori. \newline
            $\operatorname{Tree}(\operatorname{Item} v) $ \newline
            \% Legge il valore memorizzato nel nodo \newline
            $\operatorname{Item} \operatorname{read}()$ \newline
            \% Modifica il valore memorizzato nel nodo \newline
            $\operatorname{write}(\operatorname{Item} v)$ \newline
            \% Restituisce il padre, oppure \textbf{nil} se questo nodo è radice \newline
            $\operatorname{Tree} \operatorname{parent}() $ \newline
            \% Restituisce il figlio sinistro (destro) di questo nodo; restituisce \textbf{nil} se assente
            $\operatorname{Tree} \operatorname{left}()$ \newline
            $\operatorname{Tree} \operatorname{right}()$ \newline
            \% Inserisce il sotto-albero radicato in t come figlio sinistro (destro) di questo nodo\newline
            $\operatorname{insertLeft}(\operatorname{Tree} t)$ \newline
            $\operatorname{insertRight}(\operatorname{Tree} t) $\newline
            \% Distrugge (ricorsivamente) il figlio sinistro (destro) di questo nodo \newline
            $\operatorname{deleteLeft}()$ \newline
            $\operatorname{deleteRight}()$
        \end{algorithm}
    \subsection{Implementazione}
        \paragraph{Campi memorizzati nei nodi:}\begin{description}
            \item[\textit{parent}] Puntatore al nodo padre
            \item[\textit{left}] Puntatore al figlio sinistro
            \item[\textit{right}] Puntatore al figlio destro
        \end{description}
        \paragraph{Operazioni di base:} Implementazione operazioni \texttt{API}:
        \begin{algorithm}[H]
            \caption{Tree}
            \begin{algorithmic}
                \Function{Tree}{\Item $v$}
                    \State \Tree $t \gets$ \New \Tree
                    \State $t.\textit{value} \gets v$
                    \State $t.\textit{left} \gets t.\textit{right} \gets$ 
                    \State $t.\textit{parent} \gets \textbf{nil}$
                    \State \Return $t$
                \EndFunction
                \Function{insertLeft}{\Tree $t$}
                    \If{$left == \textbf{nil}$}
                        \State $left \gets t$
                        \State $t.\textit{parent} \gets \textit{this}$
                    \EndIf
                \EndFunction
                \Function{insertRight}{\Tree $t$}
                    \If{$right == \textbf{nil}$}
                        \State $right \gets t$
                        \State $t.\textit{parent} \gets \textit{this}$
                    \EndIf
                \EndFunction
                \Function{deleteLeft}{}
                    \If{$left \neq \textbf{nil}$}
                        \State $left.\textit{deleteLeft}()$
                        \State $left.\textit{deleteRight}()$
                        \State $left \gets \textbf{nil}$
                    \EndIf
                \EndFunction
                \algstore{bkbreak}
            \end{algorithmic}
        \end{algorithm}
        \begin{algorithm}[H]
            \begin{algorithmic}
                \algrestore{bkbreak}
                \Function{deleteRight}{}
                    \If{$right \neq \textbf{nil}$}
                        \State $right.\textit{deleteLeft}()$
                        \State $right.\textit{deleteRight}()$
                        \State $right \gets \textbf{nil}$
                    \EndIf
                \EndFunction
            \end{algorithmic}
        \end{algorithm}
    \subsection{Visite}
        \paragraph{Visita di un albero / ricerca} Una \textbf{visita} è una strategia per analizzare (visitare) tutti i nodi di un albero. Le visite possono essere:
        \begin{description}
            \item[Visita in profondità (\textit{Depth-First search} \texttt{DFS})] Usata per visitare un albero, si visita ricorsivamente ognuno dei suoi \textbf{sotto-alberi}.
                \subitem Tre varianti: pre/in/post visita
                \subitem Richiede uno \textbf{stack}
            \item[Visita in ampiezza (\textit{Breadth-First search} \texttt{BFS})] Usata per visitare ogni \textbf{livello} dell'albero, partendo dalla radice.
                \subitem Richiede una \textbf{queue}
        \end{description}
        \begin{algorithm}[H]
            \caption{dfs(\Tree t>)}
            \begin{algorithmic}
                \If{$t \neq \textbf{nil}$}
                    \State \% pre-order visit of $t$
                    \State \Print $t$
                    \State $\operatorname{dfs}(t.\textit{left})$
                    \State \% in-order visit of $t$
                    \State \Print $t$
                    \State $\operatorname{dfs}(t.\textit{right})$
                    \State \% post-order visit of $t$
                    \State \Print $t$
                \EndIf
            \end{algorithmic}
        \end{algorithm}
        \subsubsection{Esmempi di applicazione}
            Contare i nodi in post-visita:
            \begin{algorithm}[H]
                \caption{countNodes(\Tree $t$)}
                \begin{algorithmic}
                    \If{$t \neq \textbf{nil}$}
                        \State $c \gets 1$
                        \State $c \gets c + \operatorname{countNodes}(t.\textit{left})$
                        \State $c \gets c + \operatorname{countNodes}(t.\textit{right})$
                        \State \Return $c$
                    \Else
                        \State \Return $0$
                    \EndIf
                \end{algorithmic}
            \end{algorithm}
            Stampare espressioni con In-visita:
            \begin{algorithm}[H]
                \caption{\Int printExpression(\Tree $t$)}
                \begin{algorithmic}
                    \If{$t.\operatorname{left}() == \Nil \textbf{and} t.\operatorname{right}() == \Nil$}
                        \State \Print $t.\operatorname{read}()$
                    \Else
                        \State \Print "("
                        \State \Print $\operatorname{printExpression}(t.\operatorname{left}())$
                        \State \Print $t.\operatorname{read}()$
                        \State \Print $\operatorname{printExpression}(t.\operatorname{right}())$
                        \State \Print ")"
                    \EndIf
                \end{algorithmic}
            \end{algorithm}
\section{Alberi Generici}
    \begin{definition}
        Un \textbf{albero generico} è un albero radicato nel quale ogni nodo può avere un numero arbitrario di figli.
    \end{definition}
    \subsubsection{Definizione delle \texttt{API}}
    
        \begin{algorithm}[H]
            \caption{Tree}
            \% Costituisce unb nuovo nodo, contenente $v$, senza figli o genitori. \newline
            $\operatorname{Tree}(\operatorname{Item} v) $ \newline
            \% Legge il valore memorizzato nel nodo \newline
            $\operatorname{Item} \operatorname{read}()$ \newline
            \% Modifica il valore memorizzato nel nodo \newline
            $\operatorname{write}(\operatorname{Item} v)$ \newline
            \% Restituisce il padre, oppure \textbf{nil} se questo nodo è radice \newline
            $\operatorname{Tree} \operatorname{parent}() $ \newline
            \% Restituisce il primo figlio, oppure \textbf{nil} se è una foglia \newline
            $\operatorname{Tree} \operatorname{leftmostChild}()$ \newline
            \% Restituisce il prossimo fratello, oppure \textbf{nil} se è l'ultimo figlio \newline
            $\operatorname{Tree} \operatorname{rightSibling}()$ \newline
            \% Inserisce il sotto-albero radicato in t come primo figlio di questo nodo\newline
            $\operatorname{insertChild}(\operatorname{Tree} t)$ \newline
            \% Inserisce il sotto-albero radicato in t come prossimo fratello di questo nodo\newline
            $\operatorname{insertSibling}(\operatorname{Tree} t)$ \newline
            \% Distrugge (ricorsivamente) l'albero radicato identificato dal primo figlio di questo nodo \newline
            $\operatorname{deleteChild}()$ \newline
            \% Distrugge (ricorsivamente) l'albero radicato identificato dal prossimo fratello di questo nodo \newline
            $\operatorname{deleteSibling}()$
        \end{algorithm}
    \subsection{Visite}
        \subsubsection{\textit{Breadh-First Search}}
            \begin{algorithm}[H]
                \caption{bfs(\Tree $t$)}
                \begin{algorithmic}
                    \State \Queue $Q$ = $\operatorname{Queue}()$
                    \State $Q.\operatorname{enqueue}(t)$
                    \While{\textbf{not} $Q.\operatorname{isEmpty}()$}
                        \State \Tree $u \gets Q.\operatorname{dequeue}()$
                        \State \Print $u$
                        \State $ u \gets u.\operatorname{leftmostChild}()$
                    \algstore{bfsbreak}
                \end{algorithmic}
            \end{algorithm}
            \begin{algorithm}[H]
                \begin{algorithmic}
                    \algrestore{bfsbreak}
                        \While{$u \neq \textbf{nil}$}
                            \State $Q.\operatorname{enqueue}(u)$
                            \State $u \gets u.\operatorname{rightSibling}()$
                        \EndWhile
                    \EndWhile
                \end{algorithmic}
            \end{algorithm}
        \subsubsection{Memorizzazione}
            Esistono diversi modi per memorizzare un albero, più o meno indicati a seconda del numero massimo e medi dei figli presenti:
            \begin{enumerate}
                \item Realizzazione con vettore di figli
                \item Realizzazione con primo figlio e prossimo fratello
                \item Realizzazione con vettore dei padri
            \end{enumerate}


