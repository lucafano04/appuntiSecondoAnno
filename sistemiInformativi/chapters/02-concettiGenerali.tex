\chapter{Concetti Generali sull'Informatica Aziendale}
\thispagestyle{chapterInit}
\section{Introduzione e definizioni}
    \paragraph{Informatica Aziendale} L'\textbf{informatica aziendale} è la disciplina che studia l'applicazione dell'informatica nelle aziende, studia inoltre l'influenza di questa nelle diverse categorie di un sistema aziendale. Esistono diversi settori di applicazione trai quali:
        \begin{itemize}
            \item \textbf{Aiuto e guida operativa} - Assistenza agli operatori a seguire le corrette procedure di lavoro con un costante controllo iterativo sui dati. Facilitazione di ricerca e recupero di informazioni.
            \item \textbf{Organizzativa} - Automazione di processi da un lato, richiesta di \textbf{competenze} e \textbf{risorse} differenti dall'altro.
            \item \textbf{Controllo} - Rilevazione di caratteristiche e comportamenti di un sistema, possibilità di \textbf{analisi quantitative} e \textbf{qualitative}.
            \item \textbf{Strategia} - Supporto ai processi di trasformazione e innovazione, supporto alle decisioni strategiche. 
        \end{itemize}
    \paragraph{Sistemi Informativi aziendali} I \textbf{sistemi informativi aziendali} sono l'insieme delle procedure e delle infrastrutture che definiscono e supportano l'elaborazione, la distribuzione e l'utilizzo delle informazioni all'interno di una azienda. Molto spesso ci si basa su una infrastruttura elettronica. È importante non confondere i sistemi informativi con i sistemi informatici, infatti è vero che ogni sistema informatico è un sistema informativo, ma non è vero il contrario.
    \paragraph{Risorse e processi}
        \subparagraph{Risorsa } Una \textbf{risorsa} "è tutto ciò con cui l'organizzazione opera" sia che questo possa essere un bene fisico o che questo sia un bene immateriale
        \subparagraph{Processo } Un \textbf{processo} è un insieme di attività atte a gestire una risorsa nel suo ciclo di vita.
\section{Sistema informativo aziendale}
    \paragraph{Definizione} Un \textbf{sistema informativo aziendale} è un sistema che permette di raccogliere, elaborare, memorizzare e distribuire informazioni all'interno di un'organizzazione. Questo sistema si compone di:
    \begin{itemize}
        \item \textbf{Dati}: \begin{itemize}
            \item di Configurazione - Dati che descrivono la struttura dell'organizzazione
            \item operativi - Dati che descrivono le attività dell'organizzazione
            \item di supporto - Dati che supportano le attività dell'organizzazione
            \item di stato - Dati che descrivono lo stato dell'organizzazione
        \end{itemize}
        \item \textbf{Procedure}: \begin{itemize}
            \item acquisizione - Raccolta di dati
            \item controllo ed elaborazione - Controllo e manipolazione dei dati
            \item pianificazione
        \end{itemize}
        \item \textbf{Mezzi e strumenti}: \begin{itemize}
            \item Hardware - sever e periferiche
            \item Stazioni di lavoro
            \item \dots
        \end{itemize}
    \end{itemize}
\section{Impatto dell'informatica nelle azienda}
    \subsection{Questioni da rispettare} 
        Il sistema \textbf{informatico} aziendale deve rispettare alcuni criteri per essere considerato adeguato:
        \begin{description}
            \item[Livello di astrazione] Il sistema deve essere in grado di rappresentare la realtà aziendale in modo corretto, sintetico ma completo.
            \item[Tempestività] Il sistema deve essere in grado di fornire le informazioni in tempo utile ed appropriato al contesto dell'operazione e della mole di dati.
            \item[Livello di copertura] Il sistema deve essere in grado di coprire tutte le aree aziendali e tutti i processi aziendali nei vari livelli di dettaglio.
        \end{description}
        Allo stesso tempo il sistema informativo deve \textbf{garantire}: Accessibilità dei dati e Correttezza del flusso, flusso che si divide in:
        \begin{description}
            \item[Orizzontale] tra le varie aree aziendali
            \item[Verticale] tra i vari livelli gerarchici
        \end{description}
    \subsection{Processi classi di un sistema informativo}
        Esistono tre classici processi informatizzati comuni a tutte le aziende che adottano un sistema informativo informatico:
        \begin{description}
            \item[Sviluppo funzioni operative] - Processo che si occupa di automatizzare dei processi che sono già presenti andando a ridurre i tempi e la mano d'opera necessaria.
            \item[Pianificazione] - Processo che prende i dati inseriti nel \texttt{SI} e li elabora per automatizzare processi di pianificazione.
            \item[Controllo] - Processo che renda automatico il controllo dei dati i inseriti nel \texttt{SI} e li confronta con criteri e dati di riferimento segnalando eventuali anomalie.
        \end{description}
    \label{subsec:nuoviProcessi}
    \subsection{Nuovi processi}
        \paragraph{Introduzione dell'informatica} L'introduzione dell'informatica in azienda non si occupa semplicemente di supporto a processi già esistenti, ma col tempo si aprono nuovi processi, innovativi, che prima non erano possibili.
        \paragraph{BRP} Nasce da questa idea il concetto di \textbf{Business Process Re-engineering} o \textbf{Reingegnerizzazione dei processi aziendali} che consiste nel ripensare e ridisegnare i processi aziendali per sfruttare al meglio le nuove tecnologie informatiche. La spinta verso il processo è generata dalla vasta adozione delle reti informatiche.
        \paragraph{Contatto col cliente al tempo di internet} Con l'avvento di internet e delle reti informatiche, il contatto con il cliente assume delle modalità completamente nuove, si passa da un contatto diretto a un contatto mediato da un sistema informatico, che per certi versi può essere più efficiente e più efficace.
\section{I sistemi informativi nelle aziende Italiane}
    \paragraph{Le aziende in italia} Le aziende in Italia assumono una conformazione molto differente rispetto al panorama europeo, infatti il 99,9\% delle aziende italiane sono \textbf{PMI} (Piccole e Medie Imprese) e solo lo 0,1\% sono grandi aziende.
    \paragraph{PMI e \texttt{SI}} Le PMI sono aziende che hanno una struttura molto semplice e che spesso l'investire in un sistema informativo non è una priorità visto che i processi sono molto semplici e non richiedono un sistema informativo complesso. \newline
    Spesso quindi un \texttt{SI} potrebbe essere visto come un costo inutile, ma con l'avvento di internet e delle nuove tecnologie, anche le PMI stanno iniziando ad adottare un sistema informativo in piccola misura, ovviamente non adotteranno \texttt{SI} di grandi dimensioni, ma sistemi informativi, spesso italiani in quanto più vicini alla realtà delle PMI, più piccoli e adeguati alle loro esigenze.
