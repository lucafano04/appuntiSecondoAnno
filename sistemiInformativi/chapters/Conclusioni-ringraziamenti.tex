\chapter*{Conclusioni e ringraziamenti}
\addcontentsline{toc}{chapter}{Conclusioni e ringraziamenti}
\thispagestyle{chapterInit}

\section*{Conclusioni}
Con questo si conclude la stesura degli appunti del corso di Sistemi Informativi, tenuto dal prof. Bouquet Paolo presso l'Università degli Studi di Trento. Nell'ambito del corso sono sati trattati diversi argomenti, che vanno dalla più generale società della conoscenza, fino ad arrivare a tematiche più specifiche come l'architettura dei sistemi informativi, le differenze tra sistemi informazionali e sistemi operazionali, il \textit{data warehousing} e \textit{data mining} fino alle presentazioni integrative su \texttt{GDPR}, sistemi \texttt{CRM} e \textit{E-provisioning} elaborate dai colleghi di corso.

\section*{Ringraziamenti}
Ringrazio il prof. Bouquet Paolo per la disponibilità e la passione con cui ha tenuto il corso, rendendo le lezioni interessanti e coinvolgenti. Ringrazio inoltre i colleghi di corso per la collaborazione e la condivisione di materiale e conoscenze, che ha reso possibile la stesura di questi appunti. Particolare ringraziamento và a chi ha seguito il corso vicino a me e con cui ho condiviso le lezioni e le discussioni in aula e fuori (con partitine a carte di mezzo). 

\vfill
{\footnotesize
    \footnotesize\section*{Note}
    Questi appunti sono stati scritti durante il corso di ``Sistemi Informativi'', tenuto dal prof. Bouquet Paolo presso l'Università degli Studi di Trento nell'anno accademico 2024/2025. Gli appunti sono stati scritti in \LaTeX{} e sono disponibili su \href{https://github.com/lucafano04/appuntisecondoanno}{GitHub} e sono rilasciati sotto licenza \href{https://creativecommons.org/licenses/by-nc-sa/4.0/}{CC BY-NC-SA 4.0} come conseguenza sono liberamente utilizzabili e modificabili, ma non possono essere utilizzati a scopi commerciali e devono mantenere la stessa licenza, il materiale rimane liberamente usabile e modificabile nell'ambito accademico, della formazione e della divulgazione scientifica e tecnologica. L'utilizzatore è tenuto a citare l'autore originale e a mantenere la stessa licenza per le opere derivate. Ognuno è libero di usare questi come punto di partenza per lo studio in funzione delle proprie esigenze e di condividerli con chiunque ne possa trarre beneficio, anzi è incoraggiato a farlo.
    L'autore (Luca Facchini) non si assume nessuna responsabilità sull'uso che verrà fatto di questi appunti e non garantisce la completa correttezza e completezza degli stessi, inoltre non si assume nessuna responsabilità per eventuali errori o imprecisioni presenti negli appunti, questi vengono infatti distribuiti \textit{as is} e possono contenere errori o imprecisioni, l'utilizzatore è tenuto a verificare e a correggere eventuali errori presenti negli appunti. Nell'eventualità di errori o imprecisioni si prega di contattare l'autore e/o di aprire una \textit{issue} sul repository di GitHub. (Ultimo aggiornamento: \today)
}