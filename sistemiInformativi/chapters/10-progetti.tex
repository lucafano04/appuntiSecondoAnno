\chapter{Presentazioni}
\thispagestyle{chapterInit}

In questo capitolo andremo ad analizzare e riassumere le presentazioni elaborate da alcuni miei colleghi durante il corso dell'anno accademico 2024/2025, tali presentazioni sono riportate in quanto ritenute dal professore come materiale d'esame.

\section[Il regolamento europeo sulla privacy (\texttt{GDPR})]{Obbiettivi, obblighi e diritti, principali figure di riferimento nel regolamento europeo sulla privacy (\texttt{GDPR})}
\label{sec:gdpr}
{\footnotesize Bellissima presentazione a cura dei miei bellissimi colleghi: \textsc{Cerlenco} Sara, \textsc{Compagni} Davide e \textsc{La Rosa} Stefania.\footnote{Contatti degli autori: Cerlenco S. - \href{mailto:sara.cerlenco@studenti.unitn.it}{sara.cerlenco@studenti.unitn.it}, Compagni D. \href{mailto:davide.compagi@studenti.unitn.it}{davide.compagi@studenti.unitn.it}, La Rosa S. \href{mailto:stefania.larosa@studenti.unitn.it}{stefania.larosa@studenti.unitn.it}
}
    \subsection{Introduzione}
        \paragraph{Introduzione} In generale il \texttt{GDPR} è un regolamento europeo (\texttt{UE 2016/679}) che riguarda il trattamento dei dati personali dei cittadini europei, esso si pone come obbiettivo la protezione ed il controllo di come i dati vengono gestiti, cerca di assicurare chiarezza e trasparenza nel trattamento dei dati personali, e di facilitare la movimentazione dei dati personali all'interno dell'Unione Europea.
        \paragraph{Dati personali e sensibili} Nel regolamento viene definito il concetto di \textit{dati personali} come qualsiasi informazione relativa ad una persona fisica identificata o identificabile, mentre i \textit{dati sensibili} sono dati personali che rivelano l'origine razziale o etnica, le opinioni politiche, le convinzioni religiose o filosofiche, l'appartenenza sindacale, nonché dati genetici, biometrici, relativi alla salute o alla vita sessuale. Su questa tipologia di dati il regolamento è molto più severo e prevede delle restrizioni più stringenti.
        \paragraph{Quando il \texttt{GDPR} si applica e quando no} Il regolamento si applica a tutte le aziende che trattano dati personali di cittadini europei (indipendentemente dalla sede dell'azienda), ma non si applica se non vengono trattati dati personali di cittadini europei (ad esempio se si tratta di un'azienda che non ha clienti europei o usa dati di sole persone giuridiche). Inoltre se i dati trattati riguardano politiche estere, sicurezza nazionale, reati, minacce alla sicurezza pubblica, istituzioni, organi pubblici, ecc. il regolamento non si applica.
        \paragraph{Consenso} Il regolamento prevede che il consenso, sulla quale si basa tutto il regolamento, per il trattamento dei dati personali debba essere \textbf{inequivocabile} ovvero espresso in modo chiaro e non equivocabile, \textbf{libero} ovvero non condizionato, \textbf{specifico} ovvero per uno scopo ben preciso, \textbf{informato} ovvero l'utente deve essere informato su come verranno trattati i dati, \textbf{revocabile} ovvero l'utente può revocare il consenso in qualsiasi momento. Inoltre il consenso deve essere registrato in un apposito registro.
        \paragraph{Trasferimento dati all'estero} Il regolamento prevede che i dati personali non possano essere trasferiti al di fuori dell'Unione Europea se non in presenza di adeguate garanzie che possono essere o le norme dello stato estero, o le \textit{Standard Contractual Clauses} (SCC) ovvero clausole contrattuali standard, o le \textit{Binding Corporate Rules} (BCR) ovvero regole aziendali vincolanti, o i \textit{Consent} ovvero il consenso dell'interessato.
    \subsection{Figure di riferimento}
        \paragraph{Intestatario} Come definito da: \texttt{Art. 4 c. 1} è la persona fisica della quale sono trattati i dati personali. I principali diritti dell'intestatario sono: il diritto di accesso, il diritto di rettifica, il diritto di cancellazione (diritto all'oblio), il diritto di portabilità. Se uno di questi diritti viene violato l'intestatario può presentare un reclamo all'autorità di controllo o opporsi al trattamento dei dati.
        \paragraph{Titolare del trattamento} Come definito da: \texttt{Art. 4 c. 7} è la persona fisica o giuridica, l'autorità pubblica, il servizio o altro organismo che, singolarmente o insieme ad altri, determina le finalità e i mezzi del trattamento dei dati personali. I principali obblighi del titolare del trattamento sono: la definizione delle finalità del trattamento, l'implementazione di misure di sicurezza, la dimostrazione della conformità al regolamento e l'adesione ai codici di condotta. Dato che spesso il titolare del trattamento non è in grado di garantire la conformità al regolamento è facoltà di questo delegar determinate attività ad un \textit{Responsabile del trattamento}.
        \paragraph{Responsabile del trattamento} Come definito da: \texttt{Art. 4 c. 8} è la persona fisica o giuridica, l'autorità pubblica, il servizio o altro organismo che tratta dati personali per conto del titolare del trattamento. Questo deve rispettare le istruzioni fornitogli dal titolare del trattamento, deve garantire la sicurezza dei dati, deve rispettare la riservatezza dei dati e deve garantire la conformità al regolamento. Anche questo può delegare determinate attività ad un \textit{Sub-responsabile del trattamento}.
        \paragraph{\textit{Data Protection Officer} - \texttt{DPO}} Come definito da: \texttt{Art. 37} è una figura la quale è responsabile della protezione dei dati. Questa carica và nominata obbligatoriamente in determinate situazioni. I compiti del \texttt{DPO} sono: informare e consigliare il titolare del trattamento e i dipendenti, monitorare il rispetto del regolamento, cooperare con l'autorità di controllo, essere il punto di contatto con l'autorità di controllo. Il \texttt{DPO} può essere un dipendente dell'azienda o un consulente esterno inoltre può agire in modo indipendente e non può essere licenziato per il solo fatto di aver svolto il proprio lavoro.
        \paragraph{Autorità di controllo} Come definito da: \texttt{Art. 4 c. 21} è un'autorità pubblica indipendente che sovrintende all'applicazione del regolamento. Le principali funzioni dell'autorità di controllo sono: monitorare l'applicazione del regolamento, fornire consulenza, promuovere la consapevolezza dei diritti e degli obblighi, cooperare con le autorità di controllo di altri stati membri, partecipare alla cooperazione europea. L'autorità di controllo può svolgere ispezioni, può richiedere documentazione, può richiedere informazioni, può richiedere l'accesso ai locali, può richiedere la sospensione del trattamento, può richiedere la cancellazione dei dati, può richiedere la notifica di una violazione dei dati. In italia l'autorità di controllo è il ``Garante per la protezione dei dati personali''.
        \paragraph{Destinatari e Terzi} I destinatari sono le persone fisiche o giuridiche, l'autorità pubblica, il servizio o altro organismo che ricevono i dati personali, mentre i terzi sono le persone fisiche o giuridiche, l'autorità pubblica, il servizio o altro organismo che non sono l'intestatario, il titolare del trattamento, il responsabile del trattamento e le persone autorizzate a trattare i dati personali. I destinatari e i terzi devono rispettare le norme del regolamento e non possono trattare i dati personali per scopi diversi da quelli per cui sono stati raccolti.
    \subsection{Impatto del \texttt{GDPR} sui \texttt{SI}}
        Per chi i sistemi informativi gli usa il \texttt{GDPR} ha un impatto molto forte in quanto andranno definiti nuovi requisiti di progettazione ed operativi i quali modificheranno l'implementazione di nuove tecniche organizzative.
        \paragraph{\textit{Privacy by design \& by default}} Visto che i \texttt{SI} devono integrare a pieno nuove misure per la protezione della privacy allora questi dovranno essere progettati in modo da raccogliere solo i dati strettamente necessari alle loro funzioni minimizzando l'uso dei dati personali e garantendo la protezione dei dati personali sin dalla progettazione (\textit{privacy by design}). Inoltre per \textit{default} i \texttt{SI} dovranno garantire che i dati personali non siano resi accessibili a un numero indefinito di persone senza l'intervento dell'interessato impedendo di fatto i trattamenti non necessari (\textit{privacy by default}).
        \paragraph{Sicurezza dei dati} I \texttt{SI} dovranno garantire la sicurezza dei dati personali implementando tecniche di criptografia e pseudonimizzazione dei dati garantendo la riservatezza. Anche eventuali copie di \textit{backup} dovranno essere protette e cifrate oltre che previste per garantire una rapida ripresa in caso di perdita dei dati (\textit{disaster recovery}).
        \subsubsection{Come i \texttt{SI} rispondono ai diritti dell'intestatario}
            I \texttt{SI} dovranno garantire che i diritti dell'intestatario siano rispettati, in particolare dovranno garantire:
            \paragraph{Diritto alla cancellazione} I \texttt{SI} dovranno garantire che i dati personali siano cancellati in modo sicuro e definitivo, inoltre dovranno garantire che i dati personali non siano più accessibili e che non siano più trattati.
            \paragraph{Portabilità e trasparenza} Se richiesto dall'intestatario i \texttt{SI} dovranno garantire che i dati personali siano trasferiti in un formato leggibile e strutturato secondo le norme del regolamento, inoltre dovranno garantire che i dati personali siano trasferiti in modo sicuro e che non siano accessibili a terzi. Il tutto in tempi brevi e con costi contenuti.
            \paragraph{Tracciabilità e gestione delle attività di trattamento} I \texttt{SI} dovranno tenere traccia di \textbf{chi}, \textbf{quando} e a che \textbf{scopo} ha trattato i dati personali, inoltre dovranno garantire che i dati personali siano trattati solo per gli scopi per cui sono stati raccolti e che non siano trattati per scopi diversi.
        
        \paragraph{Valutazione d'impatto e gestione del rischio} I \texttt{SI} dovranno garantire che venga effettuata una valutazione d'impatto sul trattamento dei dati personali per valutare i rischi per i diritti e le libertà delle persone fisiche, inoltre dovranno garantire che vengano implementate misure per ridurre i rischi e che vengano monitorati i rischi.
\section{I sistemi di \texttt{CRM}: \textsc{Salesforce}}
{\footnotesize Presentazione a cura di \textsc{Mattiuz} Marco \footnote{Contatti dell'autore: Mattiuz M. - \href{mailto:marco.mattiuz@studenti.unitn.it}{marco.mattiuz@studenti.unitn.it}}
    \subsubsection{Cos'è un \texttt{CRM}}
        Un \texttt{CRM} (\textit{Customer Relationship Management}) è un sistema che permette di gestire le relazioni con i clienti, permette di raccogliere, organizzare e analizzare i dati dei clienti. Questo lavora solitamente nell'ambito di gestione dei contatti, dell'automazione dei processi quali la vendita, il marketing e il servizio clienti, dell'analisi e \textit{reporting} dei dati quali le previsioni di vendita, l'analisi dei dati dei clienti e la gestione delle campagne di marketing e dell'integrazione con altri sistemi quali i sistemi di \textit{e-commerce}, i sistemi di \textit{help desk} e i sistemi di \textit{business intelligence}.\newline
        In breve lo scopo di un \texttt{CRM} è quello di migliorare l'esperienza del cliente ed aumentare le vendite creando una relazione più profonda e duratura con il cliente.
    \subsubsection{Studio di \textsc{Gartner}}
        Secondo uno studio di \textsc{Gartner}\footnote{
            Azienda di consulenza e ricerca nel campo dell'informatica e delle telecomunicazioni, abbiamo discusso del loro \textit{Magic Quadrant} e del loro \textit{Hype Cycle} nella sezione \ref{sec:gartner} - \nameref{sec:gartner}
        } (aggiornato all'agosto 2023) i principali fornitori di \texttt{CRM} sono: \textit{Salesforce}, \textit{Microsoft}, \textit{Oracle} e \textit{PegaSystems}. Questo studio è stato condotto su 16 fornitori di \texttt{CRM} e ha valutato i fornitori in base a 15 criteri tra cui: la visione, la capacità di esecuzione, la completezza della visione e la capacità di esecuzione, oltre a come questi sistemi si adattano alle esigenze dei diversi settori e aree geografiche.
    \subsubsection{\textit{Salesforce}}
        \textit{Salesforce} è un fornitore di \texttt{CRM} che offre una vasta gamma di servizi inclusi in diversi pacchetti, di seguito andremo ad analizzare i principali servizi offerti da \textit{Salesforce}, ma prima è importante sottolineare che \textit{Salesforce} oltre ai propri prodotti \texttt{CRM} è padre anche della piattaforma \textit{MuleSoft} e della piattaforma \textit{Slack} entrambe ampiamente utilizzate nel mondo dell'informatica.
        \paragraph{\textit{Agentforce}} È un servizio di \textit{CRM} che permette di creare degli agenti autonomi che offrono supporto agli utenti e dipendenti, troviamo tra questi \textit{IT Agent} che monitora e risolve i problemi tecnici, \textit{Banking Agent} che fornisce assistenza ai clienti del settore bancario, \textit{Retail Agent} che fornisce assistenza ai clienti sui vari mercati,ecc\dots
        \paragraph{\textit{Customer 360}} È un servizio di \textit{CRM} che permette di creare una visione unificata del cliente tramite dati provenienti da diverse fonti, permette di creare profili dei clienti, di creare segmenti di clienti, di creare \textit{target} di clienti e di creare campagne di marketing personalizzate.
        \paragraph{\textit{Small buisness}} È un servizio progettato per le piccole imprese il quale consente queste di gestire i contatti, le vendite, il marketing e il servizio clienti in modo semplice ed efficace.
        \paragraph{\textit{Starter Suite}} È un insieme di servizi quali: \textit{Sales}, \textit{Service}, \textit{Marketing}, \textit{Slack} che permette di gestire le vendite, il servizio clienti, il marketing e la comunicazione inter aziendale in modo integrato e completo.
        \paragraph{\textit{Einstein 1}} È un servizio che permette di creare applicazioni ``\textit{low code}'' per automatizzare i processi aziendali integrabile con gli altri servizi di \textit{Salesforce}.
        \paragraph{\textit{Tableau}} È un servizio di \textit{business intelligence} che permette di creare \textit{dashboard} e \textit{report} per analizzare i dati dei clienti e delle vendite.
\newpage
\section{\textit{E-procurement} e pubblica amministrazione}
{\footnotesize Presentazione a cura di \textsc{Marin} Eric \footnote{Contatti dell'autore: Marin E. - \href{mailto:eric.marin@studenti.unitn.it}{eric.marin@studenti.unitn.it}}}
    L'\textit{E-procurement} anche noto come \textit{Supplier Exchange} è definito come l'insieme delle tecnologie volte all'automazione e digitalizzazione del processo di acquisto delle materie prime o servizio da parte di un'azienda.
    \subsection{Il processo in generale}
        \subsubsection{Processo di \textit{E-procurement}}
            Il processo di \textit{E-procurement} è composto da diverse fasi:
            \begin{enumerate}
                \item \textbf{\textit{E-informing}}: fase in cui l'azienda informa i fornitori della necessità di acquistare un bene o un servizio.
                \item \textbf{\textit{E-Sourcing}}: fase in cui l'azienda definisce i requisiti ed individua potenziali fornitori.
                \item \textbf{Selezione del fornitore}: fase in cui l'azienda seleziona il fornitore migliore in base ai requisiti definiti. Può avvenire tramite
                    \begin{description}
                        \item[\textit{E-Tendering}] Il fornitore viene scelto tramite un \textbf{bando} pubblicato tramite un portale web di \textit{E-Noticing}
                        \item[\textit{E-Auction}] Il fornitore viene scelto tramite un'asta online dove i fornitori competono tra di loro per offrire il prezzo più basso ed i migliori servizi.
                        \item[\textit{E-Purchasing}] Per beni a basso costo ma elevato volume si può utilizzare un sistema di acquisto automatico. Ovvero viene fatto un bando per un determinato periodo di tempo e il fornitore che offre il prezzo più basso vince e quando quel bene viene richiesto il fornitore deve consegnarlo entro un determinato tempo al prezzo stabilito.
                    \end{description}
                \item \textbf{\textit{E-Ordering}}: fase in cui l'azienda ordina il bene o il servizio al fornitore.
                \item \textbf{\textit{E-Invoicing}}: fase in cui il fornitore emette la fattura elettronica all'azienda.
                \item \textbf{\textit{E-Payment}}: fase in cui l'azienda paga il fornitore.
            \end{enumerate}
        \subsubsection{Vantaggi e svantaggi dell'\textit{E-procurement}}
            L'\textit{E-procurement} offre diversi vantaggi tra cui:
            \begin{itemize}
                \item Migliore efficienza
                \item Trasparenza e controllo del processo
                \item Migliore livello di qualità/prezzo dei beni e servizi con costi ridotti
            \end{itemize}
            Tuttavia le aziende che non possiedono i mezzi digitali necessari vengono escluse da questo processo, inoltre l'\textit{E-procurement} può portare ad una maggiore dipendenza da un numero ristretto di fornitori.
    \subsection{Caso \texttt{CONSIP} e pubblica amministrazione}
        \paragraph{Pubblica amministrazione} Anche la pubblica amministrazione può beneficiare dell'\textit{E-procurement} in quanto riduce le barriere di ingresso per i fornitori, promuovendo le \texttt{PMI} che compongono il 99\% delle imprese italiane, inoltre abbassano il rischio di corruzione grazie al maggior controllo.
        \subsubsection{\texttt{CONSIP S.p.A.}} 
            È una società per azioni interamente controllata dal Ministero dell'Economia e delle Finanze, la quale si occupa di gestire le forniture della pubblica amministrazione. \texttt{CONSIP} ha sviluppato il portale \texttt{Acquisti in rete} il quale permette di gestire le forniture della pubblica amministrazione in modo efficiente e trasparente.
            \paragraph{Acquisti in Rete} Acquisti in rete è un portale operativo dove si svolgono le gare d'appalto e le aste elettroniche, inoltre permette di gestire le forniture della pubblica amministrazione in che queste possano effettuare ordini diretti o partecipare a gare d'appalto.\newline
            All'interno del portale vengono presentate aree merceologiche ovvero categorie di beni e servizi quali Alimenti, Arredi, Energia, Lavori di manutenzione, ecc\dots
            Vengono messi a disposizione diversi strumenti di acquisto quali: Contratti pronti all'uso (ovvero contratti già pronti per l'acquisto), Gare d'appalto (ovvero gare per l'acquisto di beni e servizi), Aste elettroniche (ovvero aste online per l'acquisto di beni e servizi), Mercati elettronici (ovvero mercati online per l'acquisto di beni e servizi).
            \paragraph{Accordi Quadro} Gli accordi quadro sono contratti che vengono stipulati tra \texttt{CONSIP} e i fornitori per l'acquisto di beni e servizi, questi contratti sono pronti all'uso e permettono alle pubbliche amministrazioni di acquistare beni e servizi senza dover svolgere una gara d'appalto.
            \paragraph{Mercato Elettronico} Il mercato elettronico è un portale online dove i fornitori possono vendere i propri beni e servizi alle pubbliche amministrazioni, inoltre permette alle pubbliche amministrazioni di acquistare beni e servizi senza dover svolgere una gara d'appalto.
            \paragraph{Sistema dinamico} Il sistema dinamico è un sistema che permette alle pubbliche amministrazioni di acquistare beni e servizi senza dover svolgere una gara d'appalto, inoltre permette ai fornitori di vendere i propri beni e servizi alle pubbliche amministrazioni.
            \paragraph{Confronto} Il confronto tra i vari strumenti di acquisto mostra che il mercato elettronico è il più flessibile e il più veloce, mentre gli accordi quadro sono i più convenienti e i più sicuri.
\section{La sicurezza informatica}
{\footnotesize Intervento del dot. \textsc{Zambrini} Massimiliano - Responsabile della sicurezza informatica presso \texttt{ACI} \footnote{Contatti dell'autore: Zambrini M. - \href{mailto:m.zambrini@informatica.aci.it}{m.zambrini@informatica.aci.it}}}
\paragraph{Nota} Questa presentazione non è stata riportata in quanto riguardava argomenti trattati al corso di ``Introduction to Computer and Network Security'' del quale sono disponibili gli appunti nel mio \textit{repository} di \textit{GitHub} \href{https://raw.githubusercontent.com/lucafano04/appuntisecondoanno/main/IntroductionSecurityComputerNetwork/main.pdf}{al seguente indirizzo}, si vedano in particolare i capitoli 1,2 e 9. Inoltre si è parlato di \texttt{GDPR}, per il quale si rimanda alla sezione \ref{sec:gdpr} - \nameref{sec:gdpr}.