\chapter[Struttura aziendale e del suo \texttt{SI}]{La struttura dell’azienda e del suo sistema informativo}
\thispagestyle{chapterInit}
\section{Concetto di esigenza informativa}
    \paragraph{Funzione \texttt{SI}} La funzione primaria del \textbf{sistema informativo} è quella di aiutare e guidare chi svolge mansioni che mandano avanti l'azienda attraverso queste. Inoltre il \texttt{SI} deve essere di aiuto e guida in modo diverso per aree diverse, ciò tramite il \textbf{livello d'astrazione} che sale man mano che si sale di livello gerarchico. L'\textbf{esigenza informativa} dipende dal tipo di attività svolta e dal livello gerarchico dell'utente. (es. i livelli operativi hanno bisogno di informazioni attuali ed precise spesso il singolo dato, mentre i livelli direzionali hanno bisogno di informazioni sintetizzate anche su periodi più lunghi).
    \subsection{Schema di Anthony}
        \paragraph{Schema di Anthony} L'organizzazione aziendale è vista a forma piramidale con i livelli operativi alla base, i livelli intermedi al centro e i livelli direzionali in cima. Ogni livello ha bisogno di informazioni diverse e quindi il \texttt{SI} deve essere in grado di fornire informazioni adeguate a ciascun livello.
        \begin{figure}[H]
            \centering
            \includegraphics[scale = 0.5]{03/schemaAnthony.png}
            \caption{Schema di Anthony}
        \end{figure}\newpage
        \paragraph{profili informativi} Di seguito è riportata una tabella con i profili informativi di ciascun livello, si può notare come i livelli operativi abbiano bisogno di poche informazioni ma molto dettagliate, precise e in modo continuo, il livello direzionale tattico ha accesso ai dati con frequenza minore ma prefissata e con un livello di dettaglio minore produce quindi un volume medio di informazioni, il livello strategico ha bisogno di informazioni molto sintetizzate e con una frequenza molto bassa se non sporadica ma ha bisogno anche di informazioni esterne all'azienda.
        \begin{table}[H]
            \begin{adjustbox}{width=\textwidth}
                \begin{tabular}{|c|c|c|c|c|}
                    \hline
                    & \textbf{Frequenza} & \textbf{Dati} & \textbf{Provenienza dati} & \textbf{Volume} \\
                    \hline
                    \textbf{Livello direzionale strategico} & Sporadica & molto sintetizzati & interni ed esterni & basso \\
                    \hline
                    \textbf{Livello direzionale tattico} & Prefissata & sintetici e analitici & interni & medio \\
                    \hline
                    \textbf{Livello operativo} & Continua & analitici & interni & elevati \\
                    \hline
                \end{tabular}
            \end{adjustbox}
        \end{table}
\section{Sistemi operazionali}
    \paragraph{funzioni principali}
        \begin{itemize}
            \item Automazione di attività procedurali - In questo caso il \texttt{SI} è un supporto all'operatore
            \item Definizione di nuovi processi - come visto \hyperref[subsec:nuoviProcessi]{sottosezione 2.3.3}
            \item Aiuto nelle attività aziendali 
            \item Raccolta di dati - gli operatori inseriscono i dati nel sistema in modo continuo
            \item Guida per l'operatore - il sistema guida l'operatore nelle attività da svolgere in questo modo si riducono gli errori
        \end{itemize}
    \paragraph{Azioni sui dati}
        \begin{itemize}
            \item Accesso interattivo in inserimento, lettura, modifica - l'operatore può interagire con il sistema e modificare i dati nei limiti imposti 
            \item Trattamento di dati - il sistema tratta i dati in modo automatico e li presenta all'operatore in modo chiaro 
            \item Descrizione di eventi - il sistema descrive le transazioni e le attività svolte in modo da poterle ripetere in caso di necessità
            \item Valutazione e trattamento di informazioni utili - il sistema valuta i dati se sussistono errori e li segnala all'operatore
            \item Aggregazione per il calcolo di indicatori di stato - il sistema aggrega i dati per calcolare indicatori di stato dell'azienda
        \end{itemize}
    \paragraph{Componenti fondamentali}
        \begin{itemize}
            \item Base si dati operazionale - contiene i dati operativi dell'azienda
            \item Funzioni operative - funzioni che permettono di svolgere le attività operative
        \end{itemize}
\section{Sistemi informazionali}
    \paragraph{funzioni principali}
        \begin{itemize}
            \item Facilitazione del processo decisionale
            \item Presentazione dei dati secondo diverse aggregazioni e viste
            \item Confronto tra indicatori aziendali e indicatori esterni
        \end{itemize}
    \paragraph{Azioni sui dati}
        \begin{itemize}
            \item Accesso in lettura
            \item Aggregazione dei dati
            \item Descrizione di aree/temi
            \item Profondità temporale
            \item Multi-dimensionalità
        \end{itemize}
    \paragraph{Componenti fondamentali}
        \begin{itemize}
            \item Base dati informativa
            \item Strumenti di analisi
            \item Procedure di alimentazione (dati)
        \end{itemize}