\documentclass[twoside,a4paper]{report}
\usepackage[italian]{babel}
\usepackage[utf8]{inputenc}
\usepackage{amsmath}
\usepackage{amsthm}
\usepackage{amsfonts}
\usepackage{amssymb}
\usepackage{cancel}
\usepackage[margin=1in]{geometry}
\usepackage{hyperref}
\usepackage{bookmark}
\usepackage{setspace}
\usepackage{titlesec}
\usepackage{fancyhdr}
\usepackage{adjustbox}
\usepackage{float}
\usepackage{graphicx}

\graphicspath{{./images/}}

\setlength{\parskip}{0pt}
\titlespacing*{\subparagraph}{1em}{0em}{0em} 

\makeatletter
\renewenvironment{abstract}{%
    \if@twocolumn
        \section*{\abstractname}%
    \else
        \begin{center}%
            {\bfseries \abstractname\vspace{-.5em}\vspace{\z@}}%
        \end{center}%
        \small
        \begin{quotation}
    \fi}
    {\if@twocolumn\else\end{quotation}\fi}
\makeatother

\hypersetup{
    pdfauthor={Luca Facchini},
    pdftitle={Appunti di Sistemi Informativi},
    pdfsubject={Appunti del corso di Sistemi Informativi, tenuto dal prof. Bouquet Paolo presso l'Università degli Studi di Trento. Corso seguito nell'anno accademico 2024/2025.},
    pdfkeywords={Sistemi Informativi, Università degli Studi di Trento, Paolo Bouquet},
    pdfproducer={LaTeX},
    pdfcreator={pdflatex},
}

\fancypagestyle{chapterInit}{%
    \fancyhf{}
    \renewcommand{\headrulewidth}{0pt}
    \fancyfoot{}
    \fancyfoot[LE,RO]{\thepage}
    \fancyfoot[LO,RE]{"Appunti di Sistemi Informativi" di Luca Facchini}
}
\fancypagestyle{stdPage}{
    \fancyhead{}
    \fancyhead[LE]{\leftmark}
    \fancyhead[RO]{\rightmark}
    \setlength{\headheight}{15pt}
    \fancyfoot{}
    \fancyfoot[LE,RO]{\thepage}
    \fancyfoot[LO,RE]{"Appunti di Sistemi Informativi" di Luca Facchini}
    \renewcommand{\footrulewidth}{0.4pt}
}

\newtheorem{definition}{Definizione}[chapter]

\let\oldquote\quote
\let\endoldquote\endquote

\RenewDocumentEnvironment{quote}{om}
    {\oldquote}
    {\par\nobreak\smallskip
        \hfill(#2\IfValueT{#1}{~---~#1})\endoldquote 
        \addvspace{\bigskipamount}}

\title{Appunti di Sistemi Informativi}
\author{Luca Facchini (mat. 245965)}
\date{A.A. 2024/2025}

\begin{document}
    \begin{titlepage}
        \centering  % Center everything on the title page
        {\Huge\textbf{Appunti di Sistemi Informativi}} \\[1cm] % Title
        \vspace{0.5cm}
        
        {\Large Luca Facchini} \\ % Author name
        \vspace{0.3cm}
        {\large Matricola: 245965} \\[2cm] % Additional author info
        
        {\large Corso tenuto dal prof. Casari Paolo} \\[0.3cm] % Course information
        {\large Università degli Studi di Trento} \\[1.5cm]
        
        {\large A.A. 2024/2025} \\[3cm] % Academic year
        
        % Abstract section with spacing control
        \vfill
        \begin{abstract}
            Questo documento contiene gli appunti del corso di Sistemi Informativi, tenuto dal prof. Casari Paolo presso l'Università degli Studi di Trento. Il corso è stato seguito nell'anno accademico 2024/2025.\footnote{Dove non specificato diversamente eventuali immagini provengono dalle slide del corso (o da materiale didattico fornito dal docente) le quali sono fornite e tratte dal libro: "Sistemi Informativi Aziendali" di Pighin M. e Mazona A. solo per le immagini possono essere soggette a copyright e comunque usabili solo per fini didattici e in base alla legge italiana sul diritto d'autore (al massimo 15\%).}
        \end{abstract}
        
        \vfill  % Pushes the content to the center vertically
    \end{titlepage}

    \pagestyle{stdPage}
    
    \addtocontents{toc}{\protect\thispagestyle{stdPage}}
    \begingroup
        \tableofcontents
        \thispagestyle{stdPage}
    \endgroup
    
    \chapter{La società della conoscenza}
\thispagestyle{chapterInit}
\section{Introduzione}
    \paragraph{Cos'è una "società della conoscenza"} Il concetto di "\textbf{società della conoscenza}" si riferisce ad un modello di società nel quale la \textbf{conoscenza}, l'\textbf{informazione} e l'\textbf{innovazione} diventano i principali \textbf{mezzi della crescita economica e dello sviluppo sociale}. In questo tipo di società viene "premiato" chi ha la \textbf{capacità di apprendere, di innovare} e di \textbf{adattarsi ai rapidi cambiamenti tecnologici}.
    \paragraph{Perchè è importante?} La \textbf{conoscenza} è importante perché in una economia basata su questa la ricchezza e il potere sono determinati dalla \textbf{capacità di creare}, inoltre la \textbf{innovazione e competitività} è essenziale e la nuova idea domina i mercati. Altro asset importante è il \textbf{capitale umano e l'istruzione} in quanto le persone ben formate sono in grado di creare valore. La \textbf{globalizzazione e interconnessione} sono altri fattori importanti in quanto la conoscenza è un bene che si diffonde rapidamente e facilmente. Infine anche lo \textbf{sviluppo sostenibile} è importante in quanto la conoscenza permette di trovare soluzioni a problemi ambientali.
\section{Nuove leggi della società della conoscenza}
    \subsection{Leggi di Moore}
        \subsubsection{Prima legge di Moore}
            \paragraph{Definizione} La \textbf{legge di Moore} è un'osservazione fatta da \textbf{Gordon Moore} nel 1965, secondo la quale la \textbf{densità di transistor} nei circuiti integrati raddoppia ogni 18 mesi. Questo significa che la potenza di calcolo dei computer raddoppia ogni 18 mesi.\newline I limiti della prima legge di Moore sarebbero solo nel raggiungimento dei limiti fisici della materia.
            \paragraph{Conseguenze} Questa legge ha portato ad un \textbf{aumento esponenziale} della potenza di calcolo dei computer, portando ad un \textbf{aumento della velocità} e della \textbf{capacità di memorizzazione} dei computer. Questo ha portato ad un \textbf{aumento della diffusione} dei computer e ad una \textbf{riduzione dei costi}.
        \subsubsection{Seconda legge di Moore}
            \paragraph{Definizione} La \textbf{seconda legge di Moore} è un'osservazione fatta da \textbf{Gordon Moore} che dice: "L'investimento per realizzare una nuova tecnologia di microprocessori cresce in maniera esponenziale con il tempo".
            \paragraph{Conseguenze} Questa legge implica che per aumentare la potenza di calcolo dei computer è necessario un \textbf{investimento sempre maggiore}. Questo ha portato ad un \textbf{aumento dei costi} per la realizzazione di nuove tecnologie.\newline Questi effetti fanno sì che le società che si possono permettere di investire sono sempre meno e quelle che da sole non riescono ad investire si uniscono. Effetto economico di questa legge è l'aumento del rischio in caso di investimento sbagliata. 
    \subsection{Leggi di Sarnoff, MetCalfe, Reed}
        \subsubsection{Legge di Sarnoff}
            \paragraph{Definizione} La \textbf{legge di Sarnoff} è un'osservazione fatta da \textbf{David Sarnoff} nel 1950, secondo la quale il \textbf{valore di un sistema di comunicazione} del tipo \texttt{broadcast} è proporzionale al \textbf{numero di utenti} del sistema. Dunque: "Il valore $V$ di una rete di broadcasting è direttamente proporzionale al numero $N$ di utenti della rete". $ V = N $.
            \paragraph{Conseguenze} Il valore della rete aumenta con il numero di utenti.
        \subsubsection{Legge di MetCalfe}
            \paragraph{Definizione} "Il valore $ V $ di un sistema di comunicazione cresce con il quadrato del numero di utenti $ N $ della rete". $ V = N^2 - N $.\newline Questa legge è legata ai sistemi come il \textbf{telefono} o \textbf{fax} che permettono comunicazioni \textbf{uno a uno}.
            \paragraph{Implicazione} La connessione di reti indipendenti crea un valore più elevato rispetto al valore delle singole reti.
        \subsubsection{Legge di Reed}
            \paragraph{Definizione} "L'utilità dele grandi reti, formate da reti di reti (con particolare riferimento alle reti di relazione sociale) cresce esponenzialmente con il numero di nodi". $ V = 2^N - N - 1 $.
            \paragraph{Conseguenze} Questa legge implica che il valore di una rete sociale cresce esponenzialmente con il numero di nodi. Questo è dovuto al fatto che con ogni nodo si possono creare nuovi sottogruppi di nodi.
        \subsubsection{Conseguenze delle leggi}
            \paragraph{Conseguenze} Se si distribuisce un solo contenuto allora si ha una crescita lineare del valore, se si attivano transizione per il commercio elettronico si ha una crescita quadratica del valore, se si crea una comunità si ha una crescita esponenziale del valore. \newline Per chi investe è meglio puntare sulla legge di Reed.
\section{Hype Cycle di Gartner}
    \subsection{Premessa}
        \paragraph{Cos'è?} L'\textbf{Hype Cycle} è un modello sviluppato da \textbf{Gartner} (società) che rappresenta la \textbf{maturità, l'adozione e l'applicazione delle tecnologie emergenti}. Il modello è rappresentato da un grafico che mostra la \textbf{curva di hype} di una tecnologia, ovvero la \textbf{fase di crescita e declino} di una tecnologia.
    \subsection{La curva della domanda} 
        \paragraph{Cosa rappresenta} La \textbf{curva della domanda} rappresenta come un determinato prodotto e/o tecnologia viene adottato dal mercato. La curva è composta da tre picchi di acquisto: i \textbf{pionieri}, la \textbf{maggioranza} e i \textbf{ritardatari}. Questa curva è importante per capire come un prodotto viene adottato dal mercato e per capire come si può influenzare la domanda.
        \paragraph{I picchi} I \textbf{pionieri} sono quella cerchia ristretta di persone che sono disposte a comprare un prodotto appena uscito anche a costi elevati. La \textbf{maggioranza} è quella cerchia di persone che comprano un prodotto quando il prezzo è ragionevole ma pretendono tecnologie semplici e facili da usare e sono sensibili ai trend creati dai pionieri. Infine i \textbf{ritardatari} sono coloro che comprano un prodotto quando ormai è diventato un bene di uso comune e non si possono esimere dall'acquisto per rimanere competitivi sul mercato, questi sono molto prudenti e non amano rischiare inoltre la loro preoccupazione principale è il costo.
            \begin{figure}[H]
                \centering
                \includegraphics[width=0.5\textwidth]{01/graficoDomanda.png}
                \caption{Curva della domanda}
            \end{figure}
    
    \subsection{Hype Cycle}
        \paragraph{Cos'è?} Il grafico di seguito illustra la \textbf{curva di hype} di una tecnologia, ovvero la \textbf{fase di crescita e declino} di una tecnologia. Questo grafico è importante per capire come una nuova tecnologia viene richiesta ed adottato dal mercato.
        \paragraph{Le fasi} Questa curva ha diversi alti e bassi:
        \begin{enumerate}
            \item \textbf{Innovazione} è la fase in cui la tecnologia viene presentata al mercato e le prime startup iniziano a sviluppare prodotti basati su questa tecnologia.
            \item \textbf{Crescita esponenziale} è la fase in cui la tecnologia inizia a venire seguita da un numero sempre maggiore di persone e il mercato potenziale dietro a questa tecnologia inizia a crescere. I mass media iniziano a parlare di questa tecnologia. I primi prodotti basati su questa tecnologia iniziano ad essere venduti ad un costo molto elevato.
            \item \textbf{Picco delle aspettative inflazionate} è la fase in cui la tecnologia raggiunge il suo picco di interesse e le aspettative sono molto alte. In questa fase la tecnologia è vista come la soluzione a tutti i problemi.
            \item \textbf{Declino} è la fase in cui la tecnologia non riesce a soddisfare le aspettative e il mercato inizia a perdere interesse, in quanto la tecnologia non è in grado di risolvere i problemi del suo punto di massimo.
            \item \textbf{Risalita della produttività} è la fase in vengono realizzati la seconda/terza generazione dei prodotti basati su questa tecnologia e il mercato inizia a capire come utilizzare questa tecnologia in modo efficace. In questa fase la tecnologia inizia a diventare un bene di uso comune.
            \item \textbf{Piatto di produttività} è la fase in cui la tecnologia è diventata un bene di uso comune e il mercato inizia a saturarsi. Qui la tecnologia viene usata per risolvere problemi specifici ma non è più vista come la soluzione a tutti i problemi (20/30 \% del mercato rispetto al picco).
        \end{enumerate}
        \begin{figure}[H]
            \centering
            \includegraphics[width=0.5\textwidth]{01/hypeCycle.png}
            \caption{Curva di hype}
        \end{figure}
        Qualche volta però la tecnologia non riesce a riprendersi dal declino, vengono quindi proposte altre due curve con fasi che si sostituiscono alle fasi 5 e 6:
        \begin{enumerate}
            \item[5.] \textbf{Cimitero delle tecnologie} è la fase in cui la tecnologia non riesce a riprendersi dal declino e viene abbandonata.
            \item[5.] \textbf{Palude di uso comune} è la fase in cui la tecnologia è diventata un bene di uso comune ma non riesce a trovare nuove applicazioni e il mercato è molto più ristretto, il prodotto viene usato solo da una cerchia ristretta di persone.
        \end{enumerate}
        \begin{figure}[H]
            \centering
            \includegraphics[width=0.5\textwidth]{01/hypeCycleAlt.png}
            \caption{Curva di hype con fasi alternative}
        \end{figure}
    \subsection{Magic Quadrant di Gartner}
        \paragraph{Cos'è?} Il \textbf{Magic Quadrant} è un modello sviluppato da \textbf{Gartner} che permette di valutare le \textbf{aziende} in base alla loro \textbf{completezza della visione} e alla loro \textbf{abilità di esecuzione}. Questo modello è importante per capire come un'azienda si posiziona rispetto ai suoi concorrenti e per capire quali sono i punti di forza e di debolezza di un'azienda.
        \paragraph{I quadranti} Il modello è composto da quattro quadranti:
        \begin{itemize}
            \item \textbf{Leader} sono le aziende che hanno una \textbf{completa visione} del mercato e che sono in grado di \textbf{eseguire} in modo efficace.
            \item \textbf{Challenger} sono le aziende che hanno una \textbf{completa visione} del mercato ma che non sono in grado di \textbf{eseguire} in modo efficace.
            \item \textbf{Visionary} sono le aziende che hanno una \textbf{visione parziale} del mercato ma che sono in grado di \textbf{eseguire} in modo efficace.
            \item \textbf{Niche Player} sono le aziende che hanno una \textbf{visione parziale} del mercato e che non sono in grado di \textbf{eseguire} in modo efficace.
        \end{itemize}
        \begin{figure}[H]
            \centering
            \includegraphics[width=0.5\textwidth]{01/magicQuadrant.png}
            \caption{Magic Quadrant di Gartner. Source: (Maggio 2021) Gartner}
        \end{figure}
\section{Long Tail di Anderson}
    \subsection{Premessa}
        \paragraph{Cos'è?} La \textbf{Long Tail} è un modello sviluppato da \textbf{Chris Anderson} che rappresenta la \textbf{distribuzione delle vendite} di un prodotto (non per forza informatico o di nuova tecnologia). Questo modello descrive come la vendita di una certa categoria di prodotti è distribuita tra i prodotti più venduti e i prodotti meno venduti.
    \subsection{La coda lunga}
        \paragraph{Cos'è?} La \textbf{coda lunga} è la parte della distribuzione delle vendite che si trova dopo i prodotti più venduti. Questa parte della distribuzione è composta da molti prodotti che vendono poche copie ciascuno.
        \paragraph{Conseguenze} Questo modello ha diverse conseguenze:
        \begin{itemize}
            \item \textbf{Aumento della varietà} di prodotti disponibili sul mercato.
            \item \textbf{Aumento della disponibilità} di prodotti di nicchia.
            \item \textbf{Aumento della vendita} di prodotti di nicchia.
            \item \textbf{Aumento della vendita} di prodotti meno popolari.
            \item \textbf{Aumento della vendita} di prodotti meno venduti.
        \end{itemize}
        Conseguenza fondamentale riguarda soprattutto i motori online, in quanto questi possono permettersi di vendere prodotti di nicchia in quanto non hanno i costi di un negozio fisico.
        \begin{figure}[H]
            \centering
            \includegraphics[width=0.5\textwidth]{01/longTail.png}
            \caption{Long Tail}
        \end{figure}

    \chapter[Concetti Generali]{Concetti Generali sull'Informatica Aziendale}
\thispagestyle{chapterInit}
Nel presente capitolo verranno trattati i concetti generali dell'informatica aziendale, in particolare verranno trattati i seguenti argomenti: definizioni di informatica aziendale, di sistema informativo aziendale ed altro, verrà inoltre trattato l'impatto dell'informatica nelle aziende e come le aziende italiane si stanno adattando a queste nuove tecnologie.

\section{Introduzione e definizioni}
    Definiamo innanzitutto alcuni concetti chiave per la comprensione del corso.
    \paragraph{Informatica Aziendale} L'\textbf{informatica aziendale} è la disciplina che studia l'applicazione dell'informatica nelle aziende, studia inoltre l'influenza di questa nelle diverse categorie di un sistema aziendale. Esistono diversi settori di applicazione trai quali:
        \begin{itemize}
            \item \textbf{Aiuto e guida operativa} - Assistenza agli operatori a seguire le corrette procedure di lavoro con un costante controllo iterativo sui dati. Facilitazione di ricerca e recupero di informazioni.
            \item \textbf{Organizzativa} - Automazione di processi da un lato, richiesta di \textbf{competenze} e \textbf{risorse} differenti dall'altro.
            \item \textbf{Controllo} - Rilevazione di caratteristiche e comportamenti di un sistema, possibilità di \textbf{analisi quantitative} e \textbf{qualitative}.
            \item \textbf{Strategia} - Supporto ai processi di trasformazione e innovazione, supporto alle decisioni strategiche. 
        \end{itemize}
    \paragraph{Sistemi Informativi aziendali} I \textbf{sistemi informativi aziendali} sono l'insieme delle procedure e delle infrastrutture che definiscono e supportano l'elaborazione, la distribuzione e l'utilizzo delle informazioni all'interno di una azienda. Molto spesso ci si basa su una infrastruttura elettronica. È importante non confondere i sistemi informativi con i sistemi informatici, infatti è vero che ogni sistema informatico è un sistema informativo, ma non è vero il contrario.
    \paragraph{Risorsa} Una \textbf{risorsa} ``è tutto ciò con cui l'organizzazione opera'' sia che questo possa essere un bene fisico o che questo sia un bene immateriale
    \paragraph{Processo} Un \textbf{processo} è un insieme di attività atte a gestire una risorsa nel suo ciclo di vita.
\section{Sistema informativo aziendale}
    Un \textbf{sistema informativo aziendale} è un sistema che permette di raccogliere, elaborare, memorizzare e distribuire informazioni all'interno di un'organizzazione. Questo sistema si compone di:
    \begin{itemize}
        \item \textbf{Dati}: \begin{itemize}
            \item di Configurazione - Dati che descrivono la struttura dell'organizzazione
            \item operativi - Dati che descrivono le attività dell'organizzazione
            \item di supporto - Dati che supportano le attività dell'organizzazione
            \item di stato - Dati che descrivono lo stato dell'organizzazione
        \end{itemize}
        \item \textbf{Procedure}: \begin{itemize}
            \item acquisizione - Raccolta di dati
            \item controllo ed elaborazione - Controllo e manipolazione dei dati
            \item pianificazione
        \end{itemize}
        \item \textbf{Mezzi e strumenti}: \begin{itemize}
            \item Hardware - sever e periferiche
            \item Stazioni di lavoro
            \item \dots
        \end{itemize}
    \end{itemize}
    Da notare come in questa definizione non si parli di software, ma di mezzi e strumenti, questo perché il sistema informatico può essere o meno una parte del sistema informativo aziendale, questo perché il sistema informativo aziendale può consistere in un sistema informatico, ma non è detto che debba essere così. Questo rende di fatto il sistema informatico un ``sottoinsieme'' del sistema informativo aziendale.
\section{Impatto dell'informatica nelle azienda}
    L'informatica ha avuto un impatto molto forte nelle aziende, infatti ha portato a una serie di cambiamenti che hanno rivoluzionato il modo di lavorare delle aziende.
    \subsection{Conoscenza dei fenomeni aziendali}
        Ogni sistema \textbf{informativo} aziendale è lo strumento per diffondere la conoscenza all'interno dell'azienda. Per adempiere al suo compito questo deve rispettare alcuni criteri, che sono poi gli stessi che permettono di dividere i fenomeni aziendali:
        \begin{description}
            \item[Livello di astrazione] Il sistema deve essere in grado di rappresentare la realtà aziendale in modo corretto, sintetico ma completo. In alcuni livelli più ``alti'' si devono avere informazioni più sintetiche, mentre in livelli più ``bassi'' si devono avere informazioni più dettagliate\footnote{Più avanti nel capitolo ?? si approfondirà questo concetto}. % TODO: inserire riferimento al capitolo/sezione/paragrafo...
            \item[Tempestività] Il sistema deve essere in grado di fornire le informazioni in tempo utile ed appropriato al contesto dell'operazione e della mole di dati. Anche in questo caso si ha una differenziazione tra i vari livelli, difatti in livelli più ``alti'' le informazioni possono essere meno tempestive con attese più lunghe, mentre in livelli più ``bassi'' le informazioni devono disponibili immediatamente.
            \item[Livello di copertura] Il sistema deve essere in grado di coprire tutte le aree aziendali e tutti i processi aziendali nei vari livelli di dettaglio. Questo racchiude entrambi i concetti di \textbf{orizzontalità} e \textbf{verticalità} del flusso informativo.
        \end{description}
        Allo stesso tempo il sistema informativo deve \textbf{garantire}: Accessibilità dei dati e Correttezza del flusso, flusso che si divide in:
        \begin{description}
            \item[Orizzontale] tra le varie aree aziendali
            \item[Verticale] tra i vari livelli gerarchici
        \end{description}
    \subsection{Processi classi di un sistema informativo}
        Tra tutte le attività di una azienda possiamo distinguere tre processi i quali sono solitamente i primi ad essere informatizzati per la loro elevata ``attrattiva informatica''. Questi processi sono:
        \begin{description}
            \item[Sviluppo funzioni operative] Processo che si occupa di automatizzare dei processi che sono già presenti andando a ridurre i tempi e la mano d'opera necessaria.
            \item[Pianificazione] Processo che prende i dati inseriti nel \texttt{SI} e li elabora per automatizzare processi di pianificazione.
            \item[Controllo] Processo che renda automatico il controllo dei dati i inseriti nel \texttt{SI} e li confronta con criteri e dati di riferimento segnalando eventuali anomalie.
        \end{description}
    \subsection{Nuovi processi}
        \label{subsec:nuoviProcessi}
        
        \paragraph{Introduzione dell'informatica} L'introduzione dell'informatica in azienda non si occupa semplicemente di supporto a processi già esistenti, come nel caso dei processi classici, ma introduce nuovi processi e ne modifica altri. Questi processi particolarmente informatizzati sono impossibili o molto difficili da realizzare senza l'ausilio di un sistema informatico.
        \paragraph{BRP} Nasce da questa idea il concetto di \textbf{Business Process Re-engineering} o \textbf{Reingegnerizzazione dei processi aziendali} che consiste nel ripensare e ridisegnare i processi aziendali per sfruttare al meglio le nuove tecnologie informatiche. 
        \paragraph{Contatto col cliente al tempo di internet} La spinta verso il processo è generata dalla vasta adozione delle reti informatiche come supporto alla comunicazione e alla collaborazione tra le persone. Con l'avvento di internet, infatti, il contatto con il cliente assume delle modalità completamente nuove, si passa da un contatto diretto a un contatto mediato da un sistema informatico, che per certi versi può essere più efficiente e più efficace. Questa trasformazione ha portato ad una enorme riduzione delle tempistiche di contatto e di risposta, ma ha anche portato ad una interazione diretta tra cliente e sistema informativo, il quale é in grado di fornire al cliente informazioni in tempo reale e di rispondere alle sue richieste in modo automatico.
\section{I \texttt{SI} nelle aziende Italiane e relazione \texttt{ICT} - azienda}
    Analizziamo ora come i sistemi informativi si sono evoluti per adattarsi alle esigenze delle aziende italiane ed in che modo i servizi \texttt{ICT} hanno influenzato l'organizzazione interna ed esterna delle aziende.
    \subsection{Le aziende in italia}
        \paragraph{Le aziende in italia} Le aziende in Italia assumono una conformazione molto differente rispetto al panorama europeo, infatti il 99,9\% delle aziende italiane sono \textbf{PMI} (Piccole e Medie Imprese) e solo lo 0,1\% sono grandi aziende.
        \paragraph{\texttt{PMI} e \texttt{SI}} 
            Le \texttt{PMI} sono aziende che hanno una struttura molto semplice e che spesso l'investire in un sistema informativo non è una priorità visto che i processi sono molto semplici e non richiedono un sistema informativo complesso. \newline
            Spesso quindi un \texttt{SI} potrebbe essere visto come un costo inutile, ma con l'avvento di internet e delle nuove tecnologie, anche le \texttt{PMI} stanno iniziando ad adottare un sistema informativo in piccola misura, ovviamente non adotteranno \texttt{SI} di grandi dimensioni, ma sistemi informativi, spesso italiani in quanto più vicini alla realtà delle \texttt{PMI}, più piccoli e adeguati alle loro esigenze.
        \paragraph{Evoluzione dei \texttt{SI}}
            I sistemi informativi delle grandi aziende che in un primo luogo, come già discusso, erano molto distati dalle \texttt{PMI}, si sono evoluti e si sono adattati alle esigenze delle \texttt{PMI} creando linee di \texttt{SI} adattati alla struttura flessibile e semplice delle \texttt{PMI}. Questo ha portato ad una maggiore diffusione dei sistemi informativi anche nelle \texttt{PMI} e ad una maggiore diffusione delle nuove tecnologie informatiche.
    \subsection{Cambiamenti dei \texttt{SI}}
        I sistemi informativi aziendali stanno entrando molto più facilmente all'interno delle aziende, questo è dovuto ad una evoluzione del mondo \texttt{ICT} il quale ha favorito lo sviluppo di sistemi informativi più flessibili e adattabili alle esigenze delle aziende.\newline
        Oltre a questo lo sviluppo dei servizi \texttt{ICT} hanno portato a cambiamenti nell'organizzazione interna ed esterna delle aziende.
        \paragraph{Organizzazione Interna} L'evoluzione dei servizi \texttt{ICT} ha portato alla riduzione dei ruoli impiegatizi, ovvero ruoli che non sono direttamente legati alla produzione, ma che sono necessari per il funzionamento dell'azienda. Oltre alla necessaria riqualificazione dei ruoli aziendali ed alla riduzione dei ruoli di supporto alla produzione (controllo, amministrazione, ecc\dots). Il processo di \textit{front-office}, quali la vendita e il marketing, sono stati completamente rivoluzionati. Tutto ciò ha scaturito una revisione del modello organizzativo aziendale passando da una organizzazione ``per funzioni'' ad una organizzazione ``per processi''. 
        \paragraph{Organizzazione Esterna} L'evoluzione dei servizi \texttt{ICT} ha portato ad una maggiore collaborazione tra aziende, infatti la rete ha permesso di creare nuove forme di collaborazione tra aziende, come ad esempio la semplificazione del processo di \textit{outsourcing} e la creazione di nuove forme di collaborazione tra aziende. Questo accade al discapito della dimensione dell'azienda che non è più un fattore determinante per il successo dell'azienda stessa.
    \chapter[Struttura aziendale e del suo \texttt{SI}]{La struttura dell’azienda e del suo sistema informativo}
\thispagestyle{chapterInit}
\section{Concetto di esigenza informativa}
    \paragraph{Funzione \texttt{SI}} La funzione primaria del \textbf{sistema informativo} è quella di aiutare e guidare chi svolge mansioni che mandano avanti l'azienda attraverso queste. Inoltre il \texttt{SI} deve essere di aiuto e guida in modo diverso per aree diverse, ciò tramite il \textbf{livello d'astrazione} che sale man mano che si sale di livello gerarchico. L'\textbf{esigenza informativa} dipende dal tipo di attività svolta e dal livello gerarchico dell'utente. (es. i livelli operativi hanno bisogno di informazioni attuali ed precise spesso il singolo dato, mentre i livelli direzionali hanno bisogno di informazioni sintetizzate anche su periodi più lunghi).
    \subsection{Schema di Anthony}
        \paragraph{Schema di Anthony} L'organizzazione aziendale è vista a forma piramidale con i livelli operativi alla base, i livelli intermedi al centro e i livelli direzionali in cima. Ogni livello ha bisogno di informazioni diverse e quindi il \texttt{SI} deve essere in grado di fornire informazioni adeguate a ciascun livello.
        \begin{figure}[H]
            \centering
            \includegraphics[scale = 0.5]{03/schemaAnthony.png}
            \caption{Schema di Anthony}
        \end{figure}\newpage
        \paragraph{profili informativi} Di seguito è riportata una tabella con i profili informativi di ciascun livello, si può notare come i livelli operativi abbiano bisogno di poche informazioni ma molto dettagliate, precise e in modo continuo, il livello direzionale tattico ha accesso ai dati con frequenza minore ma prefissata e con un livello di dettaglio minore produce quindi un volume medio di informazioni, il livello strategico ha bisogno di informazioni molto sintetizzate e con una frequenza molto bassa se non sporadica ma ha bisogno anche di informazioni esterne all'azienda.
        \begin{table}[H]
            \begin{adjustbox}{width=\textwidth}
                \begin{tabular}{|c|c|c|c|c|}
                    \hline
                    & \textbf{Frequenza} & \textbf{Dati} & \textbf{Provenienza dati} & \textbf{Volume} \\
                    \hline
                    \textbf{Livello direzionale strategico} & Sporadica & molto sintetizzati & interni ed esterni & basso \\
                    \hline
                    \textbf{Livello direzionale tattico} & Prefissata & sintetici e analitici & interni & medio \\
                    \hline
                    \textbf{Livello operativo} & Continua & analitici & interni & elevati \\
                    \hline
                \end{tabular}
            \end{adjustbox}
        \end{table}
\section{Sistemi operazionali}
    \paragraph{funzioni principali}
        \begin{itemize}
            \item Automazione di attività procedurali - In questo caso il \texttt{SI} è un supporto all'operatore
            \item Definizione di nuovi processi - come visto \hyperref[subsec:nuoviProcessi]{sottosezione 2.3.3}
            \item Aiuto nelle attività aziendali 
            \item Raccolta di dati - gli operatori inseriscono i dati nel sistema in modo continuo
            \item Guida per l'operatore - il sistema guida l'operatore nelle attività da svolgere in questo modo si riducono gli errori
        \end{itemize}
    \paragraph{Azioni sui dati}
        \begin{itemize}
            \item Accesso interattivo in inserimento, lettura, modifica - l'operatore può interagire con il sistema e modificare i dati nei limiti imposti 
            \item Trattamento di dati - il sistema tratta i dati in modo automatico e li presenta all'operatore in modo chiaro 
            \item Descrizione di eventi - il sistema descrive le transazioni e le attività svolte in modo da poterle ripetere in caso di necessità
            \item Valutazione e trattamento di informazioni utili - il sistema valuta i dati se sussistono errori e li segnala all'operatore
            \item Aggregazione per il calcolo di indicatori di stato - il sistema aggrega i dati per calcolare indicatori di stato dell'azienda
        \end{itemize}
    \paragraph{Componenti fondamentali}
        \begin{itemize}
            \item Base si dati operazionale - contiene i dati operativi dell'azienda
            \item Funzioni operative - funzioni che permettono di svolgere le attività operative
        \end{itemize}
\section{Sistemi informazionali}
    \paragraph{funzioni principali}
        \begin{itemize}
            \item Facilitazione del processo decisionale
            \item Presentazione dei dati secondo diverse aggregazioni e viste
            \item Confronto tra indicatori aziendali e indicatori esterni
        \end{itemize}
    \paragraph{Azioni sui dati}
        \begin{itemize}
            \item Accesso in lettura
            \item Aggregazione dei dati
            \item Descrizione di aree/temi
            \item Profondità temporale
            \item Multi-dimensionalità
        \end{itemize}
    \paragraph{Componenti fondamentali}
        \begin{itemize}
            \item Base dati informativa
            \item Strumenti di analisi
            \item Procedure di alimentazione (dati)
        \end{itemize}
    \chapter{Le scelte organizzative}
\thispagestyle{chapterInit}
Andiamo ora ad analizzare come il sistema informativo possa essere costruito e gestito all'interno di un'azienda. In particolare andremo ad analizzare le scelte che un'azienda può fare riguardo alla costruzione e gestione del sistema informativo, quali sono le figure professionali che si occupano di questo sulla base della dimensione dell'azienda. Andiamo inoltre ad affrontare come il sistema informativo si posiziona nell'organigramma aziendale e come l'infrastruttura tecnologica può essere organizzata e gestita.
\section{Costruzione del \texttt{SI}}
    Al momento quando si parla di adottare un nuovo \texttt{SI} si possono fare tre scelte principali:
    \begin{description}
        \item[Make] costruire il \texttt{SI} internamente
        \item[Buy] acquistare il \texttt{SI} da uno o più fornitori esterni
        \item[Outsource] far gestire ad una azienda esterna il \texttt{SI}
    \end{description}
    \subsection{Opzione \textit{make}}
    \label{sec:opzMake}
        Con l'opzione \textbf{make} ovvero di \textbf{costruzione interna} si intende la costruzione del \texttt{SI} all'interno dell'azienda da parte di un gruppo di lavoro incaricato della realizzazione, manutenzione e gestione del \texttt{SI}.\newline
        Questa opzione è raramente scelta e solitamente si limita a funzioni marginali rispetto ad un \texttt{SI} completo. Il tutto però rispecchia in pieno i modelli organizzativi e si ha un controllo totale del \texttt{SI}.
        \subsubsection{Vantaggi \& Svantaggi} 
        Questa opzione comporta dei costi fissi molto elevati usati sia per il personale che è incaricato di sviluppare e mantenere l'\texttt{SI} che per l'infrastruttura sulla quale l'\texttt{SI} è installato. Inoltre quando vi è necessità di un aggiornamento importante bisogna investire molte risorse economiche. Questa soluzione ha lo svantaggio di non confrontarsi con il mercato attuale e che dunque alcune funzionalità non sono le più efficienti o efficaci. Il primo vantaggio lo si riscontra nella situazione nella quale si dovessero verificare dei problemi, allora i tempi di risoluzione saranno molto brevi per questioni banali, ma lunghi per difficoltà più complesse. Altro vantaggio importante di questa soluzione è il mantenimento interno del ``\textit{know-how}''.
    \subsection{Opzione \textit{buy}}
        \label{sec:opzBuy}
        L'opzione \textit{buy} consiste nell'acquisto di un \texttt{SI} da parte di uno o più fornitori esterni, alla quale si aggiunge l'instaurazione di un piccolo gruppo di lavoro interno per la gestione del \texttt{SI} e per la gestione dei rapporti con i fornitori del \texttt{SI}. Questa è una scelta tipica nell'economia italiana delle \texttt{PMI}.
        \subsubsection{Vantaggi \& Svantaggi}
            Questa opzione favorisce una struttura interna umana e tecnologica di molto ridotta rispetto all'\nameref{sec:opzMake} insieme ai costi fissi. L'azienda si concentra sul proprio core business senza perdere tempo e risorse nella gestione del \texttt{SI}, questo al prezzo di una stretta dipendenza dai fornitori esterni ed una perdita \underline{parziale} del \textit{know-how} aziendale oltre alla mancata proprietà del \textit{software}. Come ulteriori vantaggi si ha la maggior flessibilità del \texttt{SI} e il confronto con il mercato attuale. Il tutto però con un modello organizzativo non su misura per l'azienda e la difficoltà nel passaggio da un fornitore all'altro.
    \subsection{Opzione \textit{outsource}}
        Con l'opzione di \textbf{outsourcing} ovvero di \textbf{esternalizzazione} si delega ad una terza parte la gestione e l'organizzazione del \texttt{SI} dopo pagamento di canone periodico.\newline
        Questa opzione al giorno d'oggi sta prendendo piede tra le \texttt{PMI} in quanto permette, seppur ad un costo variabile elevato, di avere un \texttt{SI} completo e funzionante senza dover investire in \textit{hardware} e \textit{software} e senza dover adibire personale interno alla gestione del \texttt{SI}.
        \subsubsection{\textit{Hosting}}
            Con questa opzione si affida solo la parte di \textbf{infrastruttura} tecnologica, non il \textit{software} e altri servizi che solitamente vengono gestiti in ottica \hyperref[sec:opzBuy]{\textit{buy}}. Con questa opzione si può noleggiare un server fisico o una macchina virtuale.
        \subsubsection{Body rental}
            Con questo termine intendiamo l'uso di personale specialistico di una azienda esterna per trasformare costi fissi in costi variabili.
        \subsubsection{Vantaggi \& Svantaggi}
            Nell'opzione \textit{outsource} i costi sono variabili ma abbastanza alti, inoltre in caso di necessità si può aumentare o diminuire le risorse in base alle esigenze. Si è però vincolati al fornitore della soluzione utilizzata. È presente inoltre una maggiore flessibilità rispetto all'\nameref{sec:opzMake}, come al livello dell'\nameref{sec:opzBuy}. Questa opzione però non permette di avere alcun \textit{know-how} interno e quindi non si ha la possibilità di personalizzare la soluzione in base alle proprie esigenze. Inoltre si ha una maggiore dipendenza dal fornitore del servizio. Possiamo dunque definire l'opzione \textit{outsource} come una \nameref{sec:opzBuy} con costi variabili e senza controllo su nessun aspetto del \texttt{SI}.

\section{Le figure professionali}
    \paragraph{Maturità Informatica} 
        La \textbf{maturità informatica} delle aziende è un indicatore rispetto alla diversa organizzazione e alla collocazione nell'organigramma aziendale del reparto \texttt{IT}, ovvero il reparto che si occupa della gestione del \texttt{SI}. 
    \subsection{Sviluppo del reparto}
        \subsubsection{Livello 1}
            Spesso le aziende a questo livello sono alle fasi iniziali dell'automazione.
            Il \textit{team} è composto in modo da pochi elementi, spesso non specializzati ma altamente flessibili e con competenze trasversali. Questo \textit{team} ha una struttura completamente \textbf{orizzontale} e non vi è una figura di riferimento. Può capitare che i componenti del \textit{team} siano anche responsabili di altre attività aziendali e che la parte informatica sia solo una parte del loro lavoro.
        \subsubsection{Livello 2}
            \begin{figure}[H]
                \centering 
                \includegraphics[width=0.5\textwidth]{04/livello2.png}
                \caption{Schema di organizzazione del reparto \texttt{IT} a livello 2}
            \end{figure}
            In questo livello si inizia ad osservare una organizzazione gerarchica dove il \textbf{responsabile \texttt{EDP}} (Electronic Data Processing) è responsabile del reparto \texttt{IT}, sotto di lui ci sono i \textbf{sistemisti} che si occupano della gestione e dell'assistenza sull'infrastruttura. Poi ci sono gli \textbf{analisti} che supportano gli utenti e analizzano i requisiti. Infine ci sono i \textbf{programmatori} che si occupano dello sviluppo del software (non presenti nell'organizzazione \hyperref[sec:opzBuy]{\textit{buy}})
        \subsubsection{Livello 3}
            \begin{figure}[H]
                \centering 
                \includegraphics[width=0.5\textwidth]{04/livello3.png}
                \caption{Schema di organizzazione del reparto \texttt{IT} a livello 3}
            \end{figure}
            In questo livello è presente una vera e propria Direzione, sotto questa sono presenti i reparti di \texttt{EDP} e il reparto pero la ricerca su nuove tecnologie. Come mostrato della figura il primo si occupa di sviluppo di nuove applicazioni, assistenza e analisi e assistenza tecnico-sistemistica mentre nel secondo ci si occupa di Analisi dei dati, ricerca web e mobile e virtualizzazione della infrastruttura.
        \subsubsection{Livello 4}
            \begin{figure}[H]
                \centering 
                \includegraphics[width=0.5\textwidth]{04/livello4.png}
                \caption{Schema di organizzazione del reparto \texttt{IT} a livello 4}
            \end{figure}
            Livello nel quale è presente un vero e proprio \textbf{dirigente del sistema informativo} con una organizzazione gerarchica molto più complessa rispetto ai livelli precedenti. È presente una divisione \textbf{EDP} con relativo responsabile che è isolata dal reparto \textbf{nuove tecnologie} che si occupa di sviluppare nuove tecnologie e di supportare il reparto \textbf{EDP}. Inoltre sopra di questi è presente una sezione comune del sistema informativo che si occupa di coordinare i due reparti e gestire anche \textbf{privacy} e \textbf{sicurezza} o anche \textbf{pianificazione attività}. Questo livello dispone di un vero e proprio \textbf{budget} per il reparto \texttt{IT}.
    \subsection{Posizione all'interno dell'organigramma}

        Avendo definito il livello di maturità del reparto \texttt{IT} come struttura interna andiamo ora a valutare come il reparto stesso si posiziona all'interno dell'azienda, anche questo aspetto contribuisce a definire il livello di maturità dell'azienda stessa rispetto all'informatica.
        \subsubsection{Supporto amministrativo} 
            In questo caso il reparto \texttt{IT} è visto come un supporto amministrativo, questa è una visione obsoleta 
            \begin{figure}[H]
                \centering 
                \includegraphics[width=0.5\textwidth]{04/supportoAmm.png}
                \caption{Posizione del reparto \texttt{IT} come supporto amministrativo}
            \end{figure}
        \subsubsection{Servizio ad altre direzioni generali}
            In questo caso il reparto \texttt{SI} è o al pari degli altri reparti o a supporto della direzione generale. Se il reparto \texttt{IT} è al pari degli altri reparti allora questo è a supporto di tutti i reparti dell'azienda, se invece è a supporto della \texttt{DG} allora questo si posiziona più in alto rispetto agli altri reparti.
            \begin{figure}[H]
                \centering 
                \includegraphics[width=0.5\textwidth]{04/supportoTutti.png}
                \caption{Posizione del reparto \texttt{IT} come supporto a tutte le direzioni}
            \end{figure}
        \subsubsection{Organizzazione autonoma}
            In questo caso è vero che il reparto \texttt{SI} viene messo alle dipendenze del reparto \textbf{organizzazione} ma è anche vero che in questo modello il reparto \texttt{SI} è autonomo. Il suo ruolo è quello di supportare tutte le altre direzioni e di coordinare le varie aree operative dell'azienda, solitamente questo è il modello delle grandi aziende.
            \begin{figure}[H]
                \centering 
                \includegraphics[width=0.5\textwidth]{04/organizzazione.png}
                \caption{Posizione del reparto \texttt{IT} come organizzazione autonoma}
            \end{figure}
\section{Infrastruttura Tecnologica}
    Negli anni l'infrastruttura tecnologica si è evoluta siamo passati da server con vari terminali connessi fino alla virtualizzazione a archiviazione in cloud, il tutto passando dall'architettura client-server. Il \texttt{SI} deve tenersi sempre aggiornato con le tecnologie in uso.
    \paragraph{Il passaggio da server locali a soluzioni cloud}
        Recentemente molte aziende stanno eseguendo il passaggio da una infrastruttura locale ad una in cloud, in questo modo sussistono meno investimenti lato \textit{hardware} e \textit{software} che ai giorni nostri invecchiano ancora prima di essere ammortizzati. Tutto il lavoro di gestione dell'infrastruttura è spostato su una azienda esterna senza che venga a mancare l'accessibilità degli strumenti informativi aziendali. In questo modo però si è \textbf{dipendenti dalla connettività} ovvero il lavoro non può essere svolto se non si è connessi a internet.
\section{Interrompibilità del servizio informatico}
    In ogni caso bisogna sempre valutare che danno può causare l'interruzione del servizio informatico. In alcuni casi l'interruzione di esso può portare al ``blocco totale'' del lavoro, questo rischio viene spesso sottovalutato dai \textit{top manger}. Come conseguenza è importante garantire la ``continuità operativa'' ovvero la riduzione o il completo annullamento che un ``blocco totale'' possa avvenire. 
    \subsubsection{Problematiche legate all'hardware}
        L'infrastruttura dei \texttt{SI} è soggetta a guasti, bisogna dunque prevenire questi andando a implementare il concetto di ``\textit{sistema ridondato}''\footnote{La ridondanza prevede che in caso di fallimento di un singolo sistema ne esista un'altro che possa sostituirsi ad esso andando a ``coprire'' il punto difettoso} (esempio per i dischi di archiviazione: \texttt{RAID}).
        \paragraph{\textit{Hot Swap}} Importante per i supporti di archiviazione è anche il concetto di \textit{Hot Swap} che consiste nella possibilità di sostituzione di questi senza dover spegnere il sistema e quindi dover interrompere l'operabilità del \texttt{SI}. 
        \paragraph{\textit{Failure tolerancy}} L'ideale infrastruttura di un \texttt{SI} dovrebbe essere \textit{fault tolerant}, ovvero non deve esistere un oggetto\footnote{server, apparato di rete, griglia elettrica, etc\dots } che in caso di fallimento comporta al fallimento di tutto il sistema, ciò per quelle parti del sistema che sono ritenute essenziali.
        \paragraph{\textit{Backup}} In ogni ambiente si rende necessaria l'implementazione di una strategia di \textit{backup}, queste copie devono essere conservate in ambienti protetti (casseforti ignifughe, locazioni remote\dots). Una teoria suggerisce che bisogni avere sempre a disposizione tre copie dei dati su tre apparati diversi: uso, \textit{backup} interno, \textit{backup} remoto.
    \subsubsection{Problematiche legate al software}
        Maggiormente le problematiche di un \texttt{SI} sono legate al \textit{software}. I problemi legati a consistono nella ``indimostrabilità'' che un qualsiasi programma  cerchi di computare un risultato senza ``ciclare'' all'infinito. D'altra parte questo genere di malfunzionamenti raramente comporta un blocco di tutto il \texttt{SI} ma più spesso interferiscono con un processo specifico o con un gruppo di attività.
    \subsubsection{Problematiche legate ad azioni dolose}
        Molto più frequenti rispetto agli altri generi di criticità sono quelle problematiche legate ad azioni intenti a ledere l'operabilità o la segretezza dei dati. Molto spesso questo genere di problematiche è causato da vulnerabilità dell'\texttt{SI} o dell'\textit{hardware}. La causa principale però rimane l'errore umano che può essere solamente mitigato.\footnote{Più informazioni sulla sicurezza informatica in ``Appunti di Introduction to Computer and Network Security'' di Luca Facchini, capitoli 1/2}
    \chapter{I sistemi operazionali}
\thispagestyle{chapterInit}
\label{ch:operazionali}
In questo capitolo verranno esposti concetti base sui ``sistemi operazionali'' parlando dunque delle finalità di questi, dell'organizzazione della informazione operativa, della potenzialità informatiche di una azienda ed infine degli accenni alla composizione di un sistema operazionale.

\section{Finalità dei sistemi operazionali}
    Le principali finalità dei \textbf{sistemi operazionali} riguardano:
    \begin{description}
        \item[Registrazione delle transazioni] Il processo di acquisizione e memorizzazione delle informazioni relative alle transazioni aziendali.
        \item[Pianificazione e controllo] La possibilità di pianificare le operazioni aziendali e controllarne l'effettiva esecuzione.
        \item[Acquisizione ed organizzazione della conoscenza] La possibilità di acquisire e organizzare la conoscenza aziendale.
        \item[Elaborazione delle situazioni aziendali] La possibilità di elaborare le informazioni aziendali per ottenere una visione complessiva della situazione aziendale.
    \end{description}
    Per raggiungere questa finalità il sistema operazionale si compone di due sottosistemi principali:
    \begin{description}
        \item[Base di dati operazionale] Contiene le informazioni operative in forma organizzata.
        \item[Funzioni operative] Sono le funzioni che permettono di acquisire, memorizzare, elaborare e trasmettere le informazioni.
    \end{description}
    \subsection{Transazioni - Definizione e registrazione}
        \subsubsection{Cos'è una transazione}
            \paragraph{Definzione} Per definizione una \textbf{transazione} è una operazione detta \textbf{atomica} (ovvero indivisibile) che si manifesta in un certo e conosciuto momento ed è una informazione che l'azienda è interessata a registrare.
            \paragraph{Esempi} Alcuni esempi di transazioni sono: gli ordini tra cliente e fornitori, prelievi da magazzino, spedizioni, pagamenti, ecc\dots
        \subsubsection{Registrazione delle transazioni}
            Le transazioni da dover registrare, possono essere sostanzialmente di due tipi:
            \begin{description}
                \item[Semplici] Si deve registrare nel sistema solo un singolo dato.
                    \subitem Esempio di registrazione di transazione semplice è la registrazione di una movimentazione del magazzino
                \item[Complesse] Si devono registrare più operazioni elementari connesse in senso logico e spesso corelate a documenti fisici, quali ad esempio una spedizione che è correlata ad una \textit{bolla di spedizione}.
                    \subitem Esempio di registrazione di transazione complessa è la registrazione di una spedizione dove sono coinvolte più operazioni elementari quali la raccolta degli attributi (destinatario, prodotti, data e ora\dots) ed genera una bolla di spedizione.
            \end{description}
            Inoltre una transazione può generarle delle altre e quindi si parla di \textbf{transazioni a cascata}.
            \paragraph{Volume dei dati} Ogni transazione produce un volume di dati dipendentemente dalla natura dell'attività e dell'organizzazione aziendale.
    \subsection{Pianificazione e controllo delle operazioni}
        Alcuni processi aziendali sono dipendenti da altri, si rende quindi necessario usare i dati dei processi ``a monte'' per pianificare e controllare i processi ``a valle''. Tramite l'uso di \texttt{SI} è possibile adottare modelli più complessi di pianificazione e monitorare continuativamente l'andamento dello stato dei processi aziendali. 
        \paragraph{Perché pianificare e controllare} Pianificare e controllare i processi aziendali ha diversi vantaggi per l'azienda, sia per il passato che per il presente fino ad avere anche una utilità per i processi futuri. Questi vantaggi sono raggiunti tramite: La possibilità di elaborare piani e strategie di produzione, registrare e monitorare l'avanzamento delle operazioni ed infine la possibilità di misurare se e quanto i piani sono stati rispettati rispetto agli obiettivi prefissati.
            \subparagraph{Come pianificare e controllare} Il \texttt{SI} deve essere dotato di funzioni molto articolate e specifiche per l'azienda alla quale si riferisce, ad esempio quando parliamo di ``Elaborazione di piani'' il \texttt{SI} di riferimento deve essere in grado di: ottimizzare le risorse disponibili, sincronizzare le operazioni ed essere coerente con lo stato degli indicatori aziendali.
    \subsection{Organizzazione della conoscenza aziendale}
        La conoscenza aziendale (\textit{Knowledge Base} - \texttt{KB}) è l'insieme delle informazioni a supporto dell'attività di produzione aziendale, come ad esempio le informazioni sui prodotti, sui processi, sui clienti, sui fornitori, ecc\dots Queste informazioni sono essenziali per la gestione aziendale e per la pianificazione delle attività future, avere dunque all'interno del \texttt{SI} operazionale la loro versione più aggiornata sempre a disposizione è fondamentale per l'azienda.\newline
        Queste informazioni devono essere strutturate, ovvero riconducibili ad un insieme di caratteristiche predefinite, e devono essere corelate, ovvero devono essere collegate ad articoli, clienti, fornitori, ecc\dots
    \subsection{Elaborazione delle situazioni aziendali}
        Il \texttt{SI} è un sistema dinamico che serve per modellare la realtà aziendale e per fornire informazioni utili per la gestione aziendale. La conoscenza dello stato corrente, oltre che di quello passato, è fondamentale per la gestione aziendale, questa conoscenza permette di pilotare l'azienda grazie a determinati eventi. Alcuni indicatori di stato sono ad esempio: le giacenze di magazzino, i tempi di consegna, i tempi di produzione, ecc\dots\newline
        Gli indicatori dunque non rappresentano una situazione statica, ma una situazione dinamica che cambia nel tempo. Questi indicatori sono utili per la gestione aziendale e per la pianificazione delle attività future. Tutti gli indicatori di stato sono calcolati a partire dai dati inseriti, modificati e cancellati dalle transazioni aziendali e sono utili per la gestione aziendale.
\section{Informazione Operativa}
    L'informazione operativa è costituita principalmente da archivi nei quali sono presenti relazioni che coinvolgono diverse entità, questi archivi solitamente li classifichiamo in:
    \begin{description}
        \item[Movimenti] Contengono le informazioni relative alle transazioni semplici, relative ad un singolo oggetto.
        \item[Documenti] Contengono le informazioni su transazioni complesse che riguardano una lisa di oggetti (classica tabella) dove in testa troviamo anche una serie di informazioni comune a tutte le righe.
        \item[Informazioni di stato] Ovvero un insieme di indicatori di stato che permettono di avere una visione complessiva della situazione aziendale. Questi possono essere \textit{de-materializzati} e quindi calcolati al momento della richiesta o \textit{materializzati} e quindi calcolati e memorizzati in un archivio.
        \item[Informazioni Anagrafiche] Contengono le informazioni relative alle entità che partecipano alle transazioni, questi non si limitano a contenere solo dati di anagrafica di persone fisiche, ma anche di oggetti, di entità giuridiche, ecc\dots
    \end{description}
    \subsection{Qualità dei dati}
        Per qualità dei dati si fà riferimento allo standard \texttt{ISO 8402-1995}:
        \begin{quote}[\texttt{ISO 8402-1995}]{\textit{International Organization for Standardization}}
            Il possesso della totalità delle caratteristiche che portano al soddisfacimento delle esigenze espresse o implicite, dell'utente.
        \end{quote}
        \paragraph{La qualità dei dati}
            Per stabilire un indice di qualità dei dati si possono utilizzare diversi parametri quali:
            \begin{itemize}
                \item Tanto più elevata quanto più il sistema fornisce rappresentazioni degli eventi vicine alla percezione diretta della realtà
                \item La dipendenza dalla struttura del \texttt{SI} è minore quanto più i dati sono indipendenti dalla struttura del sistema
                \item La qualità è diminuita da sottosistemi non integrati e da dati ridondanti
            \end{itemize}
            In sostanza un dato per essere di qualità non deve essere ridondante, deve essere coerente con la realtà e deve essere indipendente dalla struttura del sistema.
        \paragraph{Impatto della qualità dei dati}
            Se all'interno del proprio \texttt{SI} si ha una bassa qualità dei dati, allora si avrà un forte impatto economico/organizzativo tra cui: la difficoltà nell'introduzione di innovazioni tecnologiche (adozione di una nuova tecnologia) e di processo (modificare un processo produttivo), la difficoltà nell'avvio di processi del tipo \textit{data warehousing}, inoltre dal lato umano avere una bassa qualità dei dati può portare a una scarsa soddisfazione degli utenti finali del \texttt{SI} (ovvero quelle persone che utilizzano il \texttt{SI} per svolgere il proprio lavoro).
    \subsection{Caratteristiche strutturali}
        L'informazione operativa è l'informazione che serve per svolgere le attività operative dell'azienda, questa informazione è costituita da dati che vengono acquisiti, memorizzati, elaborati e trasferiti all'interno dell'azienda. Questi dati sono utili per la gestione aziendale e per la pianificazione delle attività future. L'informazione operativa è caratterizzata da:
        \begin{description}
            \item[Aggregazione] I dati sono aggregati in base alle esigenze dell'utente, possono essere:
                \subitem \textbf{Analitici} Se si vuole avere una visione dettagliata di un singolo evento.
                \subitem \textbf{Analitici} Se si vuole avere una visione complessiva di un insieme di eventi, ottenuto aggregando i dati.
            \item[Tempificazione] I dati possono essere temporizzati in base alle esigenze dell'utente, possono essere:
                \subitem \textbf{Puntuale} Se si vuole avere una visione istantanea della situazione aziendale.
                \subitem \textbf{Cumulata} Se si vuole avere una visione della situazione aziendale in un certo periodo di tempo.
            \item[Dimensionalità] Intendiamo come dimensionalità il numero minimo di parametri necessari per estrarre una specifica informazione.
        \end{description}
        Esempio delle caratteristiche dell'informazione operativa:
        \begin{table}[H]
            \centering
            \begin{tabular}{|c|c|c|c|}
                \hline
                & \textbf{Aggregazione} & \textbf{Tempificazione} & \textbf{Dimensionalità} \\
                \hline
                \textbf{Anagrafe} & Analitica & Puntuale & unitaria \\
                \hline
                \textbf{Movimenti \& Documenti} & Analitica & Puntuale & Contenuta \\
                \hline
                \textbf{Informazioni di stato} & Analitica o aggregata & Puntuale o cumulata & Contenuta \\
                \hline
            \end{tabular}
        \end{table}
    \subsection{Caratteristiche funzionali}
        I dati operativi oltre ad essere strutturati in maniera particolare, devono anche avere delle caratteristiche funzionali che permettano di svolgere le attività operative dell'azienda. Queste caratteristiche sono:
        \begin{description}
            \item[Completezza] Estensione con cui i dati vengono registrati
            \item[Corretteza] Quanto quel dato si avvicina alla realtà
            \item[Precisione] Quanto quel dato è vicino alla realtà
            \item[Omogeneità] Se tra tutti i dati della stessa natura sono rappresentati con una stessa struttura
            \item[Fruibilità] Facilità con cui i dati possono essere reperiti, acquisiti e compresi dall'utente in relazione alle sue esigenze
        \end{description}

\footnote{
    La sezione 4.3 del libro ``Rappresentazione della realtà'' è stata omessa in quanto trattata marginalmente a lezione ed approfondita al corso di \textit{Basi di Dati}.
}
\section{Potenzialità informatica}
    \begin{definition}[Potenzialità informatica]
        La \textbf{potenzialità informatica} è costituita principalmente da due indicatori:
        \begin{description}
            \item[Intensità informativa] Misura la quantità di informazioni della quale l'azienda necessita, sia che queste vengano da fonti interne o esterne.
            \item[Attrattiva informatica] Ovvero la propensione, sulla base dei processi aziendali, dell'azienda ad utilizzare un \texttt{SI} per la gestione delle informazioni.
        \end{description}
    \end{definition}
    Inoltre la propensione del management verso l'investimento sulla infrastruttura informatica è un indicatore di potenzialità informatica. 
    \subsection{Intensità informativa}
        \begin{definition}[Intensità informatica]
            L'intensità informativa è costituita da un insieme di fattori che concorrono a determinare la quantità di informazioni di cui l'azienda necessita. 
        \end{definition}
        Questi fattori sono:
        \begin{itemize}
            \item La complessità dell'attività aziendale: oltre alla dimensione dell'azienda và presa in considerazione l'area geografica l'eventuale appartenenza ad un ``gruppo'', il livello di diversificazione dei prodotti, dei mercati e delle tecnologie. Tutti questi fattori influenzano la quantità di informazioni necessarie.
            \item L'intensità informativa
        \end{itemize}
    \begin{definition}[Intensità informativa]
        L'intensità informativa è costituita dal prodotto dell'intensità informativa di prodotto e dell'intensità informativa di processo.
    \end{definition}
    \begin{definition}[Intensità informativa di prodotto]
        L'intensità informativa di prodotto è la quantità di informazioni necessarie per la progettazione, la produzione e la commercializzazione di un prodotto.
    \end{definition}
    \begin{definition}[Intensità informativa di processo]
        L'intensità informativa di processo è la quantità di informazioni necessarie per l'avanzamento dei processi aziendali. Viene presa in considerazione anche la mole di dati che emergono dal processo e la complessità delle operazioni elementari previste dal processo.
    \end{definition}
    \subsection{Attrattiva Informatica}
        L'intensità informativa non è sufficiente a determinare se l'adozione di un \texttt{SI} possa essere vantaggiosa per l'azienda, infatti è necessario anche valutare l'attrattiva informatica.
        \begin{definition}[Attrattiva informatica]
            L'attrattiva informatica è un indicatore che misura la propensione dell'azienda ad utilizzare un \texttt{SI} per la gestione delle informazioni. Questo indicatore è costituito dall'insieme dell'attrattiva informatica dei processi aziendali.
        \end{definition}
        \begin{definition}[Attrattiva informatica di processo]
            L'attrattiva informatica di processo è costituita dai seguenti fattori:
            \begin{description}
                \item[Proceduralità] Grado di formalizzazione dei processi aziendali. Più un processo è formalizzato, più è attrattivo.
                \item[Complessità] Grado di difficoltà delle azioni elementari previste dal processo. Meno un processo è complesso e più è attrattivo.
                \item[Ripetitività] Frequenza con cui un processo viene ripetuto. Più un processo è ripetuto, più è attrattivo.
                \item[Volume] Quantità di dati e informazioni elaborate dal processo. Alti volumi di dati rendono un processo attrattivo         
            \end{description}
        \end{definition}
\section[Composizione dei \texttt{SI} Operazionali]{Composizione dei sistemi informativi operazionali}
    I \texttt{SI} operazionali sono composti da diversi sotto-sistemi che si occupano di diverse funzioni. Al momento non esiste una classificazione standard dei sotto-sistemi in quanto varia in base all'azienda e al settore di appartenenza. I criteri principali usati per la distinzione tra diversi sotto-sistemi sono: la funzione svolta, il processo aziendale coinvolto, l'architettura tecnologica, ecc\dots
    \subsubsection{Portafoglio Applicativo}
        Definiamo come \textbf{portafoglio applicativo} l'insieme delle applicazioni software che costituiscono il \texttt{SI} operativo, possono essere individuate due aree principali, ovvero il \textbf{portafoglio operativo} e il \textbf{portafoglio istituzionale}.
        \paragraph{Portafoglio Operativo} È costituito da applicazioni informatiche che trattano di processi legati al \textit{core-business} dell'azienda. Questo genere di portafoglio è caratterizzato da una elevata specializzazione ad un settore specifico oltre ad una elevata variabilità tra aziende dello stesso settore di appartenenza. Inoltre questo portafoglio è caratterizzato da una forte verticalizzazione e da una elevata specializzazione delle funzioni implementate.
        \paragraph{Portafoglio istituzionale} È costituito da applicazioni informatiche che trattano di processi a sostegno delle principali attività aziendali. Questo genere di portafoglio è caratterizzato da una elevata attrattiva informatica e da una alta proceduralita (e ripetitività) dei processi. Inoltre questo genere di portafoglio è caratterizzato da una elevata omogeneità tra aziende anche di settori diversi e la poca variabilità tra servizi e prodotti offerti.
    \subsubsection{Sistema gestionale classico}
        Nel modello del sistema gestionale classico solitamente sono presenti delle isole informatiche autonome e molto specializzate, questo genere di sviluppo è causato principalmente da uno sviluppo incrementale (ovvero lo sviluppo viene eseguito a compartimenti ``stagni'' uno per volta), da una rigidità delle organizzazioni aziendali, dalla specializzazione dei produttori di software ecc\dots
        \paragraph{Problematiche} Questo genere di sviluppo porta a diverse problematiche quando si discute sulla gestione dei dati, in questo sistema infatti i dati sono ridondanti, disomogenei e spesso incoerenti. Ad aggiungersi a ciò questo genere di gestionale rende difficile avere una visione complessiva dell'azienda.
        \begin{figure}[H]
            \centering
            \includegraphics[width=0.5\textwidth]{05/gestionaleClassico.png}
            \caption{Schema dei settori di un sistema gestionale classico}
        \end{figure}
    \subsubsection{Sistema \texttt{ERP} - \textit{Enterprise Resource Planning}}
        Un sistema \texttt{ERP} è un sistema informativo aziendale integrato, ovvero un sistema che permette di gestire in maniera integrata e coordinata tutte le informazioni aziendali. Questo genere di sistema, grazie ad una basi di dati unica ed a processi integranti e cooperanti, punta a trattare i dati in modo ottimale e ottimizzato, oltre a gestire il controllo dei processi aziendali.
        \paragraph{Vantaggi} Questo genere di sistema così implementato è molto flessibile ed in grado di assecondare l'azienda in ogni sua esigenza, inoltre permette di avere una visione complessiva dell'azienda e di avere una gestione ottimale dei processi aziendali. Inoltre questo si adatta molto rispetto all'organizzazione aziendale e all'architettura tecnologica.
        \begin{figure}[H]
            \centering
            \includegraphics[width=0.33\textwidth]{05/gestionaleERP.png}
            \caption{Schema dei settori di un sistema gestionale \texttt{ERP}}
        \end{figure}
    \subsubsection{Ambiti applicativi}
        La presenza di moduli indipendenti presente negli \texttt{ERP} rendono questi sistemi molto flessibile anche a tipologie diverse di aziende, infatti un \texttt{ERP} può essere utilizzato in diversi ambiti applicativi, anche molto diversi tra loro. Tra questi ambiti troviamo: Servizi Finanziari, Produzione, Distribuzione, Commercio, Servizi, ecc\dots
        \paragraph{Flussi di base} I flussi di base di un \texttt{ERP} sono:
        \begin{description}
            \item[Amministrativo] Flusso di prima applicazione che riguarda la gestione amministrativa dell'azienda.
            \item[Logistico] Flusso che riguarda la gestione dei processi logistici dell'azienda.
            \item[Attivo (vendite)] Flusso che riguarda la gestione delle vendite dell'azienda.
            \item[Passivo (acquisti)] Flusso che riguarda la gestione degli acquisti dell'azienda.
            \item[Produttivo] Uno trai flussi più complessi, riguarda la gestione della produzione dell'azienda, questo può variare anche di molto da azienda ad azienda. 
        \end{description}
    \subsubsection{Sistemi operazionali complementari}
        I sistemi operazionali complementari sono sistemi che vengono aggiunti al sistema operativo principale per coprire delle funzioni che non sono presenti nel sistema operativo principale. Questi sistemi sono caratterizzati da una forte specializzazione e da una forte integrazione con il sistema operativo principale.
        \paragraph{Esempi di sistemi operazionali complementari} Alcuni esempi di sistemi operazionali complementari sono: i sistemi di supporto alle decisioni, i sistemi di gestione della qualità, i sistemi di gestione ambientale, le estensioni dell'\texttt{ERP} ecc\dots
    \chapter{\texttt{ERP}}
\thispagestyle{chapterInit}
In questo capitolo affrontiamo le varee aree di un \texttt{ERP} e le funzionalità che queste offrono.
\section{Area Amministrativa}
    In questo capitolo si analizzeranno le funzionalità dell'area amministrativa di un sistema \texttt{ERP}. Questa area è fondamentale per la gestione delle risorse finanziarie e contabili dell'azienda. In particolare, si analizzeranno le seguenti funzionalità: Piano dei conti, gestione di anagrafiche e movimenti contabili, finanziari ed IVA.

    La parte amministrativa di un \texttt{SI} si basa su delle strutture base, queste possono essere o per la gestione dell'anagrafica, o per la gestione dei movimenti contabili. In particolare nella parte anagrafica ritroviamo: \begin{itemize}
        \item Piano dei conti
        \item Clienti
        \item Fornitori
        \item Istituti bancari
    \end{itemize}
    Nella parte contabile invece troviamo: \begin{itemize}
        \item Movimenti contabili
        \item Movimenti finanziari
        \item Movimenti IVA
    \end{itemize}
    Questa separazione di vari aspetti ci permette di avere una visione più chiara e ordinata delle informazioni.
    \paragraph{Piano dei conti} Il piano dei conti è una struttura gerarchica che permette di classificare i conti contabili in modo ordinato. In questa struttura troviamo sia conti di attivo che di passivo, e conti di costo e ricavo. Solitamente questa è organizzata in tre livelli ma in moduli più avanzati si possono trovare anche più (o meno) livelli. Questa struttura permette di avere una visione chiara e ordinata delle informazioni contabili andando a raggruppare i conti in base alla loro natura. Il piano dei conti è rappresentato Solitamente tramite una tabella, i cui campi principali sono: Codice, Descrizione, Livello, Classe e Tipo. Il codice, anche questo organizzato in modo gerarchico permette di identificare in modo univoco il conto, es: 1.1.001 significa il conto 001 di livello 3, sotto il conto 1.1 di livello 2, sotto il conto 1 di livello 1. La descrizione è il nome del conto come ``Crediti'', ``Acquisto merci'', ``Vendita merci'' ecc\dots Il livello indica il livello gerarchico del conto, la classe indica se il conto è un mastro, un conto o un sotto/conto, infine il tipo indica se il conto è patrimoniale o economico.
    \paragraph{Anagrafiche} Le anagrafiche sono una raccolta delle informazioni riguardanti i clienti, i fornitori, ecc\dots Queste informazioni sono fondamentali per la gestione delle attività dell'azienda. In particolare possiamo raccogliere informazioni proprie del cliente quali il nome, la ragione sociale, il codice fiscale, la partita IVA, ecc\dots queste non riguardano dei flussi in particolare ma sono proprie del cliente e difficilmente subiranno modifiche. Inoltre possiamo raccogliere informazioni di natura contabile e/o finanziaria che riguardano a pieno i flussi di denaro, come il limite di credito, la valuta, il pagamento, ecc\dots oltre ai dati in se per sè possono essere presenti informazioni riguardo ad accordi commerciali e/o sospensione del servizio.
    \paragraph{Movimentazione contabile} La movimentazione contabile è la parte più importante dell'area amministrativa di un \texttt{ERP}. Mentre la parte di anagrafiche permette di raccogliere informazioni riguardanti i clienti, i fornitori, ecc\dots la movimentazione contabile permette di raccogliere informazioni riguardanti i flussi di denaro. La struttura dei movimenti contabili si divide in: voce contabile (quale azienda, l'erario IVA, la cassa, ecc\dots), il Dare (uscite), l'avere (entrate) il Saldo ed il Segno (positivo o negativo). In questa struttura andiamo a registrare tutte le operazioni che vengono eseguite dall'azienda, in modo da avere un quadro chiaro e preciso della situazione finanziaria dell'azienda. Questa struttura è fondamentale per la gestione delle risorse finanziarie dell'azienda, in quanto permette di avere un quadro chiaro e preciso della situazione finanziaria dell'azienda. 
    \paragraph{Movimentazione finanziaria} La movimentazione finanziaria traccia i debiti e i crediti rateizzati, i pagamenti e le scadenze. In generale quando si deve ricevere/effettuare un pagamento si crea un movimento finanziario, questo rimane con stato ``Aperto'' fino a quando non viene saldato e/o viene emessa una nota di credito o di debito da parte della stessa controparte che vada a saldare tutta o parte della somma dovuta. In sostanza la parte di movimentazione finanziaria deve tenere traccia del debito/credito di un'azienda nei confronti di un'altra e aggiustare di conseguenza i saldi contabili.
    \paragraph{Movimentazione \texttt{IVA}} La movimentazione \texttt{IVA} è una parte molto complessa ed in continuo aggiornamento in quanto deve tenere traccia della attuale normativa fiscale. In generale la movimentazione \texttt{IVA} deve tenere conto dell'imponibile e dell'aliquota \texttt{IVA} applicata, la quale varia in base alla natura del prodotto e se il cliente rientra nella categoria di ``non soggetto'' o ``esente''. Inoltre la movimentazione \texttt{IVA} deve tenere conto delle fatture emesse e ricevute, e delle note di credito e debito emesse e ricevute. In generale la movimentazione \texttt{IVA} deve tenere traccia di tutte le operazioni che riguardano l'IVA, in modo da poter calcolare in modo corretto l'IVA da versare all'erario. Tutte le righe di movimentazione \texttt{IVA} devono essere collegate ad una riga di movimentazione contabile.
\section{L'area logistica}

\section{L'area vendite}

\section{L'area acquisti}

\section{L'area produttiva}

\section{Sistemi Operazionali complementari}
    \chapter{I sistemi Informazionali}
\thispagestyle{chapterInit}
In questo capitolo si affrontano i sistemi informazionali, ovvero quei sistemi che permettono di estrarre informazioni utili dai dati aziendali. Questi sistemi sono fondamentali per la gestione delle risorse aziendali e per la pianificazione delle attività aziendali.
\section{Gli Obbiettivi}
    L'obbiettivo generale che un buon sistema informazionale dovrebbe avere è quello di poter sfruttare tutti i dati raccolti tramite i flussi operazionali per poterli usare a supporto delle attività operative e per identificare informazioni utili per la gestione dell'azienda. In particolare un sistema informazionale segue i seguenti obbiettivi:
    \begin{itemize}
        \item Il sistema informazionale come strumento a supporto delle decisioni superando sistemi di reporting e fogli di calcolo
        \item Interrogazione dei dati in maniera puntuale e complessa 
        \item Base di dati 
        \item Strumenti di analisi dei dati
    \end{itemize}
    \paragraph{Sistema informazionale come supporto alle decisioni} Il sistema informazionale deve essere uno strumento al supporto delle decisioni tramite l'elaborazione di dati, prima dei sistemi informazionali questi venivano forniti tramite \textit{report} i quali però presentano \textit{bias} basati sulla persona che lo prepara (come dati omessi o punti di enfasi), inoltre risultano statici e difficili nella loro creazione. Altri strumenti superati grazie ai sistemi informazionali sono i fogli di calcolo, i quali sono soggetti a errori, è difficile collaborare e sono difficili da mantenere in quanto se si vuole modificare una formula bisogna modificarla in tutti i fogli di calcolo distribuiti.
    \paragraph{Sistema informazionale come strumento di interrogazione dei dati} Le interrogazioni si suddividono principalmente in: interrogazioni puntuali e interrogazioni complesse. Esempio di interrogazione puntuale è la ricerca della modalità di pagamento di un dato cliente, mentre un esempio di interrogazione complessa è l'analisi dell'aumento del margine operativo per una serie di prodotti rispetto all'anno precedente.
    \paragraph{Sistema informazionale come strumento di accesso alla base di dati} Il sistema informazionale deve offrire agli utenti finali un modello intuitivo ed efficiente per l'analisi, inoltre deve garantire la possibilità di integrare da fonti diverse i dati ed offrire processi di aggiornamento di questi.
    \paragraph{Sistema informazionale come strumento di analisi dei dati} In questo caso il sistema informazionale deve offrire strumenti di analisi dei dati, come \textit{reporting} di dati, oppure offre dei sistemi per l'analisi interattiva da una ipotesi iniziale oppure offre sistemi di \textit{Data Mining}
\section{Concetti Generali}
    \subsection{Terminologia}
        \paragraph{\textit{Data Warehouse}} Il \textit{Data Warehouse} è quell'insieme di tecniche per definire, costruire e mantenere un \textit{Data Base} che sia orientato all'analisi dei dati. Questo \textit{Data Base} deve essere molto strutturato e i dati organizzati in maniera efficiente per permettere l'analisi dei dati.
        \paragraph{\textit{Decision Support System} - \texttt{DSS}} Il \texttt{DSS} è una parte del sistema informazionale che integrato nel processo decisionale dell'azienda, permette di visualizzare ed estrarre le informazioni da basi di dati ben organizzate
        \paragraph{\textit{Data mining}} Il \textit{Data Mining} è una tecnica che permette di estrarre informazioni utili da un grande insieme di dati, questa tecnica è utilizzata per scoprire relazioni tra i dati e per fare previsioni.
        \paragraph{\textit{Buisness Intelligence}} Ovvero tutte le attività di estrazione di informazioni dai dati di \textit{business} generati dai processi operativi aziendali.
        \paragraph{\textit{Knowledge Management}} Ovvero l'insieme della conoscenze che ogni individuo possiede i quali dovrebbero essere distribuiti in maniera efficiente all'interno del sistema informazionale. Spesso questa conoscenza è derivata da esperienze passate, da informazioni acquisite e da competenze acquisite, alcune volte questa conoscenza è difficile da trasferire ed è difficile da codificare.
        \paragraph{\textit{Big Data}} Estensione del concetto di \textit{Data Warehouse} in cui si considerano anche dati provenienti da flussi continui e dati non strutturati. Vengono considerati, sotto il concetto di \textit{Big Data}, anche tutti quei dati la quale produzione non ha costi aggiuntivi (stringhe di un motore di ricerca, tempo medio di visita ad una pagina web, ecc\dots). 
    \subsection{Le caratteristiche}
        \paragraph{Finalità} Il sistema informazionale deve essere in grado di fornire un substrato informativo per la conoscenza dell'azienda, inoltre deve essere in grado di descrivere il passato ed aiutare ad identificare i problemi e le loro cause. Inoltre deve poter suggerire i cambiamenti da apportare e fornire anticipazioni sui scenari futuri. 
        \paragraph{Struttura} I dati devono essere articolati intorno ai soggetti di cui si vuol conoscere l'apporto alla vita aziendale. Si organizzano dunque le informazioni per ``tema'' e non per ``funzione'' come nei sistemi operazionali.
        \paragraph{Utenza} I principali utenti del sistema informazionale sono i manager e i decisori che devono avere una visione e conoscenza ampia dell'azienda
        \paragraph{Storicità} Il sistema deve mantenere e fornire uno storico dei dati con un arco temporale adeguato e molto più esteso rispetto a quello dei sistemi operazionale, inoltre deve fornire l'evoluzione storia dei soggetti di interesse.
        \paragraph{Dettaglio} I dati sono quasi esclusivamente in forma aggregata e devono essere disponibili a diversi livelli di aggregazione.
        \paragraph{Accesso} L'accesso ai dati è principalmente solo in lettura, eventuali aggiornamenti sono solo periodici e in momenti nei quali l'attività aziendale è ferma.
\section{Modello multidimensionale}
    Il modello multidimensionale è un modello di rappresentazione dei dati che permette di rappresentare i dati in maniera più intuitiva rispetto al modello relazionale. In questo modello il processo di analisi viene posto al centro del sistema e non più il processo di inserimento dei dati (i quali vengono inseriti non in maniera diretta ma vengono estratti da un \texttt{SI} Operazionale\footnote{Si rimanda al capitolo \ref{ch:operazionali}: \nameref{ch:operazionali} per ulteriori informazioni sui sistemi operazionali}).

    In questo modello lo spazio delle informazioni viene rappresentato come insieme di matrici multidimensionali, dove ogni matrice rappresenta un tipo di evento (quale ad esempio l'immatricolazioni degli studenti), ogni elemento della matrice rappresenta un singolo evento (la singola immatricolazione) e ogni coordinata della matrice rappresenta una dimensione dell'evento (come ad esempio il corso di laurea, l'anno di immatricolazione, ecc\dots).
    \subsection{Caratteristiche}
        \paragraph{Ipercubo} Il modello multidimensionale è rappresentato da un ipercubo, ovvero una matrice multidimensionale con $n$ dimensioni, dove ogni cella rappresenta un singolo \textbf{fatto elementare}, ogni \textbf{dimensione} costituisce una coordinata del fatto e ogni \textbf{misura} è il valore numerico del fatto.
        \paragraph{Fatti} Un fatto è un evento che si vuole analizzare e misurare, quale una vendita, un acquisto, ecc\dots. I fatto sono caratterizzati da un insieme di dimensioni che lo collocano nel tempo (quando è avvenuto), nello spazio aziendale (dove è avvenuto) e inoltre sono presenti altre dimensioni che lo \underline{quantificano} e altre informazioni descrittive. 
        \paragraph{Misure} La misura è una caratteristica (e non dimensione) del fatto che si vuole analizzare, e ne descrive un aspetto \underline{quantitativo}. Ogni fatto può contenere una o più misure, le quali possono essere: \textbf{effettive} se memorizzate in maniera diretta, \textbf{calcolate a \textit{run-time}} usando i valori delle misure effettive o \textbf{implicite} ovvero che indicano la presenza o meno di un fatto.
    \subsection{Aggregabilità} 
        A partire dai \textbf{fatti} elementari si possono ricavare dei \textbf{fatti} sintetici quando si procede all'eliminazione di una o più dimensioni, in questo caso si parla di \textbf{aggregazione} dei dati. Questo processo di aggregazione viene fatto tramite opportuni operatori sui dati, in base a cosa rappresenta la dimensione aggregata, ad esempio se aggreghiamo per mese non ha "senso" avere a fine anno la somma del totale dei prodotti in magazzino a fine mese, ma ha senso avere la media mensile.
        \paragraph{Operatori di aggregazione} Per ogni coppia costituita da (misura, aggregazione) possono essere definiti operatori e regole di aggregazione, ad esempio può essere presente una misura non aggregabile lungo una dimensione (vedi esempio precedente), oppure un operatore può essere utilizzato per aggregare lungo determinate dimensioni ma non in altre. Definiamo quindi con il termine \textbf{aggregabilità} \begin{definition}[Aggregabilità]
            La possibilità di usare un operatore di aggregazione su una misura o su una coppia (misura, dimensione).
        \end{definition}
        Nel caso più specifico definiamo anche il termine \textbf{additività} ovvero \begin{definition}[Additività]
            La possibilità di usare l'operatore di aggregazione ``somma'' su una misura o su una coppia (misura, dimensione).
        \end{definition}.
        \paragraph{Tipi di misura} Le misure possono essere di diversi tipi, tra i quali troviamo:
        \begin{description}
            \item[di Livello] si prende in considerazione il valore proprio del fatto, ed il momento nel quale è stato registrato. Non viene mai usata l'aggregazione additiva sulla dimensione temporale.
            \item[Unitaria] si prende in considerazione il valore di uno dei soggetti della misura, e non è mai aggregabile con l'additività.
            \item[di Flusso] si prende in considerazione il valore proprio del fatto ed un intervallo di riferimento, questa misura è aggregabile con l'additività lungo una qualunque dimensione.
        \end{description}
        \paragraph{Esempio}
        \begin{table}[H]
            \begin{tabular}{|c|c|c|c|c|}
                \hline
            \multicolumn{2}{|c|}{\textbf{Articolo}} & \textbf{Deposito} & \textbf{Data} & \textbf{Misura (quantità)} \\ \hline
            PP1007015 & Pannello di polistirolo 100x70x15 & 1 & 01/01/2019 & 100 \\ \hline
            PP1007015 & Pannello di polistirolo 100x70x15 & 2 & 01/01/2019 & 200 \\ \hline
            VA1010 & Vite autofilettante 10x10 & 1 & 01/01/2019 & 1000 \\ \hline
            &\dots&\dots&\dots&\dots\\ \hline
            PP1007015 & Pannello di polistirolo 100x70x15 & 1 & 01/02/2019 & 150 \\ \hline
            PP1007015 & Pannello di polistirolo 100x70x15 & 2 & 01/02/2019 & 250 \\ \hline
            VA1010 & Vite autofilettante 10x10 & 1 & 01/02/2019 & 1100 \\ \hline
            \end{tabular}
            \caption{Esempio di tabella di fatti}
        \end{table}
        In questo esempio abbiamo una tabella di fatti, dove ogni riga rappresenta un singolo fatto, ogni colonna rappresenta una dimensione del fatto e l'ultima colonna rappresenta la misura del fatto. In questo caso la misura è la quantità di prodotto in magazzino. Questa è additiva rispetto alla dimensione "Deposito" ma non lo è rispetto alla dimensione "Data", inoltre non è aggregabile rispetto alla dimensione "Articolo".
\section{Caratteristiche strutturali e funzionali}
    Andiamo ad analizzare come sono strutturati i dati all'interno di un sistema informazionale, in particolare analizziamo le caratteristiche strutturali e funzionali di un sistema informazionale.
    \subsection{Caratteristiche strutturali}
        Le caratteristiche appena descritte sono le fondamenta dell'organizzazione dei dati in un sistema informazionale. Esistono però altre caratteristiche che sono fondamentali per la corretta organizzazione dei dati in un sistema informazionale.
        \paragraph{Multidimensionalità} Per dimensionalità, in generale, si intente il numero di dimensioni necessarie per identificare un fatto, mentre nei sistemi operazionali abbiamo una dimensionalità puntuale (ad esempio il codice fiscale di un cliente), nei sistemi informazionali, vista la loro natura, per identificare un fatto servono tutte le dimensioni che lo caratterizzano.
        \paragraph{Granularità} La granularità misura quanto l'informazione proposta è sintetica rispetto agli eventi sulla quale si basa, nel caso dell'ipercubo questa è una granularità minima in quanto ogni fatto corrisponde alla cella di origine dell'ipercubo. Avere diversi gradi di granularità permette di stilare analisi più efficiente ed avere una visione più chiara dei dati.
        \paragraph{Arco temporale} L'arco temporale rappresenta il periodo di tempo che si vuole analizzare, in generale questo arco temporale è molto più ampio rispetto a quello dei sistemi operazionali, inoltre deve essere possibile analizzare i dati in maniera retrospettiva. Deve essere possibile avere almeno uno storico di 10-15 anni.
        \paragraph{Profondità storica} Mentre i sistemi operazionali non tengono traccia solitamente di come un particolare dato e/o anagrafica si è evoluto nel tempo. Questa caratteristica è fondamentale nei sistemi informazionali in quanto anche l'evoluzione del fatto nel tempo è una dimensione del fatto stesso.
    \subsection{Caratteristiche funzionali}
        Oltre ad avere una struttura ben definita i sistemi informazionali devono avere delle caratteristiche funzionali ben definite, queste sono fondamentali per avere una analisi complete e coerente dei dati rilevati.
        \paragraph{Integrazione dei dati} I dati proposti al \texttt{SI} informazionale spesso provengono da fonti diverse con diverse strutture e formati, è quindi fondamentale avere un sistema che permetta di integrare questi dati in maniera efficiente e coerente in modo da evitare errori, duplicazioni, ecc\dots
        \paragraph{Accessibilità} In quanto i sistemi a supporto delle decisioni spesso vengono usati da persone con poca esperienza informatica è fondamentale che l'accesso ai dati sia il più semplice ed intuitivo possibile. Inoltre è fondamentale che il tempo di accesso ai dati sian in tempo utile per l'attività decisionale.
        \paragraph{Flessibilità} Nel contesto dei sistemi informazionali intendiamo con flessibilità la capacità del sistema di adattarsi alle diverse interrogazioni possibili sui dati. Quindi il sistema deve essere in grado di articolare le richieste, aggregare dati a più livelli e con criteri non predefiniti, dare la possibilità di mettere in relazione misure diverse, ecc\dots Questo è fondamentale perché ogni azienda ha esigenze e dati diversi, e visto che è compito dell'\texttt{SI} informazionale supportare le decisioni ed suggerire cambiamenti è fondamentale che questo possa rappresentare i dati come richiesto dall'utente.
        \paragraph{Correttezza} I dati proposti dal sistema informazionale devono essere necessariamente corretti, ciò visto lo scopo del sistema informazionale è fondamentale in quanto decisioni prese su dati errati possono portare a conseguenze disastrose. I dati necessari all'analisi non sono spesso di utilità operativa, quindi non sono controllati come i dati operativi, inoltre i dati necessari all'analisi provengono da più fonti ed è quindi possibile che la stessa entità sia memorizzata in maniera diversa in due fonti diverse.
        \paragraph{Completezza} I dati proposti dal sistema informazionale devono essere completi, ovvero non devono mancare dati utili al processo decisionale, per garantire questo la \textit{data warehouse} deve essere popolata con tutti i dati necessari all'analisi.
    \subsection{\textit{Data warehouse} e \textit{Data mart}}
        Come già affrontato la \textit{data warehouse} è un insieme di dati raccolti su decenni di informazioni aziendali, organizzati in maniera efficiente all'analisi. Questi dati provengono da fonti relazionali diverse (\texttt{SI} operazionali, fonti esterne, ecc\dots) e vengono integrati in un unico sistema. Dato che la \textit{data warehouse} può raggiungere dimensioni molto grandi quando si deve andare ad analizzare un particolare aspetto dell'azienda l'utente può creare un \textit{data mart}, ovvero una porzione della \textit{data warehouse} che contiene solo i dati necessari all'analisi. Le \textit{data mart} possono essere "telematiche" ovvero possono essere create in tempo reale selezionando aspetti specifici della \textit{data warehouse} oppure possono essere "fisiche" ovvero possono essere create in maniera fisica e mantenute in maniera separata dalla \textit{data warehouse}.
    \chapter{\textit{Data Warehousing}}
\thispagestyle{chapterInit}

I sistemi di \textit{data warehousing} sono alla base di quasi tutti i \texttt{DSS} (\textit{Decision Support System}) e \texttt{BI} (\textit{Business Intelligence}) sono progettati per gestire grandi quantità di dati, per fornire un accesso rapido e per supportare le operazioni di analisi e di reporting.

\section{\textit{Data Warehouse} e metodologia \texttt{OLAP}}
    Prima della metodologia \texttt{OLAP} nasce la metodologia \texttt{OLTP} (\textit{On-Line Transaction Processing}) che è una tecnica di gestione dei dati per sistemi operazionali basata su 12 criteri.
    La metodologia \texttt{OLAP} (\textit{On-Line Analytical Processing}) è una tecnica di analisi dei dati iterativa.
    Negli anni i criteri per definire se un prodotto è \texttt{OLAP} o meno sono stati semplificati in 5 punti caratterizzati dall'acronimo \texttt{FASMI}:
    \begin{description}
        \item[\texttt{F}] \textit{Fast Analysis}: l'analisi deve essere veloce. (Tempo di risposta medio inferiore a 5 secondi)
        \item[\texttt{A}] \textit{Analytical} (Analitico): l'analisi deve essere analitica e deve:
            \subitem Dare la possibilità di eseguire nuovi calcoli a partire dai calcoli già eseguiti.
            \subitem Fornire risposte a richieste specifiche e non solo a domande predefinite.
            \subitem Fornire i dati in rappresentazioni molteplici.
        \item[\texttt{S}] \textit{Shared} (Condiviso): i dati devono essere utilizzabili da più utenti contemporaneamente che condividono la stessa base di dati di analisi.
        \item[\texttt{M}] \textit{Multidimensional} (Multidimensionale): i dati devono essere organizzati in modo multidimensionale.
        \item[\texttt{I}] \textit{Informational} (Informativo): il sistema deve contenere tutte le informazioni necessarie per l'analisi indipendentemente da dove e come sono memorizzate.
    \end{description}
\section{Architettura dei sistemi di \textit{data warehousing}}
    Un sistema di \textit{data warehousing} è costituito da basi di dati poste a livelli diversi, ognuno di questi ha una finalità, una struttura ed una tipologia di dati contenuti. Questi livelli sono:
    \begin{description}
        \item[Sorgenti] Sono le basi di dati operative (o esterne) da cui si estraggono i dati.
        \item[\textit{Staging Area}] È una area intermedia in cui i dati estratti dalle sorgenti vengono trasformati e caricati in una forma adatta per l'immagazzinamento presso la \textit{Data Warehouse}.
        \item[\textit{Data Warehouse}] È il livello in cui i dati vengono immagazzinati in modo da poter essere utilizzati per l'analisi.
        \item[\textit{Data Mart}] È un sottoinsieme del \textit{Data Warehouse} che contiene dati specifici per un determinato settore o per un determinato gruppo di utenti.
    \end{description}
    Elementi propri di un sistema sono oltre alle varie basi di dati:
    \begin{itemize}
        \item Procedure  che permettono di estrarre, trasformare e caricare i dati.
        \item Strumenti di analisi
    \end{itemize}
    \begin{figure}[H]
        \centering
        \includegraphics[width=0.7\textwidth]{08/livelliDW.png}
        \caption{Livelli di un sistema di \textit{data warehousing}}
    \end{figure}

\section{Modelli concettuali per il \textit{data warehouse}}
    I sistemi informazionali vista la quantità di informazioni contenute all'interno necessitano di modelli dati a descrivere la composizione del sistema e della sua base di dati. Viene dunque definito uno schema dei fatti, i \texttt{DFM} (\textit{Dimensional Fact Model}).
    \subsection{\textit{Dimensional Fact Model} - \texttt{DFM}}
        Il modello \texttt{DFM} viene usato per descrivere un fatto, tutte le sue misure e le dimensioni usabili per l'analisi (quelle sulle quali il fatto è aggregabile). Un \texttt{DFM} è composto da:
        \begin{itemize}
            \item Un fatto rappresentato tramite un rettangolo
            \item Le misure contenute nel fatto (sia quelle proprie che quelle derivate)
            \item Le dimensioni base collegate al fatto (le coordinate del fatto) rappresentate da cerchi
            \item Gli attributi descrittivi collegati o al fatto o alla dimensione.
            \item Le gerarchie tra le dimensioni sono rappresentate come un albero con la radice in alto.
                \subitem Se una gerarchia è condivisa tra più dimensioni si rappresenta con un cerchio pieno con $n$ collegamenti al fatto e alle dimensioni
            \item É possibile inoltre rappresentare le relazioni di aggregabilità tra misure e dimensioni. Queste relazioni sono rappresentate con una linea tratteggiata (se non è possibile aggregare) o continua (se è possibile aggregare).
        \end{itemize}
        \begin{figure}[H]
            \centering
            \includegraphics[width=0.5\textwidth]{08/DFM.png}
            \caption{Esempio di \texttt{DFM} con tutte le caratteristiche descritte}
        \end{figure}
        C'è da aggiungere che il presente modello è riconducibile ad un modello \texttt{E-R} (\textit{Entity-Relationship}) in cui le dimensioni sono le entità sono le entità e le gerarchie sono le relazioni tra le entità ed il fatto è l'entità principale.
\section{Modelli logici per il \textit{data warehouse}}
    Mentre la modellazione concettuale riguarda come i fatti e le dimensioni sono collegati tra loro, la modellazione logica riguarda come i fatti e le dimensioni sono effettivamente memorizzati nel \textit{Data Warehouse}. I modelli logici si differenziano sulla base della scelta del \texttt{DBMS} e della struttura di memorizzazione dei dati. Oltre a definire come i dati sono memorizzati, i modelli logici definiscono anche come i dati sono interrogati e analizzati.
    \subsection{\texttt{ROLAP}}
        Il modello \texttt{ROLAP} (\textit{Relational OLAP}) è un modello logico che prevede l'utilizzo di un \texttt{DBMS} relazionale e la memorizzazione viene eseguita tramite tabelle. Questo modello inoltre prevede che le interrogazioni avvengano tramite \textit{query} \texttt{SQL} con l'uso di viste e/o funzioni di aggregazione.
        \paragraph{Vantaggi} Ridotto uso di spazi di memorizzazione e la maggiore conoscenza degli strumenti relazionali portano ad una maggiore facilità di gestione.
        \paragraph{Svantaggi} Le prestazioni delle interrogazioni possono essere peggiori rispetto ad altri modelli e la complessità delle interrogazioni può aumentare se lavoriamo con diverse aggregazioni ed analisi.
    \subsection{\texttt{MOLAP}}
        Il modello \texttt{MOLAP} (\textit{Multidimensional OLAP}) è un modello logico che prevede l'utilizzo di strutture dati intrinsecamente multidimensionali. Questo modello prevede che i dati siano memorizzati in strutture multidimensionali (quali vettori, matrici, \dots) i sistemi allocano spazio per tutte le possibili combinazioni di dimensioni e misure.
        \paragraph{Vantaggi} Le prestazioni delle interrogazioni sono migliori rispetto ad altri modelli in quanto il fatto ricercato viene trovato in un'unica posizione e non è necessario ``simulare'' le dimensioni. Inoltre le operazioni di aggregazione sono più veloci in quanto si considera solo il livello di aggregazione richiesto.
        \paragraph{Svantaggi} L'uso di spazio di memorizzazione è maggiore rispetto ad altri modelli, solitamente solo il $20\%$ dello spazio allocato viene utilizzato. Inoltre non esiste uno standard, tutte le implementazioni sono proprietarie e non inter compatibili.
    \subsection{\texttt{HOLAP}} 
        Il modello \texttt{HOLAP} (\textit{Hybrid OLAP}) è un modello logico che prevede l'utilizzo di un \texttt{DBMS} relazionale e di strutture multidimensionali. Questo modello prevede che i dati siano memorizzati in strutture multidimensionali (per le aggregazioni di livello più alto) e in tabelle relazionali (per le aggregazioni di livello più basso). Le interrogazioni possono essere eseguite sia tramite \textit{query} \texttt{SQL} che tramite \textit{query} multidimensionali.
        \paragraph{Vantaggi} Questo modello permette di sfruttare i vantaggi di entrambi i modelli, inoltre permette di utilizzare le strutture multidimensionali per le aggregazioni di livello più alto e le tabelle relazionali per le aggregazioni di livello più basso. Infatti le \textit{data mart} possono essere memorizzati in modo multidimensionale mentre il \textit{data warehouse} può essere memorizzato in modo relazionale.
        \paragraph{Svantaggi} La complessità di gestione è maggiore rispetto ad altri modelli e le prestazioni delle interrogazioni possono essere peggiori rispetto ad altri modelli.
    \subsection{Schemi multidimensionali su basi di dati relazionali}
        Visto che la maggior parte delle \textit{data warehouse} vengono memorizzati in basi di dati relazionali, vediamo come è possibile implementare uno schema multidimensionale su una base di dati relazionale.
        \subsubsection{Schema a stella}
            Lo schema a stella è un modello multidimensionale in cui il fatto, le sue misure semplici e elementi ``chiave'' riguardanti le dimensioni sono memorizzati in una tabella centrale (il fatto), le dimensioni vere e proprie sono memorizzate in tabelle separate collegate al fatto tramite una chiave esterna. Questo modello è molto semplice e facile da implementare, inoltre è molto efficiente per le interrogazioni.
            \begin{figure}[H]
                \centering
                \includegraphics[width=0.5\textwidth]{08/stella.png}
                \caption{Schema a stella}
            \end{figure}
        \subsubsection{Schema a fiocco di neve} 
            Lo schema a fiocco di neve è un'estensione dello schema a stella in cui le tabelle delle dimensioni sono normalizzate. Questo modello permette di risparmiare spazio di memorizzazione in quanto le tabelle delle dimensioni sono più piccole, inoltre permette di ridurre la ridondanza dei dati. Tuttavia le interrogazioni possono essere più complesse e le prestazioni possono essere peggiori rispetto allo schema a stella.
            \begin{figure}[H]
                \centering
                \includegraphics[width=0.5\textwidth]{08/fioccoDiNeve.png}
                \caption{Schema a fiocco di neve}
            \end{figure}
        \subsubsection{Costellazione di fatti}
            Lo schema a costellazione di fatti è un'estensione dello schema a stella in cui sono presenti più fatti collegati tra loro. Questo modello permette di analizzare più fatti contemporaneamente e di effettuare analisi più complesse. Tuttavia le interrogazioni possono essere più complesse e le prestazioni possono essere peggiori rispetto allo schema a stella.
            \begin{figure}[H]
                \centering
                \includegraphics[width=0.5\textwidth]{08/costellazione.png}
                \caption{Schema a costellazione di fatti}
            \end{figure}
\section{Ciclo di vita di \texttt{DWH} e Popolazione del \texttt{DWH}}
    Andiamo ora a vedere come viene costruito un \texttt{DWH} e come vengono popolati i vari livelli.
    \subsection{Ciclo di vita di un \texttt{DWH}}
        La costruzione di una \texttt{DWH} è un processo iterativo che prevede diverse fasi:
        \begin{enumerate}
            \item Costruzione del primo ipercubo multidimensionale sul fatto più importante.
            \item Integrazione progressiva degli altri fatti
            \item Rilascio di \textit{data mart} per i vari settori aziendali
        \end{enumerate}
        Il vantaggio principale di questo approccio è che i primi risultati sono visibili in tempi brevi, gli investimenti sulla \texttt{DWH} sono diluiti nel tempo ed si può tarare il modello sulla base delle esigenze degli utenti.
    \subsection{Popolazione della \texttt{DWH}}
        La popolazione della \texttt{DWH} prevede diverse fasi:
        \begin{enumerate}
            \item Estrazione dei dati dalle sorgenti - i dati vengono estratti dalle sorgenti e trasferiti nella \texttt{DWH}
            \item Integrazione e trasformazione dei dati - associazione dei dati estratti con i dati già presenti nella \texttt{DWH}
            \item Pulizia dei dati - aumento della qualità
            \item Caricamento dei dati nella \texttt{DWH} - i dati vengono caricati nella \texttt{DWH}
        \end{enumerate}
        % TODO: Ci sarebbe da espandere un po' di più questa sezione ma non ho trovato tempo per farlo prima dell'esame ;)
\section{L'analisi \texttt{OLAP} e principali operatori \texttt{OLAP}}
    L'analisi \texttt{OLAP} è un'analisi iterativa che si basa sul paradigma dell'``esplorazione guidata delle ipotesi''. In sostanza una analisi \texttt{OLAP} prevede che in una stessa sessione di analisi ciascun passo è conseguenza dei risultati ottenuti al passo precedente, inoltre tutte le interrogazioni operano per differenza rispetto alla sessione precedente, ovvero il risultato di una interrogazione è la differenza tra il risultato dell'interrogazione corrente e il risultato dell'interrogazione precedente. I risultati finali sono presentati sotto forma di tabelle, grafici o mappe.
    \paragraph{Operatori \texttt{OLAP}}
    Gli operatori \texttt{OLAP} sono operatori che permettono di effettuare analisi sui dati. Gli operatori principali sono:
    \begin{description}
        \item[\textit{Drill Down}] Consente di passare da un livello di aggregazione ad un livello di dettaglio inferiore.
        \item[\textit{Roll Up}] Consente di passare da un livello di dettaglio ad un livello di aggregazione superiore.
        \item[\textit{Slice}] Consente di selezionare un sottoinsieme di dati su una dimensione.
        \item[\textit{Dice}] Consente di selezionare un sottoinsieme di dati su più dimensioni.
        \item[\textit{Pivot}] Consente di scambiare le righe con le colonne di una tabella oppure ruotare una rappresentazione grafica su un asse.
    \end{description}
    \subsubsection{\textit{Drill Down}}
        L'operatore \textit{Drill Down} consente di passare da un livello di aggregazione ad un livello di dettaglio inferiore. Questo operatore permette di visualizzare i dati in modo più dettagliato. Si può aggiungere una dimensione oppure si scende lungo una gerarchia. Graficamente andiamo a ``separare'' i dati in più sottoinsiemi, esempio possiamo passare da una tabella che mostra per la dimensione zona Nord Centro Sud il totale delle vendite per il 2020 ad una tabella che mostra per la dimensione delle regioni il totale delle vendite per il 2020. Siamo andati a ``separare'' i dati per regione e non più per zona.
    \subsubsection{\textit{Roll Up}}
        L'operatore \textit{Roll Up} è l'inverso dell'operatore \textit{Drill Down}, consente di passare da un livello di dettaglio ad un livello di aggregazione superiore. Questo operatore permette di visualizzare i dati in modo più aggregato. Si può rimuovere una dimensione oppure si sale lungo una gerarchia. Graficamente andiamo a ``unire'' i dati in un unico insieme, esempio possiamo passare da una tabella che mostra per la dimensione delle regioni il totale delle vendite per il 2020 ad una tabella che mostra per la dimensione zona Nord Centro Sud il totale delle vendite per il 2020. Siamo andati a ``unire'' i dati per zona e non più per regione.
    \subsubsection{\textit{Slice}}
        L'operatore \textit{Slice} consiste nel selezionare un sottoinsieme di dati su una dimensione. Questo significa che una volta determinato il valore che vogliamo selezionare, andiamo a ``tagliare'' i dati in modo che vengano visualizzati solo i dati che rispettano il valore selezionato. Tutte le altre dimensioni rimangono invariate.
    \subsubsection{\textit{Dice}}
        L'operatore \textit{Dice} è di base una combinazione di più operatori \textit{Slice}. Consente di selezionare un sottoinsieme di dati su più dimensioni. Questo significa che una volta determinati i valori che vogliamo selezionare, andiamo a ``tagliare'' i dati in un cubo in modo che vengano visualizzati solo i dati che per ogni dimensione rispettano il valore selezionato. Tutte le altre dimensioni rimangono invariate.
    \subsubsection{\textit{Pivot}}
        L'operatore \textit{Pivot} inverte la relazione tra le dimensioni andando a ruotare il cubo dell'analisi su un asse. Questo operatore consente di visualizzare i dati in modo differente, ad esempio possiamo passare da una tabella che mostra per la dimensione zona Nord Centro Sud il totale delle vendite per il 2020 ad una tabella che mostra per la dimensione delle regioni il totale delle vendite per il 2020. Siamo andati a ``ruotare'' i dati su un asse.
    
    \chapter{Introduzione al \textit{Data Mining}}
\thispagestyle{stdPage}

{\footnotesize \textbf{Nota dell'autore}: Questo capitolo è stato tagliato in quanto durante il corso dell'anno accademico 2024/2025 il prof. Bouquet per mancanza di tempo non ha potuto trattare interamente l'argomento, si riporta dunque solo la parte che è stata trattata.}

\par

In questo capitolo verranno introdotti i concetti fondamentali del Data Mining, ovvero l'insieme di tecniche e metodologie che permettono di estrarre informazioni utili da grandi quantità di dati in modo automatico. 

\section{Limiti analisi \texttt{OLAP}}
    I sistemi di \textit{data warehouse} con analisi \texttt{OLAP} permettono di analizzare i dati in modo interattivo, tuttavia presentano queste analisi sono basate su supposizioni che l'utente fà sui dati, e non permettono di trovare informazioni che l'umano non è in grado di trovare. Inoltre, l'analisi di grandi quantità di dati può essere molto dispendiosa in termini di tempo e risorse. Questo accade perché i sistemi \texttt{OLAP} per la loro natura sono progettati per essera a supporto delle decisioni umane e non si basa sui dati oggettivi per trovare correlazioni e \textit{pattern}.
    \paragraph{Il \textit{Data mining}}
        Il \textit{Data Mining} è stato introdotto per rispondere alle problematiche dei sistemi \texttt{OLAP}, permettendo di trovare informazioni riguardanti correlazioni nascoste e supporta modelli descrittivi e predittivi. \newline
        Il \textit{data mining} permette dopo aver pulito, integrato, selezionato e trasformato i dati viene applicato un algoritmo di \textit{data mining} che permette di trovare \textit{pattern} e relazioni nascoste nei dati. A questo punto i risultati vengono valutati ed presentati a chi deve prendere decisioni.
        Possiamo notare come i primi due passaggi combacino con quelli del popolamento del \textit{data warehouse}, infatti il \textit{data mining} può essere visto come un ampliamento del \textit{data warehouse} ed in alcuni casi un suo completamento.

    \paragraph{Da \texttt{OLAP} a \texttt{OLAM}} 
        Partendo da un \textit{data warehouse} possiamo estrarre i dati da sottoporre a \textit{data mining} anche se il processo di \textit{data mining} non deve essere completamente automatico in quanto potremmo incorrere in \textit{pattern} non significativi. Lavorando con uno strumento iterativo possiamo ottenere risultati migliori.
\section{Architettura e tipi di analisi con \textit{data mining}}
    \subsubsection{Architettura}
        L'architettura di un sistema di \textit{data mining} è composta da:
        \begin{description}
            \item[\textit{data warehouse} - sorgente dati] 
                Il \textit{data warehouse} è la sorgente dei dati da analizzare, i dati vengono estratti e trasformati in modo da essere pronti per l'analisi.
            \item[\textit{Knowledge Base} - base di conoscenza] 
                La base di conoscenza contiene i modelli e le regole che vengono utilizzate sia per l'analisi dei dati che per la valutazione dei risultati.
            \item[\textit{Data Mining Engine} - motore di \textit{data mining}] 
                Il motore di \textit{data mining} è il cuore del sistema, contiene gli algoritmi che permettono di trovare i \textit{pattern} nei dati.
            \item[\textit{pattern evaluation} - valutazione delle condizioni] 
                Questo modulo interagisce coi moduli di \textit{mining} per focalizzare la ricerca sui \textit{pattern} più interessanti.
            \item[\textit{User Interface} - sistema di presentazione] 
                Questo modulo permette all'utente di interagire con il sistema, visualizzando i risultati e permettendo di modificare i parametri di ricerca.
        \end{description}
    \subsubsection{Tipi di analisi}
        Le attività che possono essere eseguite sono molteplici, troviamo due macro-categorie:
        \begin{description}
            \item[\textit{Mining} descrittivo] estrae informazioni che descrivono le proprietà dei dati.
            \item[\textit{Mining} predittivo] determina regole che permettono di fare previsioni sui dati. 
        \end{description}
        Oltre a questi due tipi di analisi possiamo avere anche analisi diagnostiche ovvero che permettono di capire le cause di un determinato fenomeno, ma anche prescrittiva che permette di suggerire azioni da intraprendere per ottenere un determinato risultato.
\end{document}