\chapter{Cinematica}
Nel seguente capitolo andremo ad analizzare la cinematica, ovvero la branca della fisica che si occupa di descrivere il moto di un punto nello spazio. Per fare ciò andremo ad analizzare le grandezze fisiche che descrivono il moto di un punto nello spazio e come queste siano collegate tra loro.

\section{Nozioni Preliminari}

    \paragraph{Sistema di riferimento} Un sistema di riferimento è un insieme di regole che permettono di determinare la posizione di un punto nello spazio. Un sistema di riferimento è composto da un'origine, da un insieme di assi e da un'unità di misura. Definiamo un sistema di riferimento in quattro assi: $x$, $y$, $z$ e $t$ dove $t$ rappresenta il tempo.
    \begin{definition}[Spazio-Tempo Euclideo]
        Lo spazio-tempo euclideo ($S$) è un sistema di riferimento in quattro assi, $x$, $y$, $z$ e $t$, dove $t$ rappresenta il tempo. Lo spazio-tempo euclideo è definito come:
        $$
            S(O_z, x, y, z; O_t, t)
        $$
        dove $O_z$ è l'origine degli assi spaziali e $O_t$ è l'origine dell'asse temporale.
    \end{definition}

\section{Moti in una dimensione e grafico orario}
    Per descrivere i moti in una dimensione possiamo utilizzare un grafico non affine, quindi non lineare, che rappresenta la posizione di un punto in funzione del tempo. Questo grafico è detto \textit{grafico orario}.
    \begin{definition}[Grafico Orario]
        Il grafico orario è un grafico cartesiano che esprime la posizione di un punto che si muove in una dimensione in funzione del tempo.
    \end{definition}
    \begin{tikzpicture}
        \begin{axis}
            [axis lines = left, xlabel = $t$, ylabel = $x$, xmin = 0, xmax = 11, ymin = 0, ymax = 11, xtick = {0, 2, 10}, ytick = {0, 2, 10}, xticklabels = {$0$, $t_i$, $t_f$}, yticklabels = {$0$, $x_i\ P_i$, $x_f\ P_f$}, legend pos = north west]
            \addplot[mark = *, color = red] coordinates {(2, 2)} node[above] {$E_i$};
            \addplot[mark = *, color = blue] coordinates {(10, 10)} node[above] {$E_f$};
        \end{axis}
    \end{tikzpicture}
    Vediamo come al momento $t_i$ il punto sia in posizione $P_i$ e di verifica come l'evento $E_i$ sia in posizione $x_i$. Al momento $t_f$ il punto è in posizione $P_f$ e l'evento $E_f$ è in posizione $x_f$. Ora possiamo definite lo spostamento:
    \begin{definition}[Spostamento]
        Lo spostamento è la variazione di posizione di un punto in un intervallo di tempo. Lo spostamento è definito come:
        $$
            \begin{aligned}
                S_{i \to f} &= x_f - x_i\\
                \Delta x_{i \to f} &= x_f - x_i
            \end{aligned}
        $$
    \end{definition}
    Da notare come lo spostamento non descrive ne' la traiettoria ne' la distanza percorsa dal punto ma solo la variazione di posizione, infatti il punto potrebbe aver compiuto un percorso "non diretto". Inoltre nello spostamento ha un verso definito e come conseguenza scrivere $S_{i \to f} \neq S_{f \to i}$.
    \begin{definition}[Distanza Percorsa]
        La distanza percorsa è la lunghezza della traiettoria percorsa da un punto in un intervallo di tempo. La distanza percorsa è definita come:
        $$
            d(P_i, P_f) = \left| x_f - x_i \right|
        $$
    \end{definition}
    Notiamo come la distanza percorsa sia sempre positiva in quanto è la lunghezza della traiettoria percorsa dal punto. Inoltre la distanza percorsa non ha un verso definito, infatti $d(P_i, P_f) = d(P_f, P_i)$.\newline
    Ora per descrivere il moto di un punto possiamo definire la velocità media:
    \begin{definition}[Velocità media]
        La velocità media è la variazione di posizione di un punto in funzione del tempo. La velocità media è definita come:
        $$
            \begin{aligned}
                v_m &= \frac{\Delta x}{\Delta t} = \frac{\Delta x_{i \to f}}{\Delta t_{i \to f}}\\
            \end{aligned}
        $$
    \end{definition}
    Da notare come la velocità media non tiene conto del moto del punto in un intervallo di tempo, ma solo della variazione di posizione. Inoltre la velocità media ha un verso definito in quanto trattiamo lo spostamento (il quale ha un verso definito).\newline
    Per descrivere il moto di un punto in un instante $t$ di tempo possiamo definire la velocità istantanea:
    \begin{definition}[Velocità istantanea]
        La velocità istantanea è la variazione di posizione di un punto in un istante di tempo. La velocità istantanea è definita come:
        $$
            v_i(t)=\lim\limits_{\Delta t \to 0} \frac{\Delta x}{\Delta t} = \frac{dx}{dt}
        $$
    \end{definition}
    Dal punto di vista matematico la velocità istantanea è la derivata della posizione rispetto al tempo. Dunque i punti dove si passa da un movimento ``in avanti'' ad un movimento ``all'indietro'' sono i punti in cui la velocità istantanea è nulla ovvero i punti di massimo e minimo della funzione posizione, inoltre il punto in cui la velocità istantanea è nulla è detto punto di inversione. Inoltre la velocità istantanea è una funzione continua in quanto la derivata di una funzione continua è anch'essa continua.\newline
    È vero che in un determinato periodo di tempo io possa aumentare o diminuire la velocità, per questo motivo definiamo la funzione di accelerazione:
    \begin{definition}[Accelerazione]
        L'accelerazione è la variazione di velocità di un punto in funzione del tempo. L'accelerazione è definita come:
        $$
            \begin{aligned}
                a &= \frac{\Delta v}{\Delta t} = \frac{\Delta v_{i \to f}}{\Delta t_{i \to f}}\\
            \end{aligned}
        $$
    \end{definition}
    Da notare come l'accelerazione non tiene conto del moto del punto in un intervallo di tempo, ma solo della variazione di velocità. Inoltre l'accelerazione ha un verso definito in quanto trattiamo la variazione di velocità (la quale ha un verso definito).
    \paragraph{Relazione tra posizione, velocità e accelerazione}
        Come già detto la velocità è la derivata della posizione rispetto al tempo e l'accelerazione è la derivata della velocità rispetto al tempo, è vero inoltre che la posizione è l'integrale della velocità rispetto al tempo e questa è l'integrale dell'accelerazione rispetto al tempo. Dunque possiamo scrivere:
        \begin{align}
            x(t) &\text{ posizione} \\
            v(t) = \frac{dx}{dt} &\text{ velocità} \\
            a(t) = \frac{dv}{dt} = \frac{d^2x}{dt^2} &\text{ accelerazione}
        \end{align} 
        Al contrario possiamo scrivere:
        \begin{align}
            v(t) &= v_0 + \int_{t_0}^{t} a(T) dT\\
            x(t) &= x_0 + \int_{t_0}^{t} v(T) dT = x_0 + \int_{t_0}^{t} dT \left[v_0+\int_{t_0}^{T} a(\tau) d\tau\right]
        \end{align}
        Ora le dimensioni fisiche (e non le unità di misura) di queste grandezze sono:
        $$
            \begin{aligned}
                \left[x\right] &= \left[L\right]\\
                [v] &= \left[\frac{L}{T}\right]\\
                [a] &= \left[\frac{L}{T^2}\right]
            \end{aligned}
        $$
        ed le rispettive unità di misura sono:
        $$
            \begin{aligned}
                \left[x\right] &= \left[m\right]\\
                [v] &= \left[\frac{m}{s}\right]\\
                [a] &= \left[\frac{m}{s^2}\right]
            \end{aligned}
        $$
    \subsection{Esercizio sulle grandezze fisiche}
        Andiamo ora ad analizzare uno strumento importante per la risoluzione di esercizi fisici, ovvero l'analisi dimensionale. L'analisi dimensionale è uno strumento che ci permette di capire se un'equazione è corretta o meno e ci può suggerire come risolvere un problema. Vediamo un esempio:
        \begin{problem}
            Sia un punto che si muove lungo un asse $x$ in funzione del tempo $t$, questo al momento $t_0=0$ si trova al punto $x(t_0)=x(0)=x_0=0$ e la sua velocità in questo punto è $v(t_0)=v(0)=v_0>0$. La sua accelerazione è descritta dalla funzione $a(x)=-Ax-B$ con $A,B>0$.\newline
            Determinare il momento in cui il punto si ferma($x_{\text{stop}}$)
        \end{problem}
        Analizziamo l'equazione dell'accelerazione:
        $$
            a(x) = -Ax-B
        $$
        notiamo come questa abbia come dimensioni fisiche:
        $$
            \begin{aligned}
                a(x) &= -Ax&-B\\
                \left[\frac{L}{T^2}\right] &= \left[?\right]\left[L\right]&-\left[?\right]
            \end{aligned}
        $$
        da queste possiamo dedurre che $A$ ha dimensioni fisiche $\left[\frac{1}{T^2}\right]$ e $B$ ha dimensioni fisiche $\left[\frac{L}{T^2}\right]$. Ora l'equazione dello spazio in funzione del tempo è: $$
            \frac{d^2x}{dt^2}=-Ax-B
        $$
        dunque dobbiamo trovare una funzione $x(t)$ tale che soddisfi questa equazione differenziale, in quanto dobbiamo mantenere mantenere una funzione sullo spazio ed ottenere il parametro $A$ all'esterno ipotizziamo che la funzione sia:
        $$  
            \begin{aligned}
                x(t)=&X_0\sin(\sqrt{A}t + \varphi) = \\
                =& \mathcal{A}\sin(\sqrt{A}t + \varphi)
            \end{aligned}
        $$
        Prima di calcolare le derivate di questa funzione verifichiamo che effettivamente questa funzione soddisfi la dimensionalità:
        $$
            \begin{aligned}
                \left[x\right] &= \left[L\right]\\
                \left[\mathcal{A}\right] &= \left[L\right]\\
                \left[\sqrt{A}\right] &= \left[\frac{1}{T}\right]\\
                \left[t\right] &= \left[T\right]\\
                \left[\varphi\right] &= \left[1\right]\\
                \left[\sqrt{A}t + \varphi\right] &= \left[1\right]\checkmark\\
                \left[\mathcal{A}\right]\left[\sin(\left[1\right])\right]&=\left[L\right]\checkmark
            \end{aligned}
        $$
        ora verifichiamo che questa funzione soddisfi l'equazione differenziale, calcoliamo la derivata prima e la derivata seconda:
        $$
            \begin{aligned}
                \frac{dx}{dt} &= \mathcal{A}\sqrt{A}\cos(\sqrt{A}t + \varphi)\\
                \frac{d^2x}{dt^2} &= -A\underbrace{\mathcal{A}\sin(\sqrt{A}t + \varphi)}_{x(t)}
            \end{aligned}
        $$
        verifichiamo come anche queste derivate soddisfino la dimensionalità:
        $$
            \begin{aligned}
                \left[\frac{dx}{dt}\right] &\stackrel?= \left[\frac{L}{T}\right] \\
                \left[\frac{L}{T}\right] &= \left[\mathcal{A}\right]\left[\sqrt{A}\right]\left[\cos(\left[1\right])\right]\\
                \left[\frac{L}{T}\right] &= \left[L\right]\left[\frac{1}{T}\right]\left[1\right]\checkmark\\
                \\ 
                \left[\frac{d^2x}{dt^2}\right] &\stackrel?= \left[\frac{L}{T^2}\right] \\
                \left[\frac{L}{T^2}\right] &= \left[A\right]\left[\mathcal{A}\right]\left[\sin(\left[1\right])\right]\\
                \left[\frac{L}{T^2}\right] &= \left[\frac{1}{T^2}\right]\left[L\right]\left[1\right]\checkmark
            \end{aligned}
        $$
        Dunque queste derivate soddisfano la dimensionalità, ma manca ancora il parametro $-B$ nell'equazione dell'accelerazione, dunque dobbiamo modificare la funzione $x(t)$ in modo che soddisfi anche questa condizione, partiamo dal fatto nella prima funzione possiamo aggiungere/sottrarre solo una quantità di dimensionalità $\left[L\right]$ e notiamo come in quanto $\left[B\right]=\left[\frac{L}{T^2}\right]$ ed $\left[A\right]=\left[\frac{1}{T^2}\right]$ allora: 
        $$
            \left[\frac{B}{A}\right] = \left[\frac{\frac{L}{\cancel{T^2}}}{\frac{1}{\cancel{T^2}}}\right] = \left[L\right]
        $$
        Dunque $\frac{B}{A}$ può essere aggiunto alla funzione $x(t)$ in quanto ha le stesse dimensioni fisiche di $x(t)$, dunque la funzione $x(t)$ diventa:
        $$
            x(t) = \mathcal{A}\sin(\sqrt{A}t + \varphi) - \frac{B}{A}
        $$
        questo non comporta alcun cambiamento alle derivate in quanto queste sono in funzione di $t$ ed $A$ e $B$ sono delle costanti. Possiamo però notare come la derivata seconda di questa può essere riscritta come:
        $$
            \begin{aligned}
                \frac{d^2x}{dt^2} =& -A\left(\mathcal{A}\sin(\sqrt{A}t + \varphi)\right)\\
                =&-A\left(x(t)+\frac{B}{A}\right)\\
                =&-Ax(t)-B
            \end{aligned}
        $$
        Abbiamo quindi trovato la funzione $x(t)$ che soddisfa l'equazione differenziale, verifichiamo che nel punto $x_0$ questa soddisfi le condizioni iniziali per la posizione e la velocità:
        \begin{align}
            x(0) &= \mathcal{A}\sin(\varphi) - \frac{B}{A} = 0 \label{eq:P1pos}\\
            \frac{dx}{dt}(0) = v(0) &= \mathcal{A}\sqrt{A}\cos(\varphi) > 0 \label{eq:P1vel}
        \end{align}
        Per $\mathcal{A}>0$ allora per \ref{eq:P1pos} $\sin(\varphi) = \frac{B}{A\mathcal{A}}$ e per \ref{eq:P1vel} $\cos(\varphi) > 0$ queste condizioni in aggiunta a quelle di $ A,B>0 $ ci permettono di dire che $$
            \begin{cases}
                \mathcal{A}>0\\
                0<\varphi<\frac{\pi}{2}
            \end{cases}
        $$
        ora dobbiamo trovare il momento in cui il punto si ferma ($x_{\text{stop}}$), ovvero il momento in cui la velocità è nulla ($v(t_{\text{stop}})=v_{\text{stop}}=0$). 
        $$
            \begin{aligned}
                v(t_{\text{stop}}) =&\sqrt{A}\mathcal{A}\cos(\sqrt{A}t_{\text{stop}} + \varphi) = 0\\
                \sqrt{A}\mathcal{A}\neq 0\Rightarrow & \cos(\sqrt{A}t_{\text{stop}} + \varphi) = 0\\
                \sqrt{A}t_{\text{stop}} + \varphi &= \frac{\pi}{2}\\
                t_{\text{stop}} &= \left(\frac{\pi}{2}-\varphi\right)\frac{1}{\sqrt{A}}\\
            \end{aligned}
        $$
        ed al momento $t_{\text{stop}}$ la posizione del punto è:
        $$
            \begin{aligned}
                x_{\text{stop}} &= \mathcal{A}\sin\left[\cancel{\sqrt{A}}\frac{1}{\cancel{\sqrt{A}}}\left(\frac{pi}2-\cancel\varphi\right) + \cancel\varphi\right] - \frac{B}{A} \\
                &= \mathcal{A}\sin\left(\frac{\pi}{2}\right) - \frac{B}{A}\\
                &= \mathcal{A} - \frac{B}{A}
            \end{aligned}
        $$
        {\footnotesize Vedi \ref{lez:26-02-2025}}
    \subsection{Problema de ``Il lancio del sasso'' (M.R.U.A e M.R.U)}
        Questo classico problema viene usato per definire il Moto Rettilineo Uniformemente Accelerato ed il Moto Rettilineo Uniforme
        \begin{problem}
            Una coppia di amici vuole misurare l'altezza di un precipizio. Decidono di farlo lanciando un sasso verso il basso e misurando il tempo che impiega a raggiungere il fondo. Sappiamo che il tempo tra il rilascio del sasso grave\footnote{Soggetto alla gravità} ed il rumore dell'impatto è di $t_a$ secondi.\newline
            Calcolare l'altezza del precipizio.
        \end{problem}
        Avendo definito il sistema di riferimento con origine la cima del precipizio (dove sono gli amici) ed un verso ``puntante'' il fondo.\\
        Il problema può essere diviso in due parti:
        \begin{enumerate}
            \item[I] Il moto del masso grave
            \item[II] Il moto del suono
        \end{enumerate}
        Mentre la prima parte è descritta da una accelerazione costante $a(t)=g$, una velocità iniziale nulla $v(0)=0$, una posizione iniziale $x(0)=0$ e una posizione finale $x(t_f)=x_f$. La seconda parte è descritta da una velocità costante $v(t)=v_s$ e una posizione iniziale $x(t)=x_f$ e una posizione finale $x(t)=0$. Allora possiamo scrivere la posizione del masso grave come:
        $$
            \begin{aligned}
                a&=g\\
                v(t)&=v_0+\int_0^t a(T)dT \\
                &= v_0 + gt\\
                z(t)&=z_0+\int_0^t v(T)dT\\ 
                &= z_0 + \int_0^t(v_0 + gT )dT \\ 
                &= \underbrace{z_0}_0 + \underbrace{v_0}_0t + \frac{1}{2}gt^2\\
                &= \frac{1}{2}gt^2
            \end{aligned}
        $$
        Da notare come tutto ciò può essere scritto solo se $t<t_f$ in quanto il masso non può andare oltre il fondo del precipizio.\\
        Assumendo che il sasso venga lasciato perpendicolarmente al suolo allora possiamo scrivere la posizione del suono come:
        $$
            \begin{aligned}
                v(t)&=v_s\\
                z(t)|_{t>t_f}&=z_f + \int_{t_f}^t v_s dT\\
                &= z_f + v_s(t-t_f)
            \end{aligned}
        $$
        A questo punto definendo come $t_a$ il tempo tra il rilascio del sasso e l'impatto (amico), $t_f$ come il tempo del fondo del precipizio e $t_0$ come il tempo di rilascio del sasso, e dunque definito che $\Delta t_a = \Delta t_{mg} + \Delta t_{ms} $ possiamo scrivere:
        $$
            \begin{cases}
                Z_f = \frac{1}{2}gt_f^2\\
                \frac{1}{2}gt_f^2 - v_s(t_s-t_f) = 0\\
            \end{cases} = 
            \begin{cases}
                /\\
                t_f^2+2\frac{v_s}{g}t_f - 2\frac{Z_f}{g} = 0
            \end{cases}
        $$
        Questa equazione di secondo grado ha come soluzione:
        $$
            t_f = -\frac{v_s}{g} \pm \sqrt{\left(\frac{v_s}{g}\right)^2 + 2\frac{Z_f}{g}}
        $$
        la cui unica soluzione valida è:
        $$
            t_f = -\frac{v_s}{g} + \sqrt{\left(\frac{v_s}{g}\right)^2 + 2\frac{Z_f}{g}}
        $$
        Questo è il tempo che impiega il sasso a raggiungere il fondo del precipizio, ora possiamo calcolare l'altezza del precipizio:
        $$
            \begin{aligned}
                Z_f &= \frac{1}{2}g\left(-\frac{v_s}{g} + \sqrt{\left(\frac{v_s}{g}\right)^2 + 2\frac{Z_f}{g}}\right)^2\\
                &= \frac{1}{2}g\left(\frac{v_s}{g} - \sqrt{\left(\frac{v_s}{g}\right)^2 + 2\frac{Z_f}{g}}\right)^2\\
                &= \frac{1}{2}g\left({\frac{v_s}{g}}^2 - 2\frac{v_s}{g}\sqrt{\left(\frac{v_s}{g}\right)^2 + 2\frac{Z_f}{g}} + \left(\left(\frac{v_s}{g}\right)^2 + 2\frac{Z_f}{g}\right)\right) \\
            \end{aligned}
        $$
    \subsection{Moto armonico}
        Il moto armonico è descritto dalla seguente funzione di posizione:
        $$
            x(t) = A\sin(\omega t + \varphi)
        $$
        dove:
        \begin{itemize}
            \item $A$ è l'ampiezza del moto
            \item $\omega$ è la pulsazione del moto
            \item $\varphi$ è la fase iniziale del moto
        \end{itemize}
        come conseguenza possiamo ricavare la posizione in funzione del tempo:
        \begin{table}[H]
            \centering
            \begin{tabular}{c|c}
                $L$ & $X$\\
                \hline
                $0$ & $A\sin(\varphi)$\\
                $-\frac{\varphi}{\omega}$ & $A\sin(\frac{\varphi}{\cancel{\omega}}\cancel{\omega}-\varphi) = 0$\\
                $-\frac{\varphi}{\omega}+\frac{\pi}{2\omega}$ & $A$\\
                $-\frac{\varphi}{\omega}+\frac{\pi}{\omega}$ & $0$\\
                $-\frac{\varphi}{\omega}+\frac{3\pi}{2\omega}$ & $-A$\\
            \end{tabular}
            \caption{Posizione in funzione del tempo del moto armonico}
        \end{table}
        come possiamo notare dall'equazione di posizione in funzione del tempo il moto armonico è periodico con periodo $T=\frac{2\pi}{\omega}$ e frequenza $f=\frac{1}{T}=\frac{\omega}{2\pi}$.\newline
        Al variare del parametro $\varphi$ (fase) il moto armonico può essere traslato lungo l'asse $t$, se invece è il parametro $A$ (ampiezza) a variare il moto armonico può essere ``allargato'' o ``restringo'' lungo l'asse $x$, ovvero i massimi saranno $A$ e i minimi $-A$. Infine se è il parametro $\omega$ (pulsazione) a variare il moto armonico può essere ``allungato'' o ``accorciato'' lungo l'asse $t$, ovvero la distanza (temporale) tra due massimi consecutivi sarà $T=\frac{2\pi}{\omega}$.
        \paragraph{Derivate}
            Come accennato la velocità è la derivata della posizione rispetto al tempo:
            $$
                v(t) = \frac{dx}{dt} = A\omega\cos(\omega t + \varphi)
            $$
            e l'accelerazione è la derivata della velocità rispetto al tempo:
            $$
                a(t) = \frac{dv}{dt} = \frac{d^2x}{dt^2} = -A\omega^2\sin(\omega t + \varphi)
            $$
        \paragraph{Analisi dimensionale} 
            Possiamo notare come la funzione di posizione del moto armonico soddisfi la dimensionalità:
            $$
                \begin{aligned}
                    \left[A\right]=\left[x(t)\right]=\left[L\right]\\
                    \left[\omega t\right]=\left[1\right]\Rightarrow \left[w\right] = \left[\frac{1}{T}\right]\\
                    \left[\omega\right] = \left[\frac{1 \text{ radiante}}{\text{sec}}\right]
                \end{aligned}
            $$
        \begin{problem}[1.26]
            In un moto armonico semplice, con pulsazione $\omega=1.55\ \frac{\texttt{rad}}{\texttt{s}}$ e ampiezza $A=7\ \texttt{cm}$, si osserva che al tempo $t=0$ il punto si trova in posizione $x(0)=2.72\ \texttt{cm}$.\newline
            Calcolare:\begin{enumerate}
                \item[a.] la fase iniziale $\varphi$ 
                \item[b.] il periodo d'oscillazione $T$
                \item[c.] la velocità iniziale $v(0)$
            \end{enumerate}
        \end{problem}
        Per prima calcoliamo la fase iniziale $\varphi$ andando a sfruttare la posizione iniziale ed la legge di posizione del moto armonico:
        $$
            \begin{aligned}
                x(0) &= A\sin(\varphi)\\
                \frac{x(0)}{A} &= \sin(\varphi)\\
                \arcsin\left(\frac{x(0)}{A}\right) &= \varphi\\
                \arcsin\left(\frac{2.72}{7}\right) &= \varphi\\
                \varphi &\approx 0.39
            \end{aligned}
        $$
        Ora possiamo calcolare il periodo d'oscillazione $T$ andando a sfruttare la definizione di periodo:
        $$
            \begin{aligned}
                T &= \frac{2\pi}{\omega}\\
                \frac{2\pi}{\omega} &= \frac{2\pi}{1.55}\\
                T &\approx 4.07\ \text{4 secondi e spicci \textasciitilde prof. Iuppa}
            \end{aligned}
        $$
        Infine avendo calcolato la fase iniziale $\varphi$ possiamo calcolare la velocità iniziale $v(0)$ andando ad utilizzare la derivata prima della posizione:
        $$
            \begin{aligned}
                v(0) &= \frac{dx}{dt}(0)\\
                &= A\omega\cos(\omega t + \varphi)\\
                &= 7\cdot1.55\cos(0.39)\\
                v(0) &\approx 10.5\ \text{cm/s}
            \end{aligned}
        $$
\section{Moti in due dimensioni}
    Andiamo ora ad analizzare i moti in due dimensioni, ovvero i moti di un punto nello spazio.
    \subsection{Vettori e definizioni}
        Per descrivere i moti in due dimensioni possiamo utilizzare i vettori, in particolare per descrivere come un punto si muove nello spazio in funzione del tempo usiamo il \textbf{vettore posizione}:
        $$
            \vec{r}(t) = \begin{bmatrix} x(t)\\ y(t) \end{bmatrix}
        $$
        \begin{definition}[Vettore Posizione]
            Il vettore posizione ($\vec{r}$) è un vettore che descrive la posizione di un punto nello spazio in funzione del tempo, questo è un vettore di due dimensioni.
        \end{definition}
        Definiamo ora il \textbf{modulo} di un vettore:
        \begin{definition}[Modulo di un vettore]
            Il modulo di un vettore ($\left|\vec{v}\right|$) è la lunghezza del vettore, ovvero la distanza tra l'origine e la coda del vettore. Questa è definita come:
            $$
                \left|\vec{v}\right|^2 = \sum_{i=1}^n v_i^2
            $$
            dove $v_i$ è la componente $i$ del vettore ed $n$ è la dimensione del vettore.
        \end{definition}
        Molto spesso sarà necessario avere un vettore di lunghezza unitaria, ovvero un vettore che ha modulo $1$, ma che mantenga la direzione ed il verso del vettore originale, questo vettore è detto \textbf{versore}:
        \begin{definition}[Versore]
            Il versore ($\hat{v}$) è un vettore di lunghezza unitaria che mantiene la direzione e il verso del vettore originale. Il versore è definito come:
            $$
                \hat{v} = \frac{\vec{v}}{\left|\vec{v}\right|}
            $$
        \end{definition}
        Ora possiamo definire le operazioni tra vettori:
        \begin{definition}[Somma di vettori]
            La somma di due vettori ($\vec{v}+\vec{u}$) è un vettore che ha come componenti la somma delle componenti dei due vettori. La somma di due vettori è definita come:
            $$
                \vec{v}+\vec{u} = \begin{bmatrix} v_x\\ v_y \end{bmatrix} + \begin{bmatrix} u_x\\ u_y \end{bmatrix} = \begin{bmatrix} v_x+u_x\\ v_y+u_y \end{bmatrix}
            $$
        \end{definition}
        Graficamente la somma di due vettori è la costruzione di un parallelogramma con i due vettori come lati, il vettore somma è la diagonale del parallelogramma. Questo lo possiamo fare grazie alla \textbf{regola de trasporto} ovvero la regola che ci permette di trasportare un vettore in un altro punto dello spazio mantenendo direzione e verso per poi sommarlo ad un altro vettore.
        \begin{definition}[Prodotto scalare]
            Il prodotto scalare tra un vettore e uno scalare ($\vec{v}\cdot\alpha$) è un vettore che ha come componenti il prodotto delle componenti del vettore per lo scalare. Il prodotto scalare è definito come:
            $$
                \vec{v}\cdot\alpha = \begin{bmatrix} v_x\\ v_y \end{bmatrix}\cdot\alpha = \begin{bmatrix} v_x\alpha\\ v_y\alpha \end{bmatrix}
            $$
            Questa operazione è usata per il calcolo del versore.
        \end{definition}
        \begin{definition}[Angolo tra due vettori]
            L'angolo tra due vettori ($\vec{v}$ e $\vec{u}$) è l'angolo tra i due vettori, ovvero l'angolo tra i due versori. L'angolo tra due vettori è definito come:
            $$
                \cos(\theta) = \frac{\vec{v}\cdot\vec{u}}{\left|\vec{v}\right|\left|\vec{u}\right|} \Rightarrow \theta = \arccos\left(\frac{\vec{v}\cdot\vec{u}}{\left|\vec{v}\right|\left|\vec{u}\right|}\right)
            $$
        \end{definition}
        In questa definizione usiamo il prodotto scalare tra due vettori, questo prodotto è definito come:
        \begin{definition}[Prodotto scalare]
            Il prodotto scalare tra due vettori ($\vec{v}\cdot\vec{u}$) è uno scalare che ha come valore la somma dei prodotti delle componenti dei due vettori. Il prodotto scalare è definito come:
            $$
                \vec{v}\cdot\vec{u} = \sum_{i=1}^n v_iu_i
            $$
            dove $v_i$ e $u_i$ sono le componenti $i$ dei due vettori ed $n$ è la dimensione dei vettori.
        \end{definition}
        Quando descriviamo il movimento del punto nello spazio possiamo descrivere il moto con il \textbf{raggio vettore}:
        \begin{definition}[Raggio vettore]
            Il raggio vettore ($\vec{r}$) è un vettore che descrive la posizione di un punto nello spazio in funzione del tempo. Il raggio vettore è definito come:
            $$
                \vec{v}(t) = \begin{bmatrix} x(t)\\ y(t) \end{bmatrix}=x\hat{v}_x+y\hat{v}_y
            $$
        \end{definition}
        Di conseguenza possiamo definire la velocità e l'accelerazione come:
        \begin{definition}[Velocità]
            La velocità ($\vec{v}$) è la derivata del raggio vettore rispetto al tempo. La velocità è definita come:
            $$
                \vec{v}(t) = \frac{d\vec{r}}{dt} = \begin{bmatrix} \frac{dx}{dt}\\ \frac{dy}{dt} \end{bmatrix} = \begin{bmatrix} v_x\\ v_y \end{bmatrix} = v_x\hat{v}_x+v_y\hat{v}_y
            $$
        \end{definition}
        \begin{definition}[Accelerazione]
            L'accelerazione ($\vec{a}$) è la derivata della velocità rispetto al tempo. L'accelerazione è definita come:
            $$
                \vec{a}(t) = \frac{d\vec{v}}{dt} = \frac{d^2\vec{r}}{dt^2} = \begin{bmatrix} \frac{dv_x}{dt}\\ \frac{dv_y}{dt} \end{bmatrix} = \begin{bmatrix} a_x\\ a_y \end{bmatrix} = a_x\hat{v}_x+a_y\hat{v}_y
            $$
        \end{definition}
        Anche per i vettori la posizione in funzione del tempo può essere scritta come l'integrale della velocità rispetto al tempo:
        $$
            \rho(t) = \vec{\rho}_0 + \int_{t_0}^{t} \vec{v}(T)dT
        $$
        e la velocità in funzione del tempo può essere scritta come l'integrale dell'accelerazione rispetto al tempo:
        $$
            \vec{v}(t) = \vec{v}_0 + \int_{t_0}^{t} \vec{a}(T)dT
        $$
        \paragraph{Coordinate Polari}
            Le coordinate polari sono un sistema di coordinate che permette di descrivere un punto nello spazio con due coordinate: il raggio ($\rho$) e l'angolo ($\theta$). Queste coordinate sono collegate alle coordinate cartesiane tramite le relazioni:
            $$
                \begin{aligned}
                    x &= \rho\cos(\theta)\\
                    y &= \rho\sin(\theta)
                \end{aligned}
            $$
            e le coordinate cartesiane sono collegate alle coordinate polari tramite le relazioni:
            $$
                \begin{aligned}
                    \rho &= \sqrt{x^2+y^2}\\
                    \theta &= \arctan\left(\frac{y}{x}\right)
                \end{aligned}
            $$
            Per passare dalle coordinate polari alle coordinate cartesiane usiamo le seguenti relazioni:
            $$
                \begin{aligned}
                    \vec{r}(x,y) & \rightarrow\left(\rho,\varphi)=(\sqrt{x^2+y^2},\arctan\left(\frac{y}{x}\right)\right)\\
                    \vec{r}(x,y)=(\rho\cos{\varphi},\rho\sin{\varphi}) & \leftarrow(\rho,\varphi)
                \end{aligned}
            $$
            Il vettore velocità può anche questo essere espresso in coordinate polari: $$
                \begin{aligned}
                    \frac{d}{dt}(\rho\cos(\varphi)) &= \left(\frac{d}{dt}\rho\right)\cos(\varphi) + \rho\left(\frac{d}{dt}\cos(\varphi)\right) \\
                    &= \rho\cos(\varphi)-\rho(\sin\varphi)\varphi\\
                    \frac{d}{dt}(\rho\sin(\varphi)) &= \left(\frac{d}{dt}\rho\right)\sin(\varphi) + \rho\left(\frac{d}{dt}\sin(\varphi)\right) \\
                    &= \rho\sin(\varphi)+\rho(\cos\varphi)\varphi
                \end{aligned}
                $$
            $$
                \vec{v}=\left(\rho\cos(\varphi)-\rho\varphi\sin(\varphi), \rho\sin(\varphi)+\rho\varphi\cos(\varphi)\right)
            $$
            A partire da questa possiamo calcolare l'accelerazione in coordinate polari:
            $$
                \begin{aligned}
                    \frac{d}{dt}(\rho\cos(\varphi)-\rho\varphi\sin(\varphi)) &= \left(\frac{d}{dt}\rho\right)\cos(\varphi)-\rho\varphi\sin(\varphi)+\rho\left(\frac{d}{dt}\cos(\varphi)\right)-\rho\left(\frac{d}{dt}\varphi\right)\sin(\varphi)-\rho\varphi\cos(\varphi)\\
                    &= \rho\cos(\varphi)-\rho\varphi\sin(\varphi)-\rho\varphi\sin(\varphi)-\rho\varphi\cos(\varphi)-\rho\varphi\cos(\varphi)\\
                    &= \rho\cos(\varphi)-2\rho\varphi\sin(\varphi)-2\rho\varphi\cos(\varphi)\\
                    \frac{d}{dt}(\rho\sin(\varphi)+\rho\varphi\cos(\varphi)) &= \left(\frac{d}{dt}\rho\right)\sin(\varphi)+\rho\varphi\cos(\varphi)+\rho\left(\frac{d}{dt}\sin(\varphi)\right)+\rho\left(\frac{d}{dt}\varphi\right)\cos(\varphi)-\rho\varphi\sin(\varphi)\\
                    &= \rho\sin(\varphi)+\rho\varphi\cos(\varphi)+\rho\varphi\cos(\varphi)-\rho\varphi\sin(\varphi)+\rho\varphi\cos(\varphi)\\
                    &= \rho\sin(\varphi)+2\rho\varphi\cos(\varphi)-\rho\varphi\sin(\varphi)
                \end{aligned}
            $$
            $$
                \vec{a}=\left(\rho\cos(\varphi)-2\rho\varphi\sin(\varphi)-2\rho\varphi\cos(\varphi), \rho\sin(\varphi)+2\rho\varphi\cos(\varphi)-\rho\varphi\sin(\varphi)\right)
            $$
        \subsubsection{Tangente alla traiettoria}
            La tangente alla traiettoria è un vettore che ha come direzione la direzione della velocità in un punto della traiettoria, la quale è \textbf{sempre} tangente alla traiettoria in quel punto. Questo vettore è definito come:
            $$
                \begin{aligned}
                    \frac{V_y}{V_x} &= \frac{\frac{dy}{\cancel{dt}}}{\frac{dx}{\cancel{dt}}} = y'(x)
                \end{aligned}
            $$
            La direzione viene espressa tramite il versore: $ \hat{v_t}=\vec{v}=\left|\vec{v}\right|\cdot\hat{v_t}$
    \subsection{Accelerazione nel piano e ascissa curvilinea}
        Come per il moto in una dimensione l'accelerazione è definita come la derivata della velocità rispetto al tempo, nel caso delle $n$ dimensioni si verifica la seguente relazione:
        $$
            \begin{aligned}
                \vec{a}=&\frac{d\vec{v}}{dt}=\frac{d}{dt}\left(\left|\vec{v}\right|\cdot\hat{v_t}\right)\\ 
                =&\frac{d\left|\vec{v}\right|}{dt}\cdot\hat{v_t}+\left|\vec{v}\right|\cdot\frac{d\hat{v_t}}{dt}
            \end{aligned}
        $$
        L'ultima parte di questa equazione, la quale corrisponde all'accelerazione, ci fà notare come nei moti su $n$ dimensioni l'accelerazione non sia solo puramente correlata alla variazione della velocità: $ \frac{d|\vec{v}|}{dt}\cdot\hat{v_t} $ ma anche alla variazione della direzione della velocità: $ \left|\vec{v}\right|\cdot\frac{d\hat{v_t}}{dt} $. Questa seconda parte dell'accelerazione è detta \textbf{accelerazione centripeta} che dipende dalla \textbf{velocità normale} e dalla \textbf{curvatura} della traiettoria.
        Definiamo quindi alcuni concetti che ci permettono di descrivere meglio il moto in due dimensioni:
        \begin{definition}[Velocità radiale]
            La velocità radiale ($\vec{v_r}$) è la componente della velocità che è parallela al raggio vettore. La velocità radiale è definita come:
            $$
                \vec{v_r} = \vec{v}\cdot\hat{r}
            $$
            {\footnotesize Vettore rosso}
        \end{definition}
        \begin{definition}[Versore radiale]
            Il versore radiale ($\hat{r}$) è il versore che ha come direzione il raggio vettore. Il versore radiale è definito come:
            $$
                \hat{r} = \frac{\vec{r}}{\left|\vec{r}\right|}
            $$
            {\footnotesize Vettore verde}
        \end{definition}
        \begin{definition}[Versore tangente alla traiettoria]
            Il versore tangente alla traiettoria ($\hat{v_t}$) è il versore che ha come direzione la tangente alla traiettoria. Il versore tangente alla traiettoria è definito come:
            $$
                \hat{v_t} = \frac{d\vec{r}}{dt} = \frac{d\vec{r}}{ds}\frac{ds}{dt} = \frac{d\vec{r}}{ds}\left|\vec{v}\right|
            $$
            Questo descrive la direzione della \textbf{velocità tangenziale} in un punto della traiettoria.
            {\footnotesize Vettore blu}
        \end{definition}
        \begin{definition}[Versore tangente al versore radiale]
            Il versore tangente al versore radiale ($\hat{v_n}$) è il versore che ha come direzione la tangente al versore radiale. Il versore tangente al versore radiale è definito come:
            $$
                \hat{v_n} = \frac{d\hat{r}}{dt}
            $$
            Questo descrive la direzione della \textbf{velocità normale} in un punto della traiettoria.
            {\footnotesize Vettore viola}
        \end{definition}
        \begin{definition}[Versore perpendicolare alla tangente della traiettoria]
            Il versore perpendicolare alla tangente della traiettoria ($\hat{v_b}$) è il versore che ha come direzione la perpendicolare alla tangente della traiettoria. Il versore perpendicolare alla tangente della traiettoria è definito come:
            $$
                \hat{v_b} = \hat{v_t}\times\hat{r}
            $$
            Questo descrive la direzione della \textbf{velocità perpendicolare} in un punto della traiettoria.
            {\footnotesize Vettore giallo}
        \end{definition}
        \begin{tikzpicture}
            % Assi 
            \draw[->] (0,0) -- (0,5) node[anchor=south]{$y$};
            \draw[->] (0,0) -- (5,0) node[anchor=west]{$x$};
            % Traiettoria
            \draw[red, thick] (0,0) to [out=30,in=180] (4,4) to [out=0,in=180] (6,2) to [out=0,in=180] (8,4);
            % Punti
            \draw[fill] (0,0) circle (0.05) node[anchor=north east]{$O$};
            \draw[fill] (2,2.6) circle (0.05) node[anchor=north]{$P$};
            % Vettori
            \draw[->, red] (0,0) -- (2,2.6) node[anchor=south]{$\vec{r}$};
            \draw[->, green] (0,0) -- (1,1.3) node[anchor=south]{$\hat{r}$};
            \draw[->, blue] (2,2.6) -- (3,4.25) node[anchor=south]{$\hat{v_t}$};
            \draw[->, purple] (2,2.6) -- (1.2,3.25) node[anchor=south]{$\hat{v_n}$};
            \draw[->, yellow] (2,2.6) -- (3,1.85) node[anchor=south]{$\hat{v_b}$};
        \end{tikzpicture}
        Dunque la \textbf{accelerazione tangenziale} in un punto è descritta dalla componente $ \frac{d|\vec{v}|}{dt}\cdot\hat{v_t} = \vec{a_t} $ mentre la \textbf{accelerazione normale} in un punto è descritta dalla componente $ \left|\vec{v}\right|\cdot\frac{d\hat{v_t}}{dt} = \vec{a_n} $.
        \subsubsection{Ascissa curvilinea}
            L'ascissa curvilinea ($s$) è la derivata dello spazio sul tempo, ovvero approssimando per tratti infinitesimi la traiettoria come una circonferenza allora:
            $$
                \frac{d\varphi}{dt}= \frac{d\varphi}{ds}\frac{ds}{dt} = \frac{1}{R}\left|\vec{v}\right|
            $$
            dove $R$ è il \textbf{raggio di curvatura} il quale è definito come il raggio della circonferenza che meglio approssima la traiettoria in un punto.\footnote{Viene usata la lettera maiuscola $R$ in quanto questo non può essere stabilito da noi ma dipende direttamente dalla traiettoria e dunque dal problema, si dice infatti che il centro della circonferenza è \underline{IMPOSTO} dalla traiettoria.}
            Dunque in un punto da un punto della traiettoria possiamo ricavare l'accelerazione normale ed la velocità tangenziale.\newline
            Analizziamo la situazione nella quale l'accelerazione tangenziale è nulla, ovvero il modulo della velocità è costante ma la direzione cambia costantemente, in questo caso $$
            \frac{dv}{dt} = 0 \Rightarrow \vec{a} = \frac{v^2}{R}\hat{V_n}
            $$
            Dunque $v$ è costante, $|\vec{a}| = \frac{v^2}{R}$ è costante e quindi $R$ è costante, ovvero la traiettoria è una circonferenza.
    \subsection{Moto Circolare Uniforme (M.C.U.)}
        Nel moto circolare uniforme la velocità è costante, la direzione cambia costantemente e l'accelerazione è costante e diretta verso il centro della circonferenza. In questo moto viene definito il concetto di \textbf{velocità angolare}:
        \begin{definition}[Velocità angolare]
            La velocità angolare ($\omega$) è la velocità con la quale un punto si muove lungo la circonferenza. La velocità angolare è definita come:
            $$
                \omega = \frac{d\varphi}{dt}
            $$
        \end{definition}
        ne consegue che $ds=Rd\varphi$ e dunque $\omega=\frac{v}{R}$.
        \begin{proof}
            Dimostriamo che $\omega=\frac{v}{R}$:
            $$
                \begin{aligned}
                    \omega &= \frac{d\varphi}{dt} = \frac{d\varphi}{ds}\frac{ds}{dt} = \frac{1}{R}v = \frac{v}{R} \Rightarrow v = R\omega
                \end{aligned}
            $$
        \end{proof}
        Il moto è descritto dalle seguenti equazioni che compongono il vettore posizione:
        $$
            \begin{cases}
                x=R\cos(\varphi)= R\cos(\omega t)\\
                y=R\sin(\varphi)= R\sin(\omega t)
            \end{cases}
        $$
        e dunque la velocità in quanto derivata della posizione rispetto al tempo è:
        $$
            \begin{cases}
                v_x = -R\sin(\omega t)\cdot \omega = -R\omega\sin(\omega t)\\
                v_y = R\cos(\omega t)\cdot \omega = R\omega\cos(\omega t)
            \end{cases}
        $$
        e l'accelerazione in quanto derivata della velocità rispetto al tempo è:
        $$
            \begin{cases}
                a_x = -R\omega\cos(\omega t)\cdot \omega = -R\omega^2\cos(\omega)\\
                a_y = -R\omega\sin(\omega t)\cdot \omega = -R\omega^2\sin(\omega t)
            \end{cases}
        $$
        Notiamo come le somme dei quadrati delle velocità $v_x^2+v_y^2$ siano uguali al prodotto dei quadrati del raggio e della velocità angolare $R^2\omega^2$.\newline
        \subsubsection{Moto circolare uniforme come somma di moti armonici}
            Il moto circolare uniforme può essere descritto come la somma di due moti armonici ortogonali della stessa ampiezza e della stessa frequenza, infatti se:
            $$
                \begin{cases}
                    \varphi = \omega t\\
                    \rho = R
                \end{cases} \Rightarrow \vec{r} = R\begin{bmatrix} \cos(\omega t)\\ \sin(\omega t) \end{bmatrix}
            $$\footnote{Dimostrazione omessa}
    \subsection{Moto Parabolico}
        Anche il moto parabolico può essere descritto come la somma di due moti, in questo caso però viene sommato il moto rettilineo uniforme parallelo all'asse delle $x$ con il moto uniformemente accelerato lungo l'asse delle $y$. Questo moto è descritto dalle seguenti equazioni:
        $$
            \vec{v}=\begin{cases}
                x=v_{0x}t\\
                y=v_{0y}t-\frac{1}{2}gt^2
            \end{cases}
        $$
        entrambe dipendono dalla velocità iniziale la quale può essere divisa nelle sue componenti come:
        $$
            \begin{cases}
                v_{0x} = |v_0|\cos(\alpha)\\
                v_{0y} = |v_0|\sin(\alpha)
            \end{cases}
        $$
        dove $\alpha$ è l'angolo che la velocità iniziale forma con l'asse delle $x$, ci redimo conto che ci basta conoscere la velocità iniziale e l'angolo per descrivere il moto parabolico.\footnote{Anche se nella realtà dovremmo considerare la resistenza dell'aria e la rotazione terrestre per la componente $v_{0x}$}.\newline
        Scrivendo le equazioni del moto parabolico con la $y$ in funzione del tempo possiamo ottenere la seguente equazione:
        $$
            \begin{aligned}
                y=&v_{oy}\frac{x}{v_{ox}}-\frac12g\left(\frac{x}{v_{ox}}\right)^2\\
                =&\underbrace{-\frac{g}{2v_{ox}^2}}_{a}x^2+\underbrace{\frac{v_{oy}}{v_{ox}}}_{b}x+\underbrace{0}_{c}
            \end{aligned}
        $$
        Questa è proprio l'equazione di una parabola, ora $a$ sarà sempre minore di $0$ a meno di un cambio del sistema di riferimento e dunque la parabola sarà sempre concava verso il basso [1].\newline
        Inoltre il $\lim_{ a\to \frac{\pi}2}y\rightarrow Indef$, ma il $\lim_{ g\to 0}y\rightarrow \text{M.R.U.}$[2], infine se la derivata prima della $y$ fosse $=0$ e dunque $2ax+b=0$ allora:
        $$
            \begin{aligned}
                x=&-\frac{b}{2a}\\
                =&-\frac{\frac{v_{oy}}{\cancel{v_{ox}}}}{\frac{g}{v_{ox}^{\cancel{2}}}}\\
                =&-\frac{v_{oy}v_{ox}}{g}\\
                =&\frac{v^2\sin(\alpha)\cos(\alpha)}{g}\\
                =&\frac{v^2}{2g}\sin(2\alpha)
            \end{aligned}
        $$
        Potrebbe essere utile la ``gittata'' [4] ovvero la distanza percorsa dal punto, questa la possiamo ricavare imponendo $y=0$: \dots
