\chapter{Seconda legge della termodinamica}
    La seconda legge della termodinamica è divisa in due enunciati: l'enunciato di Kelvin recita:
    \begin{quote}{Kelvin - Planck}
        Non non è possibile costruire una macchina il cui unico risultato sia il calore trasferito da una sorgete in lavoro 
    \end{quote}
    mentre il secondo principio di Claus recita:
    \begin{quote}{Claus}
        Non esiste alcuna trasformazione termodinamica il cui unico risultato sia il trasferimento di calore da una sorgente ad una più calda.
    \end{quote}
    In quanto principi non è possibile dimostrare la loro verità, ma è possibile dimostrare che questi sono equivalenti
    \begin{proof}
        \textbf{Da Claus a Kelvin - Planck:} Si prenda una macchina termica che violi il principio di Claus, ovvero che trasferisca calore da una sorgente a temperatura $T_1$ ad una sorgente a temperatura $T_2$ con $T_1 < T_2$, questa assorbe calore $\mathcal{Q}_1$ dalla superficie $T_1$ e cede calore $\mathcal{Q}_2$ alla superficie $T_2$. In generale è possibile costruire una macchina che assorba la stessa quantità di calore che la macchina che viola il principio di Claus, ($\cancel{C}$), cede alla sorgente a temperatura $T_2$, questa seconda macchina produce lavoro $W$ e cede calore $\mathcal{Q}_1^{DISI}$ alla sorgente a temperatura $T_1$ e assorbe calore $\left|\mathcal{Q}_2^{DISI}\right| = \left|\mathcal{Q}_2\right|$ dalla sorgente a temperatura $T_2$. Ora combinando le due macchine si ottiene che il calore scambiato dalla sorgente a temperatura $T_1$ è: $\mathcal{Q}_1 - \mathcal{Q}_1^{DISI}$, il calore scambiato dalla sorgente a temperatura $T_2$ è $\mathcal{Q}_2 - \mathcal{Q}_2^ = 0$ e il lavoro prodotto è $W$. Quindi abbiamo costruito una macchina che produce lavoro assorbendo calore ($\mathcal{Q}_1 - \mathcal{Q}_1^{DISI}$) dalla sorgente a temperatura $T_1$ e non cede calore alla sorgente a temperatura $T_2$. Questa macchina è in contraddizione con il principio di Kelvin - Planck, quindi il principio di Claus è equivalente al principio di Kelvin - Planck. (La quantità $\mathcal{Q}_1 - \mathcal{Q}_1^{DISI}$ è positiva in quanto visto che la macchina $DISI$ produce lavoro allora $-\mathcal{Q}_1^{DISI}+\mathcal{Q}_2-W=0$ e dato che $W>0$ allora $\mathcal{Q}_1^{DISI} < \mathcal{Q}_2$ e dato che nella prima macchina l'energia interna è data da $\mathcal{Q}_1 - \mathcal{Q}_2 = 0$ allora $\mathcal{Q}_1 = \mathcal{Q}_2$ e quindi $\mathcal{Q}_1 - \mathcal{Q}_1^{DISI} = \mathcal{Q}_2 - \mathcal{Q}_1^{DISI} > 0$).\newline
        \textbf{Da Kelvin - Planck a Claus:} Si prenda una macchina termica che violi il principio di Kelvin - Planck, ovvero che produca lavoro $W$ assorbendo calore $\mathcal{Q}_1$ dalla sorgente a temperatura $T_2$ e non ceda calore a nessuna sorgente a temperatura $T_2$. Consideriamo dunque una macchina che dallo stesso lavoro prodotto $W$ e tramite una sorgente a temperatura $T_1<T_2$ ceda una certa quantità di calore $-\left|\mathcal{Q}_2^{DISI}\right|$ alla superficie $T_2$ e assorba una certa quantità di calore $\mathcal{Q}_1^{DISI}$ dalla sorgente a temperatura $T_1$. Ora combinando le due macchine si ottiene che questa macchina non produce né assorbe lavoro, inoltre assorbe del calore $\left|\mathcal{Q}_1^{DISI}\right|$ dalla sorgente a temperatura $T_1$ e scambia del calore $\mathcal{Q}_2 - \left|\mathcal{Q}_2^{DISI}\right|$ alla sorgente a temperatura $T_2$. Questa ultima quantità scambiata è uguale in modulo a $\mathcal{Q}_1^{DISI}$, ciò in quanto $\mathcal{Q}_2 = W$ vito che la prima macchina produce solamente lavoro, inoltre vale che $\mathcal{Q}_1^{DISI}-\mathcal{Q}_2^{DISI} - (- W) = 0$ e dunque $W= -\mathcal{Q}_1^{DISI} + \mathcal{Q}_2^{DISI}$, sostituendo si ottiene che:
        $-\mathcal{Q}_1^{DISI} + \cancel{\mathcal{Q}_2^{DISI}} - \cancel{\mathcal{Q}_2^{DISI}}$ dunque la quantità di calore scambiata dalla sorgente a temperatura $T_1$ è $\mathcal{Q}_1^{DISI}$ e la quantità di calore scambiata dalla sorgente a temperatura $T_2$ è $-\mathcal{Q}_1^{DISI}$. Quindi abbiamo costruito una macchina che non produce/assorbe lavoro e che assorbe calore dalla sorgente a temperatura $T_1 < T_2$ e cede calore alla sorgente a temperatura $T_2$. Questa macchina è in contraddizione con il principio di Claus, quindi il principio di Kelvin - Planck è equivalente al principio di Claus.
    \end{proof}

\section{Teorema di Carnot}
    Il teorema di Carnot afferma che:
    \begin{theorem}[Teorema di Carnot]
        Presa una qualsiasi macchina termica $x$ che lavori tra due sorgenti a temperatura $T_1$ e $T_2$ con $T_1 > T_2$ questa avrà rendimento $\eta_x$ minore o uguale a quello di una macchina reversibile $r$ che lavora tra le stesse sorgenti. In particolare il rendimento sarà minore se la macchina $x$ non è reversibile e uguale se la macchina $x$ è reversibile.
        \begin{equation}
            \eta_{x,(T_1,T_2)} \leq \eta_{r,(T_1,T_2)}
        \end{equation}
    \end{theorem}
    In particolare sarà vero, per la definizione di rendimento che $1-\frac{\left|\mathcal{Q}_1\right|}{\mathcal{Q}_2} \leq 1 - \frac{T_1}{T_2}$, quindi che $\frac{\mathcal{Q}_1}{T_1} + \frac{\mathcal{Q}_2}{T_2} \leq 0$.
    \begin{proof}
        Assumiamo per assurdo che esista una macchina $x$ che lavora tra le sorgenti a temperatura $T_1$ e $T_2>T_1$ con rendimento $\eta_x$ maggiore di quello di una macchina reversibile $r$ che lavora tra le stesse sorgenti. Prendiamo dunque in considerazione questa macchina $r$ e la impostiamo in modo che il lavoro prodotto sia uguale a quello prodotto dalla macchina $x$, quindi $W_x = W_r$. Ora invertiamo la macchina $r$ in modo che questa funzioni da macchina frigorifera questa assorbirà calore dalla sorgente a temperatura $T_1$ e cederà calore alla sorgente a temperatura $T_2$ assorbendo lavoro $W$. Consideriamo la combinazione delle due macchine, la risultante non scambia lavoro in quanto $W_x = W_r$ per costruzione, inoltre scambia calore $\left|\mathcal{Q}_1^r\right|-\left|\mathcal{Q}_1^x\right|$ con la sorgente a temperatura $T_1$ e $\left|\mathcal{Q}_2^x\right|-\left|\mathcal{Q}_2^r\right|$ con la sorgente a temperatura $T_2$. Queste due quantità di calore si equivalgono per costruzione della macchina $r$ e quindi si ha che $\left|\mathcal{Q}_1^r\right|-\left|\mathcal{Q}_1^x\right| = \left|\mathcal{Q}_2^x\right|-\left|\mathcal{Q}_2^r\right|$. Chiamiamo la quantità $\left|\mathcal{Q}_1^r\right|-\left|\mathcal{Q}_1^x\right| = -\left|\mathcal{Q}_{abs}\right|$. L'ipotesi di assurdo implica che $\eta_x > \eta_r$ e quindi che $\frac{W}{\mathcal{Q}_2^x} > \frac{W}{\mathcal{Q}_2^r}$, quindi $\mathcal{Q}_2^x < \mathcal{Q}_2^r$, il che implica che $\mathcal{Q}_2^x-\mathcal{Q}_2^r < 0$. Quindi in quanto $\mathcal{Q}_2^x-\mathcal{Q}_2^r = \left|\mathcal{Q}_{abs}\right|$ si ha che $\left|\mathcal{Q}_{abs}\right| < 0$, il che significherebbe che in assenza di lavoro la superficie a temperatura $T_1$ cede calore alla sorgente a temperatura $T_2$, il che è in contraddizione con il principio di Claus. Quindi l'ipotesi di partenza è falsa e quindi il teorema di Carnot è vero per qualsiasi macchina termica $x$ che lavora tra due sorgenti a temperatura $T_1$ e $T_2$ con $T_1 > T_2$.\footnote{La dimostrazione per il caso nel quale la macchina $x$ è reversibile è immediata in quanto è possibile replicare la seguente dimostrazione invertendo la macchina $x$ in modo che questa funzioni da macchina frigorifera e infine mettendo a sistema i due risultati ottenuti.}
    \end{proof}
    Questo teorema implica che il rendimento di una macchina termica lavora con un rendimento limitato superiormente da $1-\frac{T_1}{T_2}$, quindi il rendimento di una macchina termica non può essere uguale a $1$, anzi anche se fosse reversibile il rendimento sarebbe comunque minore di $1$ in quanto il rendimento dipende dalla temperatura delle sorgenti.
    \paragraph{Teorema di Clausius}
        Il teorema di Clausius è una estensione del teorema di Carnot, esso afferma che per $N$ sorgenti termiche $T_1$, $T_2$, ..., $T_N$ con le quali una macchina termica $x$ scambia calore $\mathcal{Q}_1$, $\mathcal{Q}_2$, ..., $\mathcal{Q}_N$ rispettivamente, allora la somma $\displaystyle\sum_{i=1}^N \frac{\mathcal{Q}_i}{T_i} < 0$. La stessa cosa vale su $\infty$ sorgenti termiche, quindi $\displaystyle\oint\frac{d\mathcal{Q}}{T} < 0$.
\section{Entropia}
    A partire dalla seconda legge della termodinamica ed il teorema di clausius e considerando infinite sorgenti di un ciclo \textbf{reversibile} allora possiamo scrivere:
    \begin{align*}
        \oint \frac{d\mathcal{Q}}T=0
    \end{align*}
    Dunque prendendo due qualunque stati del ciclo $A$, $B$ ed i percorsi del ciclo $I$ e $II$ che collegano gli stati $A$ e $B$ si ha che:
    \begin{align*}
        \oint \frac{d\mathcal{Q}}T =& \int_{A, I\text{ Rev.}}^B \frac{d\mathcal{Q}}T + \int_{A, II\text{ Rev.}}^B \frac{d\mathcal{Q}}T = 0\\
        \int_{A, I\text{ Rev.}}^B \frac{d\mathcal{Q}}T =& -\int_{A, II\text{ Rev.}}^B \frac{d\mathcal{Q}}T\\
        \int_{A, I\text{ Rev.}}^B \frac{d\mathcal{Q}}T =& \int_{A, II\text{ Rev.}}^B \frac{d\mathcal{Q}}T \stackrel{\text{def.}}{\equiv} F(B) - F(A)
    \end{align*}
    Visto che il risultato non dipende dal percorso scelto visto che $I_{Rev.}$ e $II_{Rev.}$ sono entrambi reversibili, allora la funzione primitiva dell'integrale $\int_{A}^{B} \frac{d\mathcal{Q}}T$ è una funzione di stato, la quale descrive il cambiamento di calore tra due stati $A$ e $B$ in un ciclo reversibile ad una data temperatura $T$. Questa funzione di stato è chiamata \textbf{entropia} e viene indicata con $S$, formalmente si ha che:
    \begin{align}
        S|\Delta S_{A\to B}\stackrel{\text{def.}}{=}&S_B - S_A = \int_{A, \text{Rev}}^{B} \frac{d\mathcal{Q}}T\\
        dS\stackrel{\text{def.}}{=}&\left.\frac{d\mathcal{Q}}T\right|_{\text{Rev.}}
    \end{align}
    Notiamo come l'entropia non sia definita (al momento) come assoluta ma bensì solo come differenza di entropia tra due stati. Inoltre questa ha dimensionalità di $[S] = \left[\frac{E}{T}\right] = \frac{J}{K}$. Questa sta a significare che cedere una certa quantità di calore ad una certa temperatura \textbf{non} equivale all'opposto di assorbire la stessa quantità di calore ad una temperatura diversa.
    \paragraph{Entropia e ciclo di Carnot}
        Avendo definito l'entropia possiamo applicare questa nozione al ciclo di Carnot, possiamo dunque considerare il grafico $T/S$ del ciclo di Carnot, in esso le due isoterme verranno rappresentate come rette orizzontali, mentre le due adiabatiche come rette verticali, in quanto nelle prime non avviene una variazione di temperatura, mentre nelle seconde non avviene una variazione di entropia in quanto non avviene scambio di calore.
        \begin{figure}[H]
            \centering
            \begin{tikzpicture}
                \begin{axis}[
                    xlabel={$T$},
                    ylabel={$S$},
                    axis lines=middle,
                    axis line style={->},
                    xmin=0, xmax=3,
                    ymin=0, ymax=3,
                    xtick={0.5,2.5},
                    ytick={0.5,2.5},
                    xticklabels={$S_1$,$S_2$},
                    yticklabels={$T_1$,$T_2$},
                ]
                    \addplot[domain=0.5:2.5, samples=100, smooth] {2.5};
                    \addplot[domain=0.5:2.5, samples=100, smooth] {0.5};
                    \addplot[smooth] coordinates {(0.5,0.5) (0.5,2.5)};
                    \addplot[smooth] coordinates {(2.5,0.5) (2.5,2.5)};
                    \node at (axis cs:0.5,2.5) [anchor=south west] {$A$};
                    \node at (axis cs:2.5,2.5) [anchor=south east] {$B$};
                    \node at (axis cs:2.5,0.5) [anchor=south east] {$C$};
                    \node at (axis cs:0.5,0.5) [anchor=south west] {$D$};
                \end{axis}
            \end{tikzpicture}
        \end{figure}
        Mentre nel grafico $P/V$ l'area compresa tra le curve rappresenta il lavoro prodotto dalla macchina, nel grafico $T/S$ l'area compresa tra le curve rappresenta il calore scambiato dalla macchina. Calcoliamo dunque l'entropia scambiata dalla macchina durante il ciclo di Carnot, essa è data da:
        \begin{align*}
            \Delta S_{\circ} =& \Delta S_{AB} + \Delta S_{BC} + \Delta S_{CD} + \Delta S_{DA}\\
            =& \Delta S_{AB} + 0 + \Delta S_{CD} + 0\\
            =& \int_{A}^{B} \frac{d\mathcal{Q}}{T} + \int_{C}^{D} \frac{d\mathcal{Q}}{T} + 0\\
            =& \int_{A}^{B} \frac{nR\cancel{T}d[\operatorname{ln} V]}{\cancel{T}} + \int_{C}^{D} \frac{nR\cancel{T}d[\operatorname{ln} V]}{\cancel{T}}\\
            =& nR\operatorname{ln}\left(\frac{V_B}{V_A}\right) + nR\operatorname{ln}\left(\frac{V_D}{V_C}\right)\\
            =& nR\left[\operatorname{ln}\left(\frac{V_B}{V_A}\right) + \operatorname{ln}\left(\frac{V_D}{V_C}\right)\right]\\
        \end{align*}
        ma visto che $\frac{V_B}{V_A} = \frac{V_C}{V_D}$, e dunque $\frac{V_B}{V_A} = -\frac{V_D}{V_C}$, allora si ha che:
        \begin{align*}
            nR\left[\operatorname{ln}\left(\frac{V_B}{V_A}\right) + \operatorname{ln}\left(\frac{V_D}{V_C}\right)\right] = 0
        \end{align*}
        questo è indipendente dalla temperatura, quindi l'entropia scambiata dalla macchina durante il ciclo di Carnot è nulla, abbiamo quindi dimostrato che il ciclo di Carnot è un ciclo reversibile.
    \paragraph{Entropia nei sistemi isolati}
        Considerando un sistema isolato, esso non scambia calore con l'esterno, inoltre non scambia lavoro con l'esterno, quindi il suo stato non cambia al variare del tempo, quindi il suo stato è stazionario. Dato che il sistema universo è per eccellenza un sistema isolato, allora \textbf{l'entropia dell'universo non può diminuire} il che $\frac{dS_U}{dt} \geq 0$, quindi l'entropia nell'universo è positiva. Possiamo definire \textbf{l'entropia assoluta} di un sistema come: $S = k_B \operatorname{ln} [N_{pr}]$ dove $N_{pr}$ è il numero di possibili realizzazioni del sistema considerato! In questo modo l'entropia è sempre positiva, inoltre l'entropia di un sistema isolato non può diminuire, quindi l'entropia di un sistema isolato è sempre crescente. Questo ci permette di stabilire quando il nostro universo è stato originato, visto che il tempo non è assoluto ma relativo all'osservatore.  