\chapter{Nozioni di base}
Nel seguente capitolo definiamo alcune nozioni di base che verranno utilizzate all'interno del corso.

\paragraph{Sistema di riferimento} Un sistema di riferimento è un insieme di regole che permettono di determinare la posizione di un punto nello spazio. Un sistema di riferimento è composto da un'origine, da un insieme di assi e da un'unità di misura. Definiamo un sistema di riferimento in quattro assi: $x$, $y$, $z$ e $t$ dove $t$ rappresenta il tempo.
\begin{definition}[Spazio-Tempo Euclideo]
    Lo spazio-tempo euclideo ($S$) è un sistema di riferimento in quattro assi, $x$, $y$, $z$ e $t$, dove $t$ rappresenta il tempo. Lo spazio-tempo euclideo è definito come:
    $$
        S(O_z, x, y, z; O_t, t)
    $$
    dove $O_z$ è l'origine degli assi spaziali e $O_t$ è l'origine dell'asse temporale.
\end{definition}

\section{Moti in una dimensione e grafico orario}
    Per descrivere i moti in una dimensione possiamo utilizzare un grafico non affine, quindi non lineare, che rappresenta la posizione di un punto in funzione del tempo. Questo grafico è detto \textit{grafico orario}.
    \begin{definition}[Grafico Orario]
        Il grafico orario è un grafico cartesiano che esprime la posizione di un punto che si muove in una dimensione in funzione del tempo.
    \end{definition}
    \begin{tikzpicture}
        \begin{axis}
            [axis lines = left, xlabel = $t$, ylabel = $x$, xmin = 0, xmax = 11, ymin = 0, ymax = 11, xtick = {0, 2, 10}, ytick = {0, 2, 10}, xticklabels = {$0$, $t_i$, $t_f$}, yticklabels = {$0$, $x_i\ P_i$, $x_f\ P_f$}, legend pos = north west]
            \addplot[mark = *, color = red] coordinates {(2, 2)} node[above] {$E_i$};
            \addplot[mark = *, color = blue] coordinates {(10, 10)} node[above] {$E_f$};
        \end{axis}
    \end{tikzpicture}
    Vediamo come al momento $t_i$ il punto sia in posizione $P_i$ e di verifica come l'evento $E_i$ sia in posizione $x_i$. Al momento $t_f$ il punto è in posizione $P_f$ e l'evento $E_f$ è in posizione $x_f$. Ora possiamo definite lo spostamento:
    \begin{definition}[Spostamento]
        Lo spostamento è la variazione di posizione di un punto in un intervallo di tempo. Lo spostamento è definito come:
        $$
            \begin{aligned}
                S_{i \to f} &= x_f - x_i\\
                \Delta x_{i \to f} &= x_f - x_i
            \end{aligned}
        $$
    \end{definition}
    Da notare come lo spostamento non descrive ne' la traiettoria ne' la distanza percorsa dal punto ma solo la variazione di posizione, infatti il punto potrebbe aver compiuto un percorso "non diretto". Inoltre nello spostamento ha un verso definito e come conseguenza scrivere $S_{i \to f} \neq S_{f \to i}$.
    \begin{definition}[Distanza Percorsa]
        La distanza percorsa è la lunghezza della traiettoria percorsa da un punto in un intervallo di tempo. La distanza percorsa è definita come:
        $$
            d(P_i, P_f) = \left| x_f - x_i \right|
        $$
    \end{definition}
    Notiamo come la distanza percorsa sia sempre positiva in quanto è la lunghezza della traiettoria percorsa dal punto. Inoltre la distanza percorsa non ha un verso definito, infatti $d(P_i, P_f) = d(P_f, P_i)$.\newline
    Ora per descrivere il moto di un punto possiamo definire la velocità media:
    \begin{definition}[Velocità media]
        La velocità media è la variazione di posizione di un punto in funzione del tempo. La velocità media è definita come:
        $$
            \begin{aligned}
                v_m &= \frac{\Delta x}{\Delta t} = \frac{\Delta x_{i \to f}}{\Delta t_{i \to f}}\\
            \end{aligned}
        $$
    \end{definition}
    Da notare come la velocità media non tiene conto del moto del punto in un intervallo di tempo, ma solo della variazione di posizione. Inoltre la velocità media ha un verso definito in quanto trattiamo lo spostamento (il quale ha un verso definito).\newline
    Per descrivere il moto di un punto in un instante $t$ di tempo possiamo definire la velocità istantanea:
    \begin{definition}[Velocità istantanea]
        La velocità istantanea è la variazione di posizione di un punto in un istante di tempo. La velocità istantanea è definita come:
        $$
            v_i(t)=\lim\limits_{\Delta t \to 0} \frac{\Delta x}{\Delta t} = \frac{dx}{dt}
        $$
    \end{definition}
    Dal punto di vista matematico la velocità istantanea è la derivata della posizione rispetto al tempo. Dunque i punti dove si passa da un movimento ``in avanti'' ad un movimento ``all'indietro'' sono i punti in cui la velocità istantanea è nulla ovvero i punti di massimo e minimo della funzione posizione, inoltre il punto in cui la velocità istantanea è nulla è detto punto di inversione. Inoltre la velocità istantanea è una funzione continua in quanto la derivata di una funzione continua è anch'essa continua.\newline
    È vero che in un determinato periodo di tempo io possa aumentare o diminuire la velocità, per questo motivo definiamo la funzione di accelerazione:
    \begin{definition}[Accelerazione]
        L'accelerazione è la variazione di velocità di un punto in funzione del tempo. L'accelerazione è definita come:
        $$
            \begin{aligned}
                a &= \frac{\Delta v}{\Delta t} = \frac{\Delta v_{i \to f}}{\Delta t_{i \to f}}\\
            \end{aligned}
        $$
    \end{definition}
    Da notare come l'accelerazione non tiene conto del moto del punto in un intervallo di tempo, ma solo della variazione di velocità. Inoltre l'accelerazione ha un verso definito in quanto trattiamo la variazione di velocità (la quale ha un verso definito).
    \paragraph{Relazione tra posizione, velocità e accelerazione}
        Come già detto la velocità è la derivata della posizione rispetto al tempo e l'accelerazione è la derivata della velocità rispetto al tempo, è vero inoltre che la posizione è l'integrale della velocità rispetto al tempo e questa è l'integrale dell'accelerazione rispetto al tempo. Dunque possiamo scrivere:
        \begin{align}
            x(t) &\text{ posizione} \\
            v(t) = \frac{dx}{dt} &\text{ velocità} \\
            a(t) = \frac{dv}{dt} = \frac{d^2x}{dt^2} &\text{ accelerazione}
        \end{align} 
        Al contrario possiamo scrivere:
        \begin{align}
            v(t) &= v_0 + \int_{t_0}^{t} a(T) dT\\
            x(t) &= x_0 + \int_{t_0}^{t} v(T) dT = x_0 + \int_{t_0}^{t} dT \left[v_0+\int_{t_0}^{T} a(\tau) d\tau\right]
        \end{align}
        Ora le dimensioni fisiche (e non le unità di misura) di queste grandezze sono:
        $$
            \begin{aligned}
                \left[x\right] &= \left[L\right]\\
                [v] &= \left[\frac{L}{T}\right]\\
                [a] &= \left[\frac{L}{T^2}\right]
            \end{aligned}
        $$
        ed le rispettive unità di misura sono:
        $$
            \begin{aligned}
                \left[x\right] &= \left[m\right]\\
                [v] &= \left[\frac{m}{s}\right]\\
                [a] &= \left[\frac{m}{s^2}\right]
            \end{aligned}
        $$