\appendix 
\renewcommand{\thesection}{A.\arabic{section}}
\chapter{Appendice A: Note delle lezioni}
Di seguito sono riportate delle note delle lezioni ulteriori agli appunti stessi del corso.

\section{24 febbraio 2025}
\label{lez:24-02-2025}
Le tre regole del grafico orario:
\begin{itemize}
    \item Il tempo non si ferma;
    \item Il tempo scorre sempre allo stesso, uniforme, modo per tutti;
    \item Non si può andare più veloce della luce ($c$), non possono esistere dunque rette con pendenza maggiore di $c$.
    \item Non esiste ancora il teletrasporto.
\end{itemize}
\section{26 febbraio 2025}
\label{lez:26-02-2025}
Si noti come lo scopo del problema non fosse strettamente quello di trovare il punto nel quale il punto si ferma, ma di capire come l'analisi dimensionale possa aiutarci a risolvere un problema fisico.