\documentclass[a4paper,twoside]{book}
\usepackage[italian]{babel}
\usepackage[utf8]{inputenc}
\usepackage{amsmath}
\usepackage{amsthm}
\usepackage{amsfonts}
\usepackage{amssymb}
\usepackage{cancel}
\usepackage[margin=1in]{geometry}
\usepackage{hyperref}
\usepackage{bookmark}
\usepackage{setspace}
\usepackage{titlesec}
\usepackage{fancyhdr}
\usepackage{adjustbox}
\usepackage{float}
\usepackage{graphicx}
\usepackage{float}
\usepackage{algpseudocode}
\usepackage[linesnumbered,ruled,vlined]{algorithm2e}
\usepackage{xcolor}
\usepackage{appendix}
\usepackage{xparse}
\usepackage{tikz}
\usepackage{physics}
\usepackage[outline]{contour} % glow around text
\usepackage{pgfplots}
\usepackage{subcaption}

\usetikzlibrary{arrows.meta} % for arrow size
\usetikzlibrary{snakes,calc,patterns,angles,quotes}

\colorlet{xcol}{blue!70!black}
\colorlet{darkblue}{blue!40!black}
\colorlet{myred}{red!65!black}
\tikzstyle{mydashed}=[xcol,dashed,line width=0.25,dash pattern=on 2.2pt off 2.2pt]
\tikzstyle{axis}=[->,thick] %line width=0.6
\tikzstyle{ell}=[{Latex[length=3.3,width=2.2]}-{Latex[length=3.3,width=2.2]},line width=0.3]
\tikzstyle{dx}=[-{Latex[length=3.3,width=2.2]},darkblue,line width=0.3]
\tikzstyle{ground}=[preaction={fill,top color=black!10,bottom color=black!5,shading angle=20},
                    fill,pattern=north east lines,draw=none,minimum width=0.3,minimum height=0.6]
\tikzstyle{mass}=[line width=0.6,red!30!black,fill=red!40!black!10,rounded corners=1,
                    top color=red!40!black!20,bottom color=red!40!black!10,shading angle=20]
\tikzstyle{spring}=[line width=0.8,blue!7!black!80,snake=coil,segment amplitude=5,segment length=5,line cap=round]
\tikzset{>=latex} % for LaTeX arrow head
\tikzstyle{force}=[->,myred,very thick,line cap=round]
\def\tick#1#2{\draw[thick] (#1)++(#2:0.1) --++ (#2-180:0.2)}



\pgfplotsset{width=10cm,compat=1.9}

\setlength{\parskip}{0pt}
\titlespacing*{\subparagraph}{1em}{0em}{0em} 

\makeatletter
\newenvironment{abstract}{%
    \if@twocolumn
        \section*{\abstractname}%
    \else
        \begin{center}%
            {\bfseries \abstractname\vspace{-.5em}\vspace{\z@}}%
        \end{center}%
        \small
        \begin{quotation}
    \fi}
    {\if@twocolumn\else\end{quotation}\fi}
\makeatother

\let\oldquote\quote
\let\endoldquote\endquote

\RenewDocumentEnvironment{quote}{om}
    {\oldquote}
    {\par\nobreak\smallskip
        \hfill(#2\IfValueT{#1}{~---~#1})\endoldquote 
        \addvspace{\bigskipamount}}

\hypersetup{
    pdfauthor={Luca Facchini},
    pdftitle={Appunti di Fisica},
    pdfsubject={Appunti del corso di Fisica, tenuto dal prof. Iuppa Roberto presso l'Università degli Studi di Trento. Corso seguito nell'anno accademico 2024/2025.},
    pdfkeywords={Reti, Università degli Studi di Trento, Iuppa Roberto},
    pdfproducer={LaTeX},
    pdfcreator={pdflatex},
}

\fancypagestyle{chapterInit}{%
    \fancyhf{}
    \renewcommand{\headrulewidth}{0pt}
    \renewcommand{\footrulewidth}{0.4pt}
    \fancyfoot{}
    \fancyfoot[LE,RO]{\thepage}
    \fancyfoot[LO,RE]{``Appunti di Fisica" di Luca Facchini}
}
\fancypagestyle{stdPage}{
    \setlength{\headheight}{14.5pt}
    \fancyhead{}
    \fancyhead[LE]{\leftmark}
    \fancyhead[RO]{\rightmark}
    \fancyfoot{}
    \renewcommand{\footrulewidth}{0.4pt}
    \fancyfoot[LE,RO]{\thepage}
    \fancyfoot[LO,RE]{``Appunti di Fisica" di Luca Facchini}
}

\fancypagestyle{tocStyle}{
    \pagestyle{stdPage}
    \fancyhead[RE,LO]{}
}
\fancypagestyle{plain}{
    \pagestyle{chapterInit}
}

\graphicspath{{./images/}}

\newtheorem{definition}{Definizione}[chapter]
\newtheorem{problem}{Problema}[chapter]
\newtheorem{law}{Legge}[chapter]

\title{Appunti di Fisica}
\author{Luca Facchini (mat. 245965)}
\date{A.A. 2024/2025}

\begin{document}
    
    % Page numeration to roman
    \frontmatter
    \pagenumbering{Roman}

    \begin{titlepage}
        \centering  % Center everything on the title page
        {\Huge\textbf{Appunti di Fisica}} \\[1cm] % Title
        \vspace{1.5cm}
        
        {\normalsize di: } \\[.3cm]
        {\Large Facchini Luca} \\ % Author name
        \vspace{1.5cm}
        

        {\normalsize Corso tenuto dal prof. Iuppa Roberto} \\[0.3cm] % Course information
        {\large Università degli Studi di Trento} \\[1.5cm]
        
        {\large A.A. 2024/2025} \\[3cm] % Academic year
        
        % Abstract section with spacing control
        \vfill
        \begin{minipage}[t]{0.4\textwidth}
            \begin{flushleft} \normalsize
                \emph{Autore:}\\
                \textsc{Facchini} Luca \\ % Author name
                Mat. 245965 \\
                \vspace{-\baselineskip}
                \begin{tabbing}
                    Email:\= \href{mailto:luca.facchini-1@studenti.unitn.it}{luca.facchini-1@studenti.unitn.it} \\
                        \>  \href{mailto:luca@fc-software.it}{luca@fc-software.it}
                \end{tabbing}
            \end{flushleft}
        \end{minipage}%
        \hfill
        \begin{minipage}[t]{0.4\textwidth}
            \begin{flushleft} \normalsize
                \emph{Corso:}\\
                Fisica [145011] \\
                \textsc{CdL}: Laurea Triennale in Informatica \\
                Prof. \textsc{Iuppa} Roberto \\
                Email: \href{mailto:roberto.iuppa@unitn.it}{roberto.iuppa@unitn.it}
            \end{flushleft}
        \end{minipage}
        \vfill
        \begin{abstract}
            Appunti del corso di Reti, tenuto dal prof. Iuppa Roberto presso l'Università degli Studi di Trento. Corso seguito nell'anno accademico 2024/2025.
        \end{abstract}
        
        % Pushes the content to the center vertically
    \end{titlepage}
    \begingroup
        \pagestyle{tocStyle}
        \addtocontents{toc}{\protect\thispagestyle{tocStyle}}
        \addtocontents{toc}{\protect\pagestyle{tocStyle}}
        \tableofcontents
    \endgroup
    \thispagestyle{tocStyle}
    \pagestyle{stdPage}
    
    % Page numeration back to arabic
    \newpage

    \mainmatter
    \pagenumbering{arabic}
    \chapter{Cinematica}
Nel seguente capitolo andremo ad analizzare la cinematica, ovvero la branca della fisica che si occupa di descrivere il moto di un punto nello spazio. Per fare ciò andremo ad analizzare le grandezze fisiche che descrivono il moto di un punto nello spazio e come queste siano collegate tra loro.

\section{Nozioni Preliminari}

    \paragraph{Sistema di riferimento} Un sistema di riferimento è un insieme di regole che permettono di determinare la posizione di un punto nello spazio. Un sistema di riferimento è composto da un'origine, da un insieme di assi e da un'unità di misura. Definiamo un sistema di riferimento in quattro assi: $x$, $y$, $z$ e $t$ dove $t$ rappresenta il tempo.
    \begin{definition}[Spazio-Tempo Euclideo]
        Lo spazio-tempo euclideo ($S$) è un sistema di riferimento in quattro assi, $x$, $y$, $z$ e $t$, dove $t$ rappresenta il tempo. Lo spazio-tempo euclideo è definito come:
        $$
            S(O_z, x, y, z; O_t, t)
        $$
        dove $O_z$ è l'origine degli assi spaziali e $O_t$ è l'origine dell'asse temporale.
    \end{definition}

\section{Moti in una dimensione e grafico orario}
    Per descrivere i moti in una dimensione possiamo utilizzare un grafico non affine, quindi non lineare, che rappresenta la posizione di un punto in funzione del tempo. Questo grafico è detto \textit{grafico orario}.
    \begin{definition}[Grafico Orario]
        Il grafico orario è un grafico cartesiano che esprime la posizione di un punto che si muove in una dimensione in funzione del tempo.
    \end{definition}
    \begin{tikzpicture}
        \begin{axis}
            [axis lines = left, xlabel = $t$, ylabel = $x$, xmin = 0, xmax = 11, ymin = 0, ymax = 11, xtick = {0, 2, 10}, ytick = {0, 2, 10}, xticklabels = {$0$, $t_i$, $t_f$}, yticklabels = {$0$, $x_i\ P_i$, $x_f\ P_f$}, legend pos = north west]
            \addplot[mark = *, color = red] coordinates {(2, 2)} node[above] {$E_i$};
            \addplot[mark = *, color = blue] coordinates {(10, 10)} node[above] {$E_f$};
        \end{axis}
    \end{tikzpicture}
    Vediamo come al momento $t_i$ il punto sia in posizione $P_i$ e di verifica come l'evento $E_i$ sia in posizione $x_i$. Al momento $t_f$ il punto è in posizione $P_f$ e l'evento $E_f$ è in posizione $x_f$. Ora possiamo definite lo spostamento:
    \begin{definition}[Spostamento]
        Lo spostamento è la variazione di posizione di un punto in un intervallo di tempo. Lo spostamento è definito come:
        $$
            \begin{aligned}
                S_{i \to f} &= x_f - x_i\\
                \Delta x_{i \to f} &= x_f - x_i
            \end{aligned}
        $$
    \end{definition}
    Da notare come lo spostamento non descrive ne' la traiettoria ne' la distanza percorsa dal punto ma solo la variazione di posizione, infatti il punto potrebbe aver compiuto un percorso "non diretto". Inoltre nello spostamento ha un verso definito e come conseguenza scrivere $S_{i \to f} \neq S_{f \to i}$.
    \begin{definition}[Distanza Percorsa]
        La distanza percorsa è la lunghezza della traiettoria percorsa da un punto in un intervallo di tempo. La distanza percorsa è definita come:
        $$
            d(P_i, P_f) = \left| x_f - x_i \right|
        $$
    \end{definition}
    Notiamo come la distanza percorsa sia sempre positiva in quanto è la lunghezza della traiettoria percorsa dal punto. Inoltre la distanza percorsa non ha un verso definito, infatti $d(P_i, P_f) = d(P_f, P_i)$.\newline
    Ora per descrivere il moto di un punto possiamo definire la velocità media:
    \begin{definition}[Velocità media]
        La velocità media è la variazione di posizione di un punto in funzione del tempo. La velocità media è definita come:
        $$
            \begin{aligned}
                v_m &= \frac{\Delta x}{\Delta t} = \frac{\Delta x_{i \to f}}{\Delta t_{i \to f}}\\
            \end{aligned}
        $$
    \end{definition}
    Da notare come la velocità media non tiene conto del moto del punto in un intervallo di tempo, ma solo della variazione di posizione. Inoltre la velocità media ha un verso definito in quanto trattiamo lo spostamento (il quale ha un verso definito).\newline
    Per descrivere il moto di un punto in un instante $t$ di tempo possiamo definire la velocità istantanea:
    \begin{definition}[Velocità istantanea]
        La velocità istantanea è la variazione di posizione di un punto in un istante di tempo. La velocità istantanea è definita come:
        $$
            v_i(t)=\lim\limits_{\Delta t \to 0} \frac{\Delta x}{\Delta t} = \frac{dx}{dt}
        $$
    \end{definition}
    Dal punto di vista matematico la velocità istantanea è la derivata della posizione rispetto al tempo. Dunque i punti dove si passa da un movimento ``in avanti'' ad un movimento ``all'indietro'' sono i punti in cui la velocità istantanea è nulla ovvero i punti di massimo e minimo della funzione posizione, inoltre il punto in cui la velocità istantanea è nulla è detto punto di inversione. Inoltre la velocità istantanea è una funzione continua in quanto la derivata di una funzione continua è anch'essa continua.\newline
    È vero che in un determinato periodo di tempo io possa aumentare o diminuire la velocità, per questo motivo definiamo la funzione di accelerazione:
    \begin{definition}[Accelerazione]
        L'accelerazione è la variazione di velocità di un punto in funzione del tempo. L'accelerazione è definita come:
        $$
            \begin{aligned}
                a &= \frac{\Delta v}{\Delta t} = \frac{\Delta v_{i \to f}}{\Delta t_{i \to f}}\\
            \end{aligned}
        $$
    \end{definition}
    Da notare come l'accelerazione non tiene conto del moto del punto in un intervallo di tempo, ma solo della variazione di velocità. Inoltre l'accelerazione ha un verso definito in quanto trattiamo la variazione di velocità (la quale ha un verso definito).
    \paragraph{Relazione tra posizione, velocità e accelerazione}
        Come già detto la velocità è la derivata della posizione rispetto al tempo e l'accelerazione è la derivata della velocità rispetto al tempo, è vero inoltre che la posizione è l'integrale della velocità rispetto al tempo e questa è l'integrale dell'accelerazione rispetto al tempo. Dunque possiamo scrivere:
        \begin{align}
            x(t) &\text{ posizione} \\
            v(t) = \frac{dx}{dt} &\text{ velocità} \\
            a(t) = \frac{dv}{dt} = \frac{d^2x}{dt^2} &\text{ accelerazione}
        \end{align} 
        Al contrario possiamo scrivere:
        \begin{align}
            v(t) &= v_0 + \int_{t_0}^{t} a(T) dT\\
            x(t) &= x_0 + \int_{t_0}^{t} v(T) dT = x_0 + \int_{t_0}^{t} dT \left[v_0+\int_{t_0}^{T} a(\tau) d\tau\right]
        \end{align}
        Ora le dimensioni fisiche (e non le unità di misura) di queste grandezze sono:
        $$
            \begin{aligned}
                \left[x\right] &= \left[L\right]\\
                [v] &= \left[\frac{L}{T}\right]\\
                [a] &= \left[\frac{L}{T^2}\right]
            \end{aligned}
        $$
        ed le rispettive unità di misura sono:
        $$
            \begin{aligned}
                \left[x\right] &= \left[m\right]\\
                [v] &= \left[\frac{m}{s}\right]\\
                [a] &= \left[\frac{m}{s^2}\right]
            \end{aligned}
        $$
    \subsection{Esercizio sulle grandezze fisiche}
        Andiamo ora ad analizzare uno strumento importante per la risoluzione di esercizi fisici, ovvero l'analisi dimensionale. L'analisi dimensionale è uno strumento che ci permette di capire se un'equazione è corretta o meno e ci può suggerire come risolvere un problema. Vediamo un esempio:
        \begin{problem}
            Sia un punto che si muove lungo un asse $x$ in funzione del tempo $t$, questo al momento $t_0=0$ si trova al punto $x(t_0)=x(0)=x_0=0$ e la sua velocità in questo punto è $v(t_0)=v(0)=v_0>0$. La sua accelerazione è descritta dalla funzione $a(x)=-Ax-B$ con $A,B>0$.\newline
            Determinare il momento in cui il punto si ferma($x_{\text{stop}}$)
        \end{problem}
        Analizziamo l'equazione dell'accelerazione:
        $$
            a(x) = -Ax-B
        $$
        notiamo come questa abbia come dimensioni fisiche:
        $$
            \begin{aligned}
                a(x) &= -Ax&-B\\
                \left[\frac{L}{T^2}\right] &= \left[?\right]\left[L\right]&-\left[?\right]
            \end{aligned}
        $$
        da queste possiamo dedurre che $A$ ha dimensioni fisiche $\left[\frac{1}{T^2}\right]$ e $B$ ha dimensioni fisiche $\left[\frac{L}{T^2}\right]$. Ora l'equazione dello spazio in funzione del tempo è: $$
            \frac{d^2x}{dt^2}=-Ax-B
        $$
        dunque dobbiamo trovare una funzione $x(t)$ tale che soddisfi questa equazione differenziale, in quanto dobbiamo mantenere mantenere una funzione sullo spazio ed ottenere il parametro $A$ all'esterno ipotizziamo che la funzione sia:
        $$  
            \begin{aligned}
                x(t)=&X_0\sin(\sqrt{A}t + \varphi) = \\
                =& \mathcal{A}\sin(\sqrt{A}t + \varphi)
            \end{aligned}
        $$
        Prima di calcolare le derivate di questa funzione verifichiamo che effettivamente questa funzione soddisfi la dimensionalità:
        $$
            \begin{aligned}
                \left[x\right] &= \left[L\right]\\
                \left[\mathcal{A}\right] &= \left[L\right]\\
                \left[\sqrt{A}\right] &= \left[\frac{1}{T}\right]\\
                \left[t\right] &= \left[T\right]\\
                \left[\varphi\right] &= \left[1\right]\\
                \left[\sqrt{A}t + \varphi\right] &= \left[1\right]\checkmark\\
                \left[\mathcal{A}\right]\left[\sin(\left[1\right])\right]&=\left[L\right]\checkmark
            \end{aligned}
        $$
        ora verifichiamo che questa funzione soddisfi l'equazione differenziale, calcoliamo la derivata prima e la derivata seconda:
        $$
            \begin{aligned}
                \frac{dx}{dt} &= \mathcal{A}\sqrt{A}\cos(\sqrt{A}t + \varphi)\\
                \frac{d^2x}{dt^2} &= -A\underbrace{\mathcal{A}\sin(\sqrt{A}t + \varphi)}_{x(t)}
            \end{aligned}
        $$
        verifichiamo come anche queste derivate soddisfino la dimensionalità:
        $$
            \begin{aligned}
                \left[\frac{dx}{dt}\right] &\stackrel?= \left[\frac{L}{T}\right] \\
                \left[\frac{L}{T}\right] &= \left[\mathcal{A}\right]\left[\sqrt{A}\right]\left[\cos(\left[1\right])\right]\\
                \left[\frac{L}{T}\right] &= \left[L\right]\left[\frac{1}{T}\right]\left[1\right]\checkmark\\
                \\ 
                \left[\frac{d^2x}{dt^2}\right] &\stackrel?= \left[\frac{L}{T^2}\right] \\
                \left[\frac{L}{T^2}\right] &= \left[A\right]\left[\mathcal{A}\right]\left[\sin(\left[1\right])\right]\\
                \left[\frac{L}{T^2}\right] &= \left[\frac{1}{T^2}\right]\left[L\right]\left[1\right]\checkmark
            \end{aligned}
        $$
        Dunque queste derivate soddisfano la dimensionalità, ma manca ancora il parametro $-B$ nell'equazione dell'accelerazione, dunque dobbiamo modificare la funzione $x(t)$ in modo che soddisfi anche questa condizione, partiamo dal fatto nella prima funzione possiamo aggiungere/sottrarre solo una quantità di dimensionalità $\left[L\right]$ e notiamo come in quanto $\left[B\right]=\left[\frac{L}{T^2}\right]$ ed $\left[A\right]=\left[\frac{1}{T^2}\right]$ allora: 
        $$
            \left[\frac{B}{A}\right] = \left[\frac{\frac{L}{\cancel{T^2}}}{\frac{1}{\cancel{T^2}}}\right] = \left[L\right]
        $$
        Dunque $\frac{B}{A}$ può essere aggiunto alla funzione $x(t)$ in quanto ha le stesse dimensioni fisiche di $x(t)$, dunque la funzione $x(t)$ diventa:
        $$
            x(t) = \mathcal{A}\sin(\sqrt{A}t + \varphi) - \frac{B}{A}
        $$
        questo non comporta alcun cambiamento alle derivate in quanto queste sono in funzione di $t$ ed $A$ e $B$ sono delle costanti. Possiamo però notare come la derivata seconda di questa può essere riscritta come:
        $$
            \begin{aligned}
                \frac{d^2x}{dt^2} =& -A\left(\mathcal{A}\sin(\sqrt{A}t + \varphi)\right)\\
                =&-A\left(x(t)+\frac{B}{A}\right)\\
                =&-Ax(t)-B
            \end{aligned}
        $$
        Abbiamo quindi trovato la funzione $x(t)$ che soddisfa l'equazione differenziale, verifichiamo che nel punto $x_0$ questa soddisfi le condizioni iniziali per la posizione e la velocità:
        \begin{align}
            x(0) &= \mathcal{A}\sin(\varphi) - \frac{B}{A} = 0 \label{eq:P1pos}\\
            \frac{dx}{dt}(0) = v(0) &= \mathcal{A}\sqrt{A}\cos(\varphi) > 0 \label{eq:P1vel}
        \end{align}
        Per $\mathcal{A}>0$ allora per \ref{eq:P1pos} $\sin(\varphi) = \frac{B}{A\mathcal{A}}$ e per \ref{eq:P1vel} $\cos(\varphi) > 0$ queste condizioni in aggiunta a quelle di $ A,B>0 $ ci permettono di dire che $$
            \begin{cases}
                \mathcal{A}>0\\
                0<\varphi<\frac{\pi}{2}
            \end{cases}
        $$
        ora dobbiamo trovare il momento in cui il punto si ferma ($x_{\text{stop}}$), ovvero il momento in cui la velocità è nulla ($v(t_{\text{stop}})=v_{\text{stop}}=0$). 
        $$
            \begin{aligned}
                v(t_{\text{stop}}) =&\sqrt{A}\mathcal{A}\cos(\sqrt{A}t_{\text{stop}} + \varphi) = 0\\
                \sqrt{A}\mathcal{A}\neq 0\Rightarrow & \cos(\sqrt{A}t_{\text{stop}} + \varphi) = 0\\
                \sqrt{A}t_{\text{stop}} + \varphi &= \frac{\pi}{2}\\
                t_{\text{stop}} &= \left(\frac{\pi}{2}-\varphi\right)\frac{1}{\sqrt{A}}\\
            \end{aligned}
        $$
        ed al momento $t_{\text{stop}}$ la posizione del punto è:
        $$
            \begin{aligned}
                x_{\text{stop}} &= \mathcal{A}\sin\left[\cancel{\sqrt{A}}\frac{1}{\cancel{\sqrt{A}}}\left(\frac{pi}2-\cancel\varphi\right) + \cancel\varphi\right] - \frac{B}{A} \\
                &= \mathcal{A}\sin\left(\frac{\pi}{2}\right) - \frac{B}{A}\\
                &= \mathcal{A} - \frac{B}{A}
            \end{aligned}
        $$
        {\footnotesize Vedi \ref{lez:26-02-2025}}
    \subsection{Problema de ``Il lancio del sasso'' (M.R.U.A e M.R.U)}
        Questo classico problema viene usato per definire il Moto Rettilineo Uniformemente Accelerato ed il Moto Rettilineo Uniforme
        \begin{problem}
            Una coppia di amici vuole misurare l'altezza di un precipizio. Decidono di farlo lanciando un sasso verso il basso e misurando il tempo che impiega a raggiungere il fondo. Sappiamo che il tempo tra il rilascio del sasso grave\footnote{Soggetto alla gravità} ed il rumore dell'impatto è di $t_a$ secondi.\newline
            Calcolare l'altezza del precipizio.
        \end{problem}
        Avendo definito il sistema di riferimento con origine la cima del precipizio (dove sono gli amici) ed un verso ``puntante'' il fondo.\\
        Il problema può essere diviso in due parti:
        \begin{enumerate}
            \item[I] Il moto del masso grave
            \item[II] Il moto del suono
        \end{enumerate}
        Mentre la prima parte è descritta da una accelerazione costante $a(t)=g$, una velocità iniziale nulla $v(0)=0$, una posizione iniziale $x(0)=0$ e una posizione finale $x(t_f)=x_f$. La seconda parte è descritta da una velocità costante $v(t)=v_s$ e una posizione iniziale $x(t)=x_f$ e una posizione finale $x(t)=0$. Allora possiamo scrivere la posizione del masso grave come:
        $$
            \begin{aligned}
                a&=g\\
                v(t)&=v_0+\int_0^t a(T)dT \\
                &= v_0 + gt\\
                z(t)&=z_0+\int_0^t v(T)dT\\ 
                &= z_0 + \int_0^t(v_0 + gT )dT \\ 
                &= \underbrace{z_0}_0 + \underbrace{v_0}_0t + \frac{1}{2}gt^2\\
                &= \frac{1}{2}gt^2
            \end{aligned}
        $$
        Da notare come tutto ciò può essere scritto solo se $t<t_f$ in quanto il masso non può andare oltre il fondo del precipizio.\\
        Assumendo che il sasso venga lasciato perpendicolarmente al suolo allora possiamo scrivere la posizione del suono come:
        $$
            \begin{aligned}
                v(t)&=v_s\\
                z(t)|_{t>t_f}&=z_f + \int_{t_f}^t v_s dT\\
                &= z_f + v_s(t-t_f)
            \end{aligned}
        $$
        A questo punto definendo come $t_a$ il tempo tra il rilascio del sasso e l'impatto (amico), $t_f$ come il tempo del fondo del precipizio e $t_0$ come il tempo di rilascio del sasso, e dunque definito che $\Delta t_a = \Delta t_{mg} + \Delta t_{ms} $ possiamo scrivere:
        $$
            \begin{cases}
                Z_f = \frac{1}{2}gt_f^2\\
                \frac{1}{2}gt_f^2 - v_s(t_s-t_f) = 0\\
            \end{cases} = 
            \begin{cases}
                /\\
                t_f^2+2\frac{v_s}{g}t_f - 2\frac{Z_f}{g} = 0
            \end{cases}
        $$
        Questa equazione di secondo grado ha come soluzione:
        $$
            t_f = -\frac{v_s}{g} \pm \sqrt{\left(\frac{v_s}{g}\right)^2 + 2\frac{Z_f}{g}}
        $$
        la cui unica soluzione valida è:
        $$
            t_f = -\frac{v_s}{g} + \sqrt{\left(\frac{v_s}{g}\right)^2 + 2\frac{Z_f}{g}}
        $$
        Questo è il tempo che impiega il sasso a raggiungere il fondo del precipizio, ora possiamo calcolare l'altezza del precipizio:
        $$
            \begin{aligned}
                Z_f &= \frac{1}{2}g\left(-\frac{v_s}{g} + \sqrt{\left(\frac{v_s}{g}\right)^2 + 2\frac{Z_f}{g}}\right)^2\\
                &= \frac{1}{2}g\left(\frac{v_s}{g} - \sqrt{\left(\frac{v_s}{g}\right)^2 + 2\frac{Z_f}{g}}\right)^2\\
                &= \frac{1}{2}g\left({\frac{v_s}{g}}^2 - 2\frac{v_s}{g}\sqrt{\left(\frac{v_s}{g}\right)^2 + 2\frac{Z_f}{g}} + \left(\left(\frac{v_s}{g}\right)^2 + 2\frac{Z_f}{g}\right)\right) \\
            \end{aligned}
        $$
    \chapter{Dinamica}
\label{chap:dinamica}

Nel presente capitolo si analizzerà il moto di un corso a partire dai tre principi fondamentali della dinamica fino ad arrivare ad \dots
% Il capitolo non è ancora stato trattato interamente nel corso

\section{I tre principi fondamentali}
    Andiamo ora ad enunciare i tre principi fondamentali della dinamica.
    \begin{law}[Principio d'inerzia]
        In un sistema inerziale, mantiene il suo stato di moto rettilineo uniforme o quiete finché una forza esterna non agisce su questo
    \end{law}
    Da notare come questa legge vale solo ed esclusivamente nel caso nel quale il sistema di riferimento sia inerziale. In caso contrario questo principio \underline{non si applica}.
    \begin{law}[Principio di Newton]
        \label{law:principio-newton}
        L'accelerazione che un corpo riceve è legata a questo mediante una costante numerica $(m)$
    \end{law}
    Quindi $ \vec{a}=m\vec{F}$, spesso la constante $m$ viene apposta al denominatore della formula, ottenendo $ \vec{F}=m\vec{a}$. Questa è la forma più comune della legge di Newton, dove la forza ($F$) è uguale alla massa ($m$)\footnote{Sempre positiva}, misurata in $kg$, per l'accelerazione ($a$) misurata in $m/s^2$.
    \begin{law}[Principio di azione e reazione]
        La forza che un corpo $A$ esercita sul corpo $B$ è uguale alla forza che il corpo $B$ esercita sul corpo $A$, ma di verso opposto.
    \end{law}
    Questo principio è molto importante in quanto ci permette di capire come mai un corpo si muova. Infatti, se un corpo $A$ esercita una forza su un corpo $B$, il corpo $B$ eserciterà una forza uguale e di verso opposto su $A$, l'accelerazione d'altronde sarà diversa se le masse sono diverse.
    \paragraph{Quantità di moto}
        La quantità di moto è una grandezza vettoriale che descrive ``quanto'' movimento c'è in un sistema e dipende dalla massa e dalla velocità del corpo. È definito come segue:
        $$
            \vec{p}=m\vec{v}
        $$
        Ad esempio se un corpo di massa $m=100g$ viaggia a $v=360km/h$ la sua quantità di moto sarà $|\vec{p}|=0.1 [kg] \cdot 100 [m/s] = 10 [kg \cdot m/s]$
    \paragraph{Impulso}
        L'impulso è una grandezza vettoriale che descrive ``quanto'' una forza agisce su un corpo in uno specifico istante/intervallo di tempo. È definito come segue:
        $$
            \vec{J} = \vec{P_f} - \vec{P_i} = \Delta \vec{p} = m_f\vec{v_f} - m_i\vec{v_i}
        $$
        In regime di conservazione della massa, allora $m_f=m_i$ e quindi $\vec{J} = m\vec{v_f} - m\vec{v_i} = m(\vec{v_f} - \vec{v_i}) = m\vec{\Delta v}$ dove $\vec{\Delta v}$ è la variazione di velocità del corpo.\newline
        Inoltre visto il secondo principio (\ref{law:principio-newton}) possiamo scrivere:
        $$
            \vec{F} = m\vec{a} = m\frac{d\vec{v}}{dt} \Rightarrow \vec{F} = m\frac{d\vec{v}}{dt}
        $$
        Questo è anche scrivibile usando la notazione:
        $$
            \vec{F}=m\frac{\Delta \vec{v}}{\Delta t}
        $$
        in questo caso $\vec{J} = \vec{F_{\text{imp}}} \Delta t$ dove $\vec{F_{\text{imp}}}$ è la forza impulsiva, in quanto stiamo trattando di un intervallo di tempo finito e non un infinitesimo.
    \paragraph{Forza risultante}
        La forza risultante è definita come la somma vettoriale di tutte le forze che agiscono su un corpo. $$
            \vec{F_{1\rightarrow c}} + \vec{F_{2\rightarrow c}} + \dots + \vec{F_{N\rightarrow c}} = \sum_{i=1}^{N} \vec{F_{i\rightarrow c}} = \vec{R_{\text{c}}}
        $$
        Dove $\vec{R_{\text{c}}}$ è la forza risultante che agisce sul corpo $c$, mentre $\vec{F_{i\rightarrow c}}$ è la forza che il corpo $i$ esercita sul corpo $c$.\newline
        Se un corpo fosse in quiete ovvero $\vec{S_i} = \vec{S_f}$ dunque $\vec{v_i} = \vec{v_f} = 0$ allora $\vec{a_{tot}} = \vec{0}$ e quindi $\vec{R_{\text{c}}} = \vec{0}$.
\section{Specificazione delle forze}
    \paragraph{Reazione vincolare} La reazione vincolare ($\vec{N}$) è la forza che un corpo esercita su un altro corpo per impedirne il moto. Questa forza è sempre perpendicolare alla superficie di contatto tra i due corpi e diretta verso l'interno del corpo. Questa forza è sempre uguale e di verso opposto alla componente normale della forza peso. Questa forza è detta vincolare in quanto è una forza che impedisce il moto del corpo.
    \paragraph{Forza Peso} La forza peso è la forza che la Terra esercita su un corpo, questa dipende direttamente dalla massa del corpo e dalla costante gravitazionale terrestre $g=9.81 [m/s^2]$. La forza peso è definita come segue: $$
        \vec{P} = m\vec{g}
    $$
    Dunque: $ \vec{F} = m_I \vec{a} = m \vec{g} \Rightarrow \vec{a} = \vec{g} $ possiamo semplificare la massa inerziale con la massa gravitazionale in quando la forza che la terra esercita sulla massa $\vec{F_{M\rightarrow m}} = -\frac{q_{12}q_{21}}{r^2_{12}} \hat{r_{12}} G$ dove $G$ è la costante gravitazionale universale. La formula appena scritta è la formula generale con $q_{12}$ e $q_{21}$ che sono le cariche dei due corpi (quanto un corpo è propenso a partecipare ad una interazione) e $r_{12}$ è la distanza tra i due corpi. Nel caso della forza peso $q_{12}q_{21} = m\cdot M$ dove $M$ è la massa della terra e $m$ è la massa del corpo. Dunque la formula diventa: $$
        \vec{F_{M\rightarrow m}} = -\frac{m\cdot M}{r^2} \hat{r} G
    $$ ora la distanza tra i due corpi è la somma dei raggi dei due corpi, quindi $r = R_{\text{terra}} + h$ dove $R_{\text{terra}}$ è il raggio della terra e $h$ è l'altezza del corpo rispetto al suolo, dato che queste due grandezze sono molto diverse ed $h$ è molto piccolo rispetto a $R_{\text{terra}}$ possiamo approssimare $r \approx R_T$ e quindi la formula diventa: $$
        \vec{F_{M\rightarrow m}} = -\frac{m\cdot M}{R^2} \hat{r} G
    $$
    e quindi $$
        \vec{F_{M\rightarrow m}} = -m\left(\frac{GM}{R^2}\right) \hat{r} = -m\vec{g}
    $$ In quanto questa è una forza allora in un sistema inerziale vale che $\vec{F}=m\vec{a}$ e quindi $m_I\vec{a} = -m\vec{g} \Rightarrow m_I = m$
    \paragraph{Forza Elastica}  
        La forza elastica è la forza che un corpo elastico esercita su un corpo che lo comprime o lo allunga. Questa forza è definita come segue: $$
            \begin{aligned}
                \vec{F_{\text{el}}} =& -k\vec{x}
                =& -k(\vec{x_f} - \vec{x_eq})
            \end{aligned}
        $$
        Dove $k$ è la costante elastica del corpo e $\vec{x}$ è la deformazione del corpo rispetto alla posizione di equilibrio ($\vec{x_{eq}}$). Questa forza è sempre diretta verso la posizione di equilibrio del corpo, dunque se il corpo è compresso la forza sarà diretta verso l'esterno, se il corpo è allungato la forza sarà diretta verso l'interno, in ogni caso il segno meno viene apposto in quanto è in opposizione alla deformazione.
    \paragraph{Forza di Attrito} La forza di attrito è la forza che si oppone al moto di un corpo su una superficie. Questa forza è definita come segue: $$
        \vec{F_{\text{att}}} = -\mu_s|\vec{N}|\cdot \hat{d}
    $$
    Dove $\mu_s$ è il coefficiente di attrito statico, $\vec{N}$ è la reazione vincolare (della quale ne consideriamo il modulo) e $\hat{d}$ è il versore della direzione del moto (dove il segno meno indica che la forza è in opposizione al moto).
    \subsection{Reazione Vincolare}
        Un corpo soggetto alla'azione di una forza, o della risultante non nulla di più forze, rimane fermo allora l'ambiente circostante provoca su quel corpo una forza uguale e di verso opposto alla risultante delle forze che agiscono su di esso, questo in quanto vale la terza legge di Newton. Questa forza è detta \textbf{reazione vincolare} nel caso di un corpo appoggiato su una superficie piana, la reazione vincolare è perpendicolare alla superficie di appoggio e diretta verso l'interno del corpo, opponendosi alla forza peso. La reazione vincolare si indica con $\vec{N}$ e se il corpo è in quiete allora $\vec{R} = \vec{N} + \vec{P} = \vec{0}$.
        \subsubsection{Senzazione di peso} 
            La reazione vincolare esercitata da un corpo, verso un altro corpo, (come un pavimento su una persona) è uguale e di verso opposto alla forza peso. È questa forza che dà la ``sensazione di peso'' e non la forza peso ($P$) in se. Quindi nel caso di un corpo grave appoggiato su una piattaforma orizzontale la quale si muove con accelerazione $a$ allora finché il corpo è appoggiato sulla piattaforma la reazione vincolare sarà $N + P = m\cdot a$, ovvero $N + m\cdot g = m\cdot a$ e quindi $N = m\cdot (a - g)$.\newline
            Questo comporta quattro situazioni differenti:
            \begin{enumerate}
                \item $a$ è discorde con $g$ e (piattaforma accelera verso l'alto) $N$ è maggiore di $P$ e quindi la persona avrà la sensazione di peso maggiore.
                \item $a$ è concorde con $g$ e (piattaforma accelera verso il basso) $N$ è minore di $P$ e quindi la persona avrà la sensazione di peso minore.
                \item $a$ è concorde e coincide esattamente con $g$ allora $N = P$ e quindi la persona avrà la sensazione di assenza di peso.
                \item $a$ è concorde con $g$ allora l'accelerazione sarà così elevata che il corpo ``non riesce a stare dietro'' e quindi ci sarà un distacco tra il corpo e la piattaforma.
            \end{enumerate}
    \subsection{Forza di attrito radente}
        Sperimentalmente si può osservare come se si provi a spingere una massa $m$ su un piano orizzontale, questa non si muoverà per la forza $F$ che si sta esercitando su di essa, ma si muoverà solo nel momento nel quale il modulo della forza $F$ supera un certo valore ($\mu_S N$), il coefficiente $mu_S$ è il coefficiente di attrito statico e questo è dipendente dalle due superfici che si stanno sfregando. Dunque:
        $$
            \begin{cases}
                F\leq \mu_S N & \text{condizione di quiete} \\
                F > \mu_S N & \text{condizione di moto}
            \end{cases}
        $$
        Nel caso in cui il corpo non si muova, la forza di attrito sarà uguale alla forza che si sta esercitando sul corpo, in quanto il corpo non si muove e quindi la forza risultante è nulla. Dunque $F = \mu_S N$ e quindi $N = \frac{F}{\mu_S}$, inoltre la forza di attrito è sempre diretta in opposizione al moto e quindi:
        $$
            R + F + P = 0
        $$
        Chiamiamo $A_{\text{max}}$ il valore massimo di $F$ che possiamo esercitare sul corpo affinché questo non si muova, allora $A_{\text{max}} = \mu_S N$, dove $N$ è la reazione vincolare normale al piano.
        Possiamo quindi disegnare un grafico della forza di attrito in funzione della forza applicata:
        \begin{figure}[H]
            \centering
            \begin{tikzpicture}
                \begin{axis}[
                    axis lines = left,
                    xlabel = $F$,
                    ylabel = $A$,
                    xmin = 0, xmax = 10,
                    ymin = 0, ymax = 10,
                    ytick = {0, 5},
                    yticklabels = {$0$, $A_{\text{max}}$},
                    xtick = {0, 5},
                    xticklabels = {$0$, $F = \mu_S N$},
                ]
                \addplot [
                    domain=0:5, 
                    samples=50, 
                    color=red,
                ]
                {x};
                \addplot [
                    domain=0:10, 
                    samples=100, 
                    color=blue,
                    style=dashed,
                    ]
                {5};
                \end{axis}
            \end{tikzpicture}
            \caption{Grafico della forza di attrito in funzione della forza applicata}
            \label{fig:attritoStatico}
        \end{figure}
        In figura \ref{fig:attritoStatico} possiamo vedere come la forza di attrito sia uguale alla forza applicata fino a quando questa non supera il valore massimo di attrito statico, in quel momento il corpo inizia a muoversi, quando questo accade continua ad esserci attrito ma il corpo si muove e quindi la forza di attrito diventa dinamica.\newline
        La forza che si oppone al movimento di un corpo mentre questo si muove su una superficie è detta forza di attrito radente dinamico, questa forza è definita come segue:
        $$
            F_{\text{ad}} = -\mu_d N = ma
        $$
        Dove $\mu_d$ è il coefficiente di attrito dinamico, $N$ è la reazione vincolare. Dalla definizione notiamo come la forza di attrito dinamico sia sempre diretta in opposizione al moto ed \underline{non è affetta dalla velocità del corpo}, possiamo quindi aggiungere al grafico della Figura \ref{fig:attritoStatico} la forza di attrito dinamico:
        \begin{figure}[H]
            \centering
            \begin{tikzpicture}
                \begin{axis}[
                    axis lines = left,
                    xlabel = $F$,
                    ylabel = $A$,
                    xmin = 0, xmax = 10,
                    ymin = 0, ymax = 10,
                    ytick = {0, 4.75, 5},
                    yticklabels = {$0$, $A_{\text{d}}$, $A_{\text{max}}$},
                    xtick = {0, 5},
                    xticklabels = {$0$, $F = \mu_S N$},
                ]
                \addplot [
                    domain=0:5, 
                    samples=50, 
                    color=red,
                ]
                {x};
                \addplot [
                    domain=0:10, 
                    samples=100, 
                    color=blue,
                    style=dashed,
                    ]
                {5};
                \addplot[
                    domain=5:5.25,
                    samples=50,
                    color=red,
                ]{1/((x-3.95)^21+1)+4.75};
                \addplot [
                    domain=5.25:10, 
                    samples=50, 
                    color=red,
                    ]
                {4.75};
                \end{axis}
            \end{tikzpicture}
            \caption{Grafico della forza di attrito in funzione della forza applicata}
            \label{fig:attritoDinamico}
        \end{figure}
        Notiamo come il valore di $A_{\text{d}}$ sia minore di $A_{\text{max}}$ in è vero per ogni superficie che il coefficiente di attrito dinamico sia minore di quello statico ($\mu_d < \mu_S$), altrimenti il corpo non si muoverebbe.
    \subsection{Piano Inclinato}
        Se prendiamo in considerazione un corpo di massa $m$ assimilabile ad un punto materiale e soggetto alle forze peso, reazione vincolare e forza di attrito radente, se questo è appoggiato su una superficie inclinata di angolo $\theta$ rispetto all'orizzontale, allora se è la sola forza peso che agisce sul corpo, allora possiamo scrivere la seguente equazione:
        $$
            P + R = m\cdot a
        $$
        Dove $P$ è la forza peso, $R$ è la reazione vincolare e $a$ è l'accelerazione del corpo, la reazione vincolare agisce solo nella direzione normale alla superficie, quindi possiamo scomporre questo come:
        $$
            \begin{cases}
                -mg\cos \theta + N = 0 & \text{direzione normale} \\
                -mg\sin \theta = m\cdot a & \text{direzione parallela}
            \end{cases}
        $$
        Da queste due equazioni possiamo notare come tutta l'eventuale accelerazione del corpo sia diretta lungo la superficie inclinata.\footnote{Ancora non abbiamo preso in considerazione la forza di attrito radente}
        \begin{figure}[H]
            \centering
            % Figura di un corpo su un piano inclinato (triangolo rettangolo (angolo retto in basso a sinistra)  con la punta di angolo in basso a destra) con la forza peso, la suddivisione delle forze peso in due componenti, la reazione vincolare, il corpo è un rettangolo appoggiato sul piano inclinato. Gli angoli tra le componenti della forza peso e il piano inclinato devono essere indicati.
            \begin{tikzpicture}
                % Piano inclinato
                \draw (0,0) -- (8,0) -- (0,2) -- cycle;
                % Angoli del piano inclinato
                \draw (6.5,0) arc (180:166.15:1.5);
                \node at (6.75,0.25) {$\theta$};

                % Corpo appoggiato sul piano inclinato
                \draw[fill=gray, rotate around={-14.04:(4,1)}] (3.5,2) rectangle (4.5,1);
                % Forza peso
                \draw[->] (4,1.5) -- (4,0.5) node[anchor=east] {$P$};
                % Componenti della forza peso
                \draw[->] (4,1.5) -- (3.75,0.6) node[anchor=east] {$mg\cos \theta$};
                \draw[->] (4,1.5) -- (4.25,1.425) node[anchor=west] {$mg\sin \theta$};
                % Linee tratteggiate di supporto
                \draw[dashed] (4,0.5) -- (3.75,0.6);
                \draw[dashed] (4,0.5) -- (4.25,1.425);
                % Reazione vincolare
                \draw[->] (4,1.5) -- (4.25,2.4) node[anchor=west] {$N$};
            \end{tikzpicture}
            \caption{Forze che agiscono su un corpo su un piano inclinato}
            \label{fig:pianoInclinato}
        \end{figure}
        Ora, prendendo in considerazione anche la forza di attrito radente, distinguiamo due principali casi:
        \begin{itemize}
            \item Il corpo è in stasi rispetto al piano inclinato
            \item Il corpo sta scendendo (frenato) rispetto al piano inclinato
        \end{itemize}
        Nel primo caso ciò significa che la forza di attrito statico è uguale alla componente parallela della forza peso, quindi $F_{\text{att}} = \mu_S N = mg\sin \theta$ ed inoltre questa non eccede il valore massimo di attrito statico, quindi $\mu_S N \leq \mu_S mg\cos \theta$, dunque $mg\sin \theta = \mu_S mg\cos \theta$ e quindi $ mg(\sin \theta - \mu_S \cos \theta) = 0$ e quindi $\mu_S \geq \tan \theta$.\newline
        Nel secondo caso, il corpo sta scendendo rispetto al piano inclinato, quindi viene esercitata una forza di attrito dinamico, questa forza è minore della componente parallela della forza peso, il corpo scende con accelerazione $a$ e quindi $mg\sin \theta - F_{\text{att}} = m\cdot a$ e quindi $mg\sin \theta - \mu_d N = m\cdot a$ e quindi $mg\sin \theta - \mu_d mg\cos \theta = m\cdot a$ e quindi $a = g(\sin \theta - \mu_d \cos \theta)$.
    \chapter{Lavoro, Energia e Momenti}
L'obbiettivo di questo capitolo è quello di definire il lavoro e l'energia, e di stabilire la loro relazione, andando anche ad analizzare il concetto di momento ed analizzando alcuni esempi pratici e problemi di fisica.

\section{Lavoro, Potenza ed Energia Cinetica}
    \subsection{Lavoro}
        Il lavoro $W$ è definito come l'integrale in un certo percorso $\gamma$ nel quale la forza $\vec{F}$ agisce su un corpo di massa $m$:
        \begin{equation}
            W \stackrel{\text{def}}{=} \int_{\gamma} \vec{F} \cdot d\vec{l}
        \end{equation}
        dove $d\vec{l}$ è un elemento infinitesimo del percorso $\gamma$ e $\cdot$ indica il prodotto scalare. Il lavoro è una grandezza scalare, quindi non ha direzione, ma ha un valore numerico che può essere positivo o negativo a seconda della direzione della forza rispetto al movimento del corpo.\newline
        Prendiamo in considerazione una forza $\vec{F}$ applicata nella direzione $d$ di un corpo di massa $m$ appoggiato su un piano orizzontale, e che si muove di una distanza $d$ lungo la direzione della forza. Il lavoro compiuto dalla forza $\vec{F}$ è dato da:
        $$
            \begin{aligned}
                W_{Fd} =& \int_{\gamma} \vec{F} \cdot d\vec{l}\\
                =& \int_{\gamma} F\hat{x}\cdot (dx \hat{x})\\
                =& \int_{0}^{d} F dx\\
                =& Fd \qquad [N\cdot m] = [J]
            \end{aligned}
        $$
        dove $F$ è la forza applicata, $d$ è la distanza percorsa dal corpo e $\hat{x}$ è il versore della direzione della forza. Il lavoro compiuto dalla forza $\vec{F}$ è quindi dato dal prodotto della forza per la distanza percorsa dal corpo nella direzione della forza.\newline
        Ora se la forza applicata non fosse parallela alla direzione del movimento, ma formasse un angolo $\theta$ con la direzione del movimento, il lavoro compiuto dalla forza $\vec{F}$ sarebbe dato da:
        $$
            \begin{aligned}
                W_{Fd} =& \int_{\gamma} \vec{F} \cdot d\vec{l}\\
                =& \int_{\gamma} F\hat{l}\cdot (d\vec{x})\\
                =& Fd(\hat{l}\cdot \hat{x})\\
                =& Fd\cos(\theta)
            \end{aligned}
        $$
        Da questa ultima possiamo distinguere tre casi:
        \begin{itemize}
            \item $0\leq \theta < \frac{\pi}2 \Rightarrow \cos \theta > 0$ Allora $W_{Fd} > 0$ e la forza $\vec{F}$ compie \textbf{Lavoro Motore} sul corpo.
            \item $\theta = \frac{\pi}2 \Rightarrow \cos \theta = 0$ Allora $W_{Fd} = 0$ e la forza $\vec{F}$ non compie lavoro sul corpo.
            \item $\frac{\pi}2 < \theta \leq \pi \Rightarrow \cos \theta < 0$ Allora $W_{Fd} < 0$ e la forza $\vec{F}$ compie \textbf{Lavoro Frenante} sul corpo.
        \end{itemize}
        \subsubsection{Problema 1 - Massa che scivola su un piano inclinato}
            Consideriamo una massa $m$ che scivola su un piano inclinato di angolo $\alpha$ rispetto all'orizzontale.\newline
            In questo caso la forza peso $\vec{P}$ esegue un lavoro positivo sulla massa, non consideriamo al momento l'attrito, quindi la forza peso che spinge la massa lungo il piano inclinato è data da:
            $$
                \vec{P} = m\vec{g}
            $$
            ma in quanto l'asse $z$ sul quale è applicata la forza peso e misurata l'altezza $h$ forma un angolo $\theta=\pi-\left(\frac{\pi}2-\alpha\right) = \frac{pi}{2}+\alpha$ con l'asse $x$, il l'infinitesimo lavoro compiuto dalla forza peso $\vec{P}$ è dato da:
            $$
                \begin{aligned}
                    dw=& F_{P} \cdot \hat{l} \\
                    =& -mg\hat{z} \cdot (di\cdot \hat{i})\\
                    =& -mg\ di\ \hat{z} \cdot \hat{i}\\
                    =& -mg\ di\ \cos(\theta)\\
                    =& +mg\ di\ \sin(\alpha)\\
                    =& P_{\parallel} di\\
                \end{aligned}
            $$
            dove $P_{\parallel}$ è la componente della forza peso $\vec{P}$ lungo il piano inclinato. Dunque il lavoro compiuto dalla forza peso $\vec{P}$ è dato da:
            $$
                \begin{aligned}
                    W_{P} =& \int_{\gamma} P_{\parallel} dw\\
                    =& mg\sin(\alpha) L\\
                    =& mg h
                \end{aligned}
            $$
            dove $L$ è la lunghezza del piano inclinato e $h$ è l'altezza della massa rispetto al piano orizzontale, abbiamo sostituito $L$ con $h$ in quanto la lunghezza del piano inclinato moltiplicata per il seno dell'angolo $\alpha$, che è l'opposto del triangolo rettangolo, è uguale all'altezza $h$ della massa rispetto al piano orizzontale. Quindi il lavoro compiuto dalla forza peso $\vec{P}$ sulla massa $m$ è dato dal prodotto della forza peso per l'altezza della massa rispetto al piano orizzontale, dunque il lavoro compiuto dalla forza peso $\vec{P}$ sulla massa $m$ è \underline{indipendente} dall'angolo $\alpha$ del piano inclinato, ma dipende esclusivamente dall'altezza $h$ della massa rispetto al piano orizzontale.
            \begin{figure}[H]
                \centering
                \begin{subfigure}[b]{0.45\textwidth}
                    \begin{tikzpicture}
                        % Disegna il piano inclinato con angolo retto in basso a sinistra
                        \draw[thick] (0,0) -- (5,0);
                        \draw[thick] (0,0) -- (0,3);
                        \draw[thick] (0,3) -- (5,0);
                        
                        % Massa appoggiata in diagonale sul piano inclinato
                        \draw[fill=gray, rotate around={-30:(2.5,1.5)}] (2.3,1.5) rectangle (2.7,1.9);
                        \node at (2.5,1.6) {$m$};
                        
                        % Angolo
                        \draw (4.2,0) arc (180:150:0.8) node[right] {$\alpha$};
                        
                        % Forza peso
                        \draw[->, thick, red] (2.6,1.7) --++ (0,-1.5) node[right] {$\vec{P} = mg$};
                        
                        % Componenti della forza peso
                        \draw[->, thick, blue] (2.6,1.7) --++ (1,-0.6) node[below] {$P_{\parallel}$};
                        \draw[->, thick, blue] (2.6,1.7) --++ (-0.7,-0.9) node[left] {$P_{\perp}$};
                    \end{tikzpicture}
                \end{subfigure}
                \begin{subfigure}[b]{0.45\textwidth}
                    \begin{tikzpicture}
                        % Asse di riferimento inclinato
                        \draw[thick,->] (0,0) -- (0,3) node[above] {$z$};
                        \draw[thick,->] (0,0) -- (2,-1.154) node[below] {$x$};
                        \draw[thick, dashed] (0,0) -- (0,-1);
                        \draw[thick, dashed] (0,0) -- (-1,0.5777);
                        \draw (0,0.5) arc (90:-30:0.5) node[right] {$\theta = \frac{\pi}{2}+\alpha$};
                        \draw (0,-0.75) arc (270:330:0.75) node[left] {$\pi-\alpha$};
                    \end{tikzpicture}
                \end{subfigure}
                \caption{Massa su piano inclinato e rappresentazione del sistema di riferimento.}
            \end{figure}
        \subsubsection{Problema 2 - Massa su piano inclinato con attrito}
            Consideriamo una massa $m$ che scivola su un piano inclinato di angolo $\alpha$ rispetto all'orizzontale, e che è soggetta ad un attrito dinamico di coefficiente $\mu_d$.\newline
            In questo caso la forza peso $\vec{P}$ esegue un lavoro positivo sulla massa, mentre la forza di attrito $\vec{F}_{attr}$ esegue un lavoro negativo sulla massa. La forza di attrito è data da:
            $$
                \vec{A_d}= -\mu_d \left|\vec{N}\right|\hat{v}
            $$
            dove $\vec{N}$ è la forza normale, che in questo caso è data dalla componente della forza peso $\vec{P}$ perpendicolare al piano inclinato, e $\hat{v}$ è il versore della direzione del movimento della massa. L'infinitesimo lavoro compiuto dalla forza di attrito $\vec{A_d}$ è dato da:
            $$
                \begin{aligned}
                    dw=& F_{A_d} \cdot d\hat{l} \\
                    =& -\mu_d \left|\vec{N}\right|\hat{v} \cdot (dl\cdot \hat{v})\\
                    =& -\mu_d mg\cos (\alpha) dl\\
                \end{aligned}
            $$
            dove $dl$ è l'infinitesimo spostamento della massa lungo il piano inclinato. Dunque il lavoro compiuto dalla forza di attrito $W_{A_d}$ è dato da:
            $$
                \begin{aligned}
                    W_{A_d} =& \int_{\gamma} A_d dw\\
                    =& -\mu_d mg\cos(\alpha)\int_0^L dl\\
                    =& -\mu_d mg\cos(\alpha) L\\
                \end{aligned}
            $$
            dove $L$ è la lunghezza del piano inclinato. Quindi il lavoro compiuto dalla forza di attrito $\vec{A_d}$ sulla massa $m$ è dato dal prodotto della forza di attrito per la lunghezza del piano inclinato, dunque il lavoro compiuto dalla forza di attrito $\vec{A_d}$ sulla massa $m$ è \underline{dipendente}, al contrario della forza peso $\vec{P}$, dall'angolo $\alpha$ del piano inclinato.
        \subsection{Teorema delle forze vive}
            Il teorema delle forze vive, nei sistemi inerziali, afferma che la somma dei lavori compiuti dalle forze agenti su un corpo è uguale alla variazione dell'energia cinetica del corpo stesso. In formula:
            \begin{equation}
                \sum W_{i} = \Delta K
            \end{equation}
            Questo è dimostrabile in quanto
            $$
                \begin{aligned}
                    \vec{F} =& m\vec{a} \qquad & \text{Seconda Legge di Newton}\\
                    =& m\frac{d\vec{v}}{dt} \qquad & \text{Definizione di accelerazione}\\
                    \vec{F}d\vec{s} =& m\frac{d\vec{v}}{dt}d\vec{s}\\
                    =& m(d\vec{v})\frac{d\vec{s}}{dt}\\
                    =& m(d\vec{v})\vec{v}\\
                \end{aligned}
            $$
            dunque da questa otteniamo che la somma delle forze agenti su un corpo moltiplicata per l'infinitesimo spostamento del corpo è uguale alla variazione della velocità del corpo moltiplicata per la velocità del corpo stesso. Possiamo però esprimere $\vec{v}d\vec{v}$ come:
            $$
                \begin{aligned}
                    \vec{v}d\vec{v} =& v_xdv_x + v_ydv_y + v_zdv_z\\
                    =&d\left[\frac{v_x^2}2\right] + d\left[\frac{v_y^2}2\right] + d\left[\frac{v_z^2}2\right]\\
                    =& \frac12d\left[v_x^2 + v_y^2 + v_z^2\right]\\
                    =& \frac12d\left[v^2\right]\\
                \end{aligned}
            $$
            andando quindi a unire le due equazioni otteniamo:
            $$
                \begin{aligned}
                    \vec{F}d\vec{s} =& \frac12m d\left[v^2\right]\\
                    \downarrow & \qquad \text{Integrando}\\
                    \int_i^f dw =& \int_i^f \frac12m d\left[v^2\right]
                \end{aligned}
            $$
            e dunque per descrivere il lavoro esercitato su un corpo da una forza $\vec{F}$ dal punto $i$ al punto $f$ otteniamo:
            \begin{align}
                W_{F,i\to f} =& \frac12mv_f^2-\frac12mv_i^2 = \Delta E_K \\
                E_K \stackrel{def}{=}& \frac12mv^2
            \end{align}
            dove $E_K$ è l'energia cinetica del corpo, che è definita come la metà del prodotto della massa del corpo per il quadrato della sua velocità. L'energia cinetica è una grandezza scalare, quindi non ha direzione, ma ha un valore numerico che può essere positivo o negativo a seconda della direzione della velocità del corpo, questa è espressa in Joule $[J]$ ed ha dunque dimensionalità del lavoro $[J] = [N\cdot m] = [kg\cdot m^2/s^2]$.
        \subsection{Forze Conservative}
            Le forze conservative sono forze che compiono lavoro indipendentemente dal percorso seguito dal corpo, ma solo dalla posizione iniziale e finale del corpo stesso, ad esempio la forza peso $\vec{P}$ è una forza conservativa, mentre la forza di attrito $\vec{A_d}$ non lo è. Visto che dipendono solo dalla posizione iniziale e finale del corpo, possiamo scrivere le seguenti relazioni:
            \begin{align}
                \vec{F}|_{W_{i\to f}} =& \int_i^f \vec{F}\cdot d\vec{s} \label{eq:forzaConservativa}
            \end{align}
            Ovvero il lavoro compiuto dalla forza $\vec{F}$ su un corpo che si sposta da una posizione $i$ ad una posizione $f$ è indipendente dal percorso ma solo dalla posizione iniziale e finale del corpo stesso.\newline
            Questo ci permette di scrivere:
            \begin{align}
                \oint \vec{F}\cdot d\vec{s} =& 0 \label{eq:forzaConservativa2}
            \end{align}
            considerando dunque l'Equazione \ref{eq:forzaConservativa} e l'Equazione \ref{eq:forzaConservativa2} le quali sono equivalenti e semplicemente dimostrabili usando le proprietà degli integrali, possiamo definire il lavoro di una forza conservativa come:
            \begin{align}
                W_{F,i\to f} =& \mathcal{F} \label{eq:forzaConservativa3}
            \end{align}
            Se l'Equazione \ref{eq:forzaConservativa3} non risultasse verificata allora la forza $\vec{F}$ non sarebbe conservativa, e quindi il lavoro compiuto dalla forza $\vec{F}$ su un corpo che si sposta da una posizione $i$ ad una posizione $f$ dipenderebbe dal percorso seguito dal corpo stesso.
        \subsection{Energia Potenziale}
            Prendiamo in considerazione una forza conservativa $\vec{F}$, dunque il lavoro compiuto dalla forza $\vec{F}$ su un corpo che si sposta da una posizione $i$ ad una posizione $f$ è indipendente dal percorso ma solo dalla posizione iniziale e finale del corpo stesso. Fissiamo ora il punto $i\rightarrow O$ come punto di originario, e il punto $f\rightarrow P$ come punto finale del corpo stesso, quindi possiamo scrivere:
            \begin{equation}
                E_{\rho}(x,y,z)=E_{\rho,P}=-\int_{O}^{P} \vec{F}\cdot d\vec{s} \label{eq:energiaPotenziale}
            \end{equation}
            abbiamo dunque definito l'\textbf{energia potenziale} $E_{\rho}$ del punto $P$ come il lavoro compiuto dalla forza $\vec{F}$ su un corpo che si sposta dal punto $O$ al punto $P$. La funzione \ref{eq:energiaPotenziale} è definita \textbf{funzione potenziale} della forza $\vec{F}$, questa descrive l'energia potenziale del corpo in funzione della sua posizione nello spazio ($x,y,z$). L'energia potenziale è una grandezza scalare, quindi non ha direzione, ma ha un valore numerico che può essere positivo o negativo a seconda della posizione del corpo nello spazio. L'energia potenziale è espressa in Joule $[J]$ ed ha dunque dimensionalità del lavoro $[J] = [N\cdot m] = [kg\cdot m^2/s^2]$.
            \paragraph{Proprietà} Se al posto di considerare il punto $O$ come punto di origine, considerassimo un altro punto $A$ ed un altro punto $B$, allora il lavoro compiuto dalla forza $\vec{F}$ su un corpo che si sposta dal punto $A$ al punto $B$ sarebbe dato da:
            $$
                \begin{aligned}
                    W_{F,A\to B} =& \int_{A}^{B} \vec{F}\cdot d\vec{s}\\
                    =& \int_{A}^{O} \vec{F}\cdot d\vec{s} + \int_{O}^{B} \vec{F}\cdot d\vec{s}\\
                    =& -\int_{O}^{A} \vec{F}\cdot d\vec{s} - \int_{O}^{B} \vec{F}\cdot d\vec{s}
                \end{aligned}
            $$
            Da questa ricaviamo che:
            \begin{equation}
                W_{F,A\to B}=E_{\rho, A} - E_{\rho,B} = \Delta E_{\rho} \label{eq:energiaPotenziale2}
            \end{equation}
            Da notare come nel calcolo dell'energia potenziale per qualunque punto $O$ scelto come origine non influisce sul risultato dell'Equazione \ref{eq:energiaPotenziale2}. Anche in questa si nota come per le forze conservative se il punto di inizio e il punto di fine sono gli stessi, e si vuole quindi calcolare il lavoro compiuto, questo risulterà essere nullo per l'Equazione \ref{eq:forzaConservativa2}, ciò in quanto se considerassimo un qualsiasi punto intermedio $M$ nel percorso che inizia e termina nello stesso punto $A$ allora ``quello che guadagna il corpo nel percorso $A\to M$ lo perde nel percorso $M\to A$'', dunque l+integrale chiuso risulterebbe essere nullo.
            \subsubsection{Energia potenziale forza peso}
                Applicando l'Equazione \ref{eq:energiaPotenziale} alla forza peso $\vec{P}$, otteniamo:
                \begin{equation}
                    E_{\rho, P} = mgz.
                \end{equation}
                Dove $z$ è l'altezza del corpo rispetto al piano orizzontale, e $m$ è la massa del corpo. L'energia potenziale della forza peso è quindi direttamente proporzionale all'altezza del corpo rispetto al piano orizzontale, e alla massa del corpo stesso. Il segno dell'energia potenziale della forza peso è positivo, in quanto la forza peso compie lavoro positivo sul corpo quando questo si sposta verso l'alto, e negativo quando il corpo si sposta verso il basso.

    \backmatter
    \appendix 
\renewcommand{\thesection}{A.\arabic{section}}
\chapter{Appendice A: Note delle lezioni}
Di seguito sono riportate delle note delle lezioni ulteriori agli appunti stessi del corso.

\section{24 febbraio 2025}
\label{lez:24-02-2025}
Le tre regole del grafico orario:
\begin{itemize}
    \item Il tempo non si ferma;
    \item Il tempo scorre sempre allo stesso, uniforme, modo per tutti;
    \item Non si può andare più veloce della luce ($c$), non possono esistere dunque rette con pendenza maggiore di $c$.
    \item Non esiste ancora il teletrasporto.
\end{itemize}
\section{26 febbraio 2025}
\label{lez:26-02-2025}
Si noti come lo scopo del problema non fosse strettamente quello di trovare il punto nel quale il punto si ferma, ma di capire come l'analisi dimensionale possa aiutarci a risolvere un problema fisico.

\section{31 marzo 2025}
\label{lez:31-03-2025}
La scrittura delle Sotto-Sezione \ref{subsec:conservazioneEnergiaMeccanica} e il paragrafo riguardante l'energia potenziale della forza peso della Sotto-Sezione \ref{subsec:energiaPotenziale} non sono state scritte integrando gli appunti presi a lezione, ma interamente dal libro, ciò a causa di una mia impossibilità a prendere parte all'ultima parte della lezione svoltasi in data sopra indicata.
\end{document}