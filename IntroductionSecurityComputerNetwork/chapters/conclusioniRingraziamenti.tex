\chapter*{Conclusioni e ringraziamenti}
\thispagestyle{chapterInit}
\addcontentsline{toc}{chapter}{Conclusioni e ringraziamenti}

\section*{Conclusioni}
    In questo documento sono stati trattati i principali argomenti riguardanti il corso di ``\textit{Introduction to computer and network security}'' del professore Ranise Silvio, tenuto presso l'Università degli Studi di Trento nell'anno accademico 2024/2025. Il tutto nell'ambito del secondo anno del corso di laurea triennale in Informatica.
    Durante il corso sono stati affrontati i principali argomenti riguardanti la sicurezza informatica, partendo dalle basi della crittografia fino ad arrivare alle principali tecniche di sicurezza informatica. Inoltre sono stati trattati anche argomenti riguardanti la sicurezza delle reti, partendo dalle basi delle reti di calcolatori fino ad arrivare alle principali tecniche di sicurezza delle reti.\newline 
    Alcuni argomenti e/o sezioni potrebbero essere state omesse o trattate in modo superficiale, per approfondire tali argomenti si consiglia di consultare la bibliografia del corso e/o il materiale direttamente fornito dal professore.

\section*{Ringraziamenti}
    Voglio ringraziare in primo luogo il professore Ranise Silvio per il materiale e le lezioni che ha tenuto durante il corso, per l'interesse e per l'apertura al dialogo e alla discussione che ha dimostrato durante il corso.\newline
    Ringrazio inoltre i miei colleghi di corso per le discussioni e le collaborazioni che ci sono state durante il corso, è grazie anche a loro che ho trovato il tempo e la dedizione per scrivere questi appunti.\newline
    Infine ringrazio chi ha letto o sta leggendo in questo momento questi appunti, spero che siano stati utili come materiale di supporto allo studio e che abbiano aiutato a chiarire e a comprendere meglio gli argomenti trattati durante il corso, e che possano essere utili anche a chi leggerà in futuro.

\vfill
{\footnotesize
    \footnotesize\section*{Note}
    Questi appunti sono stati scritti durante il corso di ``\textit{Introduction to computer and network security}'', tenuto dal prof. Ranise Silvio presso l'Università degli Studi di Trento nell'anno accademico 2024/2025. Gli appunti sono stati scritti in \LaTeX{} e sono disponibili su \href{https://github.com/lucafano04/appuntisecondoanno}{GitHub} e sono rilasciati sotto licenza \href{https://creativecommons.org/licenses/by-nc-sa/4.0/}{CC BY-NC-SA 4.0} come conseguenza sono liberamente utilizzabili e modificabili, ma non possono essere utilizzati a scopi commerciali e devono mantenere la stessa licenza, il materiale rimane liberamente usabile e modificabile nell'ambito accademico, della formazione e della divulgazione scientifica e tecnologica. L'utilizzatore è tenuto a citare l'autore originale e a mantenere la stessa licenza per le opere derivate. Ognuno è libero di usare questi come punto di partenza per lo studio in funzione delle proprie esigenze e di condividerli con chiunque ne possa trarre beneficio, anzi è incoraggiato a farlo.
    L'autore (Luca Facchini) non si assume nessuna responsabilità sull'uso che verrà fatto di questi appunti e non garantisce la completa correttezza e completezza degli stessi, inoltre non si assume nessuna responsabilità per eventuali errori o imprecisioni presenti negli appunti, questi vengono infatti distribuiti \textit{as is} e possono contenere errori o imprecisioni, l'utilizzatore è tenuto a verificare e a correggere eventuali errori presenti negli appunti. Nell'eventualità di errori o imprecisioni si prega di contattare l'autore e/o di aprire una \textit{issue} sul repository di GitHub. (Ultimo aggiornamento: \today)
}