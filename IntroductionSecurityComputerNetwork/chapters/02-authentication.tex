\chapter{Authentication I: Passwords \& co}
\thispagestyle{chapterInit}
\section{User Authentication \& Digital Identity - Autenticazione e Identità Digitale}
    \subsection{Introduzione e Definizioni}
        \begin{description}
            \item[Identità] è un insieme di attributi relazionati ad una entità
            \item[Attributo] è una caratteristica o proprietà di una entità ch e può essere usata per descriverne lo stato, apparenza o ogni altro aspetto.  
            \item[Identità digitale] è ua identità i quali attributi sono conservati e trasmessi in forma digitale 
        \end{description}
        Ma perché la identità digitale è importante? In pratica questa può essere una soluzione per diversi aspetti che i governi mondiali vorrebbero risolvere.
    \subsection{Digital Identity Lifecycle - Ciclo di Vita dell'Identità Digitale}
        \begin{description}
            \item[Creazione] L'identità viene creata, in genere da un ente di certificazione che assicura l'identità dell'utente e fornisce/chiede una forma di accesso per riconoscere in futuro l'utente. Viene creato un ID univoco per l'utente
            \item[Autenticazione] L'identità viene autenticata tramite il metodo concordato in fase di creazione
            \item[Autorizzazione/ Accesso] L'utente ha accesso a risorse e servizi in base ai permessi assegnati, i dati necessari dell'utente vengono trasmessi dall'ente di certificazione al servizio.
            \item[Cancellazione o Disattivazione] L'identità viene cancellata o disattivata quando non è più necessaria e i dati devono essere cancellati dopo un certo periodo di tempo per garantire la privacy dell'utente.
        \end{description}
        \subsubsection{Enrollment / On-Boarding - Creazione}
            La fase di creazione di una utenza porta un \textbf{Applicant} a diventare un \textbf{Subscriber} tramite una serie di passaggi:
            \begin{description}
                \item[Resolution] Vengono raccolti gli \textbf{attributi} essenziali dell'utente (nome, indirizzo, data e luogo di nascita, \dots) vengono ora raccolti anche due documenti di identità (patente, passaporto, C.I.,\dots) - ora l'utente viene distinto univocamente su un contesto
                \item[Validation] Gli \textbf{attributi} raccolti vengono validati sulla base di \textbf{prove} tramite una fonte autorevole- gli attributi vengono ora associati ad una persona fisica 
                \item[Verification] Le \textbf{prove} vengono verificate, in questo punto si verifica la corrispondenza tra le diverse foto, viene mandato un codice ai contatti per verificare che siano i suoi, etc - l'utente viene ora confermato e la sua identità è certificata 
            \end{description}
            \paragraph{Identity Assurance Levels - Livelli di Assicurazione dell'Identità}
                \begin{description}
                    \item[IAL1] GLi attributi, se presenti, vengono auto-dichiarati, o considerati come tali
                    \item[IAL2] Una persona o di remoto o in presenza, verifica gli attributi
                    \item[IAL3] Gli attributi vengono verificati da una persona autorizzata e i documenti fisici vengono esaminati
                \end{description}  
        \subsubsection{Authentication - Autenticazione}
            \paragraph{Definzione} L'\textbf{autenticazione} è il processo di verifica dell'identità di un utente, processo o dispositivo.

            Il \textbf{richiedente} deve dimostrare al \textbf{verificante} che è chi dice di essere.
            \paragraph{Base} Quando esegui il login solitamente inserisci un \textbf{username} e una \textbf{password} che vengono confrontati con quelli memorizzati nel sistema.

            L'autenticazione tramite \textbf{password} è molto diffusa e semplice da implementare.
\section{An Introduction to Passwords - Introduzione alle Password}
    \subsection{User Authn - Autenticazione Utente}
        Quando esegui il login inserisci un \textbf{username} per annunciare chi sei e una \textbf{password} per dimostrare chi dici di essere, questo tipo di autenticazione si chiama \textbf{user authn} ovvero il processo di verifica dell'identità di un utente.
        \subsubsection{Entollment - Creazione}
            Le \textbf{password} dovrebbero essere conosciute solo dall'utente e dal sistema. Spesso però, soprattutto nelle grandi aziende, le password vengono mandata via mail o scritte dall'utente su una pagina web. In questi casi da chi potrebbe essere intercettata la password?
        \subsubsection{La teoria rispetto alla realtà}
            Le password hanno dominato il mondo dell'informatica sin dall'avvento dei computer, ma da allora la richiesta verso queste è che siano più sicure ma allo stesso tempo più "User Friendly", siamo in ricerca di alternative ma per ora delle proposte alternative come \textit{criteri di complessità} ma studi hanno rilevato che spesso non vengono rispettati e/o che non sono efficaci. 
    \subsection{Password Security - Sicurezza delle Password}
        All'inizio di internet le password venivano conservate in file appositi in chiaro, ma questo comportava un rischio molto alto, infatti se un attaccante riusciva ad accedere a quel file poteva facilmente accedere a tutte le password.

        \subsubsection{Hashing}
            Per risolvere questo problema si è pensato di "\textbf{hash-are}" le password, ovvero di applicare una funzione di hash alla password e conservare il risultato. In questo modo se un attaccante accede al file delle password non può risalire alla password originale.
            \paragraph{Proprietà della funzione di hash}
                Una funzione di hash ottimale dovrebbe avere le seguenti proprietà:
                \begin{description}
                    \item[Deterministica] dato un input la funzione restituisce sempre lo stesso output
                    \item[Rapida] la funzione deve essere veloce da calcolare
                    \item[Difficile da invertire] data l'output è difficile risalire all'input
                    \item[Biettiva] due input diversi devono avere output diversi
                    \item[Lunghezza fissa] l'output deve avere una lunghezza fissa indipendentemente dalla lunghezza dell'input
                \end{description}
                Molto spesso però capita che i bit disponibili per l'hash siano limitati rispetto a tutti i possibili input, per questo motivo si è cercato di creare funzioni di hash che generano due output uguali solo in rarissimi casi, inoltre si aggiunge a questo che per input simili l'output sia molto diverso.
            \subsubsection{Crack of Passwords}
                In caso di un "leak" di password che sono state opportunamente codificate con una funzione di hash, un attaccante può tentare di decifrarle ma questo è molto difficile. Questo diventa più semplice se vengono usate password poco sicure e/o comuni. infatti si stimi che per  trovare una password da 8 caratteri composti da lettere minuscole e numeri ci vogliano "solo" 155€ su una istanza di AWS.
        \subsection{Hash \& Salt}
            Per rendere più difficile il lavoro degli attaccanti si è pensato di aggiungere un \textbf{salt} alle password, ovvero un valore casuale che viene aggiunto alla password prima di calcolare l'hash. Questo rende più difficile per l'attaccante decifrare la password.
\section{Multi Factor Authentication - Autenticazione a più fattori}
    \subsection{Definizione}
        L'autenticazione a più fattori è un metodo di autenticazione che richiede l'uso di più metodi di autenticazione per verificare l'identità di un utente.
    \subsection{Fattori di Autenticazione}
        I fattori di autenticazione sono:
        \begin{description}
            \item[Qualcosa che sai] (password, PIN, \dots)
            \item[Qualcosa che hai] (smartphone, token, \dots)
            \item[Qualcosa che sei] (impronte digitali, riconoscimento facciale, \dots)
        \end{description}
        \subsubsection{Vantaggi/Svantaggi quello che sai}
            \paragraph{Vantaggi}
                \begin{itemize}
                    \item Facile da implementare - non richiede hardware aggiuntivo
                    \item Facile da usare - Basta ricordare la password
                    \item Facile da resettare se dimenticato - basta fare il reset della password
                \end{itemize}
            \paragraph{Svantaggi}
                \begin{itemize}
                    \item Facile da rubare - se un attaccante riesce a scoprire la password può accedere al sistema 
                    \item Facile da dimenticare - se la password è complessa è facile dimenticarla
                    \item Facile da indovinare - se la password è semplice ed è stata usata in altri contesti è facile indovinarla
                \end{itemize}
        \subsubsection{Vantaggi/Svantaggi quello che hai}
            \paragraph{Vantaggi}
                \begin{itemize}
                    \item Difficile da rubare - se un attaccante non ha il dispositivo non può accedere al sistema
                    \item Difficile da indovinare - se il dispositivo è protetto da PIN o password è difficile indovinare l'accesso
                    \item Difficile da clonare - se il dispositivo è protetto da un token è difficile clonarlo inoltre la parte di autenticazione è fatta dal dispositivo stesso e non dal sistema
                \end{itemize}
            \paragraph{Svantaggi}
                \begin{itemize}
                    \item Facile da perdere - se il dispositivo viene perso l'utente non può accedere al sistema
                    \item Difficile da resettare - se il dispositivo viene perso l'utente deve contattare l'amministratore per resettare l'accesso
                    \item Costoso - i dispositivi possono essere costosi 
                \end{itemize}
        \subsubsection{Vantaggi/Svantaggi quello che sei}
            \paragraph{Vantaggi}
                \begin{itemize}
                    \item Difficile da rubare - le impronte digitali o il riconoscimento facciale sono unici e difficili da rubare
                    \item Difficile da indovinare - è difficile indovinare le impronte digitali o il riconoscimento facciale
                    \item Difficile da clonare - è difficile clonare le impronte digitali o il riconoscimento facciale
                \end{itemize}
            \paragraph{Svantaggi}
                \begin{itemize}
                    \item Se il fattore viene compromesso non può essere cambiato - se le impronte digitali vengono rubate non possono essere cambiate
                    \item Costoso - i dispositivi possono essere costosi
                    \item Non sempre preciso - il riconoscimento facciale può essere ingannato
                \end{itemize}
        \subsubsection{PSD2 Compliance}
            \paragraph{Cos'è PSD2} La \textbf{PSD2} è una direttiva europea che regola i servizi di pagamento e che richiede l'autenticazione a più fattori per i pagamenti online. Questo per ridurre il rischio di frodi.
            \paragraph{How to comply MFA in PSD2} Una idea sarebbe quella di includere nella challenge anche i dati della particolare transazione come:
                \begin{itemize}
                    \item L'identificativo del destinatario della transazione
                    \item L'importo della transazione
                    \item L'instante nella quale la transazione è stata inizializzata
                    \item \dots
                \end{itemize}
            Purtroppo però ciò non è abbastanza in quanto i dati sopra indicati possono essere ricavati dal contesto il che rende la challenge prevedibile.
    \subsection{FIDO: Phishing Resistant Authentication}
        \subsubsection{Cos'è FIDO}
            \textbf{FIDO (Fast Identity Online)} è un insieme di standard aperti per l'autenticazione a più fattori che mira a ridurre il rischio di phishing. Il suo scopo è quello di assicurare una \textbf{autenticazione forte} e di \textbf{ridurre l'uso di password}.
        \subsubsection{Come funziona}
            \begin{enumerate}
                \item Viene chiesto all'utente di scegliere un ente FIDO
                \item L'utenti sblocca il dispositivo FIDO usando un'impronta digitale, un pulsante su un dispositivo di secondo fattore, un PIN o un qualsiasi altro metodo di autenticazione supportato
                \item Il dispositivo crea una chiave pubblica e una chiave privata univoca per il dispositivo, il servizio online e l'utente
                \item La chiave pubblica è inviata al servizio online ed associata all'account dell'utente
                \item Il servizio online chiede all'utente di autenticarsi con il dispositivo precedentemente registrato
                \item L'utente sblocca il dispositivo FIDO usando lo stesso metodo di autenticazione usato in precedenza
                \item Il dispositivo FIDO usa l'account dell'utente identificato per inviare la corretta chiave al servizio online
                \item Infine il dispositivo invia la challenge ricevuta dal servizio online firmata con la chiave privata e il servizio online verifica la firma con la chiave pubblica, l'utente è autenticato.
            \end{enumerate}
\section{Outsourcing Authentication}
    \subsection{Definizione}
        L'outsourcing dell'autenticazione è il processo di affidare l'autenticazione degli utenti ad un servizio esterno, spesso chiamato \textbf{Identity Provider}. Questo servizio si occupa di verificare l'identità dell'utente e di inviare un token di autenticazione al servizio che richiede l'autenticazione. Esempio in italia per l'accesso ai servizi pubblici si usa SPID.
    \subsection{Che problema risolve}
        L'outsourcing dell'autenticazione risolve diversi problemi:
        \begin{description}
            \item[Sicurezza] L'Identity Provider è specializzato in autenticazione e può offrire un livello di sicurezza più alto
            \item[User Experience] L'Identity Provider può offrire un'esperienza utente migliore in quanto l'utente non deve ricordare le password per ogni servizio ma solo quella dell'Identity Provider - SSO
            \item[Compliance] L'Identity Provider può aiutare a rispettare le normative sulla privacy e la sicurezza
        \end{description}
    \subsection{La soluzione}
        L'outsourcing dell'autenticazione è delegato ad una terza parte, l'\textbf{Identity Provider}, che si occupa di verificare l'identità dell'utente e di inviare le informazioni di autenticazione al servizio che richiede l'autenticazione. A questo punto il servizio invia all'utente un token di autenticazione che può essere usato per accedere al servizio senza dover inserire nuovamente le credenziali del SSO.
    \subsection{Potenziali Problemi}
        L'outsourcing dell'autenticazione può comportare alcuni problemi:
        \begin{description}
            \item[Single Point of Failure] Se l'Identity Provider è compromesso tutti i servizi che usano l'Identity Provider per l'autenticazione sono compromessi
            \item[Privacy] L'Identity Provider può raccogliere informazioni sull'utente e sulle sue abitudini di navigazione
        \end{description}