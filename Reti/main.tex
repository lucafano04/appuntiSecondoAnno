\documentclass[twoside]{report}
\usepackage[italian]{babel}
\usepackage[utf8]{inputenc}
\usepackage{amsmath}
\usepackage{amsthm}
\usepackage{amsfonts}
\usepackage{amssymb}
\usepackage{cancel}
\usepackage[margin=1in]{geometry}
\usepackage{hyperref}
\usepackage{bookmark}
\usepackage{setspace}
\usepackage{titlesec}
\usepackage{fancyhdr}
\usepackage{adjustbox}
\usepackage{float}
\usepackage{graphicx}
\usepackage{float}
\usepackage{algpseudocode}
\usepackage[linesnumbered,ruled,vlined]{algorithm2e}
\usepackage{xcolor}

\setlength{\parskip}{0pt}
\titlespacing*{\subparagraph}{1em}{0em}{0em} 

\makeatletter
\renewenvironment{abstract}{%
    \if@twocolumn
        \section*{\abstractname}%
    \else
        \begin{center}%
            {\bfseries \abstractname\vspace{-.5em}\vspace{\z@}}%
        \end{center}%
        \small
        \begin{quotation}
    \fi}
    {\if@twocolumn\else\end{quotation}\fi}
\makeatother

\hypersetup{
    pdfauthor={Luca Facchini},
    pdftitle={Appunti di Reti},
    pdfsubject={Appunti del corso di Reti, tenuto dal prof. Casari Paolo presso l'Università degli Studi di Trento. Corso seguito nell'anno accademico 2024/2025.},
    pdfkeywords={Reti, Università degli Studi di Trento, Casari Paolo},
    pdfproducer={LaTeX},
    pdfcreator={pdflatex},
}

\fancypagestyle{chapterInit}{%
    \fancyhf{}
    \renewcommand{\headrulewidth}{0pt}
    \renewcommand{\footrulewidth}{0.4pt}
    \fancyfoot{}
    \fancyfoot[LE,RO]{\thepage}
    \fancyfoot[LO,RE]{``Appunti di Reti" di Luca Facchini}
}
\fancypagestyle{stdPage}{
    \setlength{\headheight}{14.5pt}
    \fancyhead{}
    \fancyhead[RE,LO]{\rightmark}
    \fancyhead[RO,LE]{\leftmark}
    \fancyfoot{}
    \renewcommand{\footrulewidth}{0.4pt}
    \fancyfoot[LE,RO]{\thepage}
    \fancyfoot[LO,RE]{``Appunti di Reti" di Luca Facchini}
}

\fancypagestyle{tocStyle}{
    \pagestyle{stdPage}
    \fancyhead[RE,LO]{}
}

\graphicspath{{./images/}}

\newtheorem{definition}{Definizione}[chapter]

\title{Appunti di Reti}
\author{Luca Facchini (mat. 245965)}
\date{A.A. 2024/2025}

\begin{document}
    \begin{titlepage}
        \centering  % Center everything on the title page
        {\Huge\textbf{Appunti di Reti}} \\[1cm] % Title
        \vspace{1.5cm}
        
        {\normalsize di: } \\[.3cm]
        {\Large Facchini Luca} \\ % Author name
        \vspace{1.5cm}
        

        {\normalsize Corso tenuto dal prof. Casari Paolo} \\[0.3cm] % Course information
        {\large Università degli Studi di Trento} \\[1.5cm]
        
        {\large A.A. 2024/2025} \\[3cm] % Academic year
        
        % Abstract section with spacing control
        \vfill
        \begin{minipage}[t]{0.4\textwidth}
            \begin{flushleft} \normalsize
                \emph{Autore:}\\
                \textsc{Facchini} Luca \\ % Author name
                Mat. 245965 \\
                \vspace{-\baselineskip}
                \begin{tabbing}
                    Email:\= \href{mailto:luca.facchini-1@studenti.unitn.it}{luca.facchini-1@studenti.unitn.it} \\
                        \>  \href{mailto:luca@fc-software.it}{luca@fc-software.it}
                \end{tabbing}
            \end{flushleft}
        \end{minipage}%
        \hfill
        \begin{minipage}[t]{0.4\textwidth}
            \begin{flushleft} \normalsize
                \emph{Corso:}\\
                Reti [145417] \\
                \textsc{CdL}: Laurea Triennale in Informatica \\
                Prof. \textsc{Casari} Paolo \\
                Email: \href{mailto:paolo.casari@unitn.it}{paolo.casari@unitn.it}
            \end{flushleft}
        \end{minipage}
        \vfill
        \begin{abstract}
            Appunti del corso di Reti, tenuto dal prof. Casari Paolo presso l'Università degli Studi di Trento. Corso seguito nell'anno accademico 2024/2025.\newline
            Dove non specificato diversamente, il le immagini e i contenuti sono tratti dalle slide del corso del prof. Casari Paolo (\href{mailto:paolo.casari@unitn.it}{paolo.casari@unitn.it})
        \end{abstract}
        
        % Pushes the content to the center vertically
    \end{titlepage}
    \begingroup
        \pagestyle{tocStyle}
        \addtocontents{toc}{\protect\thispagestyle{tocStyle}}
        \addtocontents{toc}{\protect\pagestyle{tocStyle}}
        \tableofcontents
    \endgroup
    \pagestyle{stdPage}
    
    \chapter{Nozioni di base}
Nel seguente capitolo definiamo alcune nozioni di base che verranno utilizzate all'interno del corso.

\paragraph{Sistema di riferimento} Un sistema di riferimento è un insieme di regole che permettono di determinare la posizione di un punto nello spazio. Un sistema di riferimento è composto da un'origine, da un insieme di assi e da un'unità di misura. Definiamo un sistema di riferimento in quattro assi: $x$, $y$, $z$ e $t$ dove $t$ rappresenta il tempo.
\begin{definition}[Spazio-Tempo Euclideo]
    Lo spazio-tempo euclideo ($S$) è un sistema di riferimento in quattro assi, $x$, $y$, $z$ e $t$, dove $t$ rappresenta il tempo. Lo spazio-tempo euclideo è definito come:
    $$
        S(O_z, x, y, z; O_t, t)
    $$
    dove $O_z$ è l'origine degli assi spaziali e $O_t$ è l'origine dell'asse temporale.
\end{definition}

\section{Moti in una dimensione e grafico orario}
    Per descrivere i moti in una dimensione possiamo utilizzare un grafico non affine, quindi non lineare, che rappresenta la posizione di un punto in funzione del tempo. Questo grafico è detto \textit{grafico orario}.
    \begin{definition}[Grafico Orario]
        Il grafico orario è un grafico cartesiano che esprime la posizione di un punto che si muove in una dimensione in funzione del tempo.
    \end{definition}
    \begin{tikzpicture}
        \begin{axis}
            [axis lines = left, xlabel = $t$, ylabel = $x$, xmin = 0, xmax = 11, ymin = 0, ymax = 11, xtick = {0, 2, 10}, ytick = {0, 2, 10}, xticklabels = {$0$, $t_i$, $t_f$}, yticklabels = {$0$, $x_i\ P_i$, $x_f\ P_f$}, legend pos = north west]
            \addplot[mark = *, color = red] coordinates {(2, 2)} node[above] {$E_i$};
            \addplot[mark = *, color = blue] coordinates {(10, 10)} node[above] {$E_f$};
        \end{axis}
    \end{tikzpicture}
    Vediamo come al momento $t_i$ il punto sia in posizione $P_i$ e di verifica come l'evento $E_i$ sia in posizione $x_i$. Al momento $t_f$ il punto è in posizione $P_f$ e l'evento $E_f$ è in posizione $x_f$. Ora possiamo definite lo spostamento:
    \begin{definition}[Spostamento]
        Lo spostamento è la variazione di posizione di un punto in un intervallo di tempo. Lo spostamento è definito come:
        $$
            \begin{aligned}
                S_{i \to f} &= x_f - x_i\\
                \Delta x_{i \to f} &= x_f - x_i
            \end{aligned}
        $$
    \end{definition}
    Da notare come lo spostamento non descrive ne' la traiettoria ne' la distanza percorsa dal punto ma solo la variazione di posizione, infatti il punto potrebbe aver compiuto un percorso "non diretto". Inoltre nello spostamento ha un verso definito e come conseguenza scrivere $S_{i \to f} \neq S_{f \to i}$.
    \begin{definition}[Distanza Percorsa]
        La distanza percorsa è la lunghezza della traiettoria percorsa da un punto in un intervallo di tempo. La distanza percorsa è definita come:
        $$
            d(P_i, P_f) = \left| x_f - x_i \right|
        $$
    \end{definition}
    Notiamo come la distanza percorsa sia sempre positiva in quanto è la lunghezza della traiettoria percorsa dal punto. Inoltre la distanza percorsa non ha un verso definito, infatti $d(P_i, P_f) = d(P_f, P_i)$.\newline
    Ora per descrivere il moto di un punto possiamo definire la velocità media:
    \begin{definition}[Velocità media]
        La velocità media è la variazione di posizione di un punto in funzione del tempo. La velocità media è definita come:
        $$
            \begin{aligned}
                v_m &= \frac{\Delta x}{\Delta t} = \frac{\Delta x_{i \to f}}{\Delta t_{i \to f}}\\
            \end{aligned}
        $$
    \end{definition}
    Da notare come la velocità media non tiene conto del moto del punto in un intervallo di tempo, ma solo della variazione di posizione. Inoltre la velocità media ha un verso definito in quanto trattiamo lo spostamento (il quale ha un verso definito).\newline
    Per descrivere il moto di un punto in un instante $t$ di tempo possiamo definire la velocità istantanea:
    \begin{definition}[Velocità istantanea]
        La velocità istantanea è la variazione di posizione di un punto in un istante di tempo. La velocità istantanea è definita come:
        $$
            v_i(t)=\lim\limits_{\Delta t \to 0} \frac{\Delta x}{\Delta t} = \frac{dx}{dt}
        $$
    \end{definition}
    Dal punto di vista matematico la velocità istantanea è la derivata della posizione rispetto al tempo. Dunque i punti dove si passa da un movimento ``in avanti'' ad un movimento ``all'indietro'' sono i punti in cui la velocità istantanea è nulla ovvero i punti di massimo e minimo della funzione posizione, inoltre il punto in cui la velocità istantanea è nulla è detto punto di inversione. Inoltre la velocità istantanea è una funzione continua in quanto la derivata di una funzione continua è anch'essa continua.\newline
    È vero che in un determinato periodo di tempo io possa aumentare o diminuire la velocità, per questo motivo definiamo la funzione di accelerazione:
    \begin{definition}[Accelerazione]
        L'accelerazione è la variazione di velocità di un punto in funzione del tempo. L'accelerazione è definita come:
        $$
            \begin{aligned}
                a &= \frac{\Delta v}{\Delta t} = \frac{\Delta v_{i \to f}}{\Delta t_{i \to f}}\\
            \end{aligned}
        $$
    \end{definition}
    Da notare come l'accelerazione non tiene conto del moto del punto in un intervallo di tempo, ma solo della variazione di velocità. Inoltre l'accelerazione ha un verso definito in quanto trattiamo la variazione di velocità (la quale ha un verso definito).
    \paragraph{Relazione tra posizione, velocità e accelerazione}
        Come già detto la velocità è la derivata della posizione rispetto al tempo e l'accelerazione è la derivata della velocità rispetto al tempo, è vero inoltre che la posizione è l'integrale della velocità rispetto al tempo e questa è l'integrale dell'accelerazione rispetto al tempo. Dunque possiamo scrivere:
        \begin{align}
            x(t) &\text{ posizione} \\
            v(t) = \frac{dx}{dt} &\text{ velocità} \\
            a(t) = \frac{dv}{dt} = \frac{d^2x}{dt^2} &\text{ accelerazione}
        \end{align} 
        Al contrario possiamo scrivere:
        \begin{align}
            v(t) &= v_0 + \int_{t_0}^{t} a(T) dT\\
            x(t) &= x_0 + \int_{t_0}^{t} v(T) dT = x_0 + \int_{t_0}^{t} dT \left[v_0+\int_{t_0}^{T} a(\tau) d\tau\right]
        \end{align}
        Ora le dimensioni fisiche (e non le unità di misura) di queste grandezze sono:
        $$
            \begin{aligned}
                \left[x\right] &= \left[L\right]\\
                [v] &= \left[\frac{L}{T}\right]\\
                [a] &= \left[\frac{L}{T^2}\right]
            \end{aligned}
        $$
        ed le rispettive unità di misura sono:
        $$
            \begin{aligned}
                \left[x\right] &= \left[m\right]\\
                [v] &= \left[\frac{m}{s}\right]\\
                [a] &= \left[\frac{m}{s^2}\right]
            \end{aligned}
        $$
    \chapter{Il Livello Applicazione}
\thispagestyle{chapterInit}
\section{principi delle applicazioni di rete}
    \subsection{Architetture di rete}
        Esistono varie architetture per le applicazioni di rete, tra le quali:
        \begin{itemize}
            \item Client-Server
            \item Peer-to-Peer
            \item Architetture ibride
            \item Cloud Computing 
        \end{itemize}
        \subsubsection{Client - Server}
        \label{subsubsec:clientServer}
            In questa architettura esistono due ruoli principali:
            \paragraph{Server} Il server è un host \textbf{sempre attivo} con un \textbf{indirizzo permanente} e molto spesso difficile da scalare
            \paragraph{Client} Il client \textbf{comunica col server}, inoltre a differenza del server può \textbf{disconnettersi temporaneamente} e inoltre può avere un \textbf{indirizzo IP dinamico}. Generalmente i client \textbf{non comunicano tra di loro}.
        \subsubsection{Architettura P2P pura}
            In questa architettura \textbf{non c'è} sempre un server attivo, vengono eseguire \textbf{coppie arbitrarie} di host che comunicano tra di loro. Infine i \textbf{peer} non devono necessariamente essere sempre attivi e possono avere un \textbf{indirizzo IP dinamico}.

        \subsubsection{Architetture ibride}
            In queste architetture si ha una combinazione tra client-server e P2P, ad esempio un server con peer che comunicano tra di loro. Un esempio di architettura ibrida è Skype, oppure un applicativo di messaggistica istantanea dove le chat sono P2P ma l'individuazione degli utenti è fatta tramite un server centrale.
        \subsubsection{Cloud Computing}
            In questa architettura si ha un insieme di tecnologie che permettono \textbf{di memorizzare archiviare e/o elaborare dati} tramite l'utilizzo di risorse distribuite. La creazione di copie di sicurezza dette \textbf{backup} è automatica e l'operabilità si trasferisce online. I dati sono memorizzati in \textbf{server farm} generalmente localizzate nei paesi di origine del service provider.
    \subsection{Struttura delle applicazioni di rete}
        \subsubsection{Processi del sistema operativo}
            \paragraph{Processo:} Programma in esecuzione su un host.
            
            All'interno di uno stesso host due processi comunicano utilizzando \textbf{schemi interprocesso} (definiti dal S.O.)

            Processi su host diversi comunicano tramite \textbf{messaggi} scambiati tramite la rete.
            \paragraph{Processo client} processo che inizia la comunicazione
            \paragraph{Processo server} processo che attende di essere contattato
        \subsubsection{Socket}
            \paragraph{Socket} Un processo che invia/riceve messaggi a/da il suo \textbf{socket}.
            
            Un socket è analogo ad una porta 

    \subsection{Indirizzamento}
        \paragraph{IP} Per identificare un host in modo univoco si usa un \textbf{indirizzo IP} che è formato da 32 bit. 

        \paragraph{Numeri di porta} L'indirizzo IP però non è sufficiente ad identificare un processo all'interno dell'host per questo definiamo dei \textbf{numeri di porta}.

    \subsection{Protocolli a livello applicazione}
        \subsubsection{Definizioni}
            I protocolli a livello applicazione definiscono:
                \begin{itemize}
                    \item Tipi di messaggi scambiati
                    \item Sintassi dei messaggi
                    \item Semantica dei campi dei messaggi
                    \item Regole per determinare quando e come i processi inviano e ricevono messaggi
                \end{itemize}
            \paragraph{Protocolli dominio pubblico}Alcuni protocolli sono di pubblico dominio definiti nelle \textbf{RFC} (Request for Comments) della \textbf{IETF} (Internet Engineering Task Force). Questi consentono interoperabilità tra diversi host, esempi di protocolli a pubblico dominio sono: \textbf{HTTP}, \textbf{SMTP}\dots
            \paragraph{Protocolli proprietari} Altri protocolli sono proprietari, ad esempio Skype.
    \subsection{Servizi di trasporto}
        \subsubsection{Come segliere il protocollo di trasporto}
            \begin{description}
                \item[Perdita di dati] Applicazioni che richiedono trasmissione affidabile dei dati (es. file transfer) richiedono un protocollo di trasporto affidabile
                \item[Temporizzazione] Applicazioni che richiedono bassa latenza (es. VoIP) richiedono un protocollo di trasporto con bassa temporizzazione
                \item[Throughput] Applicazioni che richiedono alto throughput richiedono un protocollo di trasporto con alto throughput
                \item[Sicurezza] Applicazioni che richiedono sicurezza (es. trasferimento di file) richiedono un protocollo di trasporto sicuro
            \end{description}
            \begin{table}[H]
                \centering
                \begin{adjustbox}{max width=\textwidth}
                    \begin{tabular}{c p{7em} p{13em} c}
                        \textbf{Applicazione} & \textbf{Tolleranza alla perdita di dati} & \textbf{Throughput} & \textbf{Sensibilità al tempo} \\
                        \hline\\
                        Trasferimento file & No & Variabile & No \\
                        \hline\\
                        Posta elettronica & No & Variabile & No \\
                        \hline \\
                        Documenti Web & No & Variabile & No \\
                        \hline \\
                        Audio/video in tempo reale & Sì & Audio: da 5kbit/s a 1Mbit/s Video: da 10kbit/s a 5Mbit/s & Sì, centinaia di ms \\
                        \hline \\
                        Audio/video memorizzati & Si & come sopra & Sì, pochi secondi \\
                        \hline \\
                        Giochi interattivi & Sì & Fino a pochi kbit/s & Sì, centinaia di ms \\
                        \hline \\
                        Messaggistica istantanea & No & Variabile & Sì e no \\
                        \hline
                    \end{tabular}
                \end{adjustbox}
            \end{table}
        \subsubsection{TCP / UDP}
            \paragraph{TCP} \textbf{Transmission Control Protocol} è un protocollo di trasporto \textbf{affidabile} e \textbf{orientato alla connessione}. TCP ha un \textbf{controllo di flusso} e \textbf{controllo di congestione}, \textbf{non offre} temporizzazione e garanzie su un'ampiezza di banda minima, sicurezza.
            \paragraph{UDP} \textbf{User Datagram Protocol} è un protocollo di trasporto \textbf{inaffidabile} fra i processi d'invio e di ricezione. UDP \textbf{non offre} controllo di flusso, controllo di congestione, temporizzazione, garanzie su un'ampiezza di banda minima, sicurezza.
            
            \begin{table}[H]
                \centering
                \begin{adjustbox}{max width=\textwidth}
                    \begin{tabular}{c c c}
                        \textbf{Applicazione} & \textbf{Protocollo a livello applicazione} & \textbf{Protocollo di trasporto} \\
                        \hline \\
                        Posta elettronica & SMTP [RFC 2821] & TCP \\
                        \hline \\
                        Accesso a terminali remoti & Telnet [RFC 854] & TCP \\
                        \hline \\
                        Web & HTTP [RFC 2616] & TCP \\
                        \hline \\
                        Trasferimento file & FTP [RFC 959] & TCP \\
                        \hline \\
                        Multimedia in streaming & HTTP [RFC 2616], RTP [RFC 3550] & TCP, UDP \\
                        \hline \\
                        Telefonia Internet & SIP [RFC 3261], RTP [RFC 3550], Proprietario & Tipicamente UDP \\  
                        \hline
                    \end{tabular}
                \end{adjustbox}
            \end{table}
\section{Web e HTTP}
    \subsection{Terminologia}
        \begin{description}
            \item[Pagina Web] Una \textbf{pagina web} è costituita da \textbf{oggetti}
            \item[Oggetto] Un \textbf{oggetto} può essere una \textbf{pagina HTML}, un'\textbf{immagine}, un'\textbf{applet}, un'\textbf{audio}, un'\textbf{video}, \dots
            \item[un file HTML] è un \textbf{file base} per formare una \textbf{pagina web}. Suddetto file è scritto tramite l'\textbf{HyperText Markup Language} che include diversi oggetti referenziati
            \item[URL] Ogni oggetto è referenziato tramite un \textbf{URL} (Uniform Resource Locator)    
        \end{description}
        \paragraph{Esempio di URL}
            \begin{center}
                \texttt{http://www.sito.com/folder/file.html}
            \end{center}
            \begin{description}
                \item[http] Protocollo di trasferimento
                \item[www.sito.com] Nome del server
                \item[folder] Cartella in cui si trova il file
                \item[file.html] Nome del file
            \end{description}
    \subsection{Introduzione a HTTP}
        \paragraph{Overview} L'\textbf{HTTP} (HyperText Transfer Protocol) è un protocollo di livello applicazione del web. Sfrutta il modello \textbf{\hyperref[subsubsec:clientServer]{client-server}} dove il \textbf{client} invia una \textbf{richiesta} al \textbf{server} che risponde con una \textbf{risposta} contenente il \textbf{contenuto richiesto} e il client visualizza il contenuto.
        
        \paragraph{Usa TCP} Il client inizializza una connessione \textbf{TCP} con il server sulla porta 80, il server accetta la connessione \textbf{TCP} del client e si scambiano messaggi HTTP tra il browser e il web-server. Quando il trasferimento è completato la connessione \textbf{TCP} viene chiusa.
        
        Si noti come il protocollo HTTP sia \textbf{stateless}, ovvero non mantiene informazioni sullo stato del client.

        \subsubsection{Connessioni HTTP}
            \paragraph{Connessioni non persistenti} Almeno un oggetto viene trasmesso su una connessione \texttt{TCP}.
                \begin{enumerate}
                    \item Il client \texttt{HTTP} inizializza una connessione \texttt{TCP} con un server \texttt{HTTP} sulla porta 80
                    \item Il server \texttt{HTTP} sul host in attesa di una connessione \texttt{TCP} alla porta 80
                    \item Il client \texttt{HTTP} trasmette un \textit{messaggio di richiesta} con l'\textit{URL} nella socket della connessione \texttt{TCP}. Il messaggio indica che oggetto si vuole
                    \item Il server \texttt{HTTP} trasmette un \textit{messaggio di risposta} con l'oggetto richiesto nella socket della connessione \texttt{TCP}
                    \item Il server chiude la connessione \texttt{TCP}
                    \item Il client riceve l'oggetto e visualizza l'oggetto richiesto e all'arrivo del messaggio di risposta chiude la connessione \texttt{TCP}
                \end{enumerate}
                \begin{itemize}
                    \item Il metodo di connessione non persistente richiede 2 round-trip time (\texttt{RTT}) per ottenere un oggetto.
                    \item Overhead di connessione \texttt{TCP} per ogni oggetto richiesto
                    \item I browser moderni spesso in caso di connessioni non persistenti aprono richieste parallele per ottenere più oggetti contemporaneamente
                \end{itemize}
            \paragraph{Connessioni persistenti} Più oggetti vengono trasmessi su una connessione TCP
        \subsubsection{Tipi dei metodi}
            \begin{description}
                \item[GET] Il client richiede un oggetto al server 
                \item[POST] Il client invia dati al server
                \item[HEAD] Il client richiede solo l'intestazione dell'oggetto
                \item[PUT] Il client invia un oggetto al server (da HTTP/1.1)
                \item[DELETE] Il client cancella un oggetto dal server (da HTTP/1.1)
            \end{description}
        \subsubsection{Messaggio di risposta HTTP}
            \texttt{HTTP/1.1 200 OK $\Rightarrow$ Versione del protocollo, codice di stato, frase di stato\\
                Connection close $\Rightarrow$ Connessione chiusa\\
                Date: Thu, 06 Aug 1998 12:00:15 GMT $\Rightarrow$ Data e ora\\
                Server: Apache/1.3.0 (Unix) $\Rightarrow$ Server web\\
                Last-Modified: Mon, 22 Jun 1998 ... $\Rightarrow$ Data ultima modifica\\
                Content-Length: 6821 $\Rightarrow$ Lunghezza del contenuto\\
                Content-Type: text/html $\Rightarrow$ Tipo di contenuto\\
                dati dati dati dati dati  ... $\Rightarrow$ Dati
            }
        \subsubsection{Codici di stato}
        \begin{description}
            \item[200] OK $\Rightarrow$ La richiesta è stata completata con successo l'oggetto richiesto è stato trasmesso
            \item[301] Moved Permanently $\Rightarrow$  Il documento richiesto è stato spostato in un'altra locazione
            \item[400] Bad Request $\Rightarrow$ La richiesta non può essere soddisfatta di errori client
            \item[404] Not Found $\Rightarrow$ Il documento richiesto non è stato trovato sul server
            \item[505] HTTP Version Not Supported $\Rightarrow$ La versione HTTP usata non è supportata dal server
        \end{description}
    \subsection{Cookies}
        I \textbf{cookies} sono composti da quattro componenti:
        \begin{enumerate}
            \item Una riga di intestazione nel messaggio di \textit{risposta HTTP} % TODO: inserire riferimento
            \item Una riga di intestazione nel messaggio di \textit{richiesta HTTP} % TODO: inserire riferimento
            \item Un file mantenuto sul \textit{client}
            \item Un database mantenuto sul \textit{server}
        \end{enumerate}
        \subsubsection{Come vengono usati cookies}
            \paragraph{Cosa contengono}
                \begin{itemize}
                    \item Autorizzazione
                    \item Carta per acquisti
                    \item Raccomandazioni
                    \item Stato della sessioni dell'utente
                \end{itemize}
            \paragraph{Lo Stato}
                \begin{itemize}
                    \item Mantengono lo stato del mittente e del ricevente per più richieste
                    \item I messaggi HTTP trasportano lo stato
                \end{itemize}
            \paragraph{Privacy}
                \begin{itemize}
                    \item I cookies possono essere usati per tracciare la navigazione dell'utente
                    \item L'utente può fornire al sito nome e l'indirizzo
                \end{itemize}
    \subsection{Cache web}
        \paragraph{Obbiettivo:} soddisfare le richieste degli utenti senza coinvolgere il server d'origine
        \paragraph{Cache} è una copia di un oggetto mantenuta da un'entità più vicina all'utente
        \paragraph{Il Procedimento} Il client invia una richiesta al server proxy, il server proxy invia la richiesta al server d'origine se l'oggetto non è in cache, altrimenti il server proxy invia l'oggetto al client.
        \paragraph{Vantaggi} Riduzione del tempo di risposta, riduzione del traffico di rete, riduzione del carico sui server d'origine
        \paragraph{Perchè viene usata} Viene usata per ridurre il tempo di risposta e il traffico di rete, in certe situazioni delle istituzioni si possono dotare di un cache interna per ridurre il traffico di rete verso l'esterno e per ridurre il tempo di risposta.
        \paragraph{\texttt{GET} condizionale} Il client può chiedere al server proxy di inviare l'oggetto solo se è stato modificato, in caso contrario il server proxy invia un messaggio di risposta con codice 304 (Not Modified) e l'oggetto non viene inviato. Il controllo viene eseguito tramite un \textbf{header} \texttt{If-Modified-Since} che contiene la data dell'ultima modifica dell'oggetto.
    \subsection{HTTP/1.0 e HTTP/1.1}
        \subsubsection{HTTP/1.0}
            \begin{itemize}
                \item Connessioni non persistenti
                \item Ogni oggetto richiede una connessione TCP separata
                \item Non supporta proxy
                \item Non supporta cache
            \end{itemize}
        \subsubsection{HTTP/1.1}
            \begin{itemize}
                \item Connessioni persistenti
                \item Pipelining
                \item Host Virtuale
                \item Cache
                \item Cookies
                \item Connessioni persistenti
                \item Pipelining
                \item Host Virtuale
                \item Cache
                \item Cookies
            \end{itemize}
    \subsection{HTTP/2.0}
        \textbf{HTTP/2} rappresenta una evoluzione di \textbf{HTTP/1.1}, il protocollo è focalizzato sulle prestazioni, specificatamente sulla latenza percepita. Obbiettivo di \texttt{HTTP/2} è di avere una unica connessione per browser.
        \subsubsection{Framing binario}
            Nuovo livello di framing binario per incapsulare i messaggi \texttt{HTTP}, in questo modo la semantica \texttt{HTTP} rimane invariata ma la codifica in transito è differente. Tutte le comunicazioni \texttt{HTTP/2} sono suddivise in messaggi più piccoli, ognuno dei quali codificano un formato binario, inoltre sia il client che il server possono inviare messaggi in qualsiasi momento.ù
        \subsubsection{Stream, messaggi e frame}
            Tutte le comunicazioni vengono eseguite all'interno di una connessione TCP bidirezionale, ogni \textbf{stream} ha un identificativo univoco con priorità. Ogni messaggio è un messaggio \texttt{HTTP} logico (richiesta/risposta). Il frame è la più piccola unità di comunicazione di un certo tipo specifico di dati.
        \subsubsection{Multiplexing di richieste e risposte}
        \subsubsection{Server Push}
    \subsection{Transport Layer Security (TLS)}
    \subsection{HTTPS}
\section{FTP - File Transfer Protocol}
    \paragraph{FTP} Il \textbf{File Transfer Protocol} è un protocollo di trasferimento di file che permette di trasferire file tra un host e un server. FTP è un protocollo \textbf{stateful} che mantiene lo stato del client e del server durante la sessione. Lo standard FTP è definito nella \textbf{RFC 959} e usa una porta standard di \textbf{21}.
    \subsection{Connessione di controllo}
        \paragraph{Connessione di controllo} La connessione di controllo è usata per inviare comandi tra il client e il server. I comandi sono inviati in \textbf{ASCII} e i comandi sono \textbf{case-insensitive}. La connessione di controllo è \textbf{stateful} e mantiene lo stato del client e del server durante la sessione. La connessione di controllo usa la porta \textbf{21}, mentre la connessione dati usa la porta \textbf{20}, questo è un esempio di protocollo con \textbf{controllo fuori banda}.
    \subsection{Comandi FTP}

\section{Posta Elettronica}
    \paragraph{Introduzione}
        Per la gestione della posta elettronica esistono 3 componenti principali:
        \begin{itemize}
            \item Agente utente
            \item Server di posta
            \item Simple Mail Transfer Protocol (SMTP)
        \end{itemize}
        \subparagraph{Agente utente} è detto anche "mail reader" e permette di comporre, modificare e leggere i messaggi di posta elettronica. I messaggi in uscita o in arrivo vengono memorizzati sul server di posta che è sempre attivo.
        \subparagraph{Server di posta} Contiene la \textbf{Casella di posta} che contiene i messaggi in arrivo, ha una \textbf{coda di messaggi} in uscita ed usa il \textbf{protocollo SMTP} tra server di posta per inviare messaggi di posta elettronica, in quanto il protocollo \textbf{SMTP} richiede che il server ricevente sia sempre in ascolto.
    \subsection{SMTP}
        Il protocollo \textbf{SMTP} (Simple Mail Transfer Protocol) è un protocollo di livello applicazione che permette di inviare messaggi di posta elettronica tra server di posta. Il protocollo \textbf{SMTP} usa la porta \textbf{25} ed è un protocollo \textbf{stateless}.
        \paragraph{Fasi del trasferimento} Il trasferimento di un messaggio di posta elettronica avviene in tre fasi:
            \begin{description}
                \item[Handshaking] Il client apre una connessione \textbf{TCP} con il server di posta, il server risponde con un messaggio di benvenuto
                \item[trasferimento] Il client invia il messaggio, il server accetta il messaggio e lo deposita nella casella di posta del destinatario
                \item[Chiusura] Il client chiude la connessione
            \end{description}
        \paragraph{Iterazione comando/risposa} I comando usano \textbf{ASCII} a 7 bit e sono \textbf{case-insensitive}, le risposte sono codificate con un codice a tre cifre.
        \paragraph{Note finali} 
            \begin{itemize}
                \item Il protocollo usa connessioni \textbf{persistenti}
                \item Il protocollo richiede che il messaggio (intestazione e corpo) sia nel formato \textbf{ASCII} a 7 bit
                \item Il protocollo prevede che \texttt{<CR><LF>.<CR><LF>} sia usato per terminare il messaggio
            \end{itemize}
        \paragraph{Formato dei messaggi di posta elettronica}
            \begin{description}
                \item[Intestazione]contiene i mittenti, i destinatari, il soggetto, la data e l'ora
                \item[riga vuota] separa l'intestazione dal corpo
                \item[Corpo] contiene il testo del messaggio
            \end{description}
    \subsection{POP3}
        Il protocollo \textbf{POP3} (Post Office Protocol 3) è un protocollo di livello applicazione che permette di scaricare i messaggi di posta elettronica dal server di posta. Il protocollo \textbf{POP3} usa la porta \textbf{110} ed è un protocollo \textbf{stateful}.
        \paragraph{Fasi del trasferimento} Il trasferimento di un messaggio di posta elettronica avviene in tre fasi:
            \begin{description}
                \item[autorizzazione] Il client apre una connessione \textbf{TCP} con il server di posta, il client si autentica con il server
                \item[trasferimento] Il client scarica i messaggi di posta elettronica
                \item[Chiusura] Il client chiude la connessione
            \end{description}
        \paragraph{Comandi POP3}
            \begin{description}
                \item[USER] Autenticazione
                \item[PASS] Password
                \item[LIST] Lista dei messaggi
                \item[RETR] Recupera un messaggio
                \item[DELE] Cancella un messaggio
                \item[QUIT] Chiude la connessione
            \end{description}
    \subsection{IMAP}
        Il protocollo \textbf{IMAP} (Internet Message Access Protocol) è un protocollo di livello applicazione che permette di scaricare i messaggi di posta elettronica dal server di posta. Il protocollo \textbf{IMAP} usa la porta \textbf{143} ed è un protocollo \textbf{stateful}.
        \paragraph{Fasi del trasferimento} Il trasferimento di un messaggio di posta elettronica avviene in tre fasi:
            \begin{description}
                \item[autorizzazione] Il client apre una connessione \textbf{TCP} con il server di posta, il client si autentica con il server
                \item[trasferimento] Il client scarica i messaggi di posta elettronica
                \item[Chiusura] Il client chiude la connessione
            \end{description}
        \paragraph{Comandi IMAP}
            \begin{description}
                \item[LOGIN] Autenticazione
                \item[SELECT] Seleziona una casella di posta
                \item[FETCH] Recupera un messaggio
                \item[STORE] Modifica lo stato di un messaggio
                \item[LOGOUT] Chiude la connessione
            \end{description}
\section{DNS}
    \paragraph{Introduzione}
        \paragraph{Domain Name Sysyem} Il \textbf{DNS} consiste in un \textit{database distribuito} implementando una gerarchia di \textit{server DNS}. Il \textit{DNS} è un protocollo a livello applicazione che consente agli host e ai router di comunicare per \textit{risolvere} i nomi degli host in indirizzi IP.
    \subsection{Servizi DNS}
        \begin{itemize}
            \item Traduzione degli hostname in indirizzi IP
            \item Host aliasing - Un host può avere più nomi
            \item Mail server aliasing - Un host può avere più server di posta
            \item Payload distribution - Distribuzione del carico tra i server
        \end{itemize}
        \paragraph{Perchè non centralizzare DNS}
        \begin{itemize}
            \item Singolo punto di fallimento
            \item Traffico di rete
            \item Database centralizzato distante
            \item Manutenzione
        \end{itemize}
    \subsection{Struttura del DNS}
        In generale i server \texttt{DNS} sono organizzati in una struttura gerarchica a \textbf{albero} dove il nodo radice è il server \texttt{DNS} radice (13 al mondo) esistono dei server di \texttt{DNS} di nomi di primo livello (com) (TLD) e infine i server di \texttt{DNS} autoritativi usati per un dominio di secondo livello (google.com)
        \paragraph{Server \texttt{DNS} locali} Ogni ISP ha un server \texttt{DNS} locale che si occupa di tradurre i nomi degli host in indirizzi IP
    \subsection{Resource Record \texttt{RR}}
        \paragraph{Resource Record} Un \texttt{RR} è una tupla che contiene i seguenti campi:
        \begin{itemize}
            \item \texttt{Name} - Il nome del dominio
            \item \texttt{Value} - Il valore del campo
            \item \texttt{Type} - Il tipo di record
            \item \texttt{TTL} - Il tempo di vita del record
        \end{itemize}
        \paragraph{Tipi di \texttt{RR}}
        \begin{itemize}
            \item \texttt{A} - Indirizzo IP - \textbf{name}: \texttt{hostname} \textbf{value}: \texttt{IP}
            \item \texttt{NS} - Server di nomi - \textbf{name}: \texttt{dominio} \textbf{value}: \texttt{hostname}
            \item \texttt{CNAME} - Nome canonico - \textbf{name}: \texttt{alias} \textbf{value}: \texttt{hostname}
            \item \texttt{MX} - Mail server - \textbf{name}: \texttt{dominio} \textbf{value}: \texttt{hostname}
        \end{itemize}
    \subsection{Inserire un record}
        \paragraph{Esempio} Abbiamo avviato la nuova società
        \begin{itemize}
            \item Registriamo il nome "foo.com" presso un \texttt{registrar}
            \item Otteniamo un indirizzo IP per il nostro server web (host)
            \item Diamo al nostro registrar l'indirizzo IP del nostro server web e il nome del nostro server web. Esempio records: \texttt{(foo.com, dns1.foo.com, NS), (dns1.foo.com, 211.211.211.211, A)}
        \end{itemize}
    \chapter{Il Livello di Trasporto}
\thispagestyle{chapterInit}
\paragraph{Obbiettivi}
    \begin{itemize}
        \item Capire i principi che sono alla base dei servizi di livello di trasporto:
            \subitem \textit{Multiplexing/de-multiplexing}
            \subitem Trasferimento dati affidabile
            \subitem Controllo di flusso
            \subitem Controllo di congestione
        \item Descrivere i protocolli del livello di trasporto di Internet:
            \subitem \Acrshort*{TCP}: Trasposto orientato alla connessione
            \subitem Controllo di congestione TCP
            \subitem \Acrshort*{UDP}: Trasporto non orientato alla connessione
    \end{itemize}
\section{Servizi a livello di trasporto}
    \paragraph{Introduzione} I \textbf{protocolli di trasporto} forniscono la comunicazione logica tra processi applicativi di host diversi. I protocolli di trasporto vengono eseguiti negli host "terminali" ovvero quelli che generano o consumano i dati. Dal lato di inviante il protocollo di trasporto divide in diversi segmenti i dati ricevuti dal livello di applicazione e li invia al livello di rete. Dal lato di ricevente il protocollo di trasporto riassembla i segmenti ricevuti e li invia al livello di applicazione.
    \paragraph{\Acrshort*{TCP}} Il \acrfull*{TCP} è un protocollo di trasporto orientato alla connessione. Il \Acrshort*{TCP} fornisce un trasferimento affidabile dei dati, controllo di flusso e controllo di congestione.
    \paragraph{\Acrshort*{UDP}} L'\acrfull*{UDP} è un protocollo di trasporto non orientato alla connessione. L'\Acrshort*{UDP} non fornisce trasferimento affidabile dei dati, controllo di flusso e controllo di congestione.
    \paragraph{Servizi non disponibili} Al momento in internet non è disponibile un servizio di garanzia su ritardi (latenza), e non è disponibile un servizio di garanzia sulla banda (velocità di trasferimento).
\section{\textit{Multiplexing e De-multiplexing}}
    \paragraph{Introduzione} Il \textit{\textbf{multiplexing}} è il processo di invio di dati da più \textit{socket} a un'unica connessione, per identificare il \textit{socket} di destinazione si utilizza un \textit{\textbf{port number}}. Il \textit{\textbf{de-multiplexing}} è il processo di invio dei dati ricevuti al socket corretto in base al port number.
    \subsection{\textit{De-multiplexing}}
        \paragraph{Come funziona} In primo luogo quando l'\textit{host} riceve un segmento \Acrshort*{IP} contenente: \Acrshort*{IP} del mittente, \Acrshort*{IP} del destinatario, protocollo di trasporto, porta di destinazione e porta di sorgente. L'\textit{host} utilizza l'indirizzo \Acrshort*{IP} del destinatario e la porta di destinazione per inviare il segmento al processo corretto.
        \subsubsection{\textit{De-multiplexing} senza connessione}
            Per eseguire il de-multiplexing senza connessione si crea un \textit{socket} per ricevere i dati. Il \textit{socket} è ora associato a una porta ed a un indirizzo \Acrshort*{IP}. Quando l'\textit{host} riceve il segmento \Acrshort*{UDP}, viene controllato che il numero di porta di destinazione sia uguale alla porta del \textit{socket}. Se il numero di porta non corrisponde il segmento viene scartato. Se invece il numero di porta corrisponde il segmento viene inviato al processo associato al \textit{socket}.
            \begin{figure}[H]
                \centering
                \includegraphics[width=0.5\textwidth]{03/DemultiplexingSenzaConnessione.png}
                \caption{\textit{De-multiplexing} senza connessione}
            \end{figure}
        \subsubsection{\textit{De-multiplexing} orientato alla connessione}
            Quando si utilizza un protocollo orientato alla connessione (\Acrshort*{TCP}), il processo di \textit{de-multiplexing} è leggermente diverso. Il \textit{socket} infatti è costituito da quattro parametri: indirizzo \Acrshort*{IP} del mittente, indirizzo \Acrshort*{IP} del destinatario, numero di porta di sorgente e numero di porta di destinazione. Quando l'\textit{host} riceve un segmento \Acrshort*{TCP} controlla che i quattro parametri del \textit{socket} corrispondano ai quattro parametri del segmento. Se i parametri non corrispondono il segmento viene scartato, altrimenti viene inviato al processo associato al \textit{socket}. Un \textit{host} può supportare più \textit{socket} contemporaneamente purché cambi almeno uno dei quattro parametri, inoltre i \textit{web server} sono un chiaro esempio di applicazione che utilizza più \textit{socket} contemporaneamente (su \texttt{HTTP/1.0} un \textit{socket} per ogni richiesta).\footnote{Se viene allocata una porta ad una connessione, la porta non può essere utilizzata da altre connessioni, quindi nel caso di un \textit{web server} è vero che questo ascolta sulla porta \texttt{80}, ma quando un client si connette al server, il server apre un \textit{socket} con una porta dinamica.}
            \begin{figure}[H]
                \centering
                \includegraphics[width=0.5\textwidth]{03/DemultiplexingConConnessione.png}
                \caption{De-multiplexing orientato alla connessione}
            \end{figure}
    \subsection[Porte \texttt*{TCP}-\texttt{UDP}]{Porte \Acrshort*{TCP}-\Acrshort*{UDP}}
        La destinazione finale di un segmento non è un host ma un processo. L'interfaccia tra l'applicazione e il livello di trasporto è chiamata \textbf{socket} o \textbf{porta} (nel caso di \Acrshort*{UDP} e \Acrshort*{TCP}). Un \textbf{socket} è un indirizzo \Acrshort*{IP} e un numero di porta. Un \textit{\textbf{port number}} è un numero a $16$ bit che identifica un processo all'interno di un host. Esiste una mappatura biunivoca tra un \textit{\textbf{port number}} e un processo. I servizi standard utilizzano porte ben note con valori tra $0$ e $1023$. I processi non-standard e le connessioni in ingresso a un client usano numeri fino a $25535$ ($16$ bit).
        \paragraph{Numeri di porte}
            I numeri di porta si classificano come segue:
            \begin{description}
                \item[Statici] Per i servizi standard, es. \Acrshort*{HTTP} (\texttt{80}), \Acrshort*{FTP} (\texttt{21}), \Acrshort*{SSH} (\texttt{22}), \texttt{Telnet} (\texttt{23}), \Acrshort*{SMTP} (\texttt{25}), \Acrshort*{POP3} (\texttt{110}), \Acrshort*{IMAP} (\texttt{143}), \Acrshort*{HTTPS} (\texttt{443}), ecc.
                \item[Dinamici] (o ``ephemeral'') per le connessioni in uscita o per porte allocate dinamicamente, es. client web, client \Acrshort*{FTP}, client \Acrshort*{SSH}, client \texttt{Telnet}, client \Acrshort*{SMTP}, client \Acrshort*{POP3}, client \Acrshort*{IMAP}, client \Acrshort*{HTTPS}, ecc.
            \end{description}
            Inoltre è importante dire che le porte di sorgente e di destinazione non sono le stesse in quanto la porta di sorgente è una porta dinamica assegnata dal sistema operativo.
\section[Trasporto senza connessione: \texttt{UDP}]{Trasporto senza connessione: \Acrshort*{UDP}}
    \paragraph{Caratteristiche} L'\acrfull*{UDP} è un protocollo di trasporto senza connessione, offre un servizio \textit{best effort} e non fornisce garanzie di consegna, ordine o duplicazione dei dati. Questo in quanto non ha \textit{handshake} iniziale e non mantiene alcuno stato di connessione. 
    \paragraph{Perché esiste \Acrshort*{UDP}} Non richiede di stabilire una connessione, è semplice e veloce, Header di segmento corti, senza controllo di congestione.
    \subsection{Header}
        \begin{figure}[H]
            \centering
            \includegraphics[width=0.3\textwidth]{03/UDPHeader.png}
            \caption{Header di un segmento \Acrshort*{UDP}}
        \end{figure}
        \begin{description}
            \item[Porta di sorgente] Numero di porta del processo mittente.
            \item[Porta di destinazione] Numero di porta del processo destinatario.
            \item[Lunghezza] Lunghezza del segmento in byte.
            \item[\textit{Checksum}] Utilizzato per rilevare errori nel segmento.
        \end{description}
        \subsubsection{\textit{Checksum}}
            Il \textit{checksum} è un campo a 16 bit che viene utilizzato per rilevare errori nel segmento. Questo viene calcolato da entrambe le parti: viene trattato il contenuto come una sequenza di $ 16 $ bit e si sommano tutti i bit (se presente riporto questo viene sommato a sua volta) e viene eseguito il complemento a 1. Il mittente invia il \textit{checksum} calcolato nel segmento e il ricevente calcola il \textit{checksum} del segmento ricevuto e lo confronta con il \textit{checksum} ricevuto. Se i due \textit{checksum} non corrispondono il segmento viene scartato.
\section[Trasferimento dati affidabile]{Principi del trasferimento dati affidabile}
    \subsection[Automatic Repeat reQuest (\texttt{ARQ})]{\acrfull*{ARQ}}
        \acrfull*{ARQ} è una classe di protocolli che ``cercano'' di recuperare i segmenti persi o danneggiati. Questa classe usa dei pacchetti speciali per notificare al mittente che un segmento è stato perso o danneggiato. Questi pacchetti speciali sono chiamati \acrfull*{ACK} e \acrfull*{NACK}.
        \paragraph{Esempi di protocolli basati su \Acrshort*{ARQ}} \begin{itemize}
            \item \textit{Stop-and-Wait}
            \item \textit{Go-Back-N}
            \item \textit{Selective Repeat}
            \item \Acrshort*{TCP}
            \item il protocollo \Acrshort*{MAC} (al livello 2) dei sistemi \Acrshort*{Wi-Fi}
        \end{itemize}
    \subsection{\textit{Stop-and-Wait}}
        Nel protocollo \textit{Stop-and-Wait} il mittente invia una \Acrshort*{PDU} e ne mantiene una copia in memoria, imposta dunque un \textit{timeout} su quel \Acrshort*{PDU}. Attende poi un \Acrshort*{ACK} dal ricevente, se non riceve l'\Acrshort*{ACK} entro il \textit{timeout} invia nuovamente la \texttt{PDU}. Se invece riceve l'\Acrshort*{ACK} controlla che questo non contenga errori, che sia il numero di sequenza corretto e che sia per la \Acrshort*{PDU} inviata. Se tutto è corretto invia la prossima \Acrshort*{PDU}. Il ricevente quando riceve una \Acrshort*{PDU} controlla che il numero di sequenza sia corretto e che la \Acrshort*{PDU} non abbia errori, se tutto è corretto invia un \Acrshort*{ACK} al mittente, de-capsula la \Acrshort*{PDU} ai livelli superiori. Se sono presenti errori nella \Acrshort*{PDU} il ricevente esegue il \textit{drop} della \Acrshort*{PDU}.
        \subsubsection{Efficienza dello \textit{Stop-and-Wait}}
            Assumendo una banda $ R = 1 G-bit/s, 15ms $ di ritardo di propagazione, lunghezza del messaggio $ L = 8000bit $, allora il tempo di trasmissione sarà: $T_{trans} = \frac{L}{R} = \frac{80000}{10^9} = 8\mu s$. Mentre il \textit{throughput} percepito a livello applicativo sarà: $ \frac{L}{T_{trans}+RTT}= \frac{8000}{8\mu s + 30ms} = 33 Kbps $.\footnote{Aggiungiamo della formula il \Acrshort*{RTT} ovvero il \acrlong*{RTT} che è il tempo che impiega un pacchetto per andare dal mittente al ricevente e ritornare indietro, nel nostro caso lo aggiungiamo per il pacchetto di \Acrshort*{ACK} ed è di $ 30ms $.} Dunque anche se la nostra banda è di $ 1 Gbps $ il \textit{throughput} percepito a livello applicativo è di $ 33 Kbps $, l'efficienza dunque è: $ \frac{T_{trans}}{T_{trans}+RTT} = \frac{0.008}{0.008+30} = 0.00027 $ ovvero $ 0.027\% $.
    \subsection{Protocolli con \textit{pipelining}}
        I protocolli con \textit{pipelining} permettono di inviare più segmenti successivi senza attendere l'\Acrshort*{ACK} del segmento precedente, si allarga dunque il range dei pacchetti di sequenza accettabili. Questo permette di aumentare l'efficienza del trasferimento dati. Esempio di protocolli con \textit{pipelining} sono \textit{Go-Back-N} e \textit{Selective Repeat}.
        \subsubsection{\textit{Throughput} in presenza di \textit{pipelining}}
            Assumiamo la stessa situazione precedente ed un \textit{pipelining} di $ N = 3 $ allora il \textit{throughput} sarà: $$ \frac{3L}{T_{trans}+RTT} = \frac{24000}{8\mu s + 30ms} = 100 Kbps $$questo è un miglioramento del $ 300\% $ rispetto allo \textit{Stop-and-Wait}. In generale il \textit{throughput} rispetto al \textit{pipelining} con $ N $ segmenti in parallelo è: $ \frac{N \cdot L}{RTT + T_{trans}} $. Il parametro $ N $ è detto \textit{window size} o "dimensione della finestra".
        \subsubsection{Definizioni}
            \begin{description}
                \item[\acrfull*{W_T}] Insieme di \Acrshort*{PDU} che il mittente può inviare senza attendere un \Acrshort*{ACK} del ricevente.
                    \subitem Grande al massimo come la memoria allocata dal sistema operativo.
                    \subitem \Acrshort*{W_T_d} indica la dimensione della finestra.
                \item[\acrfull*{W_R}] Insieme di \Acrshort*{PDU} che il ricevente può ricevere può accettare e memorizzare.
                    \subitem Grande al massimo come la memoria allocata dal sistema operativo.
                \item[\acrfull*{W_LOW}] Puntatore al primo segmento trasmesso ma non ancora confermato.
                \item[\acrfull*{W_HIGH}] Indica l'ultimo segmento già trasmesso della finestra di trasmissione \Acrshort*{W_T}.
            \end{description}
        
        \begin{figure}[H]
            \centering
            \includegraphics[width=0.5\textwidth]{03/finestrePipelining.png}
            \caption{Finestre di trasmissione e di ricezione}
        \end{figure}
        \subsubsection[ACKnowledgment (\Acrshort*{ACK})]{\acrfull*{ACK}}
            Esistono vari tipi di \Acrshort*{ACK} a seconda del protocollo utilizzato, abbiamo dunque:
            \begin{itemize}
                \item \Acrshort*{ACK} individuale il cui compito è quello di indicare la corretta ricezione di uno specifico pacchetto - \texttt{ACK($n$)} vuol dire ho ricevuto il pacchetto $n$.
                \item \Acrshort*{ACK} cumulativo il cui compito è quello di indicare la corretta ricezione di tutti i pacchetti fino a quello specificato - \texttt{ACK($n$)} vuol dire ho ricevuto tutti i pacchetti fino a $n$ (escluso), mi aspetto il pacchetto $n$.
                \item \Acrshort*{ACK} negativo o \Acrshort*{NACK} il cui compito è quello di indicare la mancata ricezione di un pacchetto - \texttt{NACK($n$)} vuol dire non ho ricevuto il pacchetto $n$, invialo nuovamente.
                \item Esiste poi la tecnica del ``\textbf{Piggybacking}", ovvero l'inserimento dell'\Acrshort*{ACK} (di un pacchetto precedente) all'interno di un pacchetto dati successivo.
            \end{itemize}
        \subsubsection{\textit{Go-Back-N}}
            Quando si sceglie di usare il protocollo del tipo \textit{Go-Back-N} questo consiste in: il mittente invia fino ad un numero $ n $ di pacchetti senza aver ricevuto prima \Acrshort*{ACK}, quando un pacchetto è stato ricevuto correttamente viene inviato un \Acrshort*{ACK} cumulativo, se un pacchetto non è stato ricevuto allora i pacchetti successivi vengono scartati in attesa del pacchetto mancante. Dopo un periodo di \textit{timeout} il mittente invia nuovamente tutti i pacchetti a partire dal pacchetto mancante, basandosi sull'ultimo \Acrshort*{ACK} ricevuto. La finestra di trasmissione \Acrshort*{W_T} è dunque composta da $ n $ pacchetti e non viene spostata finché non si riceve un \Acrshort*{ACK} cumulativo, mentre la finestra di ricezione \Acrshort*{W_T} è composta da un solo pacchetto.
        \subsubsection{\textit{Selective Repeat}}
            Nel paradigma del \textit{selective repeat} vengono usati \Acrshort*{ACK} singoli, inoltre è presente una finestra di ricezione \Acrshort*{W_R} composta da $ m $ pacchetti, ciò significa che anche se un pacchetto ricevuto fuori sequenza viene ricevuto allora questo viene comunque ``salvato" all'interno di un buffer in attesa del pacchetto nell'ordine corretto. Anche il mittente in caso di \Acrshort*{ACK} fuori sequenza conserva in memoria questo dato e non lo scarta. Quello che succede se un pacchetto non viene ricevuto ma qualche pacchetto (fino a $m-1$) successivo viene ricevuto correttamente è che il mittente invia nuovamente solo il pacchetto mancante, mentre i pacchetti successivi, se sono già stati \texttt{ACK'ati}, non vengono inviati nuovamente e la trasmissione riprende dal primo pacchetto non \texttt{ACK' ato}.
        \paragraph{Spazio dei numeri di sequenza}
            Solitamente se si hanno $ k $ bit a disposizione, per il dominio del numero di sequenza, allora si usa un periodo pari a $ 2^k $, ovvero il periodo massimo con quello spazio di bit. Le finestre di trasmissione per non avere conflitti devono avere somma inferiore al periodo, quindi \Acrshort*{W_T_d}$+$\Acrshort*{W_R_d}$< 2^k $.\footnote{I conflitti nel caso nel quale ``La somma delle finestre di trasmissione e di ricezione sia maggiore del periodo dei numeri di sequenza" sono dovuti al fatto che in questa situazione in caso di perdita di pacchetti di controllo (\Acrshort*{ACK}/\Acrshort*{NACK}) ma trasmissione corretta dei pacchetti dati, il mittente non saprebbe se il pacchetto è stato ricevuto correttamente o meno il che provoca un \textit{timeout} e la ritrasmissione dei pacchetti dati per persi, però dato che abbiamo già ricevuto i pacchetti dati abbiamo spostato la finestra di ricezione ed il ricevente potrebbe scambiare come per buoni la ritrasmissione dei pacchetti dati che il mittente ha dato per persi.} 
\section[Trasporto Orientato Connessione \texttt{TCP}]{Trasporto orientato alla connessione \Acrshort*{TCP}}
    \paragraph{\Acrshort*{TCP} - vari standard \Acrshort*{RFC}} Il \acrfull*{TCP} è un protocollo di trasporto orientato alla connessione, è stato standardizzato nel \texttt{\Acrshort*{RFC} 793} e successivamente aggiornato con il \texttt{\Acrshort*{RFC} 1122}, \texttt{\Acrshort*{RFC} 1323}, \texttt{2018}, \texttt{2581} ed è in continuo aggiornamento. Questo prevede una connessione \underline{punto-punto} tra mittente e destinatario, è presente un flusso di byte affidabile e consegnato in ordine senza limiti, è presente un meccanismo di \textit{pipelining} e di controllo di congestione per non sovraccaricare la rete. La connessione inoltre (anche se a livelli inferiori non lo è) è \textit{\underline{full-duplex}} ovvero entrambi i lati possono inviare e ricevere dati contemporaneamente. Inoltre il \Acrshort*{TCP} è un protocollo \textit{\underline{stateful}} ovvero mantiene uno stato della connessione, infatti si dice che è orientato alla connessione. Infine \Acrshort*{TCP} ha un ``flusso controllato'' ovvero il trasmettitore non può inviare dati se il ricevente non è pronto a riceverli.
    \subsection[Struttura di un pacchetto \texttt{TCP}]{Struttura di un pacchetto \Acrshort*{TCP}}
        \begin{figure}[H]
            \centering
            \includegraphics[width=0.45\textwidth]{03/pacchettoTCP.jpg}
            \caption{Struttura di un pacchetto \Acrshort*{TCP} } 
        \end{figure}
        {\footnotesize Immagine tratta da \href{https://commons.wikimedia.org/wiki/File:Ntwk_tcp_header.jpg}{Wikimedia Commons} di Gopalpaliwal at English Wikibooks il file è rilasciato sotto licenza Creative Commons \href{https://creativecommons.org/licenses/by-sa/3.0/deed.en}{Attribution-Share Alike 3.0 Unported}.}
        Nel pacchetto \Acrshort*{TCP} sono presenti i seguenti campi principali:
        \begin{description}
            \item[\textit{Source Port}] Porta di sorgente.
            \item[\textit{Destination Port}] Porta di destinazione.
            \item[\textit{Sequence Number}] Numero di sequenza del primo byte del segmento.
            \item[\Acrlong*{ACK} \textit{Number}] Numero di sequenza del prossimo byte atteso.
            \item[\textit{Data Offset}] Lunghezza dell'header in parole di 32 bit.
            \item[\Acrshort*{URG}] Flag che indica la presenza di dati urgenti.
            \item[\Acrshort*{ACK}] Flag che indica la presenza di un campo \Acrshort*{ACK}.
            \item[\Acrshort*{PSH}] Flag che indica che i dati devono essere passati al livello superiore.
            \item[\Acrshort*{PST}] Flag che indica l'inizio di una connessione.
            \item[\Acrshort*{SYN}] Flag che indica la sincronizzazione dei numeri di sequenza.
            \item[\Acrshort*{FIN}] Flag che indica la chiusura della connessione.
            \item[\textit{Window}] Dimensione della finestra di ricezione. (\Acrshort*{RWND})
            \item[\textit{checksum}] Utilizzato per rilevare errori nel segmento (contiene oltre agli header \Acrshort*{TCP} e i dati dei livelli superiori anche i campi \Acrshort*{IP} come l'indirizzo \Acrshort*{IP} del mittente e del destinatario, la lunghezza del segmento, il protocollo di trasporto, ecc.).
            \item[\textit{Urgent Pointer}] Puntatore ai dati urgenti.
            \item[\textit{\Acrshort*{TCP} Options}] Opzioni aggiuntive. (opzionali)
            \item[\textit{Padding}] Padding per allineare il segmento a 32 bit.
            \item[\textit{Data}] Dati del segmento.
        \end{description}
        \paragraph{\acrfull*{RWND}} La finestra di ricezione è un campo a $16$ bit che indica la dimensione della finestra di ricezione del ricevente. Questo campo è utilizzato per il controllo di flusso, infatti il mittente non può inviare dati se la finestra di ricezione del ricevente è piena. La finestra di ricezione è un campo a $16$ bit, quindi la dimensione massima della finestra di ricezione è di $ 2^{16} - 1 = 65535 $ byte. In base alla velocità della banda questo campo può essere modificato per evitare che il mittente non sfrutti tutta la banda disponibile.
        \paragraph{Numeri di sequenza \Acrshort*{ACK} di \Acrshort*{TCP}} I numeri di sequenza di \Acrshort*{TCP} sono a $32$ bit, questo significa che il numero di sequenza può variare tra $ 0 $ e $ 2^{32} - 1 = 4294967295 $. Il numero di sequenza nella direzione mittente-ricevente può essere diverso da quello nella direzione ricevente-mittente, questo perché i numeri di sequenza sono indipendenti nelle due direzioni, inoltre non è detto che i numeri di sequenza inizino da $ 0 $, infatti possono iniziano solitamente da un numero casuale. Durante la trasmissione di un pacchetto mittente-ricevente può essere allegato anche un campo \Acrshort*{ACK} per la conferma della ricezione del pacchetto precedente tra ricevente-mittente (stessa cosa per la direzione opposta).
        \subsubsection{Lunghezza massima segmento \Acrshort*{MSS} e \Acrshort*{MTU}}
            In quanto il \Acrshort*{TCP} lavora per byte cerca sempre di non inviare un singolo byte solo in quanto sarebbe uno spreco di risorse e di banda. Allo stesso tempo non si può inviare un segmento troppo grande in quanto potrebbe essere frammentato a livello di rete. Viene dunque introdotta una "lunghezza massima" detta \acrfull*{MSS} che indica la lunghezza massima di un segmento \Acrshort*{TCP}. La \Acrshort*{MSS} è calcolata come la \acrfull*{MTU} che è la lunghezza massima di un pacchetto che può essere inviato su una rete a livello di collegamento. A sua volta la \Acrshort*{MTU} viene calcolata da passati al livello \textit{data-link} e può variare da rete a rete. La \Acrshort*{MSS} si riferisce non alla lunghezza di tutto il segmento \Acrshort*{TCP} ma solo al \textit{payload}, ovvero il campo dati del segmento \Acrshort*{TCP}.
            \paragraph{Come si sceglie \Acrshort*{MSS}?} Non esistono meccanismi per comunicarlo, viene dunque adottato un modello del tipo \textit{trial\& error}, ovvero il mittente invia un segmento con una \Acrshort*{MSS} di dimensione $ X $ se si nota che i livelli inferiori sopportano la dimensione $ X $ allora si aumenta la dimensione della \Acrshort*{MSS}, altrimenti se si nota che qualche messaggio inizia ad essere perso si riduce la dimensione della \Acrshort*{MSS}.
                \subparagraph{Valodi di default}: \begin{itemize}
                    \item \Acrshort*{MTU} di ethernet: $ 1500 $ byte (payload inseribile al livello 2)
                    \item Header \Acrshort*{IP}: $ 20 $ byte
                    \item Header \Acrshort*{TCP}: $ 20 $ byte
                    \item \Acrshort*{MSS} di default: $ 1460 $ byte
                \end{itemize}
                \subparagraph{``\textit{Least maximum}"} La più piccola \Acrshort*{MTU} impostabile per \Acrshort*{IP} è di $ 576 $ byte, questo dunque la \Acrshort*{MSS} di minima impostabile è di $ 536 $ byte.
    \subsection[Setup della connessione \texttt{TCP} - \textit{handshake}]{Setup della connessione \Acrshort*{TCP} - \textit{handshake}}
        La procedura di apertura di una connessione \Acrshort*{TCP} è detta \textit{three-way handshake}, questa procedura è composta dai seguenti passaggi:
        \begin{enumerate}
            \item \textbf{Host A} invia un segmento \Acrshort*{TCP} con il flag \Acrshort*{SYN} impostato a \texttt{1} e la porta di sorgente $ A $ e la porta di destinazione $ B $.
            \item \textbf{Host B} riceve il segmento \Acrshort*{TCP} e invia un segmento \Acrshort*{TCP} con il flag \Acrshort*{SYN} impostato a \texttt{1} e il flag \Acrshort*{ACK} impostato a \texttt{1} (avvenuta la ricezione del segmento di \Acrshort*{SYN}) e la porta di sorgente $ B $ e la porta di destinazione $ A $.
            \item \textbf{Host A} riceve il segmento \Acrshort*{TCP} e invia un segmento \Acrshort*{TCP} con il flag \Acrshort*{ACK} impostato a \texttt{1} (avvenuta la ricezione del segmento di \Acrshort*{SYN}) e la porta di sorgente $ A $ e la porta di destinazione $ B $.
        \end{enumerate}
        In tutti questi passaggi il numero di \Acrshort*{ACK} non si riferisce al numero di sequenza del segmento ricevuto ma al numero di sequenza del prossimo segmento atteso. Questo \textit{handshake} è necessario per sincronizzare i numeri di sequenza tra mittente e ricevente.
    \subsection[Chiusura della connessione \texttt{TCP}]{Chiusura della connessione \Acrshort*{TCP}}
        La procedura di chiusura di una connessione \Acrshort*{TCP} è detta \textit{tearDown}, questa richiede che la connessione sia chiusa in tutte e due le direzioni. Esiste una maniera ``gentile'' per chiudere la connessione si segue il seguente schema:
        \begin{enumerate}
            \item Invio di un segmento \Acrshort*{TCP} con il flag \Acrshort*{FIN} impostato a \texttt{1} da \texttt{A} a \texttt{B}
            \item Ricezione da parte di \texttt{B} del segmento \Acrshort*{TCP} e invio di \Acrshort*{ACK} \& \Acrshort*{FIN} da \texttt{B} ad \texttt{A}. 
            \item Ricezione da parte di \texttt{A} di \Acrshort*{ACK} \& \Acrshort*{FIN}. La connessione è \textit{half-closed} ovvero la connessione è chiusa in una direzione ma aperta nell'altra.
            \item Trasmissione di eventuali dati rimanenti da \texttt{B} ad \texttt{A}.
            \item Invio di un segmento \Acrshort*{TCP} con il flag \Acrshort*{FIN} impostato a \texttt{1} da \texttt{B} ad \texttt{A}.
            \item Ricezione da parte di \texttt{A} del segmento \Acrshort*{TCP} e invio di \Acrshort*{ACK} da \texttt{A} a \texttt{B}.
            \item Ricezione da parte di \texttt{B} di \Acrshort*{ACK}. La connessione è chiusa in entrambe le direzioni.
        \end{enumerate}
        \paragraph{Chiusura con \Acrshort*{RST}} Se un host invia un segmento \Acrshort*{TCP} con il flag \Acrshort*{RST} (reset) impostato a \texttt{1} allora la connessione viene chiusa immediatamente senza attendere risposta dall'altra parte. Questo meccanismo è utilizzato per chiudere una connessione in modo ``brusco'' in caso di problemi.
    \subsection[Tempi \texttt{RTT} e \texttt{RTO}]{Tempi \Acrshort*{RTT} e \Acrshort*{RTO}}
        Il \Acrshort*{TCP} deve ``impostare'' un \textit{timeout} per l'invio e per la ricezione dei segmenti, questo \textit{timeout} è detto \acrfull*{RTO}, deve essere dunque un valore superiore al \acrfull*{RTT} ovvero il tempo che impiega un pacchetto per andare dal mittente al ricevente e ritornare indietro. Il \Acrshort*{RTT} può variare nel tempo, quindi il \Acrshort*{RTO} deve essere impostato in modo dinamico, se è troppo basso si rischia di re-inviare prematuramente un segmento, se è troppo alto si rischia di ``aspettare'' per troppo tempo. Quindi in sostanza và prima stimato il \Acrshort*{RTT} e poi impostato il \Acrshort*{RTO}, stimiamo il \Acrshort*{RTT} prendendone un campione: $\operatorname{sampleRTT}$ dove prendiamo in considerazione il tempo tra l'invio del pacchetto e la ricezione dell'\Acrshort*{ACK}. Per mantenere il tempo aggiornato ma non troppo sensibile ai ``picchi'' che si possono verificare sulla rete viene utilizzata la seguente formula:
        \[ \operatorname{EstimatedRTT} = (1-\alpha) \cdot \operatorname{EstimatedRTT} + \alpha \cdot \operatorname{SampleRTT} \]
        Dove $ \alpha $ è un parametro che indica la "sensibilità" del tempo, se $ \alpha $ è basso allora il tempo sarà poco sensibile ai picchi, se $ \alpha $ è alto allora il tempo sarà molto sensibile ai picchi, questa è una media mobile esponenziale ponderata. Solitamente $ \alpha = 0.125 $.\newline
        Per impostare \Acrshort*{RTO} non ci avvaliamo solamente di questo dato appena ricavato ma anche della deviazione standard del \Acrshort*{RTT} ovvero 
        $$ \operatorname{DevRTT} = (1-\beta) \cdot \operatorname{DevRTT} + \beta \cdot \left| \operatorname{SampleRTT} - \operatorname{EstimatedRTT} \right| $$
        Dove $ \beta $ è un parametro che indica la "sensibilità" della deviazione standard, se $ \beta $ è basso allora la deviazione standard sarà poco sensibile ai picchi, se $ \beta $ è alto allora la deviazione standard sarà molto sensibile ai picchi, questa è una media mobile esponenziale ponderata. Solitamente $ \beta = 0.25 $.\newline
        Infine il \Acrshort*{RTO} viene calcolato come: 
        $$ \text{RTO} = \operatorname{EstimatedRTT} + 4 \cdot \operatorname{DevRTT} $$
        In tutto questo il valore iniziale di $\operatorname{EstimatedRTT}$ è un margine di sicurezza deciso globalmente da parte di standard (\texttt{\Acrshort*{RFC} 6298}) secondo il quale al punto \texttt{2.1} di questo documento notiamo come: 
        \begin{quote}[2.2]{\texttt{\Acrshort*{RFC} 6298}}
            Until a round-trip time (RTT) measurement has been made for a segment sent between the sender and receiver, the sender SHOULD set RTO $\leftarrow 1$ second, though the "backing off" on repeated retransmission discussed in (5.5) still applies.
        \end{quote}
        Il che significa che finché non si è misurato il tempo di andata e ritorno tra mittente e ricevente il \Acrshort*{RTO} deve essere impostato ad $1$ secondo.
    \subsection[Controllo di flusso \texttt{RWND}]{Controllo di flusso \Acrshort*{RWND}}
        Il \Acrshort*{TCP} implementa un meccanismo di controllo di flusso per evitare che il mittente invii troppi dati al ricevente, questo meccanismo è basato sulla finestra di ricezione \Acrshort*{W_R}, il mittente non può inviare dati se la finestra di ricezione del ricevente è piena. La finestra di ricezione è un campo a $16$ bit, quindi la dimensione massima della finestra di ricezione è di $ 2^{16} - 1 = 65535 $ byte. Il mittente invia dati fino a $ \min(W_T, W_R) $, dove $ W_T $ è la finestra di trasmissione del mittente e $ W_R $ è la finestra di ricezione del ricevente. Il ricevente invia un segmento \Acrshort*{TCP} con il campo \texttt{Window} impostato alla dimensione della finestra di ricezione, il mittente legge questo campo e regola la dimensione della finestra di trasmissione in base a questo valore.
\section{Principi di controllo di congestione}
    Informalmente la congestione si può tradurre come ``troppi trasmettitori stanno mandando troppi dati e la \underline{\textbf{rete}} non riesce a gestirli''. Quindi il problema è nella rete e non nel ricevitore. Questo problema si può verificare come: pacchetti persi (\textit{buffer overflow}) o ritardi (\textit{queueing delay}). Il controllo di congestione è un insieme di tecniche che permettono di evitare che la rete vada in congestione. Il controllo di congestione è un problema molto complesso e non esiste una soluzione al problema, esistono però delle tecniche che permettono di mitigare il problema.
    \subsection{Cause/costi della congestione}
        \subsubsection{Scenario 1}
            Assumiamo di avere due trasmettitori e due ricevitori, un \textit{router} con una coda di dimensione infinita, la capacità del link in uscita è $ R $ e non possono esserci ritrasmissioni, allora il \textit{throughput} massimo per ogni trasmettitore è $ R/2 $, ma se entrambi i trasmettitori inviano dati contemporaneamente allora il ritardo salirà asintoticamente con l'avvicinarsi a $ R/2 $.
        \subsubsection{Scenario 2}
            Assumiamo di avere due trasmettitori e due ricevitori, un \textit{router} con una coda di dimensione finita, la capacità del link in uscita è $ R $ e il mittente ritrasmette i pacchetti in timeout, allora considerando il tasso di arrivo dall'applicazione del mittente $\lambda_{in}$ e il tasso percepito dal destinatario $\lambda_{out}$ e il fatto che il mittente invii dati solo quando il router ha spazio nel buffer allora ci troveremmo nel caso ideale ed abbiamo a disposizione un \textit{throughput} di $ R/2 $ per ogni trasmettitore. Se invece il mittente invia dati senza preoccuparsi dello stato del router invia e re-invia i pacchetti in caso di timeout allora per un input di $\lambda_{in}$ pari a $ R/2 $ il \textit{throughput} in uscita sarà di $ R/4 $ (per via delle ritrasmissioni).
\section[Controllo di congestione \texttt{TCP}]{Controllo di congestione \Acrshort*{TCP}}
    \paragraph{Alcune cose da dire} Innanzitutto bisogna dire che non esiste un solo algoritmo per il controllo di congestione, esistono infatti molte varianti, ognuna di queste è stata introdotta per rimuovere delle limitazioni della versione precedente. Inoltre l'implementazione di un algoritmo rispetto ad un altro dipende spesso dal sistema operativo. Tutte le implementazioni di \Acrshort*{TCP} ragionano in \textit{byte}.\footnote{All'interno del corso si è cercato di mantenere una notazione più ``umana'' e si è parlato di \textit{pacchetti} e \textit{segmenti} fin quando possibile ma in realtà \Acrshort*{TCP} ragiona in \textit{byte}.}
    \paragraph{Caratteristiche} Il controllo di congestione \textbf{adatta il tasso di trasmissione} sulla base delle condizioni della rete, inoltre lo scopo è quello di evitare di \textbf{saturare} e \textbf{congestionare} la rete. 
    \paragraph{Approcci possibili} Esistono due approcci possibili per il controllo di congestione: \begin{itemize}
        \item \textbf{Controllo di congestione \textit{end-to-end}} 
            \subitem Non coinvolge la rete
            \subitem Si capisce se c'è congestione osservando perdite di pacchetti o ritardi
            \subitem Metodo usato da \Acrshort*{TCP}
        \item \textbf{Controllo di congestione assistito dalla rete}
            \subitem I router forniscono feedback ai trasmettitori
            \subitem Un singolo bit per indicare la congestione
    \end{itemize}
    \subsection[\texttt{TCP CC}: \textit{additive increase multiplicative decrease} (\texttt{AIMD})]{\Acrshort*{TCPCC}: \acrfull*{AIMD}}
        \subsubsection{Approccio}
            Il mittente aumenta il tasso di trasmissione cercando di occupare la banda disponibile e diminuisce il tasso di trasmissione quando rileva una perdita. Questo algoritmo segue due passi fondamentali: \begin{itemize}
                \item \textbf{Additive Increase} Aumenta il tasso di trasmissione di 1 \Acrshort*{MSS} ogni \Acrshort*{RTT} finché non si rileva una perdita.
                \item \textbf{Multiplicative Decrease} riduce la finestra (tipicamente di un fattore $ \frac{1}{2} $) quando si rileva una perdita.
            \end{itemize}
        \subsubsection{Perché usare \Acrshort*{AIMD}}
            Per ottenere un \textit{fairness} tra i trasmettitori, infatti se $ k $ sessioni \Acrshort*{TCP} si dividono uno stesso \textit{link} ed è presente un \textit{bottleneck} allora la banda percepita da ogni trasmettitore sarà di $ \frac{R}{k} $.\newline
            Due sessioni in competizione sullo stesso \textit{link} di banda $ R $ allora poniamo un grafico sul quale l'asse delle ascisse è la banda percepita della sessione 1 e l'asse delle ordinate è la banda percepita della sessione 2. 
            \begin{figure}[H]
                \centering
                \includegraphics[width=0.5\textwidth]{03/congestione1.png}
                \caption{Grafico di congestione}
            \end{figure}
            Dal grafico si vede come nel tempo le connessioni oscillano verso il punto di intersezione, questo è dovuto al fatto che entrambe le connessioni aumentano la loro banda fino a che non si satura il link, a quel punto entrambe rilevano una perdita e riducono la loro banda dello stesso fattore, questo porta ad un \textit{fairness} tra le due connessioni.
    \subsection{Meccanismi per il controllo di congestione}
        Il controllo di congestione gestisce l'adattamento della cosiddetta finestra di congestione (\Acrshort*{CWND}$=$ numero di byte che il mittente può inviare). \newline
        In \Acrshort*{TCP} ci sono diversi algoritmi per il controllo di congestione, tra i più famosi ci sono: \begin{itemize}
            \item \textbf{In assenza di perdite}
                \subitem \Acrlong*{SS}
                \subitem \Acrlong*{CA}
            \item Per migliorare l'efficienza di \Acrshort*{TCP} in caso si verifichino perdite
                \subitem \Acrlong*{FRet}
                \subitem \Acrlong*{FRec}
        \end{itemize}
        \textbf{In ogni caso}: $\Acrshort*{W_T} = \min(\Acrshort*{CWND}, \Acrshort*{RWND}) = \min(\Acrshort*{CWND}, \Acrshort*{W_R}) $.
        \subsubsection{Slow Start}
            Il \Acrlong*{SS} è un algoritmo che prevede che per ogni \Acrshort*{ACK} ricevuto aumento di 1 \Acrshort*{MSS} la finestra di congestione, in questo modo la finestra aumenta esponenzialmente. Questo algoritmo è utilizzato quando la connessione è appena stata aperta e non si conosce la banda disponibile. Lo \Acrlong*{SS} termina quando la finestra di congestione raggiunge una soglia detta \acrfull*{SSThresh} e si passa al \textit{Congestion Avoidance}.
            \paragraph{Algoritmo della fase \Acrlong*{SS}}
            \begin{enumerate}
                \item \textbf{Inizializzazione} \begin{itemize}
                    \item \Acrshort*{CWND} = 1 \Acrshort*{MSS}
                    \item \Acrshort*{SSThresh} = \Acrshort*{RWND} (o \Acrshort*{RWND}/2)
                \end{itemize}
                \item \textbf{\Acrshort*{ACK} valido ricevuto}:\begin{itemize}
                    \item \Acrshort*{CWND} = \Acrshort*{CWND} + 1 \Acrshort*{MSS}
                    \item Sposto \Acrshort*{W_LOW} al primo segmento non \texttt{ACK'ato}
                    \item Se \Acrshort*{CWND} $ \geq $ \Acrshort*{SSThresh} allora passo alla fase di \textit{Congestion Avoidance}
                    \item Trasmetto nuovi segmenti (compresi tra \Acrshort*{W_LOW} e \Acrshort*{W_HIGH})
                \end{itemize}
                \item \textbf{Se scatta un \textit{timeout}}: \begin{itemize}
                    \item Abbasso \Acrshort*{SSThresh} a $\max(\Acrshort*{CWND}/2, 2)$
                    \item Aumento \Acrshort*{RTO} = $2\cdot \Acrshort*{RTO}$
                    \item Reimposto \Acrshort*{CWND} = 1 \Acrshort*{MSS}
                    \item Ritrasmetto il segmento in \textit{timeout} 
                \end{itemize}
            \end{enumerate}
            \subsubsection{\Acrlong*{CA}}
            Il \Acrlong*{CA} è un algoritmo che prevede che per ogni \Acrshort*{ACK} ricevuto aumento di $ \Acrshort*{MSS}\cdot \frac{\Acrshort*{MSS}}{\Acrshort*{CWND}} $ la finestra di congestione, in questo modo la finestra aumenta linearmente quando ``Trasmetto tutti i pacchetti della finestra attuale con successo''.
            Quindi per ogni \Acrshort*{RTT} in cui ricevo tutti gli \Acrshort*{ACK} attesi allora aumento di un segmento. Questo algoritmo segue un comportamento lineare e non esponenziale il che lo rende ideale per mantenere la rete stabile.
            \paragraph{Algoritmo della fase \Acrlong*{CA}}
            \begin{itemize}
                \item \textbf{Se ricevo un \Acrshort*{ACK} valido}:\begin{itemize}
                        \item \Acrshort*{CWND} = \Acrshort*{CWND} + $ \frac{\Acrshort*{MSS}}{\Acrshort*{CWND}} $ (in \underline{\textbf{byte}}!)
                        \item Sposto \Acrshort*{W_LOW} al primo segmento non \texttt{ACK'ato}
                        \item Trasmetto nuovi segmenti (compresi tra \Acrshort*{W_LOW} e \Acrshort*{W_HIGH})
                    \end{itemize}
                \item \textbf{Se scatta un \textit{timeout}}: \begin{itemize}
                        \item Passo alla fase di \Acrlong*{SS}
                        \item Abbasso $ \Acrshort*{SSThresh} = \max(\Acrshort*{CWND}/2, 2) $
                        \item Aumento \Acrshort*{RTO} = 2 \Acrshort*{RTO}
                        \item Reimposto \Acrshort*{CWND} = 1 \Acrshort*{MSS}
                        \item Ritrasmetto il segmento in timeout
                \end{itemize}
            \end{itemize}
        \subsubsection{Parametri i quali possono essere modificati}
            \begin{itemize}
                \item \acrfull*{CWND} -  Dimensione della finestra di congestione.
                \item \acrfull*{SSThresh} - Soglia di \Acrlong*{SS}.
                \item \acrfull*{RTO} - Tempo di ritrasmissione.
                \item \textbf{$\Acrshort*{W_LOW}\ \&\ \Acrshort*{W_HIGH}$} - Puntatori alla finestra di trasmissione.
            \end{itemize}
        \subsubsection{\acrfull*{FRet}}
            Il \Acrlong*{FRet} è un algoritmo facente parte della fase di \Acrlong*{CA} che prescrive la seguente condizione: se il mittente riceve tre \Acrshort*{ACK} duplicati allora ritrasmette il segmento richiesto dal \Acrshort*{ACK} duplicato, inoltre il fatto che il mittente riceva tre \Acrshort*{ACK} duplicati indica che c'è una congestione sulla rete si passa dunque alla fase di \textit{Fast Recovery}. Inoltre prendo in considerazione il valore di \texttt{RECOVER}$=\Acrshort*{W_HIGH}$ per determinare quanti segmenti sono stati trasmessi nella rete, in questo modo posso capire, una volta ricevuto un \Acrshort*{ACK} differente, quando e se il processo di \textit{Fast Retransmit} è terminato con successo.
        \subsubsection{\acrfull*{FRec}}
            Quando ricevo il 3° \Acrshort*{ACK} duplicato entro in \Acrlong*{FRec}, in questa fase avvengono diversi passaggi per evitare di saturare la rete: \begin{itemize}
                \item \textbf{Al 3° \Acrshort*{ACK} duplicato}:\begin{itemize}
                    \item \Acrshort*{SSThresh} = \Acrshort*{CWND}/2
                    \item Supponendo di aver perso solo il segmento in questione: \Acrshort*{CWND} = \Acrshort*{SSThresh} + 3 \Acrshort*{MSS}
                    \item Non sposto \Acrshort*{W_LOW}
                \end{itemize}
                \item \textbf{Se arrivano altri \Acrshort*{ACK} duplicati allora}: \begin{itemize}
                    \item \Acrshort*{CWND} = \Acrshort*{CWND} +1 \Acrshort*{MSS}
                    \item Non sposto \Acrshort*{W_LOW}
                \end{itemize}
                \item \textbf{Quando arriva un \Acrshort*{ACK} valido} (che comprende \texttt{RECOVER}): \begin{itemize}
                    \item \Acrshort*{CWND} = \Acrshort*{SSThresh}
                    \item Passo alla fase di \Acrlong*{CA}
                    \item Sposto \Acrshort*{W_LOW} al primo segmento non \texttt{ACK'ato}
                \end{itemize}
                \item \textbf{Se arriva un \Acrshort*{ACK} che \underline{non} comprende \texttt{RECOVER}}:\begin{itemize}
                    \item Ritrasmetto il primo segmento non \texttt{ACK'ato}
                    \item \Acrshort*{CWND} = \Acrshort*{CWND}-(numero di segmenti senza \Acrshort*{ACK})+1
                    \item Sposto \Acrshort*{W_LOW} al primo segmento non \texttt{ACK'ato}
                \end{itemize}
            \end{itemize}
            
            \begin{figure}[H]
                \centering
                \includegraphics[width=0.6\textwidth]{03/macchinaStatiTCP.png}
                \caption{Macchina a stati di \Acrshort*{TCP}}
            \end{figure}
        \subsubsection{Problemi di equità (\textit{fairness}) in \Acrshort*{TCP}}
            In quanto le applicazione multimediali usano di rado \Acrshort*{TCP} per la trasmissione (ma usano \Acrshort*{UDP}), queste non sono soggette ai controlli di congestione, quindi le connessioni \Acrshort*{TCP} sono penalizzate. Questo avviene in quanto le connessioni \Acrshort*{UDP} trasmettono a velocità costante indipendentemente da fattori esterni.\newline
            Se invece abbiamo due \textit{host} che usano \Acrshort*{TCP} e uno apre (ad es.) $ 9 $ connessioni mentre l'altro ne apre $ 1 $ allora il primo avrà un \textit{throughput} di $ \frac{R}2 $ mentre il secondo avrà un \textit{throughput} di $ \frac{R} {10} $.
    \subsection{Altri algoritmi più recenti per il controllo di congestione}
        Negli algoritmi visti fino ad ora il processo di "rallentamento" della trasmissione avveniva solo in caso di perdita di pacchetti, questo però permette di regolare la banda solo quando è troppo tardi, in quanto deve avvenire una perdita prima che il mittente si renda conto che c'è una congestione. Per ovviare a questo problema sono stati introdotti nuovi algoritmi che permettono di regolare la banda. Questi algoritmi sono: \begin{itemize}
            \item \acrfull*{CUBIC}
            \item \acrfull*{BBR}
            \item \acrfull*{QUIC}
        \end{itemize}
        \subsubsection{\acrfull*{CUBIC}}
            L'algoritmo \Acrshort*{CUBIC} fa variare la lunghezza della finestra di congestione secondo una funzione cubica nel tempo, questo ne migliora la scalabilità e la stabilità. Questo algoritmo è stato introdotto nel kernel di \texttt{Linux} a partire dalla versione 2.6.19, mentre in \texttt{Windows} è stato introdotto a partire dal 2017.
            \paragraph{Principi del funzionamento} Per un migliore utilizzo e stabilità della rete \Acrshort*{CUBIC} usa sia sia la parte concava che quella convessa della funzione cubica per regolare la finestra di congestione. 
            $$
                \Acrshort*{CWND}_{cubic}(t) = C(t-K)^3 + \Acrshort*{CWND}_{max}
            $$
            Dove $ C $ è una costante, $ K = \sqrt[3]{\frac{\Acrshort*{CWND}_{max}(1-\beta)}{C}} $ e $ \Acrshort*{CWND}_{max} $ è la dimensione massima della finestra di congestione. Per lo standard \texttt{\Acrshort*{RFC} 8312} $ C = 0.4 $ e $ \beta = 0.7 $, ma dopo che è stata rilevata una congestione allora $ \beta = 0.5 $. Inoltre questo algoritmo è "\Acrshort*{TCP}\textit{-friendly}" ovvero non penalizza i flussi \Acrshort*{TCP} legacy che condividono la stessa rete.
        \subsubsection{\acrfull*{BBR}}
            L'algoritmo \Acrshort*{BBR} è un algoritmo di controllo di congestione che cerca di massimizzare il \textit{throughput} e minimizzare il ritardo. Questo algoritmo è stato introdotto da \texttt{Google} nel 2016 e si basa non sul rilevamento di perdite ma su due parametri: \begin{itemize}
                \item \textbf{Bottleneck Bandwidth} La banda disponibile sul \textit{bottleneck}.
                \item \textbf{Round-trip propagation time} Il tempo di propagazione del pacchetto.
            \end{itemize}
            Il funzionamento a grandi linee prevede la trasmissione di pacchetti ad una velocità che non \textit{dovrebbe} saturare la rete. Questo infatti è progettato per ridurre la finestra di congestione prima che si verifichi una perdita, in questo modo si dovrebbe riuscire a limitare ritrasmissioni inutili. Un vantaggio di \Acrshort*{BBR} è quello che solo il \textit{server} lo deve implementare e non anche il \textit{client}. Il concetto usato è quello di \textit{pacing} ovvero inserisco nuovi pacchetti nella \Acrshort*{CWND} solo quando il nodo più lento della rete è pronto a riceverli.
            \paragraph{Migliore produttività} Secondo \textit{Google} \Acrshort*{BBR} con un \textit{link} a $10$ Gbps che invia dati lungo un percorso con \Acrshort*{RTT} di $100$ms con tasso di perdita dell'$1\%$ riesce a raggiungere un \textit{throughput} di $3,3$Mbit/s con \Acrshort*{CUBIC} e di $9100$Mbit/s con \texttt{BBR}. Questo è ideale nel caso di connessioni \texttt{\Acrshort*{HTTP}/2} che sfruttano una singola connessione per trasmettere dati.
            \paragraph{Latenza inferiore} \Acrshort*{BBR} riesce a mantenere una latenza inferiore rispetto a \Acrshort*{CUBIC} in quanto riesce a mantenere la banda costante e non satura la rete. Vari studi (sempre di \textit{Google}) hanno dimostrato che su un collegamento di $10$ Mbps con \Acrshort*{RTT} di $40$ ms ed un \textit{bottleneck} di $1000$ pacchetti la latenza di \Acrshort*{BBR} è di soli $43$ ms contro i $1090$ ms di \Acrshort*{CUBIC}.
        \subsubsection{\acrfull*{QUIC}}
            \Acrshort*{QUIC} è un protocollo di trasporto sviluppato da \texttt{Google} nel 2012 e si prefissa il raggiungimento di due obbiettivi:\begin{itemize}
                \item Evitare fenomeni di \textit{head-of-line blocking}
                \item Ridurre la latenza di \Acrshort*{TCP}
            \end{itemize}
            \Acrshort*{QUIC} può essere implementato a livello applicazione, oltre che a livello di \textit{kernel}. Lo \textit{use case} di questo dovrebbe essere quello delle connessioni \texttt{\Acrshort*{HTTP}/3}. Il principio di funzionamento di questo è che i pacchetti vengono trasmessi tramite una connessione \Acrshort*{UDP} e non \Acrshort*{TCP}, questo permette di evitare i problemi di \textit{head-of-line blocking} in quanto se un pacchetto viene perso allora non si bloccano tutti i pacchetti successivi. Inoltre \Acrshort*{QUIC} permette di ridurre l'\textit{overhead} di connessione in quando incorpora in se stesso lo scambio delle chiavi (o \textit{handshake}) di \Acrshort*{TLS}.
    \subsection{Conclusioni}
        \subsubsection{Meglio dunque \Acrshort*{TCP} o \Acrshort*{UDP}?}
        La scelta tra \Acrshort*{TCP} e \Acrshort*{UDP} dipende da cosa si vuole fare, se si vuole trasmettere dati in modo affidabile e si vuole evitare di saturare la rete allora si deve usare \Acrshort*{TCP}, se invece si vuole trasmettere dati in modo veloce e non si vuole preoccuparsi di perdite di pacchetti allora si deve usare \Acrshort*{UDP}. Questa scelta però non è così libera come sembra, in quanto se si vuole usare il protocollo \Acrshort*{QUIC} necessitiamo di connessione \Acrshort*{UDP} ma molta della nostra infrastruttura blocca le connessioni di questo tipo in quanto non avviene un controllo di congestione e quindi si rischia di saturare la rete. Google ha provato a mostrare come \Acrshort*{QUIC} sia migliore di \Acrshort*{TCP} cercando di "sbloccare" la rete per questo tipo di connessioni, detto ciò i prodotti della serie \textit{chromium} aprono in contemporanea una connessione \Acrshort*{TCP} e una connessione \Acrshort*{UDP} e scelgono quella che ha il \textit{throughput} migliore. 
        \subsubsection{Cambio di rete}
            Con le connessioni \Acrshort*{TCP} le \textit{socket} vengono identificate dalla quadrupla: \texttt{(\Acrshort*{IP} M.,\Acrshort*{IP} D., Porta M., Porta D.)}, se si cambia rete allora si cambia anche l'indirizzo \Acrshort*{IP} e quindi la connessione \Acrshort*{TCP} viene persa. Con \Acrshort*{QUIC} invece la connessione viene mantenuta in quanto la \textit{socket} è identificata da un \texttt{ID} e non dall'indirizzo \Acrshort*{IP}.
    \chapter{Il livello di rete}
\label{cap:livelloRete}
\thispagestyle{chapterInit}
\section{Visione d'insieme}
    \paragraph{Obbiettivo del livello di rete} L'obbiettivo principale del livello di rete è quello di permettere la comunicazione tramite reti diverse attraverso apparecchi detti \textit{router} i quali hanno il compito di inoltrare le informazioni verso la destinazione.
    \paragraph{Funzioni principali} Esistono due funzioni principali del livello di rete: \begin{description}
        \item[Inoltro \textit{forwarding}] Questa è una operazione a livello locale che consiste nel prendere un pacchetto in ingresso e inoltrarlo verso l'uscita corretta.
        \item[Instradamento \textit{routing}] Questa è una operazione a livello globale che consiste nel determinare il percorso migliore per inoltrare un pacchetto verso la destinazione. Per questa operazione si utilizzano degli algoritmi di \textit{routing}.
    \end{description}
    Queste due funzioni sono legate tra loro, ma possono essere isolate. Infatti convenzionalmente distinguiamo con \textit{control plane} la parte del livello di rete che si occupa dell'istradamento e con \textit{data plane} la parte che si occupa dell'inoltro. Questa distinzione è utile per capire come funzionano i router.
    \subparagraph{\textit{Data Pane}} Il \textit{data plane} ha funzione a livello locale ad ogni \textit{router}, questo è il livello che determina \underline{come} inoltrare un \textit{datagram} fornendo la funzione di \textit{forwarding}.
    \subparagraph{\textit{Control Plane}} Il \textit{control plane} ha funzione a livello globale, questo è il livello che determina \underline{dove} inoltrare un \textit{datagram} fornendo la funzione di \textit{routing}.
\section{Come è fatto un router}
    Visto a livello "alto" un router è composto da tre livelli principali: \begin{description}
        \item[Terminazione di linea] Questo è il livello più basso del router, è composto da un'interfaccia di rete che si occupa di ricevere i pacchetti e di inviarli al livello successivo.
        \item[Protocollo di livello \textit{data link}] Questo livello si occupa di ricevere i pacchetti dal livello precedente e di inviarli al livello successivo. Inoltre si occupa di fare il controllo degli errori e di gestire il flusso.
        \item[Inoltro e \textit{buffer}] Questo è il livello che si occupa di inoltrare i pacchetti verso la destinazione. Inoltre si occupa di fare il \textit{buffering} dei pacchetti in caso di congestione.
    \end{description}
    \subsection{Sistemi di commutazione}
        I sistemi di commutazione trasferiscono i pacchetti dalle porte di ingresso all'uscita appropriata. Definiamo come \textbf{tasso di comunicazione} la frequenza alla quale i pacchetti vengono portati dall'ingresso all'uscita (spesso è un multiplo della velocità di comunicazione) 
        \subsubsection{Commutazione a memoria}
            Questo è il metodo più semplice, i pacchetti vengono memorizzati in un buffer comune e poi inoltrati verso l'uscita appropriata. Questo metodo è molto semplice ma ha il problema che la velocità di inoltro è limitata dalla velocità di accesso alla memoria.
        \subsubsection{Commutazione a bus}
            Questo metodo consiste nel collegare le porte di ingresso e di uscita tramite un bus, sempre comune a tutte le porte. Questo metodo risulta lento in quanto non possono essere trasferiti più pacchetti contemporaneamente anche se le porte di ingresso e di uscita sono diverse. Il \texttt{Cisco 5600} è un esempio di router che utilizza questo metodo e riesce a trasferire fino a 32 Gbit/s.
        \subsubsection{Commutazione a matrice}
            Questo metodo consiste nel collegare le porte di ingresso e di uscita tramite una matrice di commutazione. Questo metodo è molto veloce in quanto permette di trasferire più pacchetti contemporaneamente in quanto se le porte sono differenti allora basta attivare i vari collegamenti della matrice. Questo metodo è molto veloce ed ispirato ai primi commutatori telefonici. Il \texttt{Cisco 12000} è un esempio di router che utilizza questo metodo e riesce a trasferire fino a 60 Gbit/s.
    \subsection{Accodamenti}
        Gli accodamenti sono utilizzati per evitare la perdita di pacchetti in caso di congestione dell'apparecchio di rete. Questo è un problema molto comune in quanto i router sono dispositivi molto veloci e le porte di uscita sono molto più lente. Per evitare la perdita di pacchetti si utilizzano delle code che permettono di memorizzare i pacchetti in attesa di essere inoltrati.
        Le code possono essere formate in ingresso, quando una stessa porta di uscita è condivisa da più porte di ingresso e quindi una porta di uscita può essere congestionata. Le code possono essere formate in uscita, quando una porta di uscita ha un \textit{link} più lento rispetto alla velocità di inoltro dei pacchetti tramite le porte di ingresso o il commutatore di pacchetto.
        \paragraph{Quanta memoria serve per i \textit{buffer}} Secondo \texttt{RFC 3439} la quantità di memoria necessaria per i buffer è data dalla formula: \[M = \frac{RTT \cdot C}{\sqrt{N}}\] Dove: \begin{description}
            \item[$M$] è la memoria necessaria per il buffer
            \item[$RTT$] è il tempo di round trip
            \item[$C$] è la capacità del collegamento
            \item[$N$] è il numero di connessioni
        \end{description}
        \paragraph{Meccanismi di \textit{scheduling}}
            I meccanismi di \textit{scheduling} sono utilizzati per decidere quale pacchetto inoltrare quando si ha la possibilità di inoltrare più pacchetti. Il meccanismo più semplice è il \textit{First In First Out} (\texttt{FIFO}) che inoltra i pacchetti in ordine di arrivo. Inoltre viene applicata una politica di scarto dei pacchetti in caso di buffer pieno. Questa politica può essere: \begin{description}
                \item[\textit{Drop Tail}] Questa politica scarta i pacchetti in arrivo quando il buffer è pieno.
                \item[\textit{Random Early Detection}] Questa politica scarta i pacchetti in arrivo in modo casuale quando il buffer è pieno.
                \item[\textit{Priority Drop}] Questa politica scarta i pacchetti in arrivo in base alla priorità.
            \end{description}
            La politica di scarto dei pacchetti dipende dall'implementazione del router.
\section{Il protocollo \texttt{IP}}
    \subsection{Il formato del \textit{datagram} \texttt{IP} (\texttt{IPv4})}
        \begin{figure}[H]
            \centering
            \includegraphics[width=0.36\textwidth]{04/datagramIPv4.png}
            \caption{Il formato del \textit{datagram} \texttt{IP} (\texttt{IPv4})}
            \label{fig:IPv4Header}
        \end{figure}
        Il formato del \textit{datagram} \texttt{IP} è composto da 20 byte di intestazione e da un campo dati. Il campo dati può contenere fino a 65.535 byte.  Di seguito si riportano i vari campi dell'intestazione:
        \begin{description}
            \item[VER] (4 bit) Questo campo contiene la versione del protocollo \texttt{IP} utilizzato.
            \item[Lunghezza \textit{header}] (4 bit) Questo campo contiene la lunghezza dell'intestazione in parole da 32 bit. (=5 se non ci sono opzioni)
            \item[Tipo di servizio - \texttt{ToS}] (8 bit) Questo campo contiene informazioni sul tipo di servizio richiesto.
            \item[Lunghezza totale] (16 bit) Questo campo contiene la lunghezza totale del \textit{datagram} in byte.
            \item[Identificativo] (16 bit) Questo campo contiene un numero univoco per il \textit{datagram}.
            \item[Flag] (3 bit) Questo campo contiene i flag per il frammento. Il primo bit è il bit di \textit{Don't Fragment}, il secondo bit è il bit di \textit{More Fragment} e il terzo bit è il bit di \textit{Fragment Offset}.
            \item[Offset] (13 bit) Questo campo contiene l'offset del frammento. (Espresso in multipli di 8 byte)
            \item[\textit{Time To Live} - \texttt{TTL}] (8 bit) Questo campo contiene il numero di \textit{hop} massimo che il \textit{datagram} può fare.
            \item[Protocollo] (8 bit) Questo campo contiene il protocollo di trasporto che si trova nel campo dati.
            \item[Checksum] (16 bit) Questo campo contiene il checksum dell'intestazione.
            \item[Indirizzo IP sorgente] (32 bit) Questo campo contiene l'indirizzo IP sorgente.
            \item[Indirizzo IP destinazione] (32 bit) Questo campo contiene l'indirizzo IP destinazione.
            \item[Opzioni] (variabile) Questo campo contiene le opzioni del \textit{datagram}.
            \item[Padding] (variabile) Questo campo contiene il padding per allineare l'intestazione a multipli di 32 bit.
        \end{description}
    \subsection{\texttt{MTU} e Frammentazione}
        La \texttt{MTU} (\textit{Maximum Transmission Unit}) è la dimensione massima di un pacchetto che può essere trasmesso su un collegamento, ogni \textit{hardware} specifica il proprio \texttt{MTU}. Se un \textit{datagram} è più grande della \texttt{MTU} allora il \textit{datagram} viene frammentato in pacchetti più piccoli. Questo processo è chiamato \textit{frammentazione}. I pacchetti frammentati vengono poi ricomposti alla destinazione.
        \paragraph{Valori Standard \texttt{MTU}} Per alcuni tipi di collegamenti sono stati definiti dei valori standard di \texttt{MTU}. Ad esempio per le reti Ethernet la \texttt{MTU} è di $1500$ byte, per le reti \texttt{WLAN 802.11} la \texttt{MTU} è di $2304$ byte,\dots. In un collegamento tra due \textit{host} possono essere presenti due valori di \texttt{MTU} diversi
        \subparagraph{Esempio} Supponendo che per raggiungere un host \texttt{B} si debba passare prima da una rete con $1500$ byte di \texttt{MTU} e poi da una rete con $1000$, tutto ciò tramite un router \texttt{R}. Allora quando il \textit{router} \texttt{R} riceve il datagram di $1500$ byte lo frammenta in due pacchetti di $1000$ byte e $500$ byte. I pacchetti vengono quindi inoltrati alla destinazione. Quando l'host \texttt{B} riceve i pacchetti li ricomponi e li passa al livello di trasporto.
        
    \subsection{Indirizzamento e \texttt{NAT}}
        \subsubsection{Indirizzi \texttt{IP}}
            Un indirizzo \texttt{IP} è composto da $32$ bit ed è associato ad un'interfaccia di rete. Una interfaccia è una connessione tramite mezzo fisico o logico, solitamente ogni \textit{host} ha una o più interfacce di rete.
            \paragraph{Caratteristiche indirizzi \texttt{IP}} Gli host e i router devono usare le stesse convenzioni per gli indirizzi \texttt{IP}, inoltre ogni indirizzo \texttt{IP} deve essere unico e raggiungibile da un qualsiasi punto di internet. Quando si invia un pacchetto \texttt{IP} si invia l'indirizzo \texttt{IP} sorgente e l'indirizzo \texttt{IP} destinazione. I \textit{router} sono apparati di rete che quando ricevono un pacchetto \texttt{IP} decidono dove inoltrarlo in base all'indirizzo \texttt{IP} di destinazione.
            \paragraph{Norazione indirizzo \texttt{IP}} Gli indirizzi \texttt{IP} sono composti da $4$ gruppi di $8$ bit e sono scritti in notazione decimale a punti. Ogni gruppo di $8$ bit è espresso in decimale e separato da un punto. Gli indirizzi disponibili vanno dà: $0.0.0.0$ a $255.255.255.255$ 
            \paragraph{Gerarchie indirizzi \texttt{IP}} Gli indirizzi \texttt{IP} sono organizzati in una struttura gerarchica. Inoltre solitamente sono divisi in due parti: \begin{description}
                \item[Parte di rete] Questa parte identifica la rete a cui appartiene l'indirizzo \texttt{IP}.
                \item[Parte di \textit{host}] Questa parte identifica l'\textit{host} all'interno della rete.
            \end{description}
            \subparagraph{Classi di indirizzi \texttt{IP}} Gli indirizzi \texttt{IP} sono divisi in classi sulla base dei primi bit dell'indirizzo e dalla lunghezza del prefisso: \begin{description}
                \item[Classe \texttt{A}] Gli indirizzi di classe \texttt{A} hanno il primo bit a \texttt{0} e sono composti da $8$ bit di rete e $24$ bit di \textit{host}. Gli indirizzi vanno da $0.0.0.0$ a $ 127.255.255.255$ e sono riservati per le reti molto grandi.
                \item[Classe \texttt{B}] Gli indirizzi di classe \texttt{B} hanno i primi due bit a \texttt{10} e sono composti da $16$ bit di rete e $16$ bit di \textit{host}. Gli indirizzi vanno da $128.0.0.0$ a $191.255.255.255$ e sono riservati per le reti di medie dimensioni.
                \item[Classe \texttt{C}] Gli indirizzi di classe \texttt{C} hanno i primi tre bit a \texttt{110} e sono composti da $24$ bit di rete e $8$ bit di \textit{host}. Gli indirizzi vanno da $192.0.0.0$ a $223.255.255.255$ e sono riservati per le reti di piccole dimensioni.
                \item[Classe \texttt{D}] Gli indirizzi di classe \texttt{D} hanno i primi quattro bit a \texttt{1110} e sono riservati per i \textit{multicast}.
                \item[Classe \texttt{E}] Gli indirizzi di classe \texttt{E} hanno i primi quattro bit a \texttt{1111} e sono riservati per usi futuri.
            \end{description}
        \subsubsection{Assegnazione indirizzi \texttt{IP}}
            Gli indirizzi \texttt{IP} sono assegnati dalla \texttt{ICANN} (\textit{Internet Corporation for Assigned Names and Numbers}) che riserva una intera classe ai \texttt{ISP} (\textit{Internet Service Provider}) e poi questi assegnano gli indirizzi ai propri clienti. Gli indirizzi \texttt{IP} sono assegnati in modo gerarchico e quindi un \texttt{ISP} può assegnare un intero blocco di indirizzi ad un altro \texttt{ISP} e questo può assegnare un blocco di indirizzi ad un altro \texttt{ISP} e così via.
    \subsection{Indirizzamento \textit{classless}}
        In quanto ci si è accorti che con la suddivisione degli indirizzi in classi si stava sprecando molti indirizzi, si è deciso di passare ad un indirizzamento \textit{classless}. Questo tipo di indirizzamento permette di avere una suddivisione più flessibile degli indirizzi, in quanto possiamo richiedere al nostro \texttt{ISP} solo un \textit{range} di indirizzi e non una classe intera, ad esempio l'\texttt{ISP} può usare un prefisso di $26$ bit per identificare la rete nel mondo e successivamente usare i restanti $6$ bit per identificare tutti gli \textit{host} della rete.
        In questa situazione se un \texttt{ISP} ha acquistato una rete di classe \texttt{C} e ha bisogno di dividere la rete in quattro clienti allora può dividere la rete in $4$ sotto-reti e assegnare un prefisso di $26$ bit ($24$ per la classe \texttt{C} e $2$ per le sotto-reti) e i restanti $6$ bit per gli \textit{host} di ogni sotto-rete.
        \subsubsection{Maschere di rete}
            In quanto ora non si ha più una suddivisione fissa degli indirizzi, si è deciso di introdurre le \textit{maschere di rete}. Queste maschere sono composte da $32$ bit e sono composte da una parte di $1$ quelli che identificano il prefisso di rete e da una parte di $0$ che identificano gli \textit{host} della rete. Ad esempio la maschera di rete per una rete di classe \texttt{C} è $11111111.11111111.11111111.00000000$.
            \paragraph{Peché si usa?} La maschera di rete viene usata per identificare la rete di appartenenza di un indirizzo IP. Per fare ciò si fa un'operazione di \texttt{AND} tra l'indirizzo IP di destinazione e la maschera di rete. Se il risultato è uguale all'indirizzo di rete allora l'indirizzo IP appartiene alla rete. Quindi se il prefisso di rete è $128.10.0.0$ ovvero: \[10000000.00001010.00000000.00000000\] e la maschera di rete è $255.255.0.0$ ovvero: \[11111111.11111111.00000000.00000000\] e l'indirizzo IP di destinazione di un determinato pacchetto è: $128.10.2.3$ ovvero: \[10000000.00001010.00000010.00000011\] allora facendo l'operazione di \texttt{AND} tra l'indirizzo IP e la maschera di rete si ottiene: \[\begin{aligned}
                10000000.00001010.00000010.00000011 &\\
                11111111.11111111.00000000.00000000 &\\
                \hline
                10000000.00001010.00000000.00000000&=128.10.0.0
            \end{aligned}
            \] Quindi l'indirizzo IP appartiene alla rete e il pacchetto viene inoltrato a quella determinata porta di uscita.
        \subsubsection{Notazioni \texttt{CIDR}}
            \texttt{CIDR} o \textit{Classless Inter-Domain Routing} è un metodo per rappresentare le maschere di rete. Questo metodo consiste nel rappresentare la maschera di rete con un prefisso di bit. Ad esempio la maschera di rete $/24$ è uguale alla maschera di rete $11111111.11111111.11111111.00000000$, ovvero una maschera di rete di classe \texttt{C}, quindi i primi $3$ gruppi di bit sono appartenenti all'\textit{network-id} e l'ultimo gruppo di bit è appartenente all'\textit{host-id}. La notazione prevede inoltre che gli indirizzi \texttt{IP} siano rappresentati nel seguente modo: \texttt{ddd.ddd.ddd.ddd/m} dove ogni singolo \texttt{d} rappresenta un gruppo di bit e \texttt{m} rappresenta il numero di bit del prefisso di rete.
            \paragraph{Esempio} L'indirizzo $193.168.32.199/26$ ha un prefisso di rete di $26$ bit, quindi il \textit{network-id} è: $$11000001.1010100.000000010.11$$ e l'\textit{host-id} è $$00111$$ Esistono dunque $2^6=64$ indirizzi \texttt{IP} nella rete.
            \paragraph{Inoltro con \texttt{CIDR}} L'inoltro con \texttt{CIDR} è molto semplice e non differisce dall'inoltro con le classi. Infatti si fa l'operazione di \texttt{AND} tra l'indirizzo \texttt{IP} di destinazione e la maschera di rete e si confronta il risultato con l'indirizzo di rete. Se il risultato è uguale all'indirizzo di rete allora l'indirizzo \texttt{IP} appartiene alla rete e il pacchetto viene inoltrato alla porta di uscita corretta. Nella situazione in cui ci siano più reti che corrispondono al risultato dell'operazione di \texttt{AND} allora si sceglie la rete con il prefisso più lungo in quanto si presume che questa sia la rete più specifica e quindi la più breve.
            \paragraph{Aggregazione dei percorsi} L'aggregazione dei percorsi è una tecnica che permette di ridurre il numero di percorsi che un router deve memorizzare. Questa tecnica consiste nel raggruppare più reti in un'unica rete più grande. Questa tecnica è molto utile in quanto permette di ridurre il numero di percorsi che un router deve memorizzare e quindi di velocizzare l'inoltro dei pacchetti. Esempio se un \texttt{ISP} controlla tutte le reti $200.23.16.0/23$, $200.23.18.0/23$\dots $200.23.30.0/23$ allora può raggruppare tutte queste reti in un'unica rete $200.23.16.0/20$
            \paragraph{Inoltro di default} L'inoltro di default è una tecnica "\textit{last resource}" usata nel caso in cui non si trovi nessuna corrispondenza tra l'indirizzo \texttt{IP} di destinazione e le reti memorizzate nel router. In questo caso il router inoltra il pacchetto alla porta di uscita di default definita dal router e che solitamente è la porta di uscita verso internet. In questo caso nel router è presente una rotta con \texttt{IP} di destinazione $0.0.0.0$ e maschera di rete $/0$, la cui combinazione è sempre vera per qualunque indirizzo, per la regola del "prefisso più lungo" il router inoltra il pacchetto alla porta di uscita di default solo se non trova nessuna corrispondenza tra l'indirizzo \texttt{IP} di destinazione e le reti memorizzate nel router.
        \subsubsection{Tabella di routing}
            Nella tabella di routing oltre all'associazione "maschera di rete - porta di uscita" è necessario memorizzare anche l'indirizzo \texttt{IP} del prossimo \textit{router} a cui inoltrare il pacchetto, questo in quanto sapere che il pacchetto deve essere inoltrato ad una determinata porta di uscita non è sufficiente se fossero presenti più dispositivi connessi alla stessa porta di uscita.
    \subsection{Tipi di indirizzi}
        In quanto gli indirizzi \texttt{IP} disponibili sono $2^{32}$ e il numero di dispositivi connessi a internet è molto maggiore, si è deciso di introdurre dei tipi di indirizzi pubblici e privati, in modo da risparmiare indirizzi \texttt{IP} pubblici e di proteggere la rete interna da attacchi esterni.
        \subsubsection{Indirizzi pubblici e privati}
            Gli indirizzi \texttt{IP} sono divisi in due categorie: \begin{description}
                \item[Indirizzi pubblici] Gli indirizzi pubblici sono indirizzi che possono essere raggiunti da qualsiasi punto di internet. Questi indirizzi sono assegnati dalla \texttt{ICANN} e sono unici.
                \item[Indirizzi privati] Gli indirizzi privati sono indirizzi che non possono essere raggiunti da internet. Questi indirizzi sono riservati per le reti private e non possono essere usati per comunicare con internet, vengono bloccati dai router. Gli indirizzi privati sono: \begin{itemize}
                    \item Da \texttt{10.0.0.0} a \texttt{10.255.255.255} (\texttt{10.0.0.0/8})
                    \item Da \texttt{172.16.0.0} a \texttt{172.31.255.255} (\texttt{172.16.0.0/12})
                    \item Da \texttt{192.168.0.0} a \texttt{192.168.255.255} (\texttt{192.168.255.255/16})
                \end{itemize}
            \end{description}
        \subsubsection{\textit{Network Area Translation} \texttt{NAT}}
            Il \texttt{NAT} è una tecnica che permette di tradurre gli indirizzi privati in indirizzi pubblici e viceversa. Questo permette ai dispositivi di una rete privata di accedere a internet senza avere un indirizzo \texttt{IP} pubblico. Il \texttt{NAT} è un protocollo appartenente al \textit{router} funzionante nel seguente modo:
            \begin{enumerate}
                \item Il \textit{router} sostituisce l'indirizzo \texttt{IP} di sorgente e la porta di sorgente del pacchetto con il proprio indirizzo \texttt{IP} pubblico e una porta casuale.
                \item Il \textit{router} memorizza l'associazione tra l'indirizzo \texttt{IP} e porta originale con l'indirizzo \texttt{IP} pubblico e la porta generata.
                \item Viene inoltrato il pacchetto alla rete di destinazione, seguendo la tabella di routing.
                \item Quando il pacchetto di risposta arriva al \textit{router}, il \textit{router} sostituisce l'indirizzo \texttt{IP} di destinazione e la porta di destinazione con l'indirizzo \texttt{IP} privato e la porta originale.
                \item Il \textit{router} inoltra il pacchetto alla rete privata.
            \end{enumerate}
            \paragraph{Vantaggi/svantaggi del \texttt{NAT}} In quanto il numero di porta è costituito da $16$ bit, il numero di porte disponibili è $2^{16}=65536$, quindi il \texttt{NAT} permette di avere fino a $65536$ dispositivi connessi alla stessa rete privata. Il "problema" è che il protocollo \texttt{NAT} viola l'architettura a livelli, in quanto dispositivo di rete il \textit{router} non dovrebbe agire sulle porte. La mancanza di \texttt{IP} dovrebbe essere risolta con l'introduzione di \texttt{IPv6} (anche se lentamente). La sola esistenza di \texttt{NAT} deve essere tenuta in considerazione quando si progettano applicazioni (come una rete \texttt{P2P} che non funziona con \texttt{NAT}). Infine se si vuole accedere ad un dispositivo con \texttt{NAT} da internet è necessario usare altri protocolli come \textit{Port Forwarding}, \texttt{UPnP} o altri.
            \subparagraph{\texttt{NAT} può essere utile} Oltre a risparmiare indirizzi \texttt{IP} pubblici, il \texttt{NAT} può essere utile per ovviare a problemi di routing, assumiamo per esempio che un router (al quale non abbiamo accesso) imposti una regola che impedisca l'uscita di pacchetti verso la nostra rete "interna" \texttt{B}, ma permetta che pacchetti verso la rete "pubblica" \texttt{P} vengano inoltrati. In questo caso il \texttt{NAT} può essere utile per far passare i pacchetti dalla rete "interna" \texttt{B} alla rete "interna" \texttt{A} senza modificare le regole di routing. Questo grazie al fatto che il \texttt{NAT} modifica l'indirizzo \texttt{IP} di sorgente, il router non riconosce i pacchetti come provenienti dalla rete "interna" \texttt{A} e quindi li inoltra.
        \subsubsection{Indirizzi \texttt{IP} speciali}
            Gli indirizzi \texttt{IP} speciali sono indirizzi che non possono essere assegnati ad un'interfaccia di rete. Questi indirizzi sono usati per scopi speciali e non possono essere usati per comunicare con internet. Alcuni tipi di indirizzi speciali sono: \begin{itemize}
                \item Indirizzi che identificano tutta la rete
                \item Indirizzi che permettono il \textit{broadcast} a tutti gli \textit{host} della rete
                \item Indirizzi che permettono il \textit{broadcast} in una rete locale (\textit{Limited broadcast address})
                \item Indirizzi di \textit{localhost}
                \item Indirizzi di \textit{loopback}
                \item Indirizzi di \textit{multicast}
                \item Indirizzi di \textit{link-local}
            \end{itemize}
            \paragraph{Identificativi di tutta la rete} Gli indirizzi che identificano tutta la rete sono indirizzi che identificano tutta la rete. Questi indirizzi sono usati per identificare la rete e non possono essere assegnati ad un'interfaccia di rete, questi sono identificati con tutti i bit della parte di \textit{host} a $0$. Ad esempio l'indirizzo \texttt{128.211.0.16/28} identifica tutta la rete
            \paragraph{Indirizzi di \textit{broadcast}} Per il \textit{Directed Broadcast Address} si ha che l'indirizzo di \textit{broadcast} è l'indirizzo che permette di inviare un pacchetto a tutti gli \textit{host} della rete. Questo indirizzo è identificato con tutti i bit della parte di \textit{host} a $1$. Ad esempio l'indirizzo \texttt{128.211.0.31/28} è l'indirizzo di \textit{broadcast} per la rete \texttt{128.211.0.16/28}
            \paragraph{Indirizzi di \textit{broadcast} locale} L'indirizzo di \textit{broadcast} locale è l'indirizzo che permette di inviare un pacchetto a tutti gli \textit{host} della rete locale. Questo indirizzo è identificato con tutti i bit dell'indirizzo a $1$. Quindi l'indirizzo \texttt{255.255.255.255} è l'indirizzo di \textit{broadcast} locale, anche se possa sembrare globale questo rimane locale perché non viene inoltrato alla rete pubblica.
            \paragraph{Indirizzi di \textit{localhost}} Per le regole del protocollo \texttt{TCP/IP} necessitiamo di un indirizzo \texttt{IP} anche per richiedere l'assegnazione di un indirizzo \texttt{IP} allora si è deciso di riservare un indirizzo \texttt{IP} ovvero: \texttt{0.0.0.0} che viene usato solo per le prime comunicazioni all'interno di una rete per "chiedere" l'assegnazione di un indirizzo \texttt{IP}
            \paragraph{Indirizzi di \textit{loopback}} Indirizzo \texttt{IP} riservato per il \textit{loopback} del PC o del dispositivo. Questo indirizzi sono \texttt{127.0.0.0/8}, ma il più comune è \texttt{127.0.0.1}
            \paragraph{Indirizzi di \textit{multicast}} Tutti gli indirizzi \texttt{IP} che iniziano con \texttt{1110} sono indirizzi di \textit{multicast}, molti apparati di rete bloccano il traffico di \textit{multicast} per evitare attacchi.
            \paragraph{Indirizzi \texttt{IP} \textit{local}} Sono indirizzi che non vengono assegnati pubblici ma vengono assegnati autonomamente se ci si aspettava che l'indirizzo \texttt{IP} venisse assegnato da un apparato esterno ma ciò non è avvenuto. Questi indirizzi sono appartenenti alla rete $194.254.0.0/16$ 
            \paragraph{Indirizzi \texttt{IP} dei \textit{router}} Un \textit{router} per definizione ha almeno $2$ interfacce di rete, quindi ha almeno $2$ indirizzi \texttt{IP}. Questo non limita un \textit{router} ad avere solo $1$ indirizzo \texttt{IP} per ogni interfaccia di rete. Da ricordare che un indirizzo \texttt{IP} non è associato ad un \textit{host} ma ad un'interfaccia di rete su un \textit{host}. La molteplicità di indirizzi \texttt{IP} per una sola interfaccia di rete risulta utile se ad esempio volgiamo suddividere la rete interna in più reti e impostare delle regole di \textit{firewall} tra le reti.
            \paragraph{come si integra al livello 2} Per l'architettura a strati del modello \texttt{TCP/IP} il messaggio contenete gli indirizzi \texttt{IP} è incapsulato in un frame del livello 2, questo frame contiene l'indirizzo \texttt{MAC} di sorgente e di destinazione. Per ottenere questi ci avvaliamo dell'uso di \texttt{ARP}.
    \subsection{\textit{Address Resolution Protocol} \texttt{ARP}}
        L'\texttt{ARP} è un protocollo che permette di associare un indirizzo \texttt{IP} ad un indirizzo \texttt{MAC}. Questo protocollo è molto utile in quanto i \textit{router} inoltrano i pacchetti in base all'indirizzo \texttt{MAC} e non all'indirizzo \texttt{IP}. Il funzionamento dell'\texttt{ARP} è il seguente: \begin{enumerate}
            \item Un \textit{host} vuole inviare un pacchetto ad un altro \textit{host} nella stessa rete e conosce l'indirizzo \texttt{IP} di destinazione ma non l'indirizzo \texttt{MAC}.
            \item L'\textit{host} invia un pacchetto di \texttt{ARP} in \textit{broadcast} con l'indirizzo \texttt{IP} di destinazione.
            \item Tutti gli \textit{host} della rete ricevono il pacchetto di \texttt{ARP} e solo l'\textit{host} con l'indirizzo \texttt{IP} di destinazione risponde con il proprio indirizzo \texttt{MAC}.
            \item L'\textit{host} che ha inviato il pacchetto di \texttt{ARP} riceve l'indirizzo \texttt{MAC} e può inviare il pacchetto.
            \item L'\textit{host} che ha inviato il pacchetto di \texttt{ARP} memorizza l'associazione tra l'indirizzo \texttt{IP} e l'indirizzo \texttt{MAC} per un certo periodo di tempo.
            \item Se l'\textit{host} vuole inviare un altro pacchetto allo stesso \textit{host} allora non invia un altro pacchetto di \texttt{ARP} ma usa l'associazione precedentemente memorizzata.
            \item Se l'associazione scade allora l'\textit{host} invia un altro pacchetto di \texttt{ARP} per rinnovare l'associazione.
            \item Se l'\textit{host} non riceve risposta al pacchetto di \texttt{ARP} allora il pacchetto non può essere inviato.
        \end{enumerate}
        Questo protocollo con questa procedura permette di associare un indirizzo \texttt{IP} ad un indirizzo \texttt{MAC} e quindi di inoltrare i pacchetti correttamente filtrando i pacchetti da apparati come gli \textit{switch} di rete che non lavorano a livello di rete, ma solo a livello di collegamento.\newline
        Questo protocollo non è mai usato su una rete pubblica, questo perché di base una destinazione pubblica prevede che il pacchetto passi per uno o più \textit{router} e quindi l'indirizzo \texttt{MAC} cambierebbe ad ogni \textit{hop} e quindi non si associa un indirizzo \texttt{MAC} ad un indirizzo \texttt{IP} pubblico, ma al suo posto gli \textit{header} dei pacchetti \texttt{IP} contengono l'indirizzo \texttt{MAC} del \textit{default gateway}.
        \paragraph{Incapsulamento frame \texttt{ARP}} Il pacchetto di \texttt{ARP} è incapsulato in un frame (esempio \textit{Ethernet}) questi frame sono interpretati come dati da trasportare e incapsulati nel \textit{payload} del frame. Il frame contiene anche un campo \textit{type} che indica il tipo di pacchetto contenuto nel \textit{payload} del frame.
        \paragraph{Il \textit{proxy} \texttt{ARP}} Il \textit{proxy} \texttt{ARP} è una tecnica che permette ad un \textit{router} di rispondere ai pacchetti di \texttt{ARP} al posto dell'\textit{host} di destinazione. Questa tecnica è molto utile in quanto permette di nascondere la topologia di rete e di proteggere gli \textit{host} dalla ricezione di pacchetti di \texttt{ARP} malevoli.
    \subsection{\textit{Internet Control Message Protocol} \texttt{ICMP}}
        Il protocollo \texttt{ICMP} è fondamentale per il funzionamento di internet, questo protocollo permette di inviare messaggi di errore e di controllo tra i dispositivi di rete.
        \paragraph{Interdipendenza con \texttt{IP}}\texttt{IP} e \texttt{ICMP} sono \underline{interdipendenti} infatti il protocollo \texttt{IP} dipende da \texttt{ICMP} per il segnalamento di errori ma a sua volta \texttt{ICMP} necessita di \texttt{IP} per il trasporto di messaggi.
        \paragraph{Formato messaggi \texttt{ICMP}} I messaggi di \texttt{ICMP} sono costituiti da un semplice \textit{header} composto da un campo \textit{type} da 1 byte, un codice di stato fatto da 1 byte e un campo \textit{checksum} di 2 byte. Il campo \textit{type} indica il tipo di messaggio, il codice di stato indica il dettaglio del messaggio e il campo \textit{checksum} è un campo di controllo degli errori.
        \subsubsection{Principali \textit{type} di \texttt{ICMP}}
            \begin{table}[H]
                \begin{tabular}{|c|c|}
                    \hline
                    \textbf{Type} & \textbf{Descrizione} \\ \hline
                    \textit{Destination Unreachable} & Indica che la destinazione non è raggiungibile \\ \hline
                    \textit{Port Unreachable} & Indica che la porta di destinazione non è raggiungibile \\ \hline
                    \textit{Time Exceeded} & Indica che il tempo di vita del pacchetto è 0 \\ \hline
                    \textit{Parameter Problem} & Indica che c'è un problema con i parametri del pacchetto \texttt{IP} \\ \hline
                    \textit{Source Quench} & Indica che il mittente deve rallentare l'invio di pacchetti \\ \hline
                    \textit{Redirect} & Indica che il mittente deve cambiare il percorso di inoltro \\ \hline
                    \textit{Echo \& Echo Reply} & Usato per il \textit{ping} \\ \hline
                    \textit{Timestamp request/reply} & Usato per ottenere il tempo di un dispositivo \\ \hline
                    \textit{Router Advertisement/solicitation} & Usato per proporsi come \textit{router} o scoprire i \textit{router} \\ \hline
                    \textit{Fragmentation needed} & Indica che il pacchetto è troppo grande e deve essere frammentato \\ \hline
                \end{tabular}
            \end{table}
        Sono presenti dunque due classi principali di messaggi: quelli per segnalare errori e quelli per recuperare informazioni. Da notare che i messaggi di \texttt{ICMP} vengono trasportati nel campo \textit{payload} di un pacchetto \texttt{IP}.
        \subsubsection{Sfruttare \texttt{ICMP}} \texttt{ICMP} può essere sfruttato per il "\textit{ping}" e per il "\textit{traceroute}". Il \textit{ping} è un comando che permette di verificare la connessione tra due dispositivi (\textit{echo request} e \textit{echo reply}), mentre il \textit{traceroute} permette di verificare il percorso che un pacchetto fa per arrivare ad un determinato dispositivo. Questo comando invia pacchetti con un campo \textit{time to live} incrementale e aspetta un messaggio di \textit{time exceeded} per sapere che il pacchetto è arrivato al \textit{router} di destinazione.

    \subsection{\textit{Dynamic Host Configuration Protocol} \texttt{DHCP}}
        Il protocollo \texttt{DHCP} è un protocollo che permette di assegnare automaticamente un indirizzo \texttt{IP} ad un \textit{host} che si connette ad una rete. Gli indirizzi possono essere liberati quando non vengono più usati e possono essere riassegnati ad altri \textit{host}. La sintesi del funzionamento del protocollo è la seguente: \begin{enumerate}
            \item \textit\textbf{\texttt{DHCP} discover} L'\textit{host} invia un pacchetto di \texttt{DHCP} in \textit{broadcast} per trovare un \textit{server} \texttt{DHCP}. Nel pacchetto sono contenuti una sorgente generica, una destinazione generica un mio \texttt{IP} e una \textit{transaction ID}.
            \item \textit\textbf{\texttt{DHCP} offer} Il \textit{server} \texttt{DHCP} risponde con un pacchetto di \texttt{DHCP} con un indirizzo \texttt{IP} disponibile per l'\textit{host}. Nel pacchetto sono contenuti l'indirizzo \texttt{IP} di sorgente del server, l'indirizzo \texttt{IP} generico di destinazione, il mio \texttt{IP} e la stessa \textit{transaction ID} del pacchetto di \texttt{DHCP} \textit{discover} e un \textit{lifetime} dell'indirizzo \texttt{IP} offerto.
            \item \textit\textbf{\texttt{DHCP} request} L'\textit{host} invia un pacchetto di \texttt{DHCP} per richiedere l'indirizzo \texttt{IP} offerto dal \textit{server} \texttt{DHCP}. 
            \item \textit\textbf{\texttt{DHCP} ack} Il \textit{server} \texttt{DHCP} risponde con un pacchetto di \texttt{DHCP} per confermare l'assegnazione dell'indirizzo \texttt{IP} all'\textit{host}.
        \end{enumerate}
        \paragraph{Prestiti \texttt{DHCP}} L'indirizzo \texttt{IP} assegnato ad un \textit{host} può essere riassegnato ad un altro \textit{host} quando l'\textit{host} non lo usa più. Un indirizzo può però essere rinnovato quando l'\textit{host} lo usa ancora. Se il \textit{server} non rinnova l'indirizzo allora l'indirizzo viene rilasciato e l'\textit{host} deve smettere di usarlo.
        \paragraph{Altri dettagli} Il protocollo \texttt{DHCP} usa \texttt{UDP} e quindi non è affidabile ma è progettato per essere robusto a perdite e a duplicati. Inoltre il \textit{client} memorizza l'indirizzo del \textit{server} \texttt{DHCP} per richieste successive.
        \subsubsection{Formato dei messaggi \texttt{DHCP}} 
            Di seguito lo schema dei messaggi \texttt{DHCP}:
            
            \begin{figure}[H]
                \centering
                \includegraphics[width=0.5\textwidth]{04/dhcp-message.png}
                \caption{Formato dei messaggi \texttt{DHCP}}
            \end{figure}
            
            Vediamo ora a cosa servono i principali parametri:
            \begin{description}
                \item[\texttt{OP}] Campo di 1 byte che indica se il messaggio è un \textit{request} o un \textit{reply}
                \item[\texttt{HTYPE} e \texttt{HLEN}] Campi di 1 byte l'uno specificano il tipo di \textit{hardware} e la lunghezza dell'indirizzo \texttt{MAC}
                \item[\texttt{FLAGS}] Campo di 2 byte che contiene, ad esempio, se il mittente può ricevere \textit{broadcast} o solo risposte dirette
                \item[\texttt{HOPS}] Campo di 1 byte che indica quanti \textit{server} hanno inoltrato la risposta
                \item[\texttt{TRANSACTION IDENTIFIER}] Campo di 4 byte che identifica la transazione
                \item[\texttt{SECONDS ELAPSED}] Campo di 2 byte che indica da quanto tempo il \textit{client} è in attesa
                \item[\textit{altri}] Ci sono altri campi che indicano l'indirizzo \texttt{IP} del \textit{client}, l'indirizzo \texttt{IP} del \textit{server}, l'indirizzo \texttt{IP} del \textit{router} e l'indirizzo \texttt{IP} del \textit{DNS}.
            \end{description}
        \subsubsection{Se non c'è un \textit{server} \texttt{DHCP}}
            Se non c'è un \textit{server} \texttt{DHCP} allora ci sono due situazioni: la prima è si configura un indirizzo \texttt{IP} statico, mentre, se non riceviamo risposta dopo un certo periodo di tempo, l'\textit{host} provvede all'impostazione di un indirizzo \textit{link-local} che permette la comunicazione con altri \textit{host} nella stessa rete che hanno lo stesso problema.
            \paragraph{Configurazione \textit{link-local}} Per configurare un indirizzo \textit{link-local} si seguono i seguenti passi: \begin{enumerate}
                \item Si sceglie un indirizzo \texttt{IP} compreso tra \texttt{169.254.0.1} e \texttt{169.254.255.254} con maschera di rete \texttt{/16}
                \item Si cerca se esiste una interfaccia di rete con un indirizzo \texttt{IP} \textit{link-local} scelto (tramite \texttt{ARP})
                \item[3.a.] Se esiste un'altra interfaccia con lo stesso indirizzo \texttt{IP} allora si ripete il processo
                \item[3.b.] Altrimenti si configura l'indirizzo \texttt{IP} \textit{link-local}
            \end{enumerate}
    \subsection{Il viaggio di un pacchetto}
        Analizziamo ora il viaggio di un pacchetto da un \textit{host} \texttt{A} ad un \textit{host} \texttt{B} in una rete. assumiamo che \texttt{A} conosca l'indirizzo \texttt{IP} di \texttt{B} e l'indirizzo \texttt{UP} del router \texttt{R} (che è il \textit{default gateway}) (Tramite \texttt{DHCP}), inoltre conosce già l'indirizzo \texttt{MAC} del router \texttt{R} (Tramite \texttt{ARP}).\newline
        Quindi in una rete costituita da \texttt{A}, \texttt{R} e \texttt{B}, con \texttt{R} nel mezzo, il pacchetto viaggia in questo modo: \begin{enumerate}
            \item \texttt{A} crea il datagramma \texttt{IP} con sorgente \texttt{A} e destinazione \texttt{B}
            \item \texttt{A} incapsula il datagramma in un frame di livello 2 con indirizzo \texttt{MAC} di \texttt{R} come destinazione e \texttt{MAC} di \texttt{A} come sorgente
            \item Il frame viene inviato alla rete da \texttt{A} a \texttt{R}\footnote{\label{netSwitch}
                In questo punto possono essere presenti all'interno della rete altri dispositivi di livello 2 come \textit{switch} che inoltrano il pacchetto in base all'indirizzo \texttt{MAC}}
            \item Il frame arriva a \texttt{R} che estrae il datagramma \texttt{IP} e lo passa al livello 3
            \item \texttt{R} controlla la tabella di routing e inoltra il pacchetto a \texttt{B}, per fare ciò incapsula il datagramma in un frame con indirizzo \texttt{MAC} di \texttt{B} come destinazione e \texttt{MAC} di \texttt{R} come sorgente
            \item Il frame viene inviato da \texttt{R} a \texttt{B}\footref{netSwitch}
            \item \texttt{B} estrae il datagramma \texttt{IP} e lo passa al livello 3 per l'elaborazione
        \end{enumerate}
    \subsection{\texttt{IPv6}}
        Il protocollo \texttt{IPv6} è un protocollo nato per ampliare la quantità di indirizzi \texttt{IP} disponibili rispetto quelli di \texttt{IPv4}, successivamente si è voluto standardizzarlo anche in quanto il formato dell'\textit{header} velocizza l'elaborazione dei frammenti, inoltre facilita la gestione della qualità del servizio.\subsubsection{Formato del datagramma \texttt{IPv6}}
            Il formato di un frammento è: \textbf{header} da $40$ byte e la frammentazione è proibita.
            \paragraph{Header} L'\textit{header} di un datagramma \texttt{IPv6} è composto da: \begin{itemize}
                \item \textbf{Flow label} Campo di $20$ bit che permette di identificare un flusso di dati.
                \item \textbf{Priority} Campo di $4$ bit che permette di identificare la priorità del pacchetto.
                \item \textbf{Next header} Campo di $8$ bit che permette di identificare il protocollo di trasporto.
            \end{itemize}
        \subsubsection{Cambiamenti rispetto \texttt{IPv4}}
            In quanto la rete è diventata più affidabile è stato rimosso il \textit{checksum}, vengono rimosse le \textit{options} e il \textit{header} è più corto. Inoltre viene introdotto il protocollo \textit{ICMPv6} con più funzionalità rispetto a \textit{ICMP}.
        \subsubsection{Transizione da \texttt{IPv4} a \texttt{IPv6}}
            La transizione da \texttt{IPv4} a \texttt{IPv6} è molto lenta in quanto richiede un cambiamento di infrastruttura molto grande, per questo si è deciso che se un pacchetto \texttt{IPv6} deve transitare obbligatoriamente per una rete \texttt{IPv4} allora il pacchetto viene incapsulato in un pacchetto \texttt{IPv4} e poi inviato alla rete \texttt{IPv4} e poi riconvertito in un pacchetto \texttt{IPv6} alla destinazione.
        \subsubsection{Indirizzi \texttt{IPv6}}
            La lunghezza non permette notazione \textit{dotted decimal} e quindi si usa la notazione esadecimale: Ogni gruppo di $4$ bit è scritto come una cifra $0,1,\dots,9$ o una lettera: $a,b,\dots,f$. In totale $32$ cifre esadecimali costituiscono un indirizzo \texttt{IPv6}.
            \paragraph{Esempio} Un indirizzo di esempio è \texttt{2a03:2880:f108:0083:face:b00c:0000:25de} che è un indirizzo \texttt{IPv6} valido.
            \paragraph{Raggruppamento} Si possono raggruppare $0$ o omettendoli oppure se è un intero gruppo di $0$ si può omettere tutto il gruppo. Ad esempio l'indirizzo \texttt{2a03:2880:f108:0083:face:b00c:0000:25de} può essere scritto come\texttt{2a03:2880:f108:83:face:b00c:0:25de}. 
        \subsubsection{Indirizzi speciali}
            Gli indirizzi speciali di \texttt{IPv6} sono: \begin{itemize}
                \item \textbf{Unspecified address} Indirizzo $0:0:0:0:0:0:0:0$ che indica che l'indirizzo non è assegnato.
                \item \textbf{Loopback address} Indirizzo $0:0:0:0:0:0:0:1$ che indica l'indirizzo di \textit{loopback}.
                \item \textbf{Link-local address} Indirizzo $fe80::/10$ che indica un indirizzo di \textit{link-local}.
                \item \textbf{Site-local address} Indirizzo $fec0::/10$ che indica un indirizzo di \textit{site-local}.
                \item \textbf{Multicast address} Indirizzo $ff00::/8$ che indica un indirizzo di \textit{multicast}.
            \end{itemize}
        
    \chapter{Il livello \textit{Data-Link}}
\label{ch:livelloDataLink}

    \subsection{Introduzione}
        \paragraph{Terminologia} Definiamo gli \textit{host} e i \textit{router} come \textbf{nodi} di rete, inoltre il canale di comunicazione tra due nodi adiacenti è detto \textbf{link}. Il pacchetto di livello 2 è detto \textbf{frame}, il quale incapsula il pacchetto di livello 3, detto \textbf{datagram}.
        \paragraph{Contesto} Un percorso può essere diviso con diversi tipi di \textit{link}, un datagramma trasferito lungo un percorso attraversa la rete con protocolli di livello 2 differenti. Ad esempio su un percorso tra due nodi possono esserci link \textit{Ethernet} per il promo link, \textit{Frame Relay} per gli intermedi ed \texttt{802.11} per l'ultimo link. Ognuno dei protocolli di livello 2 fornisce servizi diversi, alcuni possono eseguire controllo di errori, altri no.
    \subsection{Servizi del livello}
        La creazione di un frame di livello 2 per l'accesso al link include:
        \begin{itemize}
            \item L'incapsulamento del datagramma in un frame, con aggiunta di un \textit{header} e un \textit{trailer}.
            \item Fornisce un meccanismo di accesso al canale condiviso, se il link è condiviso.
            \item Utilizza indirizzi di livello 2 detti \texttt{MAC} negli header dei frame per identificare il mittente e il destinatario.\footnote{Gli indirizzi \texttt{MAC} sono diversi dagli indirizzi \texttt{IP}}
        \end{itemize}
        Inoltre il livello 2 può fornire un servizio di consegna affidabile ai nodi adiacenti, questo spesso non viene usato per \textit{link} con basso tasso di perdita.
        \paragraph{Controllo di flusso} Il livello 2 può fornire un controllo di flusso adattando la velocità di trasmissione del mittente alla velocità accettabile del destinatario.
        \paragraph{Rilevazione di errori} Il livello 2 può rilevare errori causati da: attenuazioni, rumore e interferenze,\dots ed il ricevente può individuarne la presenza e scegliere di avvertire il mittente e/o scartare il pacchetto
        \paragraph{Corredizione di errore} Il livello 2 può correggere piccoli errori se il canale è rumoroso, ma questo è raro.
        \paragraph{\{\textit{Half}$\mid$\textit{Full\}-Duplex}} Un link può essere \textit{half-duplex} o \textit{full-duplex}, nel primo caso i due nodi possono trasmettere e ricevere, ma non contemporaneamente, nel secondo caso possono farlo contemporaneamente. Spesso si allocano due sezioni di risorse per la trasmissione e la ricezione di fatto emulando un \textit{full-duplex}.
    \subsection{Chi implementa il livello \textit{data-link}}
        Il livello \textit{data-link} è implementato da tutti gli \textit{host}, solitamente questo avviene da parte del \textit{firmware} dell'adattatore di rete (\texttt{NIC}), oppure da un \textit{chip} dedicato. In ogni caso il livello \textit{data-link} è collegato direttamente al \texttt{BUS} del sistema dell'\textit{host} ed è una combinazione di \textit{hardware}, \textit{software} e \textit{firmware}.
    \subsection{Comunicazione tra adattatori di rete} 
        \paragraph{Trasmissione} Quando un \textit{host} vuole trasmettere un datagramma, il livello \textit{data-link} del mittente incapsula il datagramma in un frame, aggiungendo \textit{bit} per il controllo di errore, controllo di flusso, ecc\dots e lo trasmette al \textit{link}. Il frame viene trasmesso da un adattatore di rete al \textit{link} e viene ricevuto dall'adattatore di rete del destinatario.
        \paragraph{Ricezione} L'adattatore di rete del destinatario estrae il datagramma dal frame e lo passa al livello \textit{network}. Il livello \textit{data-link} del destinatario può rilevare errori e scartare il frame se necessario.
\section{Rilevamento di errori e correzione}
    Nel datagramma vengono inseriti dei \textit{bit} \texttt{EDC} (\textit{Error Detection and Correction}) per rilevare e correggere errori. Questi sono \textit{bit} ridondanti che vengono aggiunti al datagramma per rilevare e correggere errori. Questi \textit{bit} sono calcolati in base ai \textit{bit} del datagramma (\texttt{D} - sia i dati che gli \textit{header}). I \textit{bit} del \texttt{EDC} vengono accodati al datagramma \texttt{D} nella sua trasmissione.

    \subsection{Vari metodi per il rilevamento di errori}
        \paragraph{Controllo di parità singola} Esiste un singolo bit di parità, che viene calcolato in modo che il numero totale di \textit{bit} a 1 sia pari o dispari. Questo metodo è molto semplice e può rilevare errori singoli, ma non può correggerli e in caso di errori multipli non è in grado di rilevarli.
        \paragraph{Controllo di parità bidimensionale} Questo metodo è una generalizzazione del controllo di parità singola in questo caso si usano più bit di parità, questo sulla base di una matrice di dati, vengono usati quindi $n+m-1$ bit di parità per una matrice $n \times m$. Questo metodo può rilevare errori singoli e multipli (se non sono sulla stessa riga o colonna), ma non può correggerli, può infatti correggere solo un errore per matrice.
        \paragraph{Correzzione di errori tramite riddondanza} Semplicemente si aggiungono \textit{bit} ridondanti al datagramma, ovvero si trasmette più volte lo stesso messaggio. Questo metodo è molto costoso in termini di banda, ma permette di rilevare e correggere errori multipli.
    
    \subsection{\textit{Cyclic redundancy check} (\texttt{CRC})}
        Il \texttt{CRC} è il metodo più efficiente per il rilevamento di errore, considera i bit di dati come un numero binario ($D$), ed dopo aver scelto una sequenza di $r+1$ bit detto \textit{polinomio generatore} ($G$) il quale è conosciuto sia dal mittente che al destinatario. Ora il \texttt{CRC} viene composto scegliendo $r$ \textit{bit} in modo che i dati $D$ siano divisibili per $G$ modulo 2 con resto $R$ (ovvero $<D,R>$ è divisibile per $G$). Se il ricevente riceve $<D',R>$ e questi non hanno resto 0, allora c'è stato un errore. A meno che non si commettano $r$ errori, il \texttt{CRC} rileva tutti gli errori di lunghezza $r-1$ o meno.
        \subsubsection{Calcolo \texttt{CRC}}
            \begin{enumerate}
                \item Vogliamo che valga: $D\times 2^r \operatorname{XOR} R = nG$
                \item Aggiungiamo $R$ ($\operatorname{XOR} R$) ad entrambi i lati: $D\times 2^r = nG \operatorname{XOR} R$
                \item Dunque il resto della divisione di $D\times 2^r$ per $G$ è $R$
                \item Quindi il \texttt{CRC}$\rightarrow R=\operatorname{resto}\left(\frac{D\times 2^r}G\right)$
            \end{enumerate}
\section[Protocolli Accesso Multiplo Canale]{Protocolli e tecnologie per l'accesso multiplo al canale}
    \subsubsection{Tipi di collegamento}
        \paragraph{Collegamento punto-punto} Un collegamento punto-punto è un collegamento dedicato tra due nodi, questo però non è sempre rispettato a livello fisico\dots ad esempio un collegamento \textit{Ethernet} è un collegamento punto-punto, ma in realtà può essere condiviso da più nodi.
        \paragraph{Collegamento broadcast} Un collegamento broadcast è un collegamento condiviso da più nodi, i frame trasmessi da un nodo sono ricevuti da tutti gli altri nodi. Questo tipo di collegamento è tipico delle reti \textit{wLAN}.
    \subsection{Protocolli per il controllo dell'accesso multiplo}
        Questi protocolli sono tipici dei collegamenti broadcast, dove più nodi condividono lo stesso canale di trasmissione per inviare e ricevere frame. Questi protocolli si occupano di evitare che due o più trasmissioni simultanee interferiscano tra loro.
        \paragraph{Protocolli ad accesso multiplo} L'algoritmo distribuito che determina come i nodi condividono il canale, determinando quando un nodo può trasmettere. Questi accordi possono essere eseguiti sullo stesso canale (tipicamente) o su canali separati (\textit{out-of-band}).
        \subsubsection{Un protocollo \texttt{MAC} ideale}
            Un protocollo \texttt{MAC} (\textit{Multiple Access Control}) ideale deve, partendo da un canale \textit{broadcast} capace di supportare $R bit/s$ deve avere le seguenti caratteristiche:
            \begin{enumerate}
                \item Quando un nodo trasmette lo può fare alla velocità massima $R$.
                \item Quando $M$ nodi vogliono trasmettere, possiamo farlo ad un tasso medio pari a $R/M$.
                \item Il sistema dovrebbe essere completamente decentralizzato (senza "centro stella") ed inoltre non dovrebbe richiedere sincronizzazione di "\textit{clock}".
                \item Il protocollo deve essere semplice.
            \end{enumerate}
        \subsubsection{Protocolli \texttt{MAC} realmente}
            I protocolli \texttt{MAC} reali non possono soddisfare tutte le caratteristiche di un protocollo ideale, ma possono comunque essere classificati in tre categorie:
            \begin{itemize}
                \item A ripartizione delle risorse del canale
                    \subitem Dividono il canale in "sotto-canali" più piccoli ed allocando ciascuno di essi ad un nodo.
                \item Ad accesso casuale
                    \subitem I nodi trasmettono quando vogliono, ma se due nodi trasmettono contemporaneamente si verifica una collisione e i nodi devono ritrasmettere.
                \item A "turni" intelligenti
                    \subitem I nodi accedono al canale a turno, ma i nodi con molti dati da trasmettere possono ottenere turni più lunghi.
            \end{itemize}
    \subsection{\texttt{TDMA}, \texttt{DFMA}, \texttt{CDMA}}
        \subsubsection{Ripartizione delle risorse (\texttt{TDMA})}
            Il protocollo \texttt{TDMA} (\textit{Time Division Multiple Access}) è un protocollo di accesso al canale in "turni" che divide il tempo in "slot" e assegna un "slot" a ciascun nodo. I nodi trasmettono solo nei loro "slot" assegnati. Questo protocollo è usato in reti telefoniche cellulari e satellitari.
        \subsubsection{Ripartizione delle frequenze (\texttt{FDMA})}
            Il protocollo \texttt{FDMA} (\textit{Frequency Division Multiple Access}) è un protocollo di accesso che prevede la divisione della banda di frequenza in "sotto-bande" e assegna una "sotto-banda" a ciascun nodo. I nodi trasmettono solo nella loro "sotto-banda" assegnata in qualunque momento del tempo. Questo protocollo è anche per le comunicazioni tramite mezzo coassiale.
        \subsubsection{Ripartizione del codice (\texttt{CDMA})}
            Il protocollo \texttt{CDMA} (\textit{Code Division Multiple Access}) è un protocollo di accesso al canale che assegna un codice univoco a ciascun nodo e permette a tutti i nodi di trasmettere contemporaneamente. I nodi trasmettono i dati codificati con il loro codice univoco e i nodi riceventi decodificano i dati con il codice del mittente. Questo protocollo è usato nelle reti \textit{wLAN} e nelle reti cellulari.
        \begin{figure}[H]
            \centering
            \includegraphics[width=0.8\linewidth]{05/confrontoTD-DF-CDMA.png}
            \caption{Confronto tra \texttt{TDMA}, \texttt{FDMA} e \texttt{CDMA}}
        \end{figure}
    \subsection{\textit{Slotted} \texttt{ALOHA}, \texttt{ALOHA} Protocolli ad accesso casuale}
            I protocolli ad accesso casuale permettono ai nodi di trasmettere quando vogliono ed alla massima velocità di trasmissione. Ma non c'è coordinamento con gli altri nodi e quindi possono verificarsi collisioni.\newline
            Un protocollo ad accesso casuale specifica: come rilevare le collisioni, come recuperare uno stato di collisione e come ritrasmettere dopo una collisione.\newline
            Alcuni esempi di accessi casuali sono: \texttt{ALOHA}, \texttt{slotted ALOHA}, \texttt{CSMA}, \texttt{CSMA/CD}, \texttt{CSMA/CA}.
        \subsubsection{\texttt{Slotted ALOHA}}
            Assumendo che tutti i pacchetti abbiano la stessa lunghezza e che il tempo sia diviso in "\textit{slot}" lunghi quanto un pacchetto, i nodi possono trasmettere solo all'inizio di uno slot e che i nodi siano sincronizzati. Questo protocollo allora è in collisione quando due o più nodi trasmettono nello stesso slot.
            \paragraph{Come funziona} Quando un nodo ha un pacchetto da trasmettere, lo trasmette all'inizio dello slot successivo, se non ci sono collisioni il nodo può eventualmente trasmettere un altro pacchetto. Se c'è una collisione il nodo ritrasmette il pacchetto in un altro slot con probabilità $p$ finché non riesce a trasmettere il pacchetto senza collisioni.
            \paragraph{Vantaggi} Questo protocollo molto semplice, se un solo nodo trasmette il canale è usato al 100\%, inoltre è decentralizzato e basta solo sincronizzarsi sugli slot.
            \paragraph{Svantaggi} Le collisioni sono probabili e e sprecano risorse, non sempre gli slot sono usati, potrebbero esserci collisioni senza la fine di una trasmissione, infine la sincronizzazione richiede il coordinamento.
            \paragraph{Efficienza} Assumendo che tutti i pacchetti hanno la stessa dimensione, che $G\geq 0$ sia il traffico offerto allora se $P[k]$ è la probabilità che ci siano $k$ pacchetti da trasmettete in un singolo slot, allora questa segue una \textit{distribuzione di Poisson} e la probabilità di successo è $P[k]=\frac{G^ke^{-G}}{k!}$. Dunque anche se il \textit{throughput} ideale è di $1$ il \textit{throughput} reale per un pacchetto è di $P[k=1]=Ge^{-G}$. La massima efficienza si ha per $G=1$ e dunque il massimo \textit{throughput} è di $\frac{1}{e}\stackrel{\sim}{=} 0.368$. \newline
            Supponendo di avere $N$ nodi e che ciascuno trasmette con probabilità $p$ allora la probabilità che un nodo trasmetta con successo è $p(1-p)^{N-1}$ e la probabilità che un nodo qualunque abbia successo è $\binom{N}{1}p(1-p)^{N-1} = Np(1-p)^{N-1}$. Il valore di $p$ che massimizza la probabilità di successo è $p=\frac{1}{N}$, ed il limite per $N\to\infty$ è di $\lim\limits_{N\to\infty}Np(1-p)^{N-1}=\frac{1}{e}\stackrel{\sim}{=}0.368$.
        \subsubsection{\texttt{ALOHA}}
            Questo protocollo è simile al \texttt{slotted ALOHA}, ma è ancora più semplice in quanto non richiede la sincronizzazione sugli slot. I nodi trasmettono il prima possibile e se c'è una collisione aspettano un tempo casuale e con probabilità $p$ ritrasmettono.
            \paragraph{Efficenza} Anche in questo caso assumiamo che tutti i pacchetti abbiano la stessa dimensione e che abbiamo \textit{host} infiniti. Chiamiamo dunque $G\geq 0$ il traffico offerto la probabilità che ci siano $k$ pacchetti da trasmettere in un singolo slot è $P[k]=\frac{G^ke^{-G}}{k!}$ e il \textit{throughput} dato dalla probabilità che solo un \textit{frame} sia trasmesso nella finestra di vulnerabilità $[t_0-1,t_1+1]$ è $P[1]=Ge^{-G}$. La probabilità che non ci siano \textit{frame} trasmessi nel momento nel quale si vuole iniziare la trasmissione, ovvero nella finestra $[t_0-1,t_0]$ è $P[0]=e^{-G}$. Quindi la probabilità che un nodo trasmetta con successo è $P[k=1]\times P[k=0]=Ge^{-2G}$ il che ci porta ad un \textit{throughput} di $\frac{1}{2e}\stackrel{\sim}{=}0.184$.
    \subsection{\texttt{CSMA}, \texttt{CSMA/CD}, \texttt{CSMA/CA}}
        \subsubsection{\textit{Carrier Sense Multiple Access} (\texttt{CSMA})}
            Il protocollo \texttt{CSMA} è un protocollo di accesso al canale che prevede che i nodi ascoltino il canale prima di trasmettere. Se il canale è libero il nodo trasmette l'intero \textit{frame}, se il canale viene valutato come occupato, la trasmissione viene ritardata.
            \paragraph{Varianti di persistenza} Questo protocollo può essere implementato in diversi modi per determinare quando trasmettere:
            \begin{itemize}
                \item Non persistente: (0-persistente) Quando un nodo è pronto a trasmettere, ascolta il canale e se è libero trasmette, altrimenti aspetta un tempo casuale (molto più lungo del tempo di trasmissione) e poi ritenta.
                \item Persistente: (1-persistente) Quando un nodo è pronto a trasmettere, ascolta il canale e se è libero trasmette, altrimenti aspetta finché non si libera e poi trasmette.
                \item $p$-persistente: Quando un nodo è pronto a trasmettere, ascolta il canale e se è libero trasmette con probabilità $p$ e con probabilità $1-p$ aspetta un tempo casuale (molto più lungo del tempo di trasmissione) e poi ritenta.
            \end{itemize}
            Se si verifica una collisione il nodo attende un tempo casuale e poi ritrasmette.
            \paragraph{Periodo di vulnerabilità} Il periodo di vulnerabilità dipende dal tempo di propagazione $\tau$ e dale tempo richiesto per rilevare se il canale è occupato $T_a$. Se un nodo trasmette ma non raggiunge tutti gli altri nodi, allora chi non ha ricevuto il frame non può trasmettere e quindi il periodo di vulnerabilità è $T_v=\tau+T_a$. Per questo fatto il protocollo \texttt{CSMA} si usa quando $\tau\ll T$ 
            \paragraph{Collisioni} Anche con il protocollo \texttt{CSMA} possono verificarsi collisioni, infatti se due nodi trasmettono contemporaneamente, il segnale di uno può non essere rilevato dall'altro e quindi si verifica una collisione.
        \subsubsection{\textit{Carrier Sense Multiple Access} con \textit{Collision Detection} \texttt{CSMA/CD}}
            Il protocollo \texttt{CSMA/CD} usa la stessa parte di \textit{carrier sensing} e se il canale è occupato posticipa la trasmissione. Inoltre il protocollo permette di rilevare le collisioni in breve termine ed di interrompere la trasmissione per ridurre lo spreco di risorse.
            \paragraph{\textit{Collision Detection}} Nelle reti cablate è semplice l'implementazione di questo protocollo, infatti la rete è \texttt{Full-duplex} e la trasmissione e la ricezione avvengono su due canali separati. Nelle reti \textit{wLAN} invece è più complicato, infatti la trasmissione e la ricezione avvengono sullo stesso canale e quindi è più difficile rilevare le collisioni in quanto la potenza del segnale ricevuto $\ll$ potenza del segnale trasmesso.
        \subsubsection{\textit{Carrier Sense Multiple Access} con \textit{Collision Avoidance} \texttt{CSMA/CA}}
            Si usa quando non si possono rilevare le collisioni e inoltre $T\gg \tau+T_a$, quindi spesso nelle reti \textit{wLAN}. Con \texttt{CSMA 1-persistent} non funziona bene in quanto le collisioni sono più frequenti ed spesso non rilevabili. Mentre invece con \texttt{CSMA p-persistent} si ha un \textit{throughput} molto basso.
    \subsection{Protocolli \texttt{MAC} "a turni"}
        \subsubsection{\textit{Polling}}
            Il protocollo di \textit{polling} è un protocollo che prevede un nodo \textit{master} che "invita" tutti gli altri nodi detti \textit{slave} a trasmettere a turno. Questo usato viene usato in reti dove i dispositivi \textit{slave} hanno poche risorse.
            \paragraph{Problemi} I problemi principali di questo metodo riguardano l'\textit{overhead} da pagare per i messaggi del protocollo stesso, l'elevata latenza e la presenza di un singolo punto di fallimento, se infatti il nodo \textit{master} fallisce, tutta la rete fallisce.
        \subsubsection{\textit{Token passing}}
            Il protocollo \textit{token passing} garantisce il "diritto" di trasmettere a ciascun nodo sulla base della presenza o meno di un "\textit{token}". Il \textit{token} è un pacchetto che viene passato sequenzialmente da un nodo all'altro. Il nodo che possiede il \textit{token} può trasmettere, altrimenti deve passare il \textit{token} al nodo successivo. Questo protocollo è usato in reti \textit{Token Ring}, ovvero reti nelle quali i nodi sono collegati in un anello.
            \paragraph{Problemi} Su questo protocollo non abbiamo problemi di \textit{single point of failure} ma nel caso il nodo che possiede il \textit{token} fallisse ed il \textit{token} non venisse passato, la rete si blocca, cosa analoga succede se è il collegamento ad interrompersi o se il \textit{token} si perde.
    \subsection{\texttt{IEEE 802} e Ethernet}
        Il gruppo di standard \texttt{IEEE 802} sono quelli che definiscono i protocolli di livello 2 e 1. Ciò che accomuna questi standard è la "struttura" del livello, difatti un qualsiasi standard appartenente al gruppo \texttt{IEEE 802} ha lo scopo di definire ciò che sta dopo il livello di \texttt{LLC} (\textit{Logical Link Control}) ovvero il livello \texttt{MAC} (\textit{Media Access Control}) che regola l'accesso al mezzo trasmissivo (\texttt{PHY} - \textit{PHysical Layer}).
        \subsubsection{Ethernet - \texttt{IEEE 802.3}}
            Ethernet è lo standard dominante nelle reti cablate, questo permette tramite ad un unico \textit{chip} di supportare velocità di trasmissione differenti. Ethernet è in continua evoluzione, si è passati da \texttt{10Mbps} \texttt{(Cat. 3)} fino ad arrivare al \texttt{40Gbps} \texttt{(Cat. 8)}.
        \subsubsection{Prime versioni} La prima versione di Ethernet era \texttt{10BASE5}, queste prevedevano l'uso di cavi coassiali che fungevano da "bus" e che venivano collegati ai nodi tramite un connettore a "vampiro", il quale penetrava il cavo coassiale grazie ad una punta metallica. Queste versioni erano molto costose e difficili da installare, inoltre non permettevano la comunicazione punto-punto ma solo broadcast verso tutti i nodi.
        \paragraph{\texttt{10BASE-2}}  Questa versione di Ethernet prevedeva l'uso di cavi coassiali più sottili e flessibili, i quali venivano collegati ai nodi tramite un connettore a "T" e un terminatore. 
        \subsubsection{Gli hub} Gli hub sono dispositivi che permettono di collegare un singolo nodo ad un singolo cavo ma con la possibilità di mantenere tutti i nodi collegati in un unico "bus". Gli hub sono dispositivi \textit{layer 1} e non \textit{layer 2} e quindi non sono in grado di gestire le collisioni, sono quindi paragonabili ad un \textit{bus} che costa di più e che comunque non permette la comunicazione punto-punto.
        \subsubsection{I doppini incrociati} I doppini incrociati sono cavi inventati da \textit{SynOptics comms} che permettono di collegare due soli nodi, questi risolvono molti problemi riguardanti la gestione e l'installazione di reti Ethernet. Questi inoltre permettono l'eliminazione del \textit{tubo giallo} ovvero il cavo coassiale ed i vari terminatori e connettori.
            \paragraph{Denominazione dei cavi} I cavi Ethernet sono denominati attraverso una sigla del genere \texttt{X/Y TP} dove \texttt{X} è la schermatura del cavo (\texttt{U}: \textit{unshielded}, \texttt{F}: \textit{foil shielded}, \texttt{S}: \textit{shielded} ed \texttt{SF}: \textit{shielded foil}) e \texttt{Y} è la schermatura di ogni singolo doppino (\texttt{U}: \textit{unshielded}, \texttt{F}: \textit{foil shielded}) e \texttt{TP}, che è una parte fissa, sta per \textit{Twisted Pair} ovvero doppino incrociato.
            \subparagraph{\texttt{T568}\{\texttt{A|B}\}} Questi sono due standard che definiscono la disposizione dei cavi all'interno del connettore \texttt{RJ45}, entrambi vengono usati e sono compatibili tra loro. Questi standard definiscono se un cavo è del tipo \textit{straight-through} il quale può essere usato per collegare due dispositivi diversi (ad esempio un \textit{host} ad uno \textit{switch}) o essere usato come cavo patch, oppure se è del tipo \textit{cross-over} il quale è usato per collegare due dispositivi uguali (ad esempio due \textit{host}). In sintesi se si usa una combinazione di \texttt{T568A} e \texttt{T568B} si ottiene un cavo \textit{cross-over}, altrimenti se viene usato lo stesso standard su entrambi i lati si ottiene un cavo \textit{straight-through}. 
        \subsection{Il \textit{frame} Ethernet} 
            \begin{figure}[H]
                \centering
                \includegraphics[width=0.6\linewidth]{05/frameEthernet.png}
                \caption{Il \textit{frame} Ethernet}
            \end{figure}
            Il \textit{frame} Ethernet è composto da:
            \begin{description}
                \item[Preambolo] Il preambolo è un campo di 8 byte (0-6: \texttt{10101010}, 7: \texttt{10101011}) che serve per sincronizzare il \textit{clock} dei nodi.
                \item[Indirizzi] L'indirizzo di destinazione e di sorgente sono di 6 byte ciascuno, l'indirizzo di destinazione è l'indirizzo \texttt{MAC} di livello 2 del nodo destinatario, mentre l'indirizzo di sorgente è l'indirizzo \texttt{MAC} del nodo mittente.\footnote{L'indirizzo \texttt{MAC} del destinatario viene inserito prima dell'indirizzo del mittente, questo per permettere ad eventuali nodi intermedi di sapere a chi inoltrare il \textit{frame} senza doverlo decodificare, inoltre se l'indirizzo di destinazione non è per l'host corrente il \textit{frame} viene scartato e il restante del \textit{frame} non viene considerato}
                \item[Tipo] Il campo tipo è di 2 byte e specifica il tipo di protocollo di livello 3 incapsulato nel \textit{frame} Ethernet (ad esempio \texttt{IP}, \texttt{Novell IPX}, \texttt{AppleTalk}, \dots)
                \item[Dati] Il campo dati è di 46-1500 byte e contiene il datagramma di livello 3 incapsulato nel \textit{frame} Ethernet.
                \item[\texttt{CRC}] Il campo \texttt{CRC} è di 4 byte e contiene il codice di rilevamento degli errori.\footnote{Il \texttt{CRC} viene inserito per ultimo in modo che se il \textit{frame} viene scartato a livelli superiori, non venga calcolato il \texttt{CRC} per l'esecuzione del controllo di errore} 
            \end{description}
        \subsubsection{Connessione? \texttt{ACK}, \texttt{NACK}} Ethernet non prevede un meccanismo di conferma di ricezione, e non prevede il mantenimento dello stato della connessione tra due nodi. Questo perché Ethernet è un protocollo di livello 2 \texttt{connectionless} ed inaffidabile. I controlli di questo genere sono demandati a protocolli di livelli superiori.
        \subsubsection{\texttt{CSMA/CD} in Ethernet} 
            L'algoritmo di \texttt{CSMA/CD} in ethernet prevede che:
            \begin{enumerate}
                \item La scheda di rete riceve un datagramma dal livello di rete (3) e lo incapsula in un \textit{frame} Ethernet.
                \item Se una scheda di rete vede il canale libero, trasmette il \textit{frame} intero, altrimenti aspetta finché il canale non è libero.
                \item Se la trasmissione termina senza altre ricezioni la scheda di rete ritiene che il \textit{frame} sia stato trasmesso con successo.
                \item Se ciò non avviene e la scheda di rete rileva una collisione, interrompe la trasmissione e invia un segnale di \textit{abort}.
                \item Dopo una collisione la scheda di rete attende un tempo casuale (\textit{backoff}) tra $0$ e $\left(2^k-1\right)T, k\leq 7$ dove $T$ è il tempo necessario per trasmettere $512$ bit.
            \end{enumerate}
        \subsubsection{Livello fisico e data link} 
            La parte di protocollo \texttt{MAC} è di formato è comune a tutti i protocolli \texttt{IEEE 802.3} ma la parte di livello fisico è diversa. La parte di livello fisico è quella che si occupa di trasmettere i bit sul mezzo trasmissivo, questa parte è diversa per ogni standard \texttt{IEEE 802.3} e per ogni velocità di trasmissione.
\section{\textit{Ethernet Switching}}
    Gli \textbf{\textit{switch}} ethernet sono dispositivi di livello 2 (a differenza degli hub che sono di livello 1) che permettono di collegare più nodi in una rete locale. Gli \textit{switch} sono dispositivi che agiscono in modo attivo, memorizzano infatti i \textit{frame} e gli inoltrano, mentre il frame viene inoltrato si analizza il \texttt{MAC} del mittente e si esegue l'associazione porta-\texttt{MAC} in modo da poter inoltrare eventuali \textit{frame} destinati a quel \texttt{MAC} direttamente alla porta associata.\newline
    Inoltre gli \textit{switch} sono trasparenti ai nodi, ovvero i nodi non sanno di essere collegati ad uno \textit{switch} ed inoltre sono \textit{plug-and-play}, ovvero non necessitano di configurazione in quanto vale il principio dell'\textbf{auto-}\textbf{apprendimento}.
    \paragraph{Domini di collisione} Mentre nella tipologia a \textit{bus} tutti gli \textit{host} sono sullo stesso "dominio di collisione", ovvero ogni nodo può potenzialmente collidere con ogni altro nodo, nella tipologia a \textbf{stella} implementata con degli \textit{switch} ogni \textit{host} è sullo stesso dominio di collisione. Questo in quanto ogni porta dello \textit{switch} è una scheda di rete assestante.
    \paragraph{Trasmissioni simultanee} In una rete \textit{switched} dato che ogni \textit{host} ha un canale dedicato per comunicare con lo \textit{switch} ed il protocollo \textit{ethernet} è applicato individualmente ad ogni porta, è possibile che due \textit{host} trasmettano contemporaneamente senza che si verifichino collisioni, inoltre è possibile che lo \textit{switch} ed l'\textit{host} trasmettano contemporaneamente senza collisioni (In quanto per la trasmissione e la ricezione si usano due doppini separati).
    \subsection{AutoApprendimento - \textit{backward learning}}
        Gli \textit{switch} Ethernet grazie ad una tabella che associa \texttt{MAC} del destinatario ad una interfaccia di uscita, sono in grado di inoltrare i \textit{frame} diretti a quel \texttt{MAC} direttamente alla porta associata, il tutto senza dover inoltrare il \textit{frame} a tutte le porte e/o doverlo memorizzare. Per compilare questa tabella si usa il \textit{backward learning}, che consiste nel "leggere" il \texttt{MAC} del mittente di ogni pacchetto in entrata e memorizzarlo nella tabella associandolo alla porta di ingresso, se il \texttt{MAC} di destinazione non è presente nella tabella si inoltra il \textit{frame} a tutte le porte tranne a quella di ingresso. Quando il \textit{frame} arriva al destinatario se questo risponde, il \textit{frame} viene letto ed il \texttt{MAC} del destinatario viene memorizzato nella tabella associandolo alla porta di ingresso.
        \paragraph{Filtro inoltro} Nel caso nel quale un \textit{frame} arrivi ad uno \textit{switch} e il \texttt{MAC} di destinazione sia presente nella tabella, il \textit{frame} viene inoltrato solo alla porta associata al \texttt{MAC} se questa è diversa dalla porta di ingresso, altrimenti il \textit{frame} viene scartato. Questo accade per mantenere la retro-compatibilità con le reti \textit{hub-ed}. Se invece il \texttt{MAC} di destinazione non è presente nella tabella, viene eseguito il \textit{flooding}.
        \paragraph{Con \textit{switch} multipli} Nel caso di \textit{switch} multipli organizzati ad \textit{albero} la situazione non cambia molto, infatti ogni \textit{switch} esegue il \textit{flooding} se il \texttt{MAC} di destinazione non è presente nella tabella, altrimenti inoltra il \textit{frame} alla porta associata al \texttt{MAC} se questa è diversa dalla porta di ingresso, altrimenti il \textit{frame} viene scartato.
        \paragraph{\textit{Switch} in anello} Nel caso di \textit{switch} collegati in anello si possono verificare dei problemi di \textit{loop}, infatti se viene eseguito il \textit{flooding} da un \textit{switch} ad un altro \textit{switch} e questo \textit{switch} inoltra il \textit{frame} ad un terzo \textit{switch} che lo inoltra al primo \textit{switch} si crea un \textit{loop}. Questo genere di \textit{loop} saturano la rete e portano ad un sovraccarico degli apparati. Per evitare questo genere di problemi si usano topologie di rete differenti. Oppure si usano protocolli di \textit{loop prevention} come il \texttt{STP} (\textit{Spanning Tree Protocol}).
    \subsection{\textit{Spanning tree} per \textit{switch}}
        In quanto in reti complesse si usano più connessioni tra due switch e/o tra switch e \textit{host} si possono creare dei \textit{loop} viene introdotto il protocollo \textbf{\textit{Spanning Tree Protocol}}. Questo protocollo prevede che i vari \textit{switch} abbaino un numero identificativo univoco di $64$ bit per il quale i primi $16$ bit sono impostati dall'amministratore di rete ed i restanti $48$ bit è il \texttt{MAC} dello \textit{switch}. Questo numero viene usato per determinare quale \textit{switch} è il \textit{root} della rete. Il \textit{root} è lo \textit{switch} con il numero più basso.\newline
        Se infatti è probabile la presenza di \textit{loop} allora gli \textit{switch} si scambiano messaggi contenenti l'\texttt{ID} dello \textit{switch} ed il costo del \textit{link}, quando uno \textit{switch} riceve un \texttt{BPDU} (\textit{Bridge Protocol Data Unit}) controlla gli \texttt{ID} nella sua tabella e se il nuovo \texttt{ID} è più basso allora la porta sul quale è arrivato il \texttt{BPDU} diventa la porta di uscita per il \textit{root}, altrimenti la porta viene bloccata. Questo processo viene ripetuto per ogni \textit{switch} e per ogni porta.
        \paragraph{Variazioni} Se un \textit{link} si rompe o si crea un nuovo \textit{link} allora il processo di determinazione del \textit{root} viene ripetuto, ciò accade o in modo automatico, tramite l'invio periodico di \texttt{BPDU} alle porte \textit{designated}, da notare come ogni \texttt{BPDU} ricevuta abbia un \texttt{TTL} che invecchia ad ogni secondo. Se il \texttt{TTL} arriva a $0$ allora il \textit{link} viene considerato morto e il processo di determinazione del \textit{root} viene ripetuto. 
        \paragraph{Performance} Il protocollo non esclude che possano esserci percorsi più lunghi del necessario, ciò può accadere se il \textit{root} non viene impostato "intelligentemente" e se la topologia della rete è complessa. 
        \paragraph{Dominio di \textit{broadcast} v/s collisione} 
            \begin{definition}[Dominio di collisione]
                Si definisce con il termine \textbf{dominio di collisione} in una rete, quella parte tale per cui se due o più \textit{nodi} trasmettono in maniera simultanea, allora si verifica una collisione.
            \end{definition}
            \begin{definition}[Dominio di \textit{broadcast}]
                Si definisce con il termine \textbf{dominio di \textit{broadcast}} in una rete, quella parte tale per cui se un nodo trasmette un \textit{frame} del tipo \textit{broadcast} di livello 2, allora tutti i nodi presenti nel dominio ricevono il \textit{frame}.
            \end{definition}
    \subsection{VLAN}
        Se si vuole dividere una rete \texttt{LAN} in più reti diverse per necessità di separazione di intenti, carico o per separare i domini di \textit{broadcast} si possono usare le \textbf{VLAN} (\textit{Virtual Local Area Network}).\newline
        Le \textit{VLAN} sono reti logiche che permettono di separare i nodi in base al \texttt{MAC} in gruppi logici, in modo che i nodi appartenenti ad una \textit{VLAN} possano comunicare tra loro come se fossero collegati ad una rete fisica separata anche attraverso a \textit{switch} diversi.\newline
        Non tutti gli \textit{switch} però possono supportare le \textit{VLAN}, infatti per poter supportare le \textit{VLAN} uno \textit{switch} deve supportare il protocollo \texttt{IEEE 802.1Q} che permette di aggiungere un \textit{tag} al \textit{frame} Ethernet che specifica a quale \textit{VLAN} appartiene il \textit{frame} stesso.\newline
        Il vantaggio delle \textit{VLAN} è che permettono di separare i domini di \textit{broadcast} e di \textit{collisione} mantenendo comunque la possibilità di comunicare tra le varie \textit{VLAN} attraverso più \textit{switch} differenti, infatti gli \textit{switch} che supportano le \textit{VLAN} sono in grado di inoltrare i \textit{frame} tra di loro configurando i \textit{link} come \textit{trunk} ovvero \textit{link} che trasportano più \textit{VLAN} contemporaneamente. Inoltre gli \textit{host} non sono \texttt{VLAN-aware} ovvero non sanno a quale \textit{VLAN} appartengono e di conseguenza riducono la complessità di configurazione della rete.\newline
        Quest'ultima caratteristica rende le \textit{VLAN} retro-compatibili con tutti i dispositivi \textit{host}, mentre per gli \textit{switch} è necessario che supportino il protocollo o che almeno ne conoscano il funzionamento, difatti se uno \textit{switch} che non supporta le \textit{VLAN} riceve un \textit{frame} con un \textit{tag} \texttt{VLAN} questo \textit{frame} viene scartato e/o inoltrato come un normale \textit{frame} Ethernet. Per ovviare ciò è necessario configurare la porta di uscita verso quello \textit{switch} come se fosse un normale \textit{host}, in questo caso tutti gli \textit{host} collegati a quello \textit{switch} appartengono alla stessa \textit{VLAN}.
\section{\texttt{IEEE 802.11} - WiFi}
    \subsection{Terminologia ed architettura}
        \begin{definition}[Host Wireless]
            Si definisce con il termine \textbf{\textit{host wireless}} un dispositivo che può trasmettere e ricevere dati tramite onde radio. Questi possono essere sia statici che mobili.
        \end{definition}
        \begin{definition}[Stazione Base]
            Si definisce con il termine \textbf{stazione base} un dispositivo che, tipicamente connesso ad una rete cablata, esegue il \textit{relay} dei dati tra gli \textit{host} \textit{wireless} e la rete cablata.
        \end{definition}
        \begin{definition}[Collegamento \textit{wireless}]
            Si definisce con il termine \textbf{collegamento \textit{wireless}} l'insieme del collegamento senza fili tra un \textit{host} \textit{wireless} ed una stazione base ed il protocollo \texttt{MAC} che regola l'accesso al mezzo trasmissivo.
        \end{definition}
        \begin{definition}[Modalità infrastrutturata]
            Si definisce con il termine \textbf{modalità infrastrutturata} una modalità di funzionamento di una rete \textit{wireless} nella quale gli \textit{host} \textit{wireless} comunicano tra di loro attraverso una stazione base. Inoltre il processo di \textit{handover} tra più stazioni base è trasparente agli \textit{host}.
        \end{definition}
        \begin{definition}[Modalità ad hoc]
            Si definisce con il termine \textbf{modalità ad hoc} una modalità di funzionamento di una rete \textit{wireless} nella quale non sono presenti stazioni base e gli \textit{host} \textit{wireless} comunicano direttamente tra di loro. Questi possono trasmettere solo ad altri \textit{host} \textit{wireless} che si trovano nel loro raggio di copertura, oppure possono essere determinati dei protocolli detti multi-salto che permettono di trasmettere ad \textit{host} che non sono direttamente raggiungibili.
        \end{definition}
        \subsubsection{Standard \texttt{IEEE 802.11}}
            Lo standard \texttt{IEEE 802.11} definisce i protocolli di livello 2 e 1 per le reti \textit{wireless}. Esistono molte varianti di questo standard, le più comuni sono:
            \begin{description}
                \item[\texttt{802.11b}] $2.4-5.8GHz$ fino ad un massimo di $11Mbps$ ed protocollo simile ad \texttt{CDMA}.
                \item[\texttt{802.11a}] $5-6GHz$ fino ad un massimo di $54Mbps$
                \item[\texttt{802.11g}] $2.4-5.8GHz$ fino ad un massimo di $54Mbps$
                \item[\texttt{802.11n}] Antenne multiple, $2.4-5.8GHz$ con range di velocità da $72$ a $600Mbps$ per tre flussi contemporanei.
                \item[\texttt{802.11ac}] canali con più banda e più flussi, $5GHz$ con velocità fino a $6.9Gbps$ con 4 flussi contemporanei.
                \item[\texttt{802.11ax}] meglio conosciuto come \texttt{WiFi 6}, $2.4-5.8GHz$ con velocità fino a $11Gbps$ per 8 flussi contemporanei, quindi $1375Mbps$ per ogni stazione.
            \end{description}
        \subsubsection{Architettura in una rete \texttt{WLAN}}
            Un una architettura di base di una rete \texttt{WLAN} si hanno una stazione \textit{wireless} (\texttt{STA} o anche \textit{nodo}) che tramite un \textit{Basic Service Set} (\texttt{BSS}) che è un grippo di \texttt{STA} che usano lo stesso canale radio alla quale fà parte un \textit{Access Point} (\texttt{AP}) che è una stazione integrata nella \texttt{LAN} cablata attraverso un \textit{Distribution System} (\texttt{DS}) che fornisce connettività verso o altri \texttt{BSS} oppure verso \textit{internet}, l'unione che il \texttt{DS} crea tra più \texttt{BSS} è chiamata \textit{Extended Service Set} (\texttt{ESS}), infine il \textit{Portal} a monte del \texttt{DS} è l'entità che fornisce connettività verso l'esterno.\newline
            In tutto questo un \texttt{BSS} può essere isolato o connesso tramite un \texttt{AP} il quale solitamente gestisce il protocollo \texttt{MAC} e la gestione del canale radio, all'interno di un \texttt{ESS} possono essere presenti più \texttt{AP} con protocolli \texttt{MAC} differenti.
        \subsubsection{\texttt{802.11}}
            \paragraph{Canali e assocazione} In una rete \texttt{802.11b} lo spettro di frequenza da $2.4GHz$ a $2.485GHz$ è diviso in $11$ canali (o $14$ se sussistono particolari condizioni), è dunque l'amministratore di quell'\texttt{AP} a scegliere il canale, solitamente sulla presenza o meno di altri \texttt{AP} vicini e sulla presenza o meno di interferenze. Quando un \texttt{STA} deve associarsi ad un \texttt{AP} ascolta i messaggi di \texttt{beacon} inviati da tutti gli \texttt{AP} i quali contengono informazioni sul nome (\textit{service set }\texttt{ID}, \texttt{SSID}) e l'indirizzo \texttt{MAC} dell'\texttt{AP}. Allora la \texttt{STA} sceglie l'\texttt{AP} tra l'elenco di quelli disponibili, si associa ad esso (se necessario si autentica), usa il protocollo \texttt{DHCP} per ottenere un indirizzo \texttt{IP} e poi può comunicare con la rete.
            \paragraph{Scansione attiva e passiva} Nel processo di associazione distinguiamo la scansione attiva da una passiva: in una scansione passiva l'\textit{host} ascolta i \textit{beacon} degli \texttt{AP} ed l'\textit{host} con quale associarci, l'\texttt{AP} risponde con un messaggio di conferma. Nella scansione attiva invece l'\textit{host} invia a tutti gli \texttt{AP} un messaggio di \textit{probe request} e l'\texttt{AP} risponde con un messaggio di \textit{probe response}, successivamente l'\textit{host} si associa all'\texttt{AP} scelto con la solita procedura.
        \subsubsection{Caratteristiche dei collegamenti \textit{wireless}}
            A differenza delle connessioni cablate, le connessioni \textit{wireless} sono molto più soggette a interferenze, infatti telefoni, forni a microonde o altri usano la stessa frequenza di $2.4GHz$ e possono interferire con la connessione. Inoltre la presenza di ostacoli come muri, mobili o persone può ridurre la qualità e l'intensità del segnale. Inoltre è possibile che lo stesso segnale raggiunga la stazione ricevente in modo diverso, infatti il segnale può essere riflesso, rifratto o assorbito.\newline
            Tutto ciò insieme al fatto che l'attenuazione delle onde radio segue una legge quadratica inversa alla distanza rende la \textit{collision detection} impossibile. Legato allo stesso problema sussiste il fatto che una antenna non può trasmettere e ricevere contemporaneamente (auto-interferenza) e che due antenne vicine possono interferire tra di loro (interferenza tra antenne).
    \subsection{Protocolli \texttt{MAC}}
        In quanto come descritto precedentemente la \textit{collision detection} è impossibile, si usano protocolli \texttt{MAC} differenti che si basano su \texttt{CSMA/CA} (\textit{Carrier Sense Multiple Access with Collision Avoidance}). Questo protocollo prevede che le \texttt{STA} che necessitano di trasmettere si contendano il canale radio, ogni volta che una \texttt{STA} ha qualcosa da trasmettere ripete la contesa con le altre \texttt{STA}. Sono presenti tuttavia delle eccezioni a ciò, infatti in versioni più avanzate di \texttt{802.11} (come \texttt{802.11n/ac/ax}) si possono concedere periodi più lunghi rispetto alla lunghezza del frame. Questo viene chiamato \texttt{TXOP} (\textit{Transmission OPportunity}) e permette di trasmettere più frame in un unico periodo di tempo.
        \subsubsection{Trasmettitore \texttt{CSMA}}
            Il trasmettitore \texttt{CSMA} funziona nel seguente modo:
            \begin{enumerate}
                \item Se il canale rimane libero per un tempo \texttt{DIFS} (\textit{Distributed InterFrame Space}) allora la \texttt{STA} può trasmettere. Altrimenti entra in \textit{backoff}.
                \item Se il canale è occupato allora:
                    \subitem si sceglie un tempo di \textit{backoff} casuale
                    \subitem quando il tempo di \textit{backoff} è scaduto e il canale è libero si trasmette solo quando il timer è scaduto.
            \end{enumerate}
            Se non si riceve nessun \texttt{ACK} allora si aumenta il valore massimo del tempo di \textit{backoff} e si sceglie un nuovo tempo di \textit{backoff} casuale.\newline
            Se invece il messaggio viene ricevuto correttamente si attende un tempo \texttt{SIFS} (\textit{Short InterFrame Space})\footnote{\texttt{DIFS}>\texttt{SIFS} per evitare altre trasmissioni di altri \textit{host}} e si invia un \texttt{ACK} al mittente. In questo protocollo \texttt{MAC} gli \texttt{ACK} sono necessari per garantire la corretta ricezione del messaggio e per le eventuali collisioni dopo il tempo di \textit{backoff} in quanto il mittente non può ascoltare il canale mentre trasmette.
            \paragraph{Problema del terminale nascosto} Il problema del terminale nascosto è un problema che si verifica quando due \texttt{STA} non possono ascoltare il canale tra di loro e quindi non possono sapere se il canale è libero o occupato. Ovvero assumiamo la situazione dove gli host \texttt{A} e \texttt{B} sono nel range di comunicazione, inoltre \texttt{B} e \texttt{C} possono comunicare tra di loro sullo stesso canale, in questa situazione ne \texttt{A} ne \texttt{C} possono sapere se il canale è libero o occupato. Questo problema può essere risolto con l'uso di \texttt{CSMA/CA} con \textit{handshaking}.
        \subsubsection{\texttt{MAC} con messaggi \texttt{RTS} e \texttt{CTS}}
            In questa situazione si aggiunge una procedura di \textit{handshaking} dove il trasmettitore deve inviare un messaggio di \texttt{RTS} (\textit{Request To Send}) e il ricevitore se è libero risponde con un messaggio di \texttt{CTS} (\textit{Clear To Send}). Questo permette di risolvere il problema del terminale nascosto e di evitare collisioni. Questo metodo però introduce un ritardo aggiuntivo dovuto alla trasmissione dei messaggi di \texttt{RTS} e \texttt{CTS}. Nel momento in cui il trasmettitore riceve il messaggio di \texttt{CTS} può trasmettere il messaggio ed il canale viene riservato per la trasmissione del messaggio.
            \paragraph{Problema del terminale esposto} Il problema del terminale esposto è un problema che si verifica quando si usa un protocollo di \textit{handshaking} e due \textit{host} vogliono comunicare con due \texttt{AP} diversi. In questo caso il \textit{host} \texttt{A} invia un messaggio di \texttt{RTS} al \texttt{AP} \texttt{X} il quale invierà un messaggio di \texttt{CTS} di fatto bloccando il canale anche per l'\texttt{AP} \texttt{Y} che si trova nelle vicinanze. Allora l'host \texttt{B} dovrà aspettare l'\texttt{ACK} del \texttt{AP} \texttt{X} che libera l'\texttt{AP} \texttt{Y} e quindi il canale.
    \subsection{Frame ed indirizzi \texttt{802.11}}
        \begin{figure}[H]
            \centering
            \includegraphics[width=0.6\linewidth]{05/frame802.11.png}
            \caption{Il \textit{frame} \texttt{802.11}}
        \end{figure}
        Possiamo notare che oltre ai campi classici di un \textit{frame} Ethernet, il \textit{frame} \texttt{802.11} contiene:
        \begin{description}
            \item[\textit{Address 1}] L'indirizzo \texttt{MAC} del destinatario \textit{wireless}
            \item[\textit{Address 2}] L'indirizzo \texttt{MAC} del mittente \textit{wireless}
            \item[\textit{Address 3}] L'indirizzo \texttt{MAC} del \textit{router} al quale è collegato l'\texttt{AP} (destinatario \texttt{LAN})
            \item[\textit{Sequence Control}] Il numero di sequenza del \textit{frame}
            \item[\textit{Address 4}] L'indirizzo \texttt{MAC} del mittente \textit{wireless} (usato solo in modalità ad hoc)
        \end{description}
        Questi quattro indirizzi sono usati per permettere la comunicazione tra \texttt{STA} e \texttt{AP} e per permettere la comunicazione tra \texttt{STA} in modalità ad hoc.

    \chapter*{Conclusioni e Ringraziamenti}
    \thispagestyle{chapterInit}
    \addcontentsline{toc}{chapter}{Conclusioni e Ringraziamenti}
    \markboth{Conclusioni e Ringraziamenti}{Conclusioni e Ringraziamenti}
    
    \section*{Conclusioni}
    Con questo si concludono gli appunti del corso di Reti, il corso ha affrontato partendo dal livello applicazione fino al livello fisico la struttura e il funzionamento delle reti di calcolatori. Durante il percorso sono stati affrontati i principali protocolli e le tecnologie che permettono la comunicazione tra i dispositivi che compongono una rete. Da notare una particolare attenzione verso i "problemi del \textit{routing}" che sono ancora ad oggi uno dei principali problemi delle reti di calcolatori. Il mondo delle reti ad oggi è un settore dell'informatica nel quale la ricerca e lo sviluppo sono in continua, e rapida, evoluzione, e che ha un impatto sempre maggiore nella vita di tutti i giorni in quanto tutti (o per lo meno chi legge) utilizzano quotidianamente il complesso mondo di internet e delle reti, e quindi è importante conoscere e capire come funzionano e come le tecnologie attuali influenzino la nostra vita quotidiana.
    \section*{Ringraziamenti}
    Voglio ringraziare in primo luogo il prof. Casari Paolo per il materiale sulla quale si basano questi appunti e per l'interesse e la passione che ha dimostrato durante il corso.\newline
    Ringrazio anche i miei colleghi di corso per le discussioni e le collaborazioni che ci sono state durante il corso, è grazie anche ad alcuni di loro che ha avuto e trovato il tempo e la dedizione per scrivere questi appunti.\newline
    Infine ringrazio chi ha letto questi appunti, spero che siano stati utili e che abbiano aiutato a chiarire e a comprendere meglio gli argomenti trattati durante il corso, e che possano essere utili anche a chi leggerà in futuro.
    \vfill
    \footnotesize\section*{Note}
    Questi appunti sono stati scritti durante il corso di Reti, tenuto dal prof. Casari Paolo presso l'Università degli Studi di Trento nell'anno accademico 2024/2025. Gli appunti sono stati scritti in \LaTeX{} e sono disponibili su \href{https://github.com/lucafano04/appuntisecondoanno}{GitHub} e sono rilasciati sotto licenza \href{https://creativecommons.org/licenses/by-nc-sa/4.0/}{CC BY-NC-SA 4.0} come conseguenza sono liberamente utilizzabili e modificabili, ma non possono essere utilizzati a scopi commerciali e devono mantenere la stessa licenza, il materiale rimane liberamente usabile e modificabile nell'ambito accademico, della formazione e della divulgazione scientifica e tecnologica. L'utilizzatore è tenuto a citare l'autore originale e a mantenere la stessa licenza per le opere derivate. Ognuno è libero di usare questi come punto di partenza per lo studio in funzione delle proprie esigenze e di condividerli con chiunque ne possa trarre beneficio, anzi è incoraggiato a farlo.
    L'autore (Luca Facchini) non si assume nessuna responsabilità sull'uso che verrà fatto di questi appunti e non garantisce la completa correttezza e completezza degli stessi, inoltre non si assume nessuna responsabilità per eventuali errori o imprecisioni presenti negli appunti, questi vengono infatti distribuiti \textit{as is} e possono contenere errori o imprecisioni, l'utilizzatore è tenuto a verificare e a correggere eventuali errori presenti negli appunti. Nell'eventualità di errori o imprecisioni si prega di contattare l'autore e/o di aprire una \textit{issue} sul repository di GitHub. (Ultimo aggiornamento: \today)

\end{document}