\chapter*{Conclusioni e Ringraziamenti}
\thispagestyle{chapterInit}
\addcontentsline{toc}{chapter}{Conclusioni e Ringraziamenti}
\markboth{Conclusioni e Ringraziamenti}{Conclusioni e Ringraziamenti}

\section*{Conclusioni}
Con questo si concludono gli appunti del corso di Reti, il corso ha affrontato partendo dal livello applicazione fino al livello fisico la struttura e il funzionamento delle reti di calcolatori. Durante il percorso sono stati affrontati i principali protocolli e le tecnologie che permettono la comunicazione tra i dispositivi che compongono una rete. Da notare una particolare attenzione verso i "problemi del \textit{routing}" che sono ancora ad oggi uno dei principali problemi delle reti di calcolatori. Il mondo delle reti ad oggi è un settore dell'informatica nel quale la ricerca e lo sviluppo sono in continua, e rapida, evoluzione, e che ha un impatto sempre maggiore nella vita di tutti i giorni in quanto tutti (o per lo meno chi legge) utilizzano quotidianamente il complesso mondo di internet e delle reti, e quindi è importante conoscere e capire come funzionano e come le tecnologie attuali influenzino la nostra vita quotidiana.
\section*{Ringraziamenti}
Voglio ringraziare in primo luogo il prof. Casari Paolo per il materiale sulla quale si basano questi appunti e per l'interesse e la passione che ha dimostrato durante il corso.\newline
Ringrazio anche i miei colleghi di corso per le discussioni e le collaborazioni che ci sono state durante il corso, è grazie anche ad alcuni di loro che ho avuto e trovato il tempo e la dedizione per scrivere questi appunti.\newline
Infine ringrazio chi ha letto questi appunti, spero che siano stati utili e che abbiano aiutato a chiarire e a comprendere meglio gli argomenti trattati durante il corso, e che possano essere utili anche a chi leggerà in futuro.
\vfill
\footnotesize\section*{Note}
Questi appunti sono stati scritti durante il corso di Reti, tenuto dal prof. Casari Paolo presso l'Università degli Studi di Trento nell'anno accademico 2024/2025. Gli appunti sono stati scritti in \LaTeX{} e sono disponibili su \href{https://github.com/lucafano04/appuntisecondoanno}{GitHub} e sono rilasciati sotto licenza \href{https://creativecommons.org/licenses/by-nc-sa/4.0/}{CC BY-NC-SA 4.0} come conseguenza sono liberamente utilizzabili e modificabili, ma non possono essere utilizzati a scopi commerciali e devono mantenere la stessa licenza, il materiale rimane liberamente usabile e modificabile nell'ambito accademico, della formazione e della divulgazione scientifica e tecnologica. L'utilizzatore è tenuto a citare l'autore originale e a mantenere la stessa licenza per le opere derivate. Ognuno è libero di usare questi come punto di partenza per lo studio in funzione delle proprie esigenze e di condividerli con chiunque ne possa trarre beneficio, anzi è incoraggiato a farlo.
L'autore (Luca Facchini) non si assume nessuna responsabilità sull'uso che verrà fatto di questi appunti e non garantisce la completa correttezza e completezza degli stessi, inoltre non si assume nessuna responsabilità per eventuali errori o imprecisioni presenti negli appunti, questi vengono infatti distribuiti \textit{as is} e possono contenere errori o imprecisioni, l'utilizzatore è tenuto a verificare e a correggere eventuali errori presenti negli appunti. Nell'eventualità di errori o imprecisioni si prega di contattare l'autore e/o di aprire una \textit{issue} sul repository di GitHub. (Ultimo aggiornamento: \today)