\chapter{Il livello di rete}
\label{cap:livelloRete}
\thispagestyle{chapterInit}
\section{Visione d'insieme}
    \paragraph{Obbiettivo del livello di rete} L'obbiettivo principale del livello di rete è quello di permettere la comunicazione tramite reti diverse attraverso apparecchi detti \textit{router} i quali hanno il compito di inoltrare le informazioni verso la destinazione.
    \paragraph{Funzioni principali} Esistono due funzioni principali del livello di rete: \begin{description}
        \item[Inoltro \textit{forwarding}] Questa è una operazione a livello locale che consiste nel prendere un pacchetto in ingresso e inoltrarlo verso l'uscita corretta.
        \item[Instradamento \textit{routing}] Questa è una operazione a livello globale che consiste nel determinare il percorso migliore per inoltrare un pacchetto verso la destinazione. Per questa operazione si utilizzano degli algoritmi di \textit{routing}.
    \end{description}
    Queste due funzioni sono legate tra loro, ma possono essere isolate. Infatti convenzionalmente distinguiamo con \textit{control plane} la parte del livello di rete che si occupa dell'istradamento e con \textit{data plane} la parte che si occupa dell'inoltro. Questa distinzione è utile per capire come funzionano i router.
    \subparagraph{\textit{Data Pane}} Il \textit{data plane} ha funzione a livello locale ad ogni \textit{router}, questo è il livello che determina \underline{come} inoltrare un \textit{datagram} fornendo la funzione di \textit{forwarding}.
    \subparagraph{\textit{Control Plane}} Il \textit{control plane} ha funzione a livello globale, questo è il livello che determina \underline{dove} inoltrare un \textit{datagram} fornendo la funzione di \textit{routing}.
\section{Come è fatto un router}
    Visto a livello "alto" un router è composto da tre livelli principali: \begin{description}
        \item[Terminazione di linea] Questo è il livello più basso del router, è composto da un'interfaccia di rete che si occupa di ricevere i pacchetti e di inviarli al livello successivo.
        \item[Protocollo di livello \textit{data link}] Questo livello si occupa di ricevere i pacchetti dal livello precedente e di inviarli al livello successivo. Inoltre si occupa di fare il controllo degli errori e di gestire il flusso.
        \item[Inoltro e \textit{buffer}] Questo è il livello che si occupa di inoltrare i pacchetti verso la destinazione. Inoltre si occupa di fare il \textit{buffering} dei pacchetti in caso di congestione.
    \end{description}
    \subsection{Sistemi di commutazione}
        I sistemi di commutazione trasferiscono i pacchetti dalle porte di ingresso all'uscita appropriata. Definiamo come \textbf{tasso di comunicazione} la frequenza alla quale i pacchetti vengono portati dall'ingresso all'uscita (spesso è un multiplo della velocità di comunicazione) 
        \subsubsection{Commutazione a memoria}
            Questo è il metodo più semplice, i pacchetti vengono memorizzati in un buffer comune e poi inoltrati verso l'uscita appropriata. Questo metodo è molto semplice ma ha il problema che la velocità di inoltro è limitata dalla velocità di accesso alla memoria.
        \subsubsection{Commutazione a bus}
            Questo metodo consiste nel collegare le porte di ingresso e di uscita tramite un bus, sempre comune a tutte le porte. Questo metodo risulta lento in quanto non possono essere trasferiti più pacchetti contemporaneamente anche se le porte di ingresso e di uscita sono diverse. Il \texttt{Cisco 5600} è un esempio di router che utilizza questo metodo e riesce a trasferire fino a 32 Gbit/s.
        \subsubsection{Commutazione a matrice}
            Questo metodo consiste nel collegare le porte di ingresso e di uscita tramite una matrice di commutazione. Questo metodo è molto veloce in quanto permette di trasferire più pacchetti contemporaneamente in quanto se le porte sono differenti allora basta attivare i vari collegamenti della matrice. Questo metodo è molto veloce ed ispirato ai primi commutatori telefonici. Il \texttt{Cisco 12000} è un esempio di router che utilizza questo metodo e riesce a trasferire fino a 60 Gbit/s.
    \subsection{Accodamenti}
        Gli accodamenti sono utilizzati per evitare la perdita di pacchetti in caso di congestione dell'apparecchio di rete. Questo è un problema molto comune in quanto i router sono dispositivi molto veloci e le porte di uscita sono molto più lente. Per evitare la perdita di pacchetti si utilizzano delle code che permettono di memorizzare i pacchetti in attesa di essere inoltrati.
        Le code possono essere formate in ingresso, quando una stessa porta di uscita è condivisa da più porte di ingresso e quindi una porta di uscita può essere congestionata. Le code possono essere formate in uscita, quando una porta di uscita ha un \textit{link} più lento rispetto alla velocità di inoltro dei pacchetti tramite le porte di ingresso o il commutatore di pacchetto.
        \paragraph{Quanta memoria serve per i \textit{buffer}} Secondo \texttt{RFC 3439} la quantità di memoria necessaria per i buffer è data dalla formula: \[M = \frac{RTT \cdot C}{\sqrt{N}}\] Dove: \begin{description}
            \item[$M$] è la memoria necessaria per il buffer
            \item[$RTT$] è il tempo di round trip
            \item[$C$] è la capacità del collegamento
            \item[$N$] è il numero di connessioni
        \end{description}
        \paragraph{Meccanismi di \textit{scheduling}}
            I meccanismi di \textit{scheduling} sono utilizzati per decidere quale pacchetto inoltrare quando si ha la possibilità di inoltrare più pacchetti. Il meccanismo più semplice è il \textit{First In First Out} (\texttt{FIFO}) che inoltra i pacchetti in ordine di arrivo. Inoltre viene applicata una politica di scarto dei pacchetti in caso di buffer pieno. Questa politica può essere: \begin{description}
                \item[\textit{Drop Tail}] Questa politica scarta i pacchetti in arrivo quando il buffer è pieno.
                \item[\textit{Random Early Detection}] Questa politica scarta i pacchetti in arrivo in modo casuale quando il buffer è pieno.
                \item[\textit{Priority Drop}] Questa politica scarta i pacchetti in arrivo in base alla priorità.
            \end{description}
            La politica di scarto dei pacchetti dipende dall'implementazione del router.
\section{Il protocollo \texttt{IP}}
    \subsection{Il formato del \textit{datagram} \texttt{IP} (\texttt{IPv4})}
        \begin{figure}[H]
            \centering
            \includegraphics[width=0.36\textwidth]{04/datagramIPv4.png}
            \caption{Il formato del \textit{datagram} \texttt{IP} (\texttt{IPv4})}
            \label{fig:IPv4Header}
        \end{figure}
        Il formato del \textit{datagram} \texttt{IP} è composto da 20 byte di intestazione e da un campo dati. Il campo dati può contenere fino a 65.535 byte.  Di seguito si riportano i vari campi dell'intestazione:
        \begin{description}
            \item[VER] (4 bit) Questo campo contiene la versione del protocollo \texttt{IP} utilizzato.
            \item[Lunghezza \textit{header}] (4 bit) Questo campo contiene la lunghezza dell'intestazione in parole da 32 bit. (=5 se non ci sono opzioni)
            \item[Tipo di servizio - \texttt{ToS}] (8 bit) Questo campo contiene informazioni sul tipo di servizio richiesto.
            \item[Lunghezza totale] (16 bit) Questo campo contiene la lunghezza totale del \textit{datagram} in byte.
            \item[Identificativo] (16 bit) Questo campo contiene un numero univoco per il \textit{datagram}.
            \item[Flag] (3 bit) Questo campo contiene i flag per il frammento. Il primo bit è il bit di \textit{Don't Fragment}, il secondo bit è il bit di \textit{More Fragment} e il terzo bit è il bit di \textit{Fragment Offset}.
            \item[Offset] (13 bit) Questo campo contiene l'offset del frammento. (Espresso in multipli di 8 byte)
            \item[\textit{Time To Live} - \texttt{TTL}] (8 bit) Questo campo contiene il numero di \textit{hop} massimo che il \textit{datagram} può fare.
            \item[Protocollo] (8 bit) Questo campo contiene il protocollo di trasporto che si trova nel campo dati.
            \item[Checksum] (16 bit) Questo campo contiene il checksum dell'intestazione.
            \item[Indirizzo IP sorgente] (32 bit) Questo campo contiene l'indirizzo IP sorgente.
            \item[Indirizzo IP destinazione] (32 bit) Questo campo contiene l'indirizzo IP destinazione.
            \item[Opzioni] (variabile) Questo campo contiene le opzioni del \textit{datagram}.
            \item[Padding] (variabile) Questo campo contiene il padding per allineare l'intestazione a multipli di 32 bit.
        \end{description}
    \subsection{\texttt{MTU} e Frammentazione}
        La \texttt{MTU} (\textit{Maximum Transmission Unit}) è la dimensione massima di un pacchetto che può essere trasmesso su un collegamento, ogni \textit{hardware} specifica il proprio \texttt{MTU}. Se un \textit{datagram} è più grande della \texttt{MTU} allora il \textit{datagram} viene frammentato in pacchetti più piccoli. Questo processo è chiamato \textit{frammentazione}. I pacchetti frammentati vengono poi ricomposti alla destinazione.
        \paragraph{Valori Standard \texttt{MTU}} Per alcuni tipi di collegamenti sono stati definiti dei valori standard di \texttt{MTU}. Ad esempio per le reti Ethernet la \texttt{MTU} è di $1500$ byte, per le reti \texttt{WLAN 802.11} la \texttt{MTU} è di $2304$ byte,\dots. In un collegamento tra due \textit{host} possono essere presenti due valori di \texttt{MTU} diversi
        \subparagraph{Esempio} Supponendo che per raggiungere un host \texttt{B} si debba passare prima da una rete con $1500$ byte di \texttt{MTU} e poi da una rete con $1000$, tutto ciò tramite un router \texttt{R}. Allora quando il \textit{router} \texttt{R} riceve il datagram di $1500$ byte lo frammenta in due pacchetti di $1000$ byte e $500$ byte. I pacchetti vengono quindi inoltrati alla destinazione. Quando l'host \texttt{B} riceve i pacchetti li ricomponi e li passa al livello di trasporto.
        
    \subsection{Indirizzamento e \texttt{NAT}}
        \subsubsection{Indirizzi \texttt{IP}}
            Un indirizzo \texttt{IP} è composto da $32$ bit ed è associato ad un'interfaccia di rete. Una interfaccia è una connessione tramite mezzo fisico o logico, solitamente ogni \textit{host} ha una o più interfacce di rete.
            \paragraph{Caratteristiche indirizzi \texttt{IP}} Gli host e i router devono usare le stesse convenzioni per gli indirizzi \texttt{IP}, inoltre ogni indirizzo \texttt{IP} deve essere unico e raggiungibile da un qualsiasi punto di internet. Quando si invia un pacchetto \texttt{IP} si invia l'indirizzo \texttt{IP} sorgente e l'indirizzo \texttt{IP} destinazione. I \textit{router} sono apparati di rete che quando ricevono un pacchetto \texttt{IP} decidono dove inoltrarlo in base all'indirizzo \texttt{IP} di destinazione.
            \paragraph{Norazione indirizzo \texttt{IP}} Gli indirizzi \texttt{IP} sono composti da $4$ gruppi di $8$ bit e sono scritti in notazione decimale a punti. Ogni gruppo di $8$ bit è espresso in decimale e separato da un punto. Gli indirizzi disponibili vanno dà: $0.0.0.0$ a $255.255.255.255$ 
            \paragraph{Gerarchie indirizzi \texttt{IP}} Gli indirizzi \texttt{IP} sono organizzati in una struttura gerarchica. Inoltre solitamente sono divisi in due parti: \begin{description}
                \item[Parte di rete] Questa parte identifica la rete a cui appartiene l'indirizzo \texttt{IP}.
                \item[Parte di \textit{host}] Questa parte identifica l'\textit{host} all'interno della rete.
            \end{description}
            \subparagraph{Classi di indirizzi \texttt{IP}} Gli indirizzi \texttt{IP} sono divisi in classi sulla base dei primi bit dell'indirizzo e dalla lunghezza del prefisso: \begin{description}
                \item[Classe \texttt{A}] Gli indirizzi di classe \texttt{A} hanno il primo bit a \texttt{0} e sono composti da $8$ bit di rete e $24$ bit di \textit{host}. Gli indirizzi vanno da $0.0.0.0$ a $ 127.255.255.255$ e sono riservati per le reti molto grandi.
                \item[Classe \texttt{B}] Gli indirizzi di classe \texttt{B} hanno i primi due bit a \texttt{10} e sono composti da $16$ bit di rete e $16$ bit di \textit{host}. Gli indirizzi vanno da $128.0.0.0$ a $191.255.255.255$ e sono riservati per le reti di medie dimensioni.
                \item[Classe \texttt{C}] Gli indirizzi di classe \texttt{C} hanno i primi tre bit a \texttt{110} e sono composti da $24$ bit di rete e $8$ bit di \textit{host}. Gli indirizzi vanno da $192.0.0.0$ a $223.255.255.255$ e sono riservati per le reti di piccole dimensioni.
                \item[Classe \texttt{D}] Gli indirizzi di classe \texttt{D} hanno i primi quattro bit a \texttt{1110} e sono riservati per i \textit{multicast}.
                \item[Classe \texttt{E}] Gli indirizzi di classe \texttt{E} hanno i primi quattro bit a \texttt{1111} e sono riservati per usi futuri.
            \end{description}
        \subsubsection{Assegnazione indirizzi \texttt{IP}}
            Gli indirizzi \texttt{IP} sono assegnati dalla \texttt{ICANN} (\textit{Internet Corporation for Assigned Names and Numbers}) che riserva una intera classe ai \texttt{ISP} (\textit{Internet Service Provider}) e poi questi assegnano gli indirizzi ai propri clienti. Gli indirizzi \texttt{IP} sono assegnati in modo gerarchico e quindi un \texttt{ISP} può assegnare un intero blocco di indirizzi ad un altro \texttt{ISP} e questo può assegnare un blocco di indirizzi ad un altro \texttt{ISP} e così via.
    \subsection{Indirizzamento \textit{classless}}
        In quanto ci si è accorti che con la suddivisione degli indirizzi in classi si stava sprecando molti indirizzi, si è deciso di passare ad un indirizzamento \textit{classless}. Questo tipo di indirizzamento permette di avere una suddivisione più flessibile degli indirizzi, in quanto possiamo richiedere al nostro \texttt{ISP} solo un \textit{range} di indirizzi e non una classe intera, ad esempio l'\texttt{ISP} può usare un prefisso di $26$ bit per identificare la rete nel mondo e successivamente usare i restanti $6$ bit per identificare tutti gli \textit{host} della rete.
        In questa situazione se un \texttt{ISP} ha acquistato una rete di classe \texttt{C} e ha bisogno di dividere la rete in quattro clienti allora può dividere la rete in $4$ sotto-reti e assegnare un prefisso di $26$ bit ($24$ per la classe \texttt{C} e $2$ per le sotto-reti) e i restanti $6$ bit per gli \textit{host} di ogni sotto-rete.
        \subsubsection{Maschere di rete}
            In quanto ora non si ha più una suddivisione fissa degli indirizzi, si è deciso di introdurre le \textit{maschere di rete}. Queste maschere sono composte da $32$ bit e sono composte da una parte di $1$ quelli che identificano il prefisso di rete e da una parte di $0$ che identificano gli \textit{host} della rete. Ad esempio la maschera di rete per una rete di classe \texttt{C} è $11111111.11111111.11111111.00000000$.
            \paragraph{Peché si usa?} La maschera di rete viene usata per identificare la rete di appartenenza di un indirizzo IP. Per fare ciò si fa un'operazione di \texttt{AND} tra l'indirizzo IP di destinazione e la maschera di rete. Se il risultato è uguale all'indirizzo di rete allora l'indirizzo IP appartiene alla rete. Quindi se il prefisso di rete è $128.10.0.0$ ovvero: \[10000000.00001010.00000000.00000000\] e la maschera di rete è $255.255.0.0$ ovvero: \[11111111.11111111.00000000.00000000\] e l'indirizzo IP di destinazione di un determinato pacchetto è: $128.10.2.3$ ovvero: \[10000000.00001010.00000010.00000011\] allora facendo l'operazione di \texttt{AND} tra l'indirizzo IP e la maschera di rete si ottiene: \[\begin{aligned}
                10000000.00001010.00000010.00000011 &\\
                11111111.11111111.00000000.00000000 &\\
                \hline
                10000000.00001010.00000000.00000000&=128.10.0.0
            \end{aligned}
            \] Quindi l'indirizzo IP appartiene alla rete e il pacchetto viene inoltrato a quella determinata porta di uscita.
        \subsubsection{Notazioni \texttt{CIDR}}
            \texttt{CIDR} o \textit{Classless Inter-Domain Routing} è un metodo per rappresentare le maschere di rete. Questo metodo consiste nel rappresentare la maschera di rete con un prefisso di bit. Ad esempio la maschera di rete $/24$ è uguale alla maschera di rete $11111111.11111111.11111111.00000000$, ovvero una maschera di rete di classe \texttt{C}, quindi i primi $3$ gruppi di bit sono appartenenti all'\textit{network-id} e l'ultimo gruppo di bit è appartenente all'\textit{host-id}. La notazione prevede inoltre che gli indirizzi \texttt{IP} siano rappresentati nel seguente modo: \texttt{ddd.ddd.ddd.ddd/m} dove ogni singolo \texttt{d} rappresenta un gruppo di bit e \texttt{m} rappresenta il numero di bit del prefisso di rete.
            \paragraph{Esempio} L'indirizzo $193.168.32.199/26$ ha un prefisso di rete di $26$ bit, quindi il \textit{network-id} è: $$11000001.1010100.000000010.11$$ e l'\textit{host-id} è $$00111$$ Esistono dunque $2^6=64$ indirizzi \texttt{IP} nella rete.
            \paragraph{Inoltro con \texttt{CIDR}} L'inoltro con \texttt{CIDR} è molto semplice e non differisce dall'inoltro con le classi. Infatti si fa l'operazione di \texttt{AND} tra l'indirizzo \texttt{IP} di destinazione e la maschera di rete e si confronta il risultato con l'indirizzo di rete. Se il risultato è uguale all'indirizzo di rete allora l'indirizzo \texttt{IP} appartiene alla rete e il pacchetto viene inoltrato alla porta di uscita corretta. Nella situazione in cui ci siano più reti che corrispondono al risultato dell'operazione di \texttt{AND} allora si sceglie la rete con il prefisso più lungo in quanto si presume che questa sia la rete più specifica e quindi la più breve.
            \paragraph{Aggregazione dei percorsi} L'aggregazione dei percorsi è una tecnica che permette di ridurre il numero di percorsi che un router deve memorizzare. Questa tecnica consiste nel raggruppare più reti in un'unica rete più grande. Questa tecnica è molto utile in quanto permette di ridurre il numero di percorsi che un router deve memorizzare e quindi di velocizzare l'inoltro dei pacchetti. Esempio se un \texttt{ISP} controlla tutte le reti $200.23.16.0/23$, $200.23.18.0/23$\dots $200.23.30.0/23$ allora può raggruppare tutte queste reti in un'unica rete $200.23.16.0/20$
            \paragraph{Inoltro di default} L'inoltro di default è una tecnica "\textit{last resource}" usata nel caso in cui non si trovi nessuna corrispondenza tra l'indirizzo \texttt{IP} di destinazione e le reti memorizzate nel router. In questo caso il router inoltra il pacchetto alla porta di uscita di default definita dal router e che solitamente è la porta di uscita verso internet. In questo caso nel router è presente una rotta con \texttt{IP} di destinazione $0.0.0.0$ e maschera di rete $/0$, la cui combinazione è sempre vera per qualunque indirizzo, per la regola del "prefisso più lungo" il router inoltra il pacchetto alla porta di uscita di default solo se non trova nessuna corrispondenza tra l'indirizzo \texttt{IP} di destinazione e le reti memorizzate nel router.
        \subsubsection{Tabella di routing}
            Nella tabella di routing oltre all'associazione "maschera di rete - porta di uscita" è necessario memorizzare anche l'indirizzo \texttt{IP} del prossimo \textit{router} a cui inoltrare il pacchetto, questo in quanto sapere che il pacchetto deve essere inoltrato ad una determinata porta di uscita non è sufficiente se fossero presenti più dispositivi connessi alla stessa porta di uscita.
    \subsection{Tipi di indirizzi}
        In quanto gli indirizzi \texttt{IP} disponibili sono $2^{32}$ e il numero di dispositivi connessi a internet è molto maggiore, si è deciso di introdurre dei tipi di indirizzi pubblici e privati, in modo da risparmiare indirizzi \texttt{IP} pubblici e di proteggere la rete interna da attacchi esterni.
        \subsubsection{Indirizzi pubblici e privati}
            Gli indirizzi \texttt{IP} sono divisi in due categorie: \begin{description}
                \item[Indirizzi pubblici] Gli indirizzi pubblici sono indirizzi che possono essere raggiunti da qualsiasi punto di internet. Questi indirizzi sono assegnati dalla \texttt{ICANN} e sono unici.
                \item[Indirizzi privati] Gli indirizzi privati sono indirizzi che non possono essere raggiunti da internet. Questi indirizzi sono riservati per le reti private e non possono essere usati per comunicare con internet, vengono bloccati dai router. Gli indirizzi privati sono: \begin{itemize}
                    \item Da \texttt{10.0.0.0} a \texttt{10.255.255.255} (\texttt{10.0.0.0/8})
                    \item Da \texttt{172.16.0.0} a \texttt{172.31.255.255} (\texttt{172.16.0.0/12})
                    \item Da \texttt{192.168.0.0} a \texttt{192.168.255.255} (\texttt{192.168.255.255/16})
                \end{itemize}
            \end{description}
        \subsubsection{\textit{Network Area Translation} \texttt{NAT}}
            Il \texttt{NAT} è una tecnica che permette di tradurre gli indirizzi privati in indirizzi pubblici e viceversa. Questo permette ai dispositivi di una rete privata di accedere a internet senza avere un indirizzo \texttt{IP} pubblico. Il \texttt{NAT} è un protocollo appartenente al \textit{router} funzionante nel seguente modo:
            \begin{enumerate}
                \item Il \textit{router} sostituisce l'indirizzo \texttt{IP} di sorgente e la porta di sorgente del pacchetto con il proprio indirizzo \texttt{IP} pubblico e una porta casuale.
                \item Il \textit{router} memorizza l'associazione tra l'indirizzo \texttt{IP} e porta originale con l'indirizzo \texttt{IP} pubblico e la porta generata.
                \item Viene inoltrato il pacchetto alla rete di destinazione, seguendo la tabella di routing.
                \item Quando il pacchetto di risposta arriva al \textit{router}, il \textit{router} sostituisce l'indirizzo \texttt{IP} di destinazione e la porta di destinazione con l'indirizzo \texttt{IP} privato e la porta originale.
                \item Il \textit{router} inoltra il pacchetto alla rete privata.
            \end{enumerate}
            \paragraph{Vantaggi/svantaggi del \texttt{NAT}} In quanto il numero di porta è costituito da $16$ bit, il numero di porte disponibili è $2^{16}=65536$, quindi il \texttt{NAT} permette di avere fino a $65536$ dispositivi connessi alla stessa rete privata. Il "problema" è che il protocollo \texttt{NAT} viola l'architettura a livelli, in quanto dispositivo di rete il \textit{router} non dovrebbe agire sulle porte. La mancanza di \texttt{IP} dovrebbe essere risolta con l'introduzione di \texttt{IPv6} (anche se lentamente). La sola esistenza di \texttt{NAT} deve essere tenuta in considerazione quando si progettano applicazioni (come una rete \texttt{P2P} che non funziona con \texttt{NAT}). Infine se si vuole accedere ad un dispositivo con \texttt{NAT} da internet è necessario usare altri protocolli come \textit{Port Forwarding}, \texttt{UPnP} o altri.
            \subparagraph{\texttt{NAT} può essere utile} Oltre a risparmiare indirizzi \texttt{IP} pubblici, il \texttt{NAT} può essere utile per ovviare a problemi di routing, assumiamo per esempio che un router (al quale non abbiamo accesso) imposti una regola che impedisca l'uscita di pacchetti verso la nostra rete "interna" \texttt{B}, ma permetta che pacchetti verso la rete "pubblica" \texttt{P} vengano inoltrati. In questo caso il \texttt{NAT} può essere utile per far passare i pacchetti dalla rete "interna" \texttt{B} alla rete "interna" \texttt{A} senza modificare le regole di routing. Questo grazie al fatto che il \texttt{NAT} modifica l'indirizzo \texttt{IP} di sorgente, il router non riconosce i pacchetti come provenienti dalla rete "interna" \texttt{A} e quindi li inoltra.
        \subsubsection{Indirizzi \texttt{IP} speciali}
            Gli indirizzi \texttt{IP} speciali sono indirizzi che non possono essere assegnati ad un'interfaccia di rete. Questi indirizzi sono usati per scopi speciali e non possono essere usati per comunicare con internet. Alcuni tipi di indirizzi speciali sono: \begin{itemize}
                \item Indirizzi che identificano tutta la rete
                \item Indirizzi che permettono il \textit{broadcast} a tutti gli \textit{host} della rete
                \item Indirizzi che permettono il \textit{broadcast} in una rete locale (\textit{Limited broadcast address})
                \item Indirizzi di \textit{localhost}
                \item Indirizzi di \textit{loopback}
                \item Indirizzi di \textit{multicast}
                \item Indirizzi di \textit{link-local}
            \end{itemize}
            \paragraph{Identificativi di tutta la rete} Gli indirizzi che identificano tutta la rete sono indirizzi che identificano tutta la rete. Questi indirizzi sono usati per identificare la rete e non possono essere assegnati ad un'interfaccia di rete, questi sono identificati con tutti i bit della parte di \textit{host} a $0$. Ad esempio l'indirizzo \texttt{128.211.0.16/28} identifica tutta la rete
            \paragraph{Indirizzi di \textit{broadcast}} Per il \textit{Directed Broadcast Address} si ha che l'indirizzo di \textit{broadcast} è l'indirizzo che permette di inviare un pacchetto a tutti gli \textit{host} della rete. Questo indirizzo è identificato con tutti i bit della parte di \textit{host} a $1$. Ad esempio l'indirizzo \texttt{128.211.0.31/28} è l'indirizzo di \textit{broadcast} per la rete \texttt{128.211.0.16/28}
            \paragraph{Indirizzi di \textit{broadcast} locale} L'indirizzo di \textit{broadcast} locale è l'indirizzo che permette di inviare un pacchetto a tutti gli \textit{host} della rete locale. Questo indirizzo è identificato con tutti i bit dell'indirizzo a $1$. Quindi l'indirizzo \texttt{255.255.255.255} è l'indirizzo di \textit{broadcast} locale, anche se possa sembrare globale questo rimane locale perché non viene inoltrato alla rete pubblica.
            \paragraph{Indirizzi di \textit{localhost}} Per le regole del protocollo \texttt{TCP/IP} necessitiamo di un indirizzo \texttt{IP} anche per richiedere l'assegnazione di un indirizzo \texttt{IP} allora si è deciso di riservare un indirizzo \texttt{IP} ovvero: \texttt{0.0.0.0} che viene usato solo per le prime comunicazioni all'interno di una rete per "chiedere" l'assegnazione di un indirizzo \texttt{IP}
            \paragraph{Indirizzi di \textit{loopback}} Indirizzo \texttt{IP} riservato per il \textit{loopback} del PC o del dispositivo. Questo indirizzi sono \texttt{127.0.0.0/8}, ma il più comune è \texttt{127.0.0.1}
            \paragraph{Indirizzi di \textit{multicast}} Tutti gli indirizzi \texttt{IP} che iniziano con \texttt{1110} sono indirizzi di \textit{multicast}, molti apparati di rete bloccano il traffico di \textit{multicast} per evitare attacchi.
            \paragraph{Indirizzi \texttt{IP} \textit{local}} Sono indirizzi che non vengono assegnati pubblici ma vengono assegnati autonomamente se ci si aspettava che l'indirizzo \texttt{IP} venisse assegnato da un apparato esterno ma ciò non è avvenuto. Questi indirizzi sono appartenenti alla rete $194.254.0.0/16$ 
            \paragraph{Indirizzi \texttt{IP} dei \textit{router}} Un \textit{router} per definizione ha almeno $2$ interfacce di rete, quindi ha almeno $2$ indirizzi \texttt{IP}. Questo non limita un \textit{router} ad avere solo $1$ indirizzo \texttt{IP} per ogni interfaccia di rete. Da ricordare che un indirizzo \texttt{IP} non è associato ad un \textit{host} ma ad un'interfaccia di rete su un \textit{host}. La molteplicità di indirizzi \texttt{IP} per una sola interfaccia di rete risulta utile se ad esempio volgiamo suddividere la rete interna in più reti e impostare delle regole di \textit{firewall} tra le reti.
            \paragraph{come si integra al livello 2} Per l'architettura a strati del modello \texttt{TCP/IP} il messaggio contenete gli indirizzi \texttt{IP} è incapsulato in un frame del livello 2, questo frame contiene l'indirizzo \texttt{MAC} di sorgente e di destinazione. Per ottenere questi ci avvaliamo dell'uso di \texttt{ARP}.
    \subsection{\textit{Address Resolution Protocol} \texttt{ARP}}
        L'\texttt{ARP} è un protocollo che permette di associare un indirizzo \texttt{IP} ad un indirizzo \texttt{MAC}. Questo protocollo è molto utile in quanto i \textit{router} inoltrano i pacchetti in base all'indirizzo \texttt{MAC} e non all'indirizzo \texttt{IP}. Il funzionamento dell'\texttt{ARP} è il seguente: \begin{enumerate}
            \item Un \textit{host} vuole inviare un pacchetto ad un altro \textit{host} nella stessa rete e conosce l'indirizzo \texttt{IP} di destinazione ma non l'indirizzo \texttt{MAC}.
            \item L'\textit{host} invia un pacchetto di \texttt{ARP} in \textit{broadcast} con l'indirizzo \texttt{IP} di destinazione.
            \item Tutti gli \textit{host} della rete ricevono il pacchetto di \texttt{ARP} e solo l'\textit{host} con l'indirizzo \texttt{IP} di destinazione risponde con il proprio indirizzo \texttt{MAC}.
            \item L'\textit{host} che ha inviato il pacchetto di \texttt{ARP} riceve l'indirizzo \texttt{MAC} e può inviare il pacchetto.
            \item L'\textit{host} che ha inviato il pacchetto di \texttt{ARP} memorizza l'associazione tra l'indirizzo \texttt{IP} e l'indirizzo \texttt{MAC} per un certo periodo di tempo.
            \item Se l'\textit{host} vuole inviare un altro pacchetto allo stesso \textit{host} allora non invia un altro pacchetto di \texttt{ARP} ma usa l'associazione precedentemente memorizzata.
            \item Se l'associazione scade allora l'\textit{host} invia un altro pacchetto di \texttt{ARP} per rinnovare l'associazione.
            \item Se l'\textit{host} non riceve risposta al pacchetto di \texttt{ARP} allora il pacchetto non può essere inviato.
        \end{enumerate}
        Questo protocollo con questa procedura permette di associare un indirizzo \texttt{IP} ad un indirizzo \texttt{MAC} e quindi di inoltrare i pacchetti correttamente filtrando i pacchetti da apparati come gli \textit{switch} di rete che non lavorano a livello di rete, ma solo a livello di collegamento.\newline
        Questo protocollo non è mai usato su una rete pubblica, questo perché di base una destinazione pubblica prevede che il pacchetto passi per uno o più \textit{router} e quindi l'indirizzo \texttt{MAC} cambierebbe ad ogni \textit{hop} e quindi non si associa un indirizzo \texttt{MAC} ad un indirizzo \texttt{IP} pubblico, ma al suo posto gli \textit{header} dei pacchetti \texttt{IP} contengono l'indirizzo \texttt{MAC} del \textit{default gateway}.
        \paragraph{Incapsulamento frame \texttt{ARP}} Il pacchetto di \texttt{ARP} è incapsulato in un frame (esempio \textit{Ethernet}) questi frame sono interpretati come dati da trasportare e incapsulati nel \textit{payload} del frame. Il frame contiene anche un campo \textit{type} che indica il tipo di pacchetto contenuto nel \textit{payload} del frame.
        \paragraph{Il \textit{proxy} \texttt{ARP}} Il \textit{proxy} \texttt{ARP} è una tecnica che permette ad un \textit{router} di rispondere ai pacchetti di \texttt{ARP} al posto dell'\textit{host} di destinazione. Questa tecnica è molto utile in quanto permette di nascondere la topologia di rete e di proteggere gli \textit{host} dalla ricezione di pacchetti di \texttt{ARP} malevoli.
    \subsection{\textit{Internet Control Message Protocol} \texttt{ICMP}}
        Il protocollo \texttt{ICMP} è fondamentale per il funzionamento di internet, questo protocollo permette di inviare messaggi di errore e di controllo tra i dispositivi di rete.
        \paragraph{Interdipendenza con \texttt{IP}}\texttt{IP} e \texttt{ICMP} sono \underline{interdipendenti} infatti il protocollo \texttt{IP} dipende da \texttt{ICMP} per il segnalamento di errori ma a sua volta \texttt{ICMP} necessita di \texttt{IP} per il trasporto di messaggi.
        \paragraph{Formato messaggi \texttt{ICMP}} I messaggi di \texttt{ICMP} sono costituiti da un semplice \textit{header} composto da un campo \textit{type} da 1 byte, un codice di stato fatto da 1 byte e un campo \textit{checksum} di 2 byte. Il campo \textit{type} indica il tipo di messaggio, il codice di stato indica il dettaglio del messaggio e il campo \textit{checksum} è un campo di controllo degli errori.
        \subsubsection{Principali \textit{type} di \texttt{ICMP}}
            \begin{table}[H]
                \begin{tabular}{|c|c|}
                    \hline
                    \textbf{Type} & \textbf{Descrizione} \\ \hline
                    \textit{Destination Unreachable} & Indica che la destinazione non è raggiungibile \\ \hline
                    \textit{Port Unreachable} & Indica che la porta di destinazione non è raggiungibile \\ \hline
                    \textit{Time Exceeded} & Indica che il tempo di vita del pacchetto è 0 \\ \hline
                    \textit{Parameter Problem} & Indica che c'è un problema con i parametri del pacchetto \texttt{IP} \\ \hline
                    \textit{Source Quench} & Indica che il mittente deve rallentare l'invio di pacchetti \\ \hline
                    \textit{Redirect} & Indica che il mittente deve cambiare il percorso di inoltro \\ \hline
                    \textit{Echo \& Echo Reply} & Usato per il \textit{ping} \\ \hline
                    \textit{Timestamp request/reply} & Usato per ottenere il tempo di un dispositivo \\ \hline
                    \textit{Router Advertisement/solicitation} & Usato per proporsi come \textit{router} o scoprire i \textit{router} \\ \hline
                    \textit{Fragmentation needed} & Indica che il pacchetto è troppo grande e deve essere frammentato \\ \hline
                \end{tabular}
            \end{table}
        Sono presenti dunque due classi principali di messaggi: quelli per segnalare errori e quelli per recuperare informazioni. Da notare che i messaggi di \texttt{ICMP} vengono trasportati nel campo \textit{payload} di un pacchetto \texttt{IP}.
        \subsubsection{Sfruttare \texttt{ICMP}} \texttt{ICMP} può essere sfruttato per il "\textit{ping}" e per il "\textit{traceroute}". Il \textit{ping} è un comando che permette di verificare la connessione tra due dispositivi (\textit{echo request} e \textit{echo reply}), mentre il \textit{traceroute} permette di verificare il percorso che un pacchetto fa per arrivare ad un determinato dispositivo. Questo comando invia pacchetti con un campo \textit{time to live} incrementale e aspetta un messaggio di \textit{time exceeded} per sapere che il pacchetto è arrivato al \textit{router} di destinazione.

    \subsection{\textit{Dynamic Host Configuration Protocol} \texttt{DHCP}}
        Il protocollo \texttt{DHCP} è un protocollo che permette di assegnare automaticamente un indirizzo \texttt{IP} ad un \textit{host} che si connette ad una rete. Gli indirizzi possono essere liberati quando non vengono più usati e possono essere riassegnati ad altri \textit{host}. La sintesi del funzionamento del protocollo è la seguente: \begin{enumerate}
            \item \textit\textbf{\texttt{DHCP} discover} L'\textit{host} invia un pacchetto di \texttt{DHCP} in \textit{broadcast} per trovare un \textit{server} \texttt{DHCP}. Nel pacchetto sono contenuti una sorgente generica, una destinazione generica un mio \texttt{IP} e una \textit{transaction ID}.
            \item \textit\textbf{\texttt{DHCP} offer} Il \textit{server} \texttt{DHCP} risponde con un pacchetto di \texttt{DHCP} con un indirizzo \texttt{IP} disponibile per l'\textit{host}. Nel pacchetto sono contenuti l'indirizzo \texttt{IP} di sorgente del server, l'indirizzo \texttt{IP} generico di destinazione, il mio \texttt{IP} e la stessa \textit{transaction ID} del pacchetto di \texttt{DHCP} \textit{discover} e un \textit{lifetime} dell'indirizzo \texttt{IP} offerto.
            \item \textit\textbf{\texttt{DHCP} request} L'\textit{host} invia un pacchetto di \texttt{DHCP} per richiedere l'indirizzo \texttt{IP} offerto dal \textit{server} \texttt{DHCP}. 
            \item \textit\textbf{\texttt{DHCP} ack} Il \textit{server} \texttt{DHCP} risponde con un pacchetto di \texttt{DHCP} per confermare l'assegnazione dell'indirizzo \texttt{IP} all'\textit{host}.
        \end{enumerate}
        \paragraph{Prestiti \texttt{DHCP}} L'indirizzo \texttt{IP} assegnato ad un \textit{host} può essere riassegnato ad un altro \textit{host} quando l'\textit{host} non lo usa più. Un indirizzo può però essere rinnovato quando l'\textit{host} lo usa ancora. Se il \textit{server} non rinnova l'indirizzo allora l'indirizzo viene rilasciato e l'\textit{host} deve smettere di usarlo.
        \paragraph{Altri dettagli} Il protocollo \texttt{DHCP} usa \texttt{UDP} e quindi non è affidabile ma è progettato per essere robusto a perdite e a duplicati. Inoltre il \textit{client} memorizza l'indirizzo del \textit{server} \texttt{DHCP} per richieste successive.
        \subsubsection{Formato dei messaggi \texttt{DHCP}} 
            Di seguito lo schema dei messaggi \texttt{DHCP}:
            
            \begin{figure}[H]
                \centering
                \includegraphics[width=0.5\textwidth]{04/dhcp-message.png}
                \caption{Formato dei messaggi \texttt{DHCP}}
            \end{figure}
            
            Vediamo ora a cosa servono i principali parametri:
            \begin{description}
                \item[\texttt{OP}] Campo di 1 byte che indica se il messaggio è un \textit{request} o un \textit{reply}
                \item[\texttt{HTYPE} e \texttt{HLEN}] Campi di 1 byte l'uno specificano il tipo di \textit{hardware} e la lunghezza dell'indirizzo \texttt{MAC}
                \item[\texttt{FLAGS}] Campo di 2 byte che contiene, ad esempio, se il mittente può ricevere \textit{broadcast} o solo risposte dirette
                \item[\texttt{HOPS}] Campo di 1 byte che indica quanti \textit{server} hanno inoltrato la risposta
                \item[\texttt{TRANSACTION IDENTIFIER}] Campo di 4 byte che identifica la transazione
                \item[\texttt{SECONDS ELAPSED}] Campo di 2 byte che indica da quanto tempo il \textit{client} è in attesa
                \item[\textit{altri}] Ci sono altri campi che indicano l'indirizzo \texttt{IP} del \textit{client}, l'indirizzo \texttt{IP} del \textit{server}, l'indirizzo \texttt{IP} del \textit{router} e l'indirizzo \texttt{IP} del \textit{DNS}.
            \end{description}
        \subsubsection{Se non c'è un \textit{server} \texttt{DHCP}}
            Se non c'è un \textit{server} \texttt{DHCP} allora ci sono due situazioni: la prima è si configura un indirizzo \texttt{IP} statico, mentre, se non riceviamo risposta dopo un certo periodo di tempo, l'\textit{host} provvede all'impostazione di un indirizzo \textit{link-local} che permette la comunicazione con altri \textit{host} nella stessa rete che hanno lo stesso problema.
            \paragraph{Configurazione \textit{link-local}} Per configurare un indirizzo \textit{link-local} si seguono i seguenti passi: \begin{enumerate}
                \item Si sceglie un indirizzo \texttt{IP} compreso tra \texttt{169.254.0.1} e \texttt{169.254.255.254} con maschera di rete \texttt{/16}
                \item Si cerca se esiste una interfaccia di rete con un indirizzo \texttt{IP} \textit{link-local} scelto (tramite \texttt{ARP})
                \item[3.a.] Se esiste un'altra interfaccia con lo stesso indirizzo \texttt{IP} allora si ripete il processo
                \item[3.b.] Altrimenti si configura l'indirizzo \texttt{IP} \textit{link-local}
            \end{enumerate}
    \subsection{Il viaggio di un pacchetto}
        Analizziamo ora il viaggio di un pacchetto da un \textit{host} \texttt{A} ad un \textit{host} \texttt{B} in una rete. assumiamo che \texttt{A} conosca l'indirizzo \texttt{IP} di \texttt{B} e l'indirizzo \texttt{UP} del router \texttt{R} (che è il \textit{default gateway}) (Tramite \texttt{DHCP}), inoltre conosce già l'indirizzo \texttt{MAC} del router \texttt{R} (Tramite \texttt{ARP}).\newline
        Quindi in una rete costituita da \texttt{A}, \texttt{R} e \texttt{B}, con \texttt{R} nel mezzo, il pacchetto viaggia in questo modo: \begin{enumerate}
            \item \texttt{A} crea il datagramma \texttt{IP} con sorgente \texttt{A} e destinazione \texttt{B}
            \item \texttt{A} incapsula il datagramma in un frame di livello 2 con indirizzo \texttt{MAC} di \texttt{R} come destinazione e \texttt{MAC} di \texttt{A} come sorgente
            \item Il frame viene inviato alla rete da \texttt{A} a \texttt{R}\footnote{\label{netSwitch}
                In questo punto possono essere presenti all'interno della rete altri dispositivi di livello 2 come \textit{switch} che inoltrano il pacchetto in base all'indirizzo \texttt{MAC}}
            \item Il frame arriva a \texttt{R} che estrae il datagramma \texttt{IP} e lo passa al livello 3
            \item \texttt{R} controlla la tabella di routing e inoltra il pacchetto a \texttt{B}, per fare ciò incapsula il datagramma in un frame con indirizzo \texttt{MAC} di \texttt{B} come destinazione e \texttt{MAC} di \texttt{R} come sorgente
            \item Il frame viene inviato da \texttt{R} a \texttt{B}\footref{netSwitch}
            \item \texttt{B} estrae il datagramma \texttt{IP} e lo passa al livello 3 per l'elaborazione
        \end{enumerate}
    \subsection{\texttt{IPv6}}
        Il protocollo \texttt{IPv6} è un protocollo nato per ampliare la quantità di indirizzi \texttt{IP} disponibili rispetto quelli di \texttt{IPv4}, successivamente si è voluto standardizzarlo anche in quanto il formato dell'\textit{header} velocizza l'elaborazione dei frammenti, inoltre facilita la gestione della qualità del servizio.\subsubsection{Formato del datagramma \texttt{IPv6}}
            Il formato di un frammento è: \textbf{header} da $40$ byte e la frammentazione è proibita.
            \paragraph{Header} L'\textit{header} di un datagramma \texttt{IPv6} è composto da: \begin{itemize}
                \item \textbf{Flow label} Campo di $20$ bit che permette di identificare un flusso di dati.
                \item \textbf{Priority} Campo di $4$ bit che permette di identificare la priorità del pacchetto.
                \item \textbf{Next header} Campo di $8$ bit che permette di identificare il protocollo di trasporto.
            \end{itemize}
        \subsubsection{Cambiamenti rispetto \texttt{IPv4}}
            In quanto la rete è diventata più affidabile è stato rimosso il \textit{checksum}, vengono rimosse le \textit{options} e il \textit{header} è più corto. Inoltre viene introdotto il protocollo \textit{ICMPv6} con più funzionalità rispetto a \textit{ICMP}.
        \subsubsection{Transizione da \texttt{IPv4} a \texttt{IPv6}}
            La transizione da \texttt{IPv4} a \texttt{IPv6} è molto lenta in quanto richiede un cambiamento di infrastruttura molto grande, per questo si è deciso che se un pacchetto \texttt{IPv6} deve transitare obbligatoriamente per una rete \texttt{IPv4} allora il pacchetto viene incapsulato in un pacchetto \texttt{IPv4} e poi inviato alla rete \texttt{IPv4} e poi riconvertito in un pacchetto \texttt{IPv6} alla destinazione.
        \subsubsection{Indirizzi \texttt{IPv6}}
            La lunghezza non permette notazione \textit{dotted decimal} e quindi si usa la notazione esadecimale: Ogni gruppo di $4$ bit è scritto come una cifra $0,1,\dots,9$ o una lettera: $a,b,\dots,f$. In totale $32$ cifre esadecimali costituiscono un indirizzo \texttt{IPv6}.
            \paragraph{Esempio} Un indirizzo di esempio è \texttt{2a03:2880:f108:0083:face:b00c:0000:25de} che è un indirizzo \texttt{IPv6} valido.
            \paragraph{Raggruppamento} Si possono raggruppare $0$ o omettendoli oppure se è un intero gruppo di $0$ si può omettere tutto il gruppo. Ad esempio l'indirizzo \texttt{2a03:2880:f108:0083:face:b00c:0000:25de} può essere scritto come\texttt{2a03:2880:f108:83:face:b00c:0:25de}. 
        \subsubsection{Indirizzi speciali}
            Gli indirizzi speciali di \texttt{IPv6} sono: \begin{itemize}
                \item \textbf{Unspecified address} Indirizzo $0:0:0:0:0:0:0:0$ che indica che l'indirizzo non è assegnato.
                \item \textbf{Loopback address} Indirizzo $0:0:0:0:0:0:0:1$ che indica l'indirizzo di \textit{loopback}.
                \item \textbf{Link-local address} Indirizzo $fe80::/10$ che indica un indirizzo di \textit{link-local}.
                \item \textbf{Site-local address} Indirizzo $fec0::/10$ che indica un indirizzo di \textit{site-local}.
                \item \textbf{Multicast address} Indirizzo $ff00::/8$ che indica un indirizzo di \textit{multicast}.
            \end{itemize}
        