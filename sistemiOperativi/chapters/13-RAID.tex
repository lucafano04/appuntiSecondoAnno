\chapter{RAID}

Il \texttt{RAID} (\textit{Redundant Array of Independent Disks}) è una tecnologia di archiviazione dei dati che combina più dischi rigidi in un'unica unità logica per migliorare le prestazioni, la capacità e la tolleranza ai guasti. Il \texttt{RAID} è stato creato un quanto l'evoluzione tecnologica ha permesso di avere dischi rigidi sempre più piccoli (fisicamente) e sempre meno costosi, inoltre è semplice equipaggiare un sistema con più dischi rigidi, quindi è possibile sfruttare questa tecnologia per migliorare le prestazioni e la sicurezza dei dati.\newline
Gli obbiettivi principali del \texttt{RAID} sono il miglioramento dell'affidabilità e delle prestazioni. 
\subsubsection{Struttura dei dispositivi \texttt{RAID}}
    I sistemi \texttt{RAID} possono essere implementati in modi differenti: usando più dischi indipendenti collegati ad un bus ed il \texttt{SO} gestisce i dischi come un'unica unità logica, oppure usando un controller del disco \textit{hardware} che gestisce i dischi e presenta al \texttt{SO} un'unica unità logica, infine con una batteria \texttt{RAID} che è una scheda \textit{hardware} indipendente che gestisce i dischi e presenta al \texttt{SO} un'unica unità logica.
\subsubsection{Concetti base}
    Le strutture \texttt{RAID} si basano sulla copia speculare dei dati e sul sezionamento dei dati. La copia speculare dei dati è una tecnica che prevede la duplicazione dei dati su più dischi per garantire la tolleranza ai guasti. Il sezionamento dei dati è una tecnica che prevede la suddivisione dei dati in blocchi e la distribuzione di questi blocchi su più dischi per migliorare le prestazioni. Il \texttt{RAID} combinando questi permette di garantire maggior affidabilità e prestazioni.
    \paragraph{Affidabilità} Se vengono memorizzati i dati su più dischi, se uno di questi si guasta, i dati possono essere recuperati tramite la ridondanza creata. Quindi l'affidabilità cresce con il numero di dischi. Questo viene garantito tramite la copia speculare dei dati (\textit{mirroring}) dove un disco logico corrisponde a più dischi fisici, in questo caso ogni scrittura viene eseguita su entrambi i dischi ma i dati vengono persi solo se entrambi i dischi si guastano.
    \paragraph{Prestazioni} Le prestazioni possono essere migliorate tramite il sezionamento dei dati (\textit{striping}) dove i dati vengono suddivisi in blocchi e distribuiti su più dischi. In questo modo le operazioni di lettura e scrittura possono essere eseguite in parallelo su più dischi, migliorando le prestazioni complessive del sistema. Ma non si può garantire la ridondanza dei dati, quindi se un solo disco si guasta, l'intero array di dischi non è più accessibile.
    \paragraph{Sezionamento dei dati} Il sezionamento dei dati è una tecnica che prevede la suddivisione o a livello di bit o a livello di blocco. Nel \texttt{bit}-\textit{by}-\texttt{bit} ad esempio con 8 dischi l'$i$-esimo bit viene memorizzato sul $i$-esimo disco.  Mentre nel \texttt{block}-\textit{by}-\texttt{block} i dati vengono suddivisi in blocchi e distribuiti su più dischi. Ad esempio con $n$ dischi, il blocco $i$ viene memorizzato sul disco $i \mod n$. 
\section*{Livelli di \texttt{RAID}}
    I livelli di \texttt{RAID} sono delle configurazioni standardizzate che definiscono come i dati vengono distribuiti e protetti su più dischi. Ogni livello ha le proprie caratteristiche in termini di prestazioni, capacità e tolleranza ai guasti.
    \subsubsection{\texttt{RAID} 0}
        Il \texttt{RAID} 0 è una configurazione di \texttt{RAID} che utilizza il sezionamento dei dati per migliorare le prestazioni. I dati vengono suddivisi in blocchi e distribuiti su più dischi, senza alcuna ridondanza. Questo significa che se un disco si guasta, tutti i dati memorizzati nell'array sono persi. Il \texttt{RAID} 0 offre prestazioni elevate, ma non garantisce la tolleranza ai guasti. Questo livello è molto economico e permette un aumento delle prestazioni, ma non è adatto per applicazioni critiche dove la perdita di dati non è accettabile.
    \subsubsection{\texttt{RAID} 1}
        Il \texttt{RAID} 1 è una configurazione di \texttt{RAID} che utilizza la copia speculare dei dati per garantire la tolleranza ai guasti. I dati vengono duplicati su più dischi, quindi se un disco si guasta, i dati possono essere recuperati dall'altro disco. Il \texttt{RAID} 1 offre prestazioni elevate in lettura, ma le prestazioni in scrittura sono inferiori rispetto al \texttt{RAID} 0 a causa della duplicazione dei dati. Questo livello ha un alto costo in termini di capacità, ed una bassa scalabilità poiché la capacità totale dell'array è pari alla capacità del disco più piccolo. Il \texttt{RAID} 1 è adatto per piccole applicazioni critiche dove la perdita di dati non è accettabile.
    \subsubsection{\texttt{RAID} 2}
        Il livello \texttt{RAID} 2 è una configurazione di \texttt{RAID} che introduce il concetto di \texttt{bit} di parità per garantire la tolleranza ai guasti. I dati vengono suddivisi in blocchi e distribuiti su più dischi, e vengono usati tre dischi aggiuntivi per memorizzare le informazioni di parità. In questo livello su $7$ dischi $4$ memorizzano i dati, $3$ memorizzano la parità. Questo livello permette la perdita di un disco e la ricostruzione dei dati, ma non è molto efficiente in termini di costo, anche se permette una migliore prestazione rispetto al \texttt{RAID} 0. Inoltre il \texttt{RAID} 2 permette permette di risparmiare solo un disco per la parità rispetto al \texttt{RAID} 1, quindi non è molto usato.
    \subsubsection{\texttt{RAID} 3}
        Il livello \texttt{RAID} 3 è una configurazione di \texttt{RAID} che utilizza il sezionamento dei dati \textbf{in \texttt{byte}}\footnote{Tutti gli altri usano il sezionamento in blocchi} ed alloca un intero disco per memorizzare i \texttt{byte} di parità. In questo modo quando il controllore legge i dati può immediatamente calcolare la parità e verificare se i dati sono corretti. In questo modo si può garantire la tolleranza al singolo guasto di un disco e/o la corruzione dei dati di un unico \texttt{byte}. Il livello \texttt{RAID} 3 ha la stessa efficienza del \texttt{RAID} 2 in termini di lettura e scrittura, ma è più efficiente in termini di costo, in quanto non abbiamo bisogno di tre dischi per la parità. Tuttavia questo livello è meno efficiente rispetto al \texttt{RAID} 1 in termini di operazioni di \texttt{I/O} ed richiede più tempo per la scrittura dei dati dato che deve essere calcolata la parità. Infine il \texttt{RAID} 3 ha un problema di usura del disco di parità, in quanto questo non verrà usato uniformemente rispetto agli altri dischi, quindi si usurerà prima degli altri. 
    
    \subsubsection{\texttt{RAID} 4}
        Il livello \texttt{RAID} 4 è una configurazione di \texttt{RAID} che utilizza il sezionamento dei dati in blocchi e memorizza le informazioni di parità su un disco dedicato, proprio come il \texttt{RAID} 3, ma a differenza di quest'ultimo il sezionamento dei dati è a livello di blocco. Questo significa che i dati vengono suddivisi in blocchi e distribuiti su più dischi, mentre le informazioni di parità vengono memorizzate su un disco dedicato. Questo livello offre gli stessi vantaggi e svantaggi del \texttt{RAID} 3.
    \subsubsection{\texttt{RAID} 5}
        Il livello \texttt{RAID} 5 è la configurazione più comune di \texttt{RAID} e combina il sezionamento dei dati in blocchi con la distribuzione delle informazioni di parità su tutti i dischi. In questo modo si ottiene una maggiore tolleranza ai guasti e prestazioni migliori rispetto al \texttt{RAID} 3/4 poiché non c'è un disco dedicato per la parità. Il \texttt{RAID} 5 richiede almeno tre dischi e può tollerare la perdita di un disco senza perdita di dati. Tuttavia, le prestazioni in scrittura sono inferiori rispetto al \texttt{RAID} 1 come per il \texttt{RAID} 3/4, poiché è necessario calcolare la parità e scrivere i dati su più dischi. Inoltre, 
    \subsubsection{\texttt{RAID} 6}
        Il livello \texttt{RAID} 6 è una configurazione di \texttt{RAID} che estende il \texttt{RAID} 5 introducendo un secondo disco di parità. Questo significa che il \texttt{RAID} 6 può tollerare la perdita di due dischi senza perdita di dati. Il \texttt{RAID} 6 diventa però più costoso in termini di capacità e prestazioni rispetto al \texttt{RAID} 5, poiché è necessario un altro intero disco che non aumenta la capacità dell'array e le prestazioni in scrittura sono inferiori a causa del doppio calcolo della parità. Tuttavia, il \texttt{RAID} 6 è adatto per applicazioni critiche dove la perdita di dati non è accettabile e si desidera una maggiore tolleranza ai guasti.
    \subsubsection{\texttt{RAID} 0+1}
        Il livello \texttt{RAID} 0+1 è una configurazione di \texttt{RAID} che combina diverse configurazioni \texttt{RAID} 0 e le combina in un unico \texttt{RAID} 1. Ad esempio supponendo di avere a disposizione $6$ dischi allora combineremo i primi $3$ dischi in un \texttt{RAID} 0 e gli altri $3$ in un \texttt{RAID} 0 ed infine combineremo i due \texttt{RAID} 0 in un \texttt{RAID} 1, in questo modo triplichiamo la capacità e la velocità di lettura e scrittura, ed abbiamo una buona tolleranza ai guasti. Tuttavia, se perdiamo due dischi di due \texttt{RAID} 0 diversi, perdiamo tutti i dati, la tolleranza ai guasti è quindi limitata allo stesso gruppo di dischi. Inoltre, il \texttt{RAID} 0+1 ha un costo elevato in termini di capacità.
    \subsubsection{\texttt{RAID} 1+0}
        Il livello \texttt{RAID} 1+0 è una configurazione di \texttt{RAID} che combina diverse configurazioni \texttt{RAID} 1 e le combina in un unico \texttt{RAID} 0. Ad esempio supponendo di avere a disposizione $6$ dischi allora combineremo i dischi $2$ a $2$ in modo da ottenere $3$ array \texttt{RAID} 1, e poi combineremo i tre array in un \texttt{RAID} 0. In questo modo come per il \texttt{RAID} 0+1 triplichiamo la capacità e la velocità di lettura e scrittura, ed abbiamo una buona tolleranza ai guasti. Tuttavia, se perdiamo entrambi i dischi di un array \texttt{RAID} 1, perdiamo tutti i dati, la tolleranza ai guasti è quindi limitata ad al massimo un disco per array \texttt{RAID} 1. Inoltre, il \texttt{RAID} 1+0 ha un costo molto elevato in termini di capacità.