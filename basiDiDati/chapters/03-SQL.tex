\chapter[\texttt{SQL}]{\texttt{SQL} - \textit{Structured Query Language}}
\thispagestyle{chapterInit}
\section{Definizioni}
    \texttt{\textbf{SQL}} o \textbf{\textit{Structured Query Language}} è un linguaggio per la gestione di basi di dati relazionali. Questo è usato per la creazione, modifica e interrogazione dei dati. Oggi questo è standardizzato ma ne esistono diversi "dialetti" a seconda del \texttt{DBMS} utilizzato.
    \paragraph{\texttt{DDL} - \textit{Data Definition Language}} 
        Per \texttt{DDL} o \textbf{\textit{Data Definition Language}} è il sottoinsieme di comandi (o più precisamente \textit{query}) che permettono di definire la struttura di un database
    \paragraph{\texttt{DML} - \textit{Data Manipulation Language}}
        Per \texttt{DML} o \textbf{\textit{Data Manipulation Language}} intendiamo il sottoinsieme \textit{query} che permettono di:\begin{itemize}
            \item Inserire dati
            \item Aggiornare dati
            \item Cancellare dati
            \item Ottenere dati
        \end{itemize}
\section{Struttura Generale delle \textit{query} \texttt{SQL}}
    \subsection{Comando \texttt{SELECT} - interrogazione dei dati}
        Il comando \texttt{SELECT} è il comando principale per l'interrogazione dei dati. La sua struttura generale è la seguente:
        \begin{lstlisting}
SELECT [DISTINCT] <lista di attributi>
FROM <lista di relazioni>
WHERE <condizione>
        \end{lstlisting}
        \begin{description}
            \item[\<lista di relazioni\>] è la lista dei nomi delle relazioni da cui si vogliono estrarre i dati. Queste possono anche definire solo una porzione di una tabella (tramite variabili di range)
            \item[\<lista di attributi\>] è la lista degli attributi che si vogliono estrarre. Questi possono essere anche espressioni aritmetiche o funzioni
            \item[\<condizione\>] è un insieme di condizioni con predicati di confronto: ($<,>,=,\leq, \geq,\neq $). Questi predicati possono essere combinati con gli operatori logici \texttt{AND, OR, NOT}
            \item[\texttt{DISTINCT}] è un modificatore opzionale che permette di eliminare i duplicati, di default \underline{NON} vengono eliminati
        \end{description}
        \subsubsection{Esecuzione di una \textit{query} \texttt{SELECT}}
            Questo genere di \textit{query} viene eseguito seguendo i seguenti passaggi:
            \begin{enumerate}
                \item Si calcola il prodotto cartesiano delle relazioni coinvolte in \<lista di relazioni\>
                \item Si prendono ora solo le tuple che soddisfano la o le condizioni specificate in \<condizione\> (se presenti)
                \item Si proiettano ora solo gli attributi specificati in \<lista di attributi\> rimuovendo gli attributi non specificati
                \item Se presente il modificatore \texttt{DISTINCT} si eliminano i duplicati
            \end{enumerate}
        \subsubsection{Operatori di Confronto} 
            Gli operatori di confronto usabili nella clausola \texttt{WHERE} del comando \texttt{SELECT} possono essere differenti per i diversi tipi di dati e \texttt{DBMS}. I principali operatori sono:
                \begin{enumerate}
                    \item Operatori aritmetici ($=, \neq, <, \leq, >, \geq$)
                    \item \texttt{BETWEEN} - per verificare se un valore è compreso in un intervallo (chiuso)
                    \item \texttt[IN] - per verificare se un valore è contenuto in una lista di valori
                    \item \texttt{LIKE} - per verificare se un valore è simile ad una stringa contenente caratteri jolly (\texttt{\%} e \texttt{\_})
                    \item \texttt{IS [NOT] NULL} - per verificare se un valore è \texttt{NULL} o meno. In quanto \texttt{NULL} è uno stato particolare che può assumere un attributo scrivere \texttt{WHERE attributo = NULL} non ha senso ed è sbagliato, si deve usare \texttt{IS NULL}
                \end{enumerate}
        \subsubsection{Operatori di aggregazione}
            \texttt{SQL} supporta 5 operatori di aggregazione:
            \begin{description}
                \item[\texttt{count(*)}] - conta il numero di tuple risultanti dalla \textit{query}
                \item[\texttt{count([DISTINCT] A)}] - conta il numero di tuple risultanti dalla \textit{query} che hanno un valore non \texttt{NULL} per l'attributo \texttt{A}, se \texttt{DISTINCT} è presente vengono conteggiate solo le tuple con valori distinti
                \item[\texttt{sum([DISTINCT] A)}] - somma i valori dell'attributo \texttt{A} per le tuple risultanti dalla \textit{query}, se \texttt{DISTINCT} è presente vengono sommati solo i valori distinti
                \item[\texttt{avg([DISTINCT] A)}] - calcola la media dei valori dell'attributo \texttt{A} per le tuple risultanti dalla \textit{query}, se \texttt{DISTINCT} è presente vengono considerati solo i valori distinti
                \item[\texttt{max(A)}] - restituisce il valore massimo dell'attributo \texttt{A} per le tuple risultanti dalla \textit{query}
                \item[\texttt{min(A)}] - restituisce il valore minimo dell'attributo \texttt{A} per le tuple risultanti dalla \textit{query}
            \end{description}
        \subsubsection{Operatori \texttt{GROUP BY} \& \texttt{HAVING}}
            Gli operatori \texttt{GROUP BY} e \texttt{HAVING} permettono di eseguire operazioni di aggregazione solo su sottoinsiemi di tuple.
            In particolare \texttt{GROUP BY} permette di raggruppare le tuple in base ai valori di uno o più attributi, mentre \texttt{HAVING} permette di filtrare i risultati in base a condizioni di aggregazione.
            \begin{lstlisting}
SELECT <lista di attributi>, <funzione di aggregazione>
FROM <lista di relazioni>
WHERE <condizione>
GROUP BY <lista di attributi di raggruppamento>
HAVING <condizione di aggregazione>
            \end{lstlisting}
            \begin{description}
                \item[\<lista di attributi di raggruppamento\>] è la lista degli attributi su cui si vuole raggruppare le tuple
                \item[\<condizione di aggregazione\>] è la condizione che deve essere soddisfatta dal risultato della funzione di aggregazione per essere incluso nel risultato
                \item[\<funzione di aggregazione\>] è una o più funzioni di aggregazione da applicare ai gruppi di tuple
            \end{description}
        \subsubsection{Operatori insiemistici}
            Gli operatori insiemistici permettono di combinare i risultati di più \textit{query} in un unico risultato. Gli operatori insiemistici sono:
            \begin{description}
                \item[\texttt{UNION}] - restituisce l'unione dei risultati di due \textit{query} (eliminando i duplicati)\footnote{In quanto l'operatore \texttt{UNION} elimina i duplicati se si sta cercando di unire due query allora non è necessario usare il modificatore \texttt{DISTINCT} in quanto ordinamento e rimozione dei duplicati sono già garantiti dall'operatore}
                \item[\texttt{UNION ALL}] - restituisce l'unione dei risultati di due \textit{query} (senza eliminare i duplicati e ordinare i risultati)
                \item[\texttt{INTERSECT}] - restituisce l'intersezione dei risultati di due \textit{query}
                \item[\texttt{EXCEPT}] - restituisce la differenza tra i risultati di due \textit{query}
            \end{description}
        \subsubsection{Operatore \texttt{JOIN}}
            L'operatore \texttt{JOIN} permette di combinare i risultati di due \textit{query} in un unico risultato. Gli operatori \texttt{JOIN} sono:
            \begin{description}
                \item[\texttt{INNER JOIN}] - restituisce le tuple che hanno un valore comune in entrambe le relazioni
                \item[\texttt{LEFT JOIN}] - restituisce tutte le tuple della relazione di sinistra e le tuple della relazione di destra che hanno un valore comune
                \item[\texttt{RIGHT JOIN}] - restituisce tutte le tuple della relazione di destra e le tuple della relazione di sinistra che hanno un valore comune
                \item[\texttt{FULL JOIN}] - restituisce tutte le tuple delle relazioni coinvolte
            \end{description}
% Più avanti verranno aggiunti altri comandi SQL, non si creeranno nuovi capitoli ma tutto quello che riguarda SQL verrà aggiunto qui